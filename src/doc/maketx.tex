\chapter{Making Tiled MIP-Map Texture Files With {\cf maketx}}
\label{chap:maketx}
\indexapi{maketx}

\section{Overview}

The \maketx program will read an image (from any file format for
which an \ImageInput plugin can be found) and then write it in a form
in which it will have high performance when used by \TextureSystem
(Chapter~\ref{chap:texturesystem}).  This involves converting it to
tiled (versus scanline) orientation, writing multiple subimages at
different resolutions (MIP-map), and setting a variety of header or
metadata fields appropriately for texture maps.

The \maketx utility is invoked as follows:

\medskip

\hspace{0.25in} {\cf maketx} [\emph{options}] \emph{input}... -o \emph{output}

\medskip

Where \emph{input} and \emph{output} name the input image and desired
output filename.  The input files may be of any image format recognized by
\product (i.e., for which \ImageInput plugins are available).  The file
format of the output image will be inferred from the file extension of
the output filename (e.g., \qkw{foo.tif} will write a TIFF file).


\section{{\cf maketx} command-line options}

\apiitem{--help}
Prints usage information to the terminal.
\apiend

\apiitem{-v}
Verbose status messages, including runtime statistics and timing.
\apiend

\apiitem{-o {\rm \emph{outputname}}}
Sets the name of the output texture.
\apiend

\apiitem{--format {\rm \emph{formatname}}}
Specifies the image format of the output file (e.g., ``tiff'',
``OpenEXR'', etc.).  If {\cf --format} is not used, \maketx will 
guess based on the file extension of the output filename; if it
is not a recognized format extension, TIFF will be used by default.
\apiend

\apiitem{-d {\rm \emph{datatype}}}
Attempt to sets the output pixel data type to one of: {\cf uint8}, 
{\cf sint8}, {\cf uint16}, {\cf sint16}, {\cf half}, {\cf float}, 
{\cf double}.

If the {\cf -d} option is not supplied, the output data type will
be the same as the data format of the input file.

In either case, the output file format itself (implied by the file
extension of the output filename) may trump the request if the file
format simply does not support the requested data type.
\apiend

\apiitem{--tile {\rm \emph{x}} {\rm \emph{y}}}
Specifies the tile size of the output texture.  If not specified,
\maketx will make $64 \times 64$ tiles.
\apiend

\apiitem{--separate}
Forces ``separate'' (e.g., RRR...GGG...BBB) packing of channels in the
output file.  Without this option specified, ``contiguous'' (e.g.,
RGBRGBRGB...) packing of channels will be used for those file formats
that support it.
\apiend

\apiitem{--update}
Ordinarily, textures are created unconditionally (which could take
several seconds for large input files if read over a network) and will
be stamped with the current time.

The {\cf --update} option enables \emph{update mode}I if the output file
already exists and has the same time stamp as the input file, the
texture will not be recreated.  If the output file does not exist or has
a different time than the input file, then the texture will be created
be given the time stamp of the input file.
\apiend

\apiitem{--wrap {\rm \emph{wrapmode}} \\
--swrap {\rm \emph{wrapmode}} --twrap {\rm \emph{wrapmode}}}
Sets the default \emph{wrap mode} for the texture, which determines
the behavior when the texture is sampled outside the $[0,1]$ range.
Valid wrap modes are: {\cf black}, {\cf clamp}, {\cf periodic},
{\cf mirror}.  The default, if none is set, is {\cf black}.  The
{\cf --wrap} option sets the wrap mode in both directions
simultaneously, while the {\cf --swrap} and {\cf --twrap} may be used to
set them individually in the $s$ (horizontal) and $t$ (vertical)
diretions.

Although this sets the default wrap mode for a texture, note that
the wrap mode may have an override specified in the texture lookup
at runtime.
\apiend

\apiitem{--resize}
Causes the highest-resolution level of the MIP-map to be a
power-of-two resolution in each dimension
(by rounding up the resolution of the input image).  There is no
good reason to do this for the sake of OIIO's texture system, but 
some users may require it in order to create MIP-map images
that are compatible with both OIIO and other texturing systems that
require power-of-2 textures.
\apiend

\apiitem{--nomipmap}
Causes the output to \emph{not} be MIP-mapped, i.e., only will have
the highest-resolution level.
\apiend

\apiitem{--checknan}
Checks every pixel of the input image to ensure that no NaN or Inf
values are present.  If such non-finite pixel values are found, 
an error message will be printed and {\cf maketx} will terminate without
writing the output image (returning an error code).
\apiend

\apiitem{--ingamma {\rm \emph{value}} \\
--outgamma {\rm \emph{value}}}
Not currently implemented
\apiend

\apiitem{--Mcamera {\rm \emph{...16 floats...}} \\
--Mscreen {\rm \emph{...16 floats...}}}
Sets the camera and screen matrices (sometimes called {\cf Nl} and
{\cf NP}, respectively, by some renderers) in the texture file, 
overriding any such matrices that may be in the input image (and would
ordinarily be copied to the output texture).
\apiend

\apiitem{--hash}
Computes a SHA-1 hash on the input file's pixels and embeds this hash
in the ``ImageDescription'' metadata of the output texture.  This is
useful in helping the \TextureSystem identify duplicate textures at
runtime.
\apiend

% --shadow --shadcube
% --volshad --envlatl --envcube --lightprobe --latl2envcube --vertcross
% --fov
% --opaquewidth


\begin{comment}

\section{{\cf maketx} Recipes}

% FIXME

This section will give quick examples of common uses of {\cf maketx}.

\subsection*{Converting between file formats}

It's a snap to converting among image formats supported by \product
(i.e., for which \ImageInput and \ImageOutput plugins can be found).
The {\cf maketx} utility will simply infer the file format from the
file extension. The following example converts a PNG image to JPEG:

\begin{code}
    maketx lena.png lena.jpg
\end{code}

\end{comment}
