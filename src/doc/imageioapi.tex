\chapter{Image I/O API}
\label{chap:imageioapi}
\index{Image I/O API|(}


\section{Image Format Spec}
\indexapi{ImageIOFormatSpec}

An {\kw ImageIOFormatSpec} is a structure that describes the complete
format specification of a single image.  It contains:

\begin{itemize}
\item The image resolution (number of pixels).
\item The origin, if it is not located beginning at pixel (0,0).
\item The full size of any ``surrounding'' image, if this represents a crop
  window.
\item Whether the image is organized into \emph{tiles}, and if so, the
  tile size.
\item The \emph{data format} of the pixel values (e.g., float, 8-bit
  integer, etc.).
\item The number of color channels in the image (e.g., 3 for RGB
  images).
\item Which channel, if any, represents \emph{alpha} and \emph{depth}.
\item Any presumed gamma correction already applied to the pixel values.
\item Quantization parameters describing how floating point values
  should be converted to integers (if cases where users pass real values
  but integer values are stored in the file).  This is used only when
  writing images, not when reading them.
\item A user-extensible (and format-extensible) list of any other
  arbitrarily-named and -typed data that may help describe the image or
  its disk representation.
\end{itemize}

\subsection{{\kw ImageIOFormatSpec} Data Members}

The {\kw ImageIOFormatSpec} contains data fields for the values that are
required to describe nearly any image, and an extensible list of
arbitrary attributes that can hold metadata that may be user-defined or
specific to individual file formats.  Here are the hard-coded data
fields:

\apiitem{int width, height, depth}
The size of the data of this image, i.e., the number pixels in each
dimension.  If {\kw depth} is 0 or 1, it indicates a 2D image, but if
{\kw depth} is greater than 1, it's actually a 3D ``volumetric'' image.
\apiend

\apiitem{int x, y, z}
The \emph{origin} of the image.  These default to (0,0,0), but setting
them differently indicates that this image is actually a \emph{crop window}
of a larger surrounding image.
\apiend

\apiitem{int full_width, full_height, full_depth}
The size of the full surrounding image, if this image is a crop window.
The default is (0,0,0), indicating that this is the ``full'' image data
(setting these values to {\kw width}, {\kw height}, and {\kw depth},
respectively, has the same meaning).
\apiend

\apiitem{int tile_width, tile_height, tile_depth}
If nonzero, indicates that the image is stored on disk organized into
rectangular \emph{tiles} of the given dimension.  The default of 
(0,0,0) indicates that the image is stored in scanline order, rather
than as tiles.
\apiend

\apiitem{ParamBaseType format}
Indicates the format of the pixel data values themselves, as a 
{\kw ParamBaseType} (see \ref{ParamBaseType}).  Typical values would be
{\kw PT_UINT8} for 8-bit unsigned values, {\kw PT_FLOAT} for 32-bit
floating-point values, etc.

\noindent NOTE: Currently, the implementation of OpenImageIO requires
all channels to have the same data format.
\apiend

\apiitem{int nchannels}
The number of \emph{channels} (color values) present in each pixel of
the image.  For example, an RGB image has 3 channels.
\apiend

\apiitem{std:vector<std::string> channelnames}
The names of each channel, in order.  Typically this will be "R", "G",
"B", "A" (alpha), "Z" (depth), or other arbitrary names.
\apiend

\apiitem{int alpha_channel}
The index of the channel that respresents \emph{alpha} (pixel coverage
and/or transparency).  It defaults to -1 if no alpha channel is present,
or if it is not know which channel represents alpha.
\apiend

\apiitem{int z_channel}
The index of the channel that respresents \emph{z} or \emph{depth} (from
the camera).  It defaults to -1 if no depth channel is present, or if it
is not know which channel represents depth.
\apiend

\apiitem{LinearitySpec nonlinear}
Describes the type of nonlinearity, if any, that is present in the
mapping of pixel values to real-world units.  {\kw LinearitySpec} is
an enumerated type that may take on the following values:
\begin{itemize}
\item[] 
\item {\kw Linear} (the default) indicates that pixel values map
  linearly.
\item {\kw GammaCorrected} indicates that the color pixel values have
  already been gamma corrected, using the exponent given by the {\kw
    gamma} field.  (It is still assumed that non-color values, such as
  alpha and depth, are linear.)
\item {\kw sRGB} indicates that color values are encoded using the sRGB
  mapping.  (It is still assumed that non-color values are linear.)
\end{itemize}
\apiend

\apiitem{float gamma}
The gamma exponent, if the pixel values in the image have already been
gamma corrected (indicated by {\kw nonlinear} having a value of {\kw
GammaCorrected}).  The default of 1.0 indicates that no gamma
correction has been applied.
\apiend

\apiitem{int quant_black, quant_white, quant_min, quant_max;\\
  float quant_dither}
Describes the \emph{quantization}, or mapping between real
(floating-point) values and the stored integer values.
FIXME - describe this better.
\apiend

\apiitem{std::vector<ImageIOParameter> extra_attribs}
A list of arbitrarily-named and arbitrarily-typed additional attributes
of the image, for any metadata not described by the hard-coded fields
described above.  This list may be manipulated with the {\kw
attribute()} and {\kw find_attribute()} methods.
\apiend

\subsection{{\kw ImageIOFormatSpec} member functions}

\noindent {\kw ImageIOFormatSpec} contains the following methods that
manipulate format specs or compute useful information about images given
their format spec:

\apiitem{ImageIOFormatSpec (int xres, int yres, int nchans, \\
   \bigspc ParamBaseType fmt = PT_UINT8)}
Constructs an ImageIOFormatSpec with the $x$ and $y$ resolution, number
of channels, and pixel data format.

All other fields are set to the obvious defaults -- the image is an
ordinary 2D image (not a volume), the image is not offset or a crop of a
bigger image, the image is scanline-oriented (not tiled), channel names
are ``R'', ``G'', ``B,'' and ``A'' (up to and including 4 channels,
beyond that they are named ``channel \emph{n}''), the fourth channel (if
it exists) is assumed to be alpha, values are assumed to be linear, and
quantizaiton (if \emph{fmt} describes an integer type) is done in
such a way that the maximum positive integer range maps to (0.0, 1.0).
\apiend

\apiitem{void set_format (ParamBaseType fmt)}
Sets the format as described, and also sets all quantization parameters
to the default for that data type (maps the maximum positive integer
range to (0.0, 1.0)).
\apiend

\apiitem{static ParamBaseType format_from_quantize (int quant_black, int quant_white,\\
\bigspc \bigspc                          int quant_min, int quant_max)}
Utility function that, given quantization parameters, returns a data
type that may be used without unacceptable loss of significant bits.
% FIXME - elaborate?
\apiend

\apiitem{size_t channel_bytes ()}
Returns the number of bytes comprising each channel of each pixel (i.e.,
the size of a single value of the type described by the {\kw format} field).
\apiend

\apiitem{size_t pixel_bytes ()}
Returns the number of bytes comprising each pixel (i.e. the number of
channels multiplied by the channel size).
\apiend

\apiitem{size_t scanline_btes ()}
Returns the number of bytes comprising each scanline (i.e. {\kw width} pixels).
\apiend

\apiitem{size_t tile_bytes ()}
Returns the number of bytes comprising an image tile (if it's a tiled image).
\apiend

\apiitem{size_t image_bytes ()}
Returns the number of bytes comprising an image of these dimensions.
\apiend

% FIXME - document auto_stride() ?

\apiitem{void attribute (const std::string \&name, ParamBaseType type, \\
\bigspc int nvalues, const void *value)}
Add a metadata attribute to {\kw extra_attribs}, with the given name,
data type, and number of values.  The {\kw value} pointer specifies
the address of the data to be copied.
\apiend

\apiitem{void attribute (const std::string \&name, unsigned int value)\\
    void attribute (const std::string \&name, int value)\\
    void attribute (const std::string \&name, float value)\\
    void attribute (const std::string \&name, const char *value)\\
    void attribute (const std::string \&name, const std::string \&value)}
Shortcuts for passing attributes comprised of a single integer,
floating-point value, or string.
\apiend

\apiitem{ImageIOParameter * find_attribute (const std::string \&name,\\
\bigspc\bigspc                              bool casesensitive=false)}
Searches {\kw extra_attribs} for an attribute matching {\kw name}
(exactly, if {\kw casesensitive} is true, otherwise in a
case-insensitive manner) and returns the pointer to that attribute
record.
\apiend




\index{Image I/O API|)}

\chapwidthend
