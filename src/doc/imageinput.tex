\chapter{Image I/O: Reading Images}
\label{chap:imageinput}
\index{Image I/O API|(}


\section{Image Input Made Simple}
\label{sec:imageinput:simple}

Here is the simplest sequence required to open an image file, find
out its resolution, and read the pixels (converting them into
8-bit values in memory, even if that's not the way they're stored in the file):

\begin{code}
        #include "imageio.h"
        using namespace OpenImageIO;
        ...

        const char *filename = "foo.jpg";
        int xres, yres, channels;
        unsigned char *pixels;

        ImageInput *in = ImageInput::create (filename);
        ImageSpec spec;
        in->open (filename, spec);
        xres = spec.width;
        yres = spec.height;
        channels = spec.nchannels;
        pixels = new unsigned char [xres*yres*channels];
        in->read_image (TypeDesc::UINT8, pixels);
        in->close ();
        delete in;
\end{code}

\noindent Here is a breakdown of what work this code is doing:

\begin{itemize}
\item Search for an ImageIO plugin that is capable of reading the file
  (\qkw{foo.jpg}), first by trying to deduce the correct plugin from the
  file extension, but if that fails, by opening every ImageIO plugin it
  can find until one will open the file without error.  When it finds
  the right plugin, it creates a subclass instance of \ImageInput that
  reads the right kind of file format.
  \begin{code}
        ImageInput *in = ImageInput::create (filename);
  \end{code}
\item Open the file, read the header, and put all relevant metadata
  about the file in a specification structure.
  \begin{code}
        ImageSpec spec;
        in->open (filename, spec);
  \end{code}
\item The specification contains vital information such as the
  dimensions of the image, number of color channels, and data type of
  the pixel values.  This is enough to allow us to allocate enough space
  for the image.
  \begin{code}
        xres = spec.width;
        yres = spec.height;
        channels = spec.nchannels;
        pixels = new unsigned char [xres*yres*channels];
  \end{code}
  Note that in this example, we don't care what data format is used for
  the pixel data in the file --- we allocate enough space for unsigned
  8-bit integer pixel values, and will rely on \product's ability to
  convert to our requested format from the native data format of the
  file.
\item Read the entire image, hiding all details of the encoding of image
  data in the file, whether the file is scanline- or tile-based, or what
  is the native format of the data in the file (in this case, we request
  that it be automatically converted to unsigned 8-bit integers).
  \begin{code}
        in->read_image (TypeDesc::UINT8, pixels);
  \end{code}
\item Close the file, destroy and free the \ImageInput we had created,
  and perform all other cleanup and release of any resources used by
  the plugin.
  \begin{code}
        in->close ();
        delete in;
  \end{code}
\end{itemize}



\section{Advanced Image Input}
\label{sec:advancedimageinput}

Let's walk through some of the most common things you might want to do,
but that are more complex than the simple example above.


\subsection{Reading individual scanlines and tiles}
\label{sec:imageinput:scanlinestiles}

The simple example of Section~\ref{sec:imageinput:simple} read an
entire image with one call.  But sometimes you want to read a large
image a
little at a time and do not wish to retain the entire image in memory
as you process it.  \product allows you to read images
one scanline at a time or one tile at a time.

\subsubsection{Reading individual scanlines}

Individual scanlines may be read using the \readscanline API
call:

\begin{code}
        ...
        in->open (filename, spec);
        unsigned char *scanline = new unsigned char [spec.width*spec.channels];
        for (int y = 0;  y < yres;  ++y) {
            in->read_scanline (y, 0, TypeDesc::UINT8, scanline);
            ... process data in scanline[0..width*channels-1] ...
        }
        delete [] scanline;
        in->close ();
        ...
\end{code}

The first two arguments to \readscanline specify which scanline
is being read by its vertical ($y$) scanline number (beginning with 0)
and, for volume images, its slice ($z$) number (the slice number should
be 0 for 2D non-volume images).  This is followed by a \TypeDesc
describing the data type of the pixel buffer you are supplying, and a
pointer to the pixel buffer itself.  Additional optional arguments
describe the data stride, which can be ignored for contiguous data (use
of strides is explained in Section~\ref{sec:imageinput:strides}).

All \ImageInput implemenations will read scanlines in strict order
(starting with scanline 0, then 1, up to {\kw yres-1}, without skipping
any).  However, it's probably uncommon for an \ImageInput to properly
allow reading of out-of-order scanlines.

The full description of the \readscanline function may be found
in Section~\ref{sec:imageinput:reference}.

\subsubsection{Reading individual tiles}

Once you {\kw open()} an image file, you can find out if it is a tiled
image (and the tile size) by examining the \ImageSpec's {\cf
  tile_width}, {\cf tile_height}, and {\cf tile_depth} fields.
If they are zero, it's a scanline image and you should read pixels
using \readscanline, not \readtile.

\begin{code}
        ...
        in->open (filename, spec);
        if (spec.tile_width == 0) {
            ... read by scanline ...
        } else {
            // Tiles
            int tilesize = spec.tile_width * spec.tile_height;
            unsigned char *tile = new unsigned char [tilesize * spec.channels];
            for (int y = 0;  y < yres;  y += spec.tile_height) {
                for (int x = 0;  x < xres;  x += spec.tile_width) {
                    in->read_tile (x, y, 0, TypeDesc::UINT8, tile);
                    ... process the pixels in tile[] ..
                }
            }
            delete [] tile;
        }
        in->close ();
        ...
\end{code}

The first three arguments to \readtile specify which tile is
being read by the pixel coordinates of any pixel contained in the
tile: $x$ (column), $y$ (scanline), and $z$ (slice, which should always
be 0 for 2D non-volume images).  This is followed by a \TypeDesc
describing the data format of the pixel buffer you are supplying, and a
pointer to the pixel buffer.  Pixel data will be written to your buffer
in order of increasing slice, increasing
scanline within each slice, and increasing column within each scanline.
Additional optional arguments describe the data stride, which can be
ignored for contiguous data (use of strides is explained in
Section~\ref{sec:imageinput:strides}).

All \ImageInput implemenations are required to support reading tiles in
arbitrary order (i.e., not in strict order of increasing $y$ rows, and
within each row, increasing $x$ column, without missing any tiles).

The full description of the \readtile function may be found
in Section~\ref{sec:imageinput:reference}.


\subsection{Converting formats}
\label{sec:imageinput:convertingformat}

The code examples of the previous sections all assumed that your
internal pixel data is stored as unsigned 8-bit integers (i.e., 0-255
range).  But \product is significantly more flexible.  

You may request that the pixels be stored in any of several formats.
This is done merely by passing the {\cf read} function the data type
of your pixel buffer, as one of the enumerated type \TypeDesc.

%FIXME
%Individual file formats, and therefore \ImageInput implementations, may
%only support a subset of the formats understood by the \product library.
%Each \ImageInput plugin implementation should document which data
%formats it supports.  An individual \ImageInput implementation may
%choose to simply fail open {\kw open()}, though the recommended behavior
%is for {\kw open()} to succeed but in fact choose a data format
%supported by the file format that best preserves the precision and range
%of the originally-requested data format.

It is not required that the pixel data buffer passed to \readimage,
\readscanline, or \readtile actually be in the same data format as the
data in the file being read.  \product will automatically convert from
native data type of the file to the internal data format of your choice.
For example. the following code will open a TIFF and read pixels into
your internal buffer represented as {\cf float} values.  This will work
regardless of whether the TIFF file itself is using 8-bit, 16-bit, or
float values.

\begin{code}
        ImageInput *in = ImageInput::create ("myfile.tif");
        ImageSpec spec;
        in->open (filename, spec);
        ...
        int numpixels = spec.width * spec.height;
        float pixels = new float [numpixels * channels];
        ...
        in->read_image (TypeDesc::FLOAT, pixels);
\end{code}

\noindent Note that \readscanline and \readtile have a parameter that
works in a corresponding manner.

You can, of course, find out the native type of the file simply by
examining {\cf spec.format}.  If you wish, you may then allocate a
buffer big enough for an image of that type and request the native type
when reading, therefore eliminating any translation among types and
seeing the actual numerical values in the file.

%FIXME
%Please refer to Section~\ref{sec:imageinput:quantization} for more
%information on how values are translated among the supported data
%formats by default, and how to change the formulas by specifying
%quantization in the \ImageSpec.


\subsection{Data Strides}
\label{sec:imageinput:strides}

In the preceeding examples, we have assumed that the buffer passed to
the {\cf read} functions (i.e., the place where you want your pixels
to be stored) is \emph{contiguous}, that is:

\begin{itemize}
\item each pixel in memory consists of a number of data values equal to
  the number of channels in the file;
\item successive column pixels within a row directly follow each other in
  memory, with the first channel of pixel $x$ of immediately following
  last channel of pixel $x-1$ of the same row;
\item for whole images or tiles, the data for each row
  immediately follows the previous one in memory (the first pixel of row
  $y$ immediately follows the last column of row $y-1$);
\item for 3D volumetric images, the first pixel of slice $z$ immediately
  follows the last pixel of of slice $z-1$.
\end{itemize}

Please note that this implies that \readtile will write pixel data into
your buffer so that it is contiguous in the shape of a single tile, not
just an offset into a whole image worth of pixels.

The \readscanline function takes an optional {\cf xstride} argument, and
the \readimage and \readtile functions take optional {\cf xstride}, 
{\cf ystride}, and {\cf zstride} values that describe the distance, in
\emph{bytes}, between successive pixel columns, rows, and slices,
respectively, of your pixel buffer.  For any of these values that are
not supplied, or are given as the special constant {\cf AutoStride},
contiguity will be assumed.

By passing different stride values, you can achieving some surprisingly
flexible functionality.  A few representative examples follow:

\begin{itemize}
\item Flip an image vertically upon reading, by using \emph{negative}
  $y$ stride:
  \begin{code}
        unsigned char pixels[spec.width * spec.height * spec.nchannels];
        int scanlinesize = spec.width * spec.nchannels * sizeof(pixels[0]);
        ...
        in->read_image (TypeDesc::UINT8,
                        (char *)pixels + (yres-1)*scanlinesize,  // offset to last
                        AutoStride,                      // default x stride
                        -scanlinesize,                   // special y stride
                        AutoStride);                     // default z stride
  \end{code}
\item Read a tile into its spot in a buffer whose layout matches
  a whole image of pixel data,
  rather than having a one-tile-only memory layout:
  \begin{code}
        unsigned char pixels[spec.width * spec.height * spec.nchannels];
        int pixelsize = spec.nchannels * sizeof(pixels[0]);
        int scanlinesize = xpec.width * pixelsize;
        ...
        in->read_tile (x, y, 0, TypeDesc::UINT8,
                       (char *)pixels + y*scanlinesize + x*pixelsize,
                       pixelsize,
                       scanlinesize);
  \end{code}
\end{itemize}

Please consult Section~\ref{sec:imageinput:reference} for detailed
descriptions of the stride parameters to each {\cf read} function.


\subsection{Reading metadata}
\label{sec:imageinput:metadata}

The \ImageSpec that is filled in by {\cf ImageInput::open()}
specifies all the common properties that describe an image: data format,
dimensions, number of channels, tiling.  However, there may be a variety
of additional \emph{metadata} that are present in the image file and
could be queried by your application.

The remainder of this section explains how to query additional metadata
in the \ImageSpec.  It is up to the \ImageInput to read these
from the file, if indeed the file format is able to carry additional
data.  Individual \ImageInput implementations should document which
metadata they read.

\subsubsection{Channel names}

In addition to specifying the number of color channels, the
\ImageSpec also stores the names of those channels in its {\cf
  channelnames} field, which is a {\cf vector<std::string>}.  Its length
should always be equal to the number of channels (it's the
responsibility of the \ImageInput to ensure this).

Only a few file formats (and thus \ImageInput implementations) have a
way of specifying custom channel names, so most of the time you will see
that the channel names follow the default convention of being named
\qkw{R}, \qkw{G}, \qkw{B}, and \qkw{A}, for red, green, blue, and alpha,
respectively.

Here is example code that prints the names of the channels in an image:

\begin{code}
        ImageInput *in = ImageInput::create (filename);
        ImageSpec spec;
        in->open (filename, spec);
        for (int i = 0;  i < spec.nchannels;  ++i)
            std::cout << "Channel " << i << " is " 
                      << spec.channelnames[i] << "\n";
\end{code}

\subsubsection{Specially-designated channels}

The \ImageSpec contains two fields, {\cf alpha_channel} and {\cf
  z_channel}, which designate which channel numbers represent alpha and
$z$ depth, if any.  If either is set to {\cf -1}, it indicates that it
is not known which channel is used for that data.

If you are doing something special with alpha or depth, it is probably
safer to respect the {\cf alpha_channel} and {\cf z_channel}
designations (if not set to {\cf -1}) rather than merely assuming that,
for example, channel 3 is always the alpha channel.

\subsubsection{Linearity hints}

We certainly hope that you are using only modern file formats that
support high precision and extended range pixels (such as OpenEXR) and
keeping all your images in a linear color space.  But you may have to
work with file formats that dictate the use of nonlinear color values.
This is prevalent in formats that store pixels only as 8-bit values,
since 256 values are not enough to linearly represent colors without
banding artifacts in the dim values.

The \ImageSpec has a field that reveals what color space the
image file is using.  Each individual \ImageInput is responsible for
setting this properly.

The \ImageSpec field {\cf linearity} can take on any of the
following values:
\begin{description}
\item[\halfspc \rm \kw{ImageSpec::UnknownLinearity}] indicates
  indicates it's unkown what color space the image file is using.
\item[\halfspc \rm \kw{ImageSpec::Linear}] indicates that the
  color pixel values are known to be linear.
\item[\halfspc \rm \kw{ImageSpec::GammaCorrected}] indicates
  that the color pixel values (but not alpha or $z$) in the file have
  already been gamma corrected (raised to the power $1/\gamma$), and
  that the gamma exponent may be found in the {\cf gamma} field of the
  \ImageSpec.
\item[\halfspc \rm \kw{ImageSpec::sRGB}] indicates that the
  color pixel values in the file are in sRGB color space.
\end{description}

The \ImageInput sets the {\cf ImageSpec::linearity} field in a
purely advisory capacity --- the {\cf read} will not convert pixel
values among color spaces.  Many image file formats only support
nonlinear color spaces (for example, JPEG/JFIF dictates use of sRGB).
So your application should intelligently deal with gamma-corrected and
sRGB input.

The linearity only describes color channels.  You should assume that
alpha or depth ($z$) channels (designated by the {\cf alpha_channel} and
{\cf z_channel} fields, respectively) always represent linear values and
should never be transformed by your application.

\subsubsection{Arbitrary metadata}

All other metadata found in the file will be stored in the
\ImageSpec's {\cf extra_attribs} field, which is a 
\ParamValueList, which is itself essentially a vector of
\ParamValue instances.  Each \ParamValue
stores one meta-datum consisting of a name, type (specified by 
a \TypeDesc), number of values, and data pointer.

If you know the name of a specific piece of metadata you want to use,
you can find it using the {\cf ImageSpec::find_attribute()}
method, which returns a pointer to the matching \ParamValue,
or {\cf NULL} if no match was found.  An optional \TypeDesc
argument can narrow the search to only parameters that match the
specified type as well as the name.  Below is an
example that looks for orientation information, expecting it to consist 
of a single integer:

\begin{code}
        ImageInput *in = ImageInput::create (filename);
        ImageSpec spec;
        in->open (filename, spec);
        ...
        ParamValue *p = spec.find_attribute ("Orientation", TypeDesc::INT);
        if (p) {
            int orientation = * (int *) p->data();
        } else {
            std::cout << "No integer orientation in the file\n";
        }
\end{code}

By convention, \ImageInput plugins will save all integer metadata as
32-bit integers ({\cf TypeDesc::INT} or {\cf TypeDesc::UINT}), even if the file format
dictates that a particular item is stored in the file as a 8- or 16-bit
integer.  This is just to keep client applications from having to deal
with all the types.  Since there is relatively little metadata compared
to pixel data, there's no real memory waste of promoting all integer
types to int32 metadata.  Floating-point metadata and string metadata
may also exist, of course.

It is also possible to step through all the metadata, item by item.
This can be accomplished using the technique of the following example:

\begin{code}
        for (size_t i = 0;  i < spec.extra_attribs.size();  ++i) {
            const ParamValue &p (spec.extra_attribs[i]);
            printf ("    \%s: ", p.name.c_str());
            if (p.type() == TypeDesc::STRING)
                printf ("\"\%s\"", *(const char **)p.data());
            else if (p.type() == TypeDesc::FLOAT)
                printf ("\%g", *(const float *)p.data());
            else if (p.type() == TypeDesc::INT)
                printf ("\%d", *(const int *)p.data());
            else if (p.type() == TypeDesc::UINT)
                printf ("\%u", *(const unsigned int *)p.data());
            else
                printf ("<unknown data type>");
            printf ("\n");
        }
\end{code}

Each individual \ImageInput implementation should document the names,
types, and meanings of all metadata attributes that they understand.


%\subsection{Controlling quantization and encoding}
%\label{sec:imageinput:quantization}
%
%FIXME


%\subsection{Random access and repeated transmission of pixels}
%\label{sec:imageinput:randomrepeated}
%
%FIXME


\subsection{Multi-image files and MIP-maps}
\label{sec:imageinput:multiimage}

Some image file formats support multiple discrete images to be stored
in one file.  When you {\cf open()} an \ImageOutput, it will by default
point to the first (i.e., number 0) subimage in the file.  You can
switch to viewing another subimage using the {\cf seek_subimage()} 
function:

\begin{code}
        ImageInput *in = ImageOutput::create (filename);
        ImageSpec spec;
        in->open (filename, spec);
        ...
        int sub = 1;
        if (in->seek_subimage (sub, spec)) {
            ...
        } else {
            ... no such subimage ...
        }
\end{code}

The {\cf seek_subimage()} function takes two arguments: the index of the
subimage to switch to (starting with 0), and a reference to an
\ImageSpec, into which will be stored the spec of the new
subimage.  The {\cf seek_subimage()} function returns {\cf true} upon
success, and {\cf false} if no such subimage existed.  It is legal to
visit subimages out of order; the \ImageInput is responsible for making
it work properly.  It is also possible to find out which subimage is
currently being viewed, using the {\cf current_subimage()} function,
which returns the index of the current subimage.

Below is pseudocode for reading all the levels of a MIP-map (a
multi-resolution image used for texture mapping) that shows how to read
multi-image files:

\begin{code}
        ImageInput *in = ImageInput::create (filename);
        ImageSpec spec;
        in->open (filename, spec);

        int num_subimages = 0;
        while (in->seek_subimage (num_subimages, spec)) {
            // Note: spec has the format spec of the current subimage
            int npixels = spec.width * spec.height;
            int nchannels = spec.nchannels;
            unsigned char *pixels = new unsigned char [npixels * nchannels];
            in->read_image (TypeDesc::UINT8, pixels);

            ... do whatever you want with this level, in pixels ...

            delete [] pixels;
            ++num_subimages;
        }
        // Note: we break out of the while loop when seek_subimage fails
        // to find a next subimage.

        in->close ();
        delete in;
\end{code}

In this example, we have used \readimage, but of course \readscanline
and \readtile work as you would expect, on the current subimage.


\subsection{Custom search paths for plugins}
\label{sec:imageinput:searchpaths}

Please see Section~\ref{sec:imageoutput:searchpaths} for discussion
about search paths for finding plugins that implement \ImageOutput.

In a similar fashion, calls to {\cf ImageOutput::create()}
will search for plugins in each directory listed in the environment
variable {\cf IMAGEIO_LIBRARY_PATH}, in the order that they are listed.
If no adequate plugin is found, then it will check the custom searchpath
passed as the optional second argument to {\cf ImageInput::create()}.
Here is an example:

\begin{code}
        char *mysearch = "/usr/myapp/lib:\${HOME}/plugins";
        ImageInput *in = ImageInput::create (filename, mysearch);
        ...
\end{code}


\subsection{Error checking}
\label{sec:imageinput:errors}

Nearly every \ImageInput API function returns a {\cf bool} indicating
whether the operation succeeded ({\cf true}) or failed ({\cf false}).
In the case of a failure, the \ImageInput will have saved an error
message describing in more detail what went wrong, and the latest
error messages is accessible using the \ImageInput method 
{\cf error_message()}, which returns the message as a {\cf std::string}.

The exception to this rule is {\cf ImageInput::create}, which returns
{\cf NULL} if it could not create an appropriate \ImageInput.  And in
this case, since no \ImageInput exists for which you can call its {\cf
  error_message()} function, there exists a global {\cf error_message()}
function (in the {\cf OpenImageIO} namespace) that retrieves the latest
error message resulting from a call to {\cf create}.

Here is another version of the simple image reading code from
Section~\ref{sec:imageinput:simple}, but this time it is fully
elaborated with error checking and reporting:

\begin{code}
        #include "imageio.h"
        using namespace OpenImageIO;
        ...

        const char *filename = "foo.jpg";
        int xres, yres, channels;
        unsigned char *pixels;

        ImageInput *in = ImageInput::create (filename);
        if (! in) {
            std::cerr << "Could not create an ImageInput for " 
                      << filename << ", error = " 
                      << OpenImageIO::error_message() << "\n";
            return;
        }

        ImageSpec spec;
        if (! in->open (filename, spec)) {
            std::cerr << "Could not open " << filename 
                      << ", error = " << in->error_message() << "\n";
            delete in;
            return;
        }
        xres = spec.width;
        yres = spec.height;
        channels = spec.nchannels;
        pixels = new unsigned char [xres*yres*channels];

        if (! in->read_image (TypeDesc::UINT8, pixels)) {
            std::cerr << "Could read pixels from " << filename 
                      << ", error = " << in->error_message() << "\n";
            delete in;
            return;
        }

        if (! in->close ()) {
            std::cerr << "Error closing " << filename 
                      << ", error = " << in->error_message() << "\n";
            delete in;
            return;
        }
        delete in;
\end{code}



\section{\ImageInput Class Reference}
\label{sec:imageinput:reference}

\apiitem{ImageInput * {\ce create} (const std::string \&filename, \\
\bigspc\bigspc\spc   const std::string \&plugin_searchpath="")}
Create and return an \ImageInput implementation that is able
to read the given file.  The {\kw plugin_searchpath} parameter is a
colon-separated list of directories to search for ImageIO plugin
DSO/DLL's (not a searchpath for the image itself!).  This will
actually just try every ImageIO plugin it can locate, until it
finds one that's able to open the file without error.  This just
creates the \ImageInput, it does not open the file.
\apiend

\apiitem{const char * {\ce format_name} (void) const}
Return the name of the format implemented by this class.
\apiend

\apiitem{bool {\ce open} (const std::string \&name, ImageSpec \&newspec)}
Opens the file with given name.  Various file attributes are put in
{\kw newspec} and a copy is also saved internally to the
\ImageInput (retrievable via {\kw spec()}.  From examining
{\kw newspec} or {\kw spec()}, you can discern the resolution, if it's
tiled, number of channels, native data format, and other metadata about
the image.  Return {\kw true} if the file was found and opened okay,
otherwise {\kw false}.
\apiend

\apiitem {const ImageSpec \& {\ce spec} (void) const}
Returns a reference to the image format specification of the
current subimage.  Note that the contents of the spec are
invalid before {\kw open()} or after {\kw close()}.
\apiend

\apiitem{bool {\ce close} ()}
Closes an open image.
\apiend


\apiitem{int {\ce current_subimage} (void) const}
Returns the index of the subimage that is currently being read.
The first subimage (or the only subimage, if there is just one) is
number 0.
\apiend


\apiitem{bool {\ce seek_subimage} (int index, ImageSpec \&newspec)}
Seek to the given subimage.  The first subimage in the file has index 0.
Return {\kw true} on success, {\kw false} on failure (including that
there is not a subimage with that index).  The new subimage's vital
statistics are put in {\kw newspec} (and also saved internally in a way
that can be retrieved via {\kw spec()}).  The \ImageInput is
expected to give the appearance of random access to subimages --- in
other words, if it can't randomly seek to the given subimage, it should
transparently close, reopen, and sequentially read through prior
subimages.
\apiend

\apiitem{bool {\ce read_scanline} (int y, int z, TypeDesc format, void *data,\\
  \bigspc                      stride_t xstride=AutoStride)}

Read the scanline that includes pixels $(*,y,z)$ into {\kw data},
converting if necessary from the native data format of the file into the
{\kw format} specified ($z=0$ for non-volume images).  The {\kw xstride}
value gives the data spacing of adjacent pixels (in bytes).  Strides set
to the special value {\kw AutoStride} imply contiguous data, i.e., \\
  \spc {\kw xstride} $=$ {\kw spec.nchannels*format.size()} \\
The \ImageInput is expected to give the appearance of random access
--- in other words, if it can't randomly seek to the given scanline, it
should transparently close, reopen, and sequentially read through prior
scanlines.  The base \ImageInput class has a default implementation
that calls {\kw read_native_scanline()} and then does appropriate format
conversion, so there's no reason for each format plugin to override this
method.
\apiend

\apiitem{bool {\ce read_scanline} (int y, int z, float *data)}
This simplified version of {\kw read_scanline()} reads to contiguous 
float pixels.
\apiend

\apiitem{bool {\ce read_tile} (int x, int y, int z, TypeDesc format,
                            void *data, \\ \bigspc stride_t xstride=AutoStride,
                            stride_t ystride=AutoStride, \\ \bigspc stride_t
                            zstride=AutoStride)}
Read the tile that includes pixels $(*,y,z)$ into {\kw data}, converting
if necessary from the native data format of the file into the 
{\kw format} specified ($z=0$ for non-volume images).  The stride values
give the data spacing of adjacent pixels, scanlines, and volumetric
slices, respectively (measured in bytes).  Strides set to the special
value of {\kw AutoStride} imply contiguous data, i.e., \\
\spc {\kw xstride} $=$ {\kw spec.nchannels*format.size()} \\
\spc {\kw ystride} $=$ {\kw xstride*spec.tile_width} \\
\spc {\kw zstride} $=$ {\kw ystride*spec.tile_height} \\
The \ImageInput is expected to give the appearance of random access
--- in other words, if it can't randomly seek to the given tile, it
should transparently close, reopen, and sequentially read through prior
tiles.  The base \ImageInput class has a default implementation
that calls read_native_tile and then does appropriate format conversion,
so there's no reason for each format plugin to override this method.
\apiend


\apiitem{bool {\ce read_tile} (int x, int y, int z, float *data)}
Simple version of {\kw read_tile} that reads to contiguous float pixels.
\apiend

\apiitem{bool {\ce read_image} (TypeDesc format, void *data, \\
                             \bigspc stride_t xstride=AutoStride,
                             stride_t ystride=AutoStride, \\
                             \bigspc stride_t zstride=AutoStride, \\
                             \bigspc ProgressCallback progress_callback=NULL,\\
                             \bigspc void *progress_callback_data=NULL)}

Read the entire image of {\kw spec.width} $\times$ {\kw spec.height}
$\times$ {\kw spec.depth}
pixels into data (which must already be sized large enough for
the entire image) with the given strides and in the desired
format.  Read tiles or scanlines automatically.  

Strides set to the special value of {\kw AutoStride} imply contiguous
data, i.e., \\
\spc {\kw xstride} $=$ {\kw spec.nchannels * format.size()} \\
\spc {\kw ystride} $=$ {\kw xstride * spec.width} \\
\spc {\kw zstride} $=$ {\kw ystride * spec.height} \\
The function will internally either call {\kw read_scanline} or 
{\kw read_tile}, depending on whether the file is scanline- or
tile-oriented.

Because this may be an expensive operation, a progres callback may be passed.
Periodically, it will be called as follows:\\
\begin{code}
    progress_callback (progress_callback_data, float done)
\end{code}
\noindent where \emph{done} gives the portion of the image 
(between 0.0 and 1.0) that has been read thus far.
\apiend

\apiitem{bool {\ce read_image} (float *data)}
Simple version of {\kw read_image()} reads to contiguous float pixels.
\apiend

\apiitem{bool {\ce read_native_scanline} (int y, int z, void *data)}
The {\kw read_native_scanline()} function is just like {\kw
  read_scanline()}, except that it keeps the data in the native format
of the disk file and always reads into contiguous memory (no strides).
It's up to the user to have enough space allocated and know what to do
with the data.  IT IS EXPECTED THAT EACH FORMAT PLUGIN WILL OVERRIDE
THIS METHOD.
\apiend

\apiitem{bool {\ce read_native_tile} (int x, int y, int z, void *data)}
The {\kw read_native_tile()} function is just like {\kw read_tile()}, 
except that it keeps the data in the native format of the disk file and
always read into contiguous memory (no strides).  It's up to the user to
have enough space allocated and know what to do with the data.  IT IS
EXPECTED THAT EACH FORMAT PLUGIN WILL OVERRIDE THIS METHOD IF IT
SUPPORTS TILED IMAGES.
\apiend

\apiitem{int {\ce send_to_input} (const char *format, ...)}
General message passing between client and image input server.
This is currently undefined and is reserved for future use.
\apiend

\apiitem{int {\ce send_to_client} (const char *format, ...)}
General message passing between client and image input server.
This is currently undefined and is reserved for future use.
\apiend

\apiitem{std::string {\ce error_message} () const}
Returns the current error string describing what went wrong if
any of the public methods returned {\kw false} indicating an error.
(Hopefully the implementation plugin called {\kw error()} with a
helpful error message.)
\apiend



\index{Image I/O API|)}

\chapwidthend
