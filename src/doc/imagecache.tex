\chapter{Cached Images}
\label{chap:imagecache}
\index{Image Cache|(}

\section{Image Cache Introduction and Theory of Operation}
\label{sec:imagecache:intro}

\ImageCache is a utility class that allows an application to read pixels
from a large number of image files while using a remarkably small amount
of memory and other resources.  Of course it is possible for an
application to do this directly using \ImageInput objects.  But
\ImageCache offers the following advantages:

\begin{itemize}
\item \ImageCache presents an even simpler user interface than
  \ImageInput --- the only supported operations are asking for an
  \ImageSpec describing a subimage in the file, retrieving for a block
  of pixels, and locking/reading/releasing individual tiles.  You refer
  to images by filename only; you don't need to keep track of individual
  file handles or \ImageInput objects.  You don't need to explicitly
  open or close files.

\item The \ImageCache is completely thread-safe; if multiple threads
  are accessing the same file, the \ImageCache internals will handle
  all the locking and resource sharing.

\item No matter how many image files you are accessing, the \ImageCache
  will maintain a reasonable number of simultaneously-open files,
  automatically closing files that have not been needed recently.

\item No matter how large the total pixels in all the image files you
  are dealing with are, the \ImageCache will use only a small amount of
  memory.  It does this by loading only the individual tiles requested,
  and as memory allotments are approached, automatically releasing the
  memory from tiles that have not been used recently.
\end{itemize}

In short, if you have an application that will need to read pixels from
many large image files, you can rely on \ImageCache to manage all the
resources for you.  It is reasonable to access thousands of image files
totalling hundreds of GB of pixels, efficiently and using a memory
footprint on the order of 50 MB.

\newpage
Below are some simple code fragments that shows \ImageCache in action:
\medskip

\begin{code}
    #include "OpenImageIO/imagecache.h"

    // Create an image cache and set some options
    ImageCache *cache = ImageCache::create ();
    cache->attribute ("max_memory_MB", 500.0);
    cache->attribute ("autotile", 64);

    // Get a block of pixels from a file.
    // (for brevity of this example, let's assume that 'size' is the
    // number of channels times the number of pixels in the requested region)
    float pixels[size];
    cache->get_pixels ("file1.jpg", 0, 0, xbegin, xend, ybegin, yend,
                       zbegin, zend, TypeDesc::FLOAT, pixels);

    // Get information about a file
    ImageSpec spec;
    bool ok = cache->get_imagespec ("file2.exr", spec);
    if (ok)
        std::cout << "resolution is " << spec.width << "x" 
                  << "spec.height << "\n";

    // Request and hold a tile, do some work with its pixels, then release
    ImageCache::Tile *tile;
    tile = cache->get_tile ("file2.exr", 0, 0, x, y, z);
    // The tile won't be freed until we release it, so this is safe:
    TypeDesc format;
    void *p = cache->tile_pixels (tile, format);
    // Now p points to the raw pixels of the tile, whose data format
    // is given by 'format'.
    cache->release_tile (tile);  
    // Now cache is permitted to free the tile when needed

    // Note that all files were referenced by name, we never had to open
    // or close any files, and all the resource and memory management 
    // was automatic.

    ImageCache::destroy (cache);
\end{code}

\newpage
\section{ImageCache API}
\label{sec:imagecache:api}

\subsection{Creating and destroying an image cache}
\label{sec:imagecache:api:createdestroy}

\ImageCache is an abstract API described as a pure virtual class.  The
actual internal implementation is not exposed through the external API
of \product.  Because of this, you cannot construct or destroy the
concrete implementation, so two static methods of \ImageCache are
provided:

\apiitem{static ImageCache *ImageCache::{\ce create} (bool shared=true)}

Creates a new \ImageCache and returns a pointer to it.  If 
{\cf shared} is {\cf true}, {\cf create()} will return a pointer
to a shared \ImageCache (so that multiple parts of an application
that request an \ImageCache will all end up with the same one).
If {\cf shared} is {\cf false}, a completely unique \ImageCache
will be created and returned.

\apiend

\apiitem{static void ImageCache::{\ce destroy} (ImageCache *x)}
Destroys an allocated \ImageCache, including freeing all system
resources that it holds.

This is necessary to ensure that the memory is freed in a way that
matches the way it was allocated within the library.  Note that simply
using {\cf delete} on the pointer will not always work (at least,
not on some platforms in which a DSO/DLL can end up using a different
allocator than the main program).

It is safe to destroy even a shared \ImageCache, as the implementation
of {\cf destroy()} will recognize a shared one and only truly release
its resources if it has been requested to be destroyed as many times as
shared \ImageCache's were created.
\apiend

\subsection{Setting options and limits for the image cache}
\label{sec:imagecache:api:attribute}

The following member functions of \ImageCache allow you to set
(and in some cases retrieve) options that control the overall
behavior of the image cache:

\apiitem{bool {\ce attribute} (const std::string \&name, TypeDesc type,
  const void *val)}
\indexapi{attribute}

Sets an attribute (i.e., a property or option) of the \ImageCache.
The {\cf name} designates the name of the attribute, {\cf type}
describes the type of data, and {\cf val} is a pointer to memory 
containing the new value for the attribute.

If the \ImageCache recognizes a valid attribute name that matches the
type specified, the attribute will be set to the new value and {\cf
  attribute()} will return {\cf true}.  If {\cf name} is not recognized
as a valid attribute name, or if the types do not match (e.g., {\cf
  type} is {\cf TypeDesc::FLOAT} but the named attribute is a string),
the attribute will not be modified, and {\cf attribute()} will return
{\cf false}.

Here are examples:

\begin{code}
      ImageCache *ts; 
      ...
      int maxfiles = 50;
      ts->attribute ("max_open_files", TypeDesc::INT, &maxfiles);

      const char *path = "/my/path";
      ts->attribute ("searchpath", TypeDesc::STRING, &path);
\end{code}

Note that when passing a string, you need to pass a pointer to the {\cf
  char*}, not a pointer to the first character.  (Rationale: for an {\cf
  int} attribute, you pass the address of the {\cf int}.  So for a
string, which is a {\cf char*}, you need to pass the address of the
string, i.e., a {\cf char**}).

The complete list of attributes can be found at the end of this section.

\apiend

\apiitem{bool {\ce attribute} (const std::string \&name, int val) \\
bool {\ce attribute} (const std::string \&name, float val) \\
bool {\ce attribute} (const std::string \&name, double val) \\
bool {\ce attribute} (const std::string \&name, const char *val) \\
bool {\ce attribute} (const std::string \&name, const std::string \& val)}
Specialized versions of {\cf attribute()} in which the data type is
implied by the type of the argument.

For example, the following are equivalent to the example above for the
general (pointer) form of {\cf attribute()}:

\begin{code}
      ts->attribute ("max_open_files", 50);
      ts->attribute ("searchpath", "/my/path");
\end{code}

\apiend


\apiitem{bool {\ce getattribute} (const std::string \&name, TypeDesc type,
  void *val)}
\indexapi{getattribute}

Gets the current value of an attribute of the \ImageCache.
The {\cf name} designates the name of the attribute, {\cf type}
describes the type of data, and {\cf val} is a pointer to memory 
where the user would like the value placed.

If the \ImageCache recognizes a valid attribute name that matches the
type specified, the attribute value will be stored at address {\cf val}
and {\cf attribute()} will return {\cf true}.  If {\cf name} is not recognized
as a valid attribute name, or if the types do not match (e.g., {\cf
  type} is {\cf TypeDesc::FLOAT} but the named attribute is a string),
no data will be written to {\cf val}, and {\cf attribute()} will return
{\cf false}.

Here are examples:

\begin{code}
      ImageCache *ts; 
      ...
      int maxfiles;
      ts->getattribute ("max_open_files", TypeDesc::INT, &maxfiles);

      const char *path;
      ts->getattribute ("searchpath", TypeDesc::STRING, &path);
\end{code}

Note that when passing a string, you need to pass a pointer to the {\cf
  char*}, not a pointer to the first character.  Also, the {\cf char*}
will end up pointing to characters owned by the \ImageCache; the
caller does not need to ever free the memory that contains the
characters.

The complete list of attributes can be found at the end of this section.


\apiend

\apiitem{bool {\ce getattribute} (const std::string \&name, int \&val) \\
bool {\ce getattribute} (const std::string \&name, float \&val) \\
bool {\ce getattribute} (const std::string \&name, double \&val) \\
bool {\ce getattribute} (const std::string \&name, char **val) \\
bool {\ce getattribute} (const std::string \&name, std::string \& val)}
Specialized versions of {\cf getattribute()} in which the data type is
implied by the type of the argument.

For example, the following are equivalent to the example above for the
general (pointer) form of {\cf getattribute()}:

\begin{code}
      int maxfiles;
      ts->getattribute ("max_open_files", &maxfiles);
      const char *path;
      ts->getattribute ("searchpath", &path);
\end{code}

\apiend


\subsubsection*{Image cache attributes}

Recognized attributes include the following:

\apiitem{int max_open_files}
The maximum number of file handles that the image cache will
hold open simultaneously.  (Default = 100)
\apiend

\apiitem{float max_memory_MB}
The maximum amount of memory (measured in MB) that the image cache
will use for its ``tile cache.'' (Default: 256.0 MB)
\apiend

\apiitem{string searchpath}
The search path for images: a colon-separated list of
directories that will be searched in order for any image name
that is not specified as an absolute path. (Default: no search path.)
\apiend

\apiitem{int autotile}
This attributes controls how the image cache deals with images that
are not ``tiled'' (i.e., are stored as scanlines). 

If {\cf autotile} is set to 0 (the default), an untiled image will be
treated as if it were a single tile of the resolution of the whole
image.  This is simple and fast, but can lead to poor cache behavior if
you are simultaneously accessing many large untiled images.

If {\cf autotile} is nonzero (e.g., 64 is a good recommended value), any
untiled images will be read and cached as if they were constructed in
tiles of size {\cf autotile} $\times$ {\cf autotile}.  This leads to
slightly more expensive disk access if you are using only a few
images, but if you are using many untiled images, the caching will be much
more efficient.
\apiend

\apiitem{int automip}
If {\cf automip} is set to 0 (the default), an untiled single-subimage
file will only be able to utilize that single subimage.

If {\cf automip} is nonzero, any untiled, single-subimage
(un-MIP-mapped) images will have lower-resolution MIP-map levels
generated on-demand if pixels are requested from the lower-res subimages
(that don't really exist).  Essentially this makes the \ImageCache
pretend that the file is MIP-mapped even if it isn't.
\apiend

\apiitem{int forcefloat}
If set to nonzero, all image tiles will be converted to {\cf float} 
type when stored in the image cache.  This can be helpful especially
for users of \ImageBuf who want to simplify their image manipulations
to only need to consider {\cf float} data.

The default is zero, meaning that image pixels are not forced to
be {\cf float} when in cache.
\apiend

\apiitem{int accept_untiled}
When nonzero (the default), \ImageCache accepts untiled images as
usual.  When set to zero, \ImageCache will reject untiled images with
an error condition, as if the file could not be properly read.
This is sometimes helpful for applications that want to enforce use of
tiled images only.
\apiend

\apiitem{int failure_retries}
When an {\cf open()} or {\cf read_tile()} calls fails, pause and try
again, up to {\cf failure_retries} times before truly returning a
failure.  This is meant to address spooky disk or network failures.  The
default is zero, meaning that failures of open or tile reading will
immediately return as a failure.
\apiend

\bigskip

\subsection{Getting information about images}
\label{sec:imagecache:api:getimageinfo}
\label{sec:imagecache:api:getimagespec}

\apiitem{bool {\ce get_image_info} (ustring filename, int subimage, int
  miplevel, \\
  \bigspc\bigspc ustring dataname, TypeDesc datatype, void *data)}
Retrieves information about the image named by {\cf filename}.
The {\cf dataname} is a keyword indcating what information should
be retrieved, {\cf datatype} is the type of data expected, and
{\cf data} points to caller-owned memory where the results should be
placed.  It is up to the caller to ensure that {\cf data} contains
enough space to hold an item of the requested {\cf datatype}.

The return value is {\cf true} if {\cf get_image_info()} is able
to find the requested {\cf dataname} and it matched the requested
{\cf datatype}.  If the requested data was not found, or was not
of the right data type, {\cf get_image_info()} will return {\cf false}.

Supported {\cf dataname} values include:

\begin{description}
\item[\spc] \spc
\vspace{-12pt} 
\item[\rm \kw{exists}] Return 1 if the file exists and
is an image format that OpenImageIO knows how to read, otherwise return
0.  The {\cf data} pointer is not used.

\item[\rm \kw{subimages}] The number of subimages in the file, as an integer.

\item[\rm \kw{resolution}] The resolution of the image file, which
is an array of 2 integers (described as {\cf TypeDesc(INT,2)}).

\item[\rm \kw{miplevels}] The number of MIPmap levels for the specified
subimage (an integer).

\item[\rm \kw{texturetype}] A string describing the type of texture
of the given file, which describes how the texture may be used (also
which texture API call is probably the right one for it).
This currently may return one of: \qkw{unknown}, \qkw{Plain Texture},
\qkw{Volume Texture}, \qkw{Shadow}, 
or \qkw{Environment}.

\item[\rm \kw{textureformat}] A string describing the format of the
given file, which describes the kind of texture stored in the file.
This currently may return one of: \qkw{unknown}, \qkw{Plain Texture},
\qkw{Volume Texture}, \qkw{Shadow}, \qkw{CubeFace Shadow}, \qkw{Volume
  Shadow}, \qkw{LatLong Environment}, or \qkw{CubeFace Environment}.
Note that there are several kinds of shadows and environment maps,
all accessible through the same API calls.

\item[\rm \kw{channels}] The number of color channels in the file 
(an integer).

\item[\rm \kw{format}] The native data format of the pixels in the
  file (an integer, giving the {\cf TypeDesc::BASETYPE} of the data).
  Note that this is not necessarily the same as the data format stored
  in the image cache.

\item[\rm \kw{cachedformat}] The native data format of the pixels as
  stored in the image cache (an integer, giving the {\cf
    TypeDesc::BASETYPE} of the data).  Note that this is not necessarily
  the same as the native data format of the file.

\item[\rm \kw{viewingmatrix}] The viewing matrix, which is a
$4 \times 4$ matrix (an {\cf Imath::M44f}, described as {\cf
  TypeDesc(FLOAT,MATRIX)}).

\item[\rm \kw{projectionmatrix}] The projection matrix, which is a
$4 \times 4$ matrix (an {\cf Imath::M44f}, described as {\cf
  TypeDesc(FLOAT,MATRIX)}).

\item[Anything else] -- For all other data names, the
the metadata of the image file will be searched for an item that
matches both the name and data type.

\end{description}
\apiend

\apiitem{bool {\ce get_imagespec} (ustring filename, ImageSpec \&spec,\\
\bigspc  int subimage=0, int miplevel=0)}

If the named image (and the specific subimage and MIP level) is found and able to be opened by an available
image format plugin, and the designated subimage exists, this function copies
its image specification for that subimage into {\cf spec} and returns
{\cf true}.  Otherwise, if the file is not found, could not be opened,
is not of a format readable by any plugin that could be found, or
the designated subimage did not exist in the file, the return value is
{\cf false} and {\cf spec} will not be modified.

\apiend

\apiitem{const ImageSpec * {\ce imagespec} (ustring filename, int
  subimage=0, int miplevel=0)}

If the named image is found and able to be opened by an available
image format plugin, and the designated subimage exists, this function
returns a pointer to an \ImageSpec that describes it.  Otherwise, if the
file is not found, could not be opened, is not of a format readable by
any plugin that could be find, or the designated subimage did
not exist in the file, the return value is NULL.

This method is much more efficient than {\cf get_imagespec()}, since it
just returns a pointer to the spec held internally by the \ImageCache
(rather than copying the spec to the user's memory).  However, the
caller must beware that the pointer is only valid as long as nobody
(even other threads) calls {\cf invalidate()} on the file, or {\cf
  invalidate_all()}, or destroys the \ImageCache.
\apiend

\apiitem{std::string {\ce resolve_filename} (const std::string \&filename)}
Returns the true path to the given file name, with searchpath logic
applied.
\apiend

\subsection{Getting pixels}
\label{sec:imagecache:api:getpixels}

\apiitem{bool {\ce get\_pixels} (ustring filename, int subimage, int miplevel, \\
         \bigspc int xbegin, int xend, int ybegin, int yend,
                            int zbegin, int zend, \\
         \bigspc TypeDesc format, void *result)}

Retrieve the rectangle of raw pixels
of the designated {\cf subimage}, storing the pixel values
beginning at the address specified by result.  The pixel values will be
converted to the type specified by {\cf format}.  It is up to the caller
to ensure that result points to an area of memory big enough to
accommodate the requested rectangle (taking into consideration its
dimensions, number of channels, and data format).
The rectangular region to be retrieved includes {\cf begin} but does not
include {\cf end} (much like STL begin/end usage).
Requested pixels that are not part of the valid pixel data region of the
image file will be filled with zero values.
\apiend

\subsection{Dealing with tiles}
\label{sec:imagecache:api:tiles}

\apiitem{ImageCache::Tile {\ce get_tile} (ustring filename, int subimage, int miplevel, \\
  \bigspc \bigspc int x, int y, int z)}
Find a tile given by an image {\cf filename}, {\cf subimage}, and pixel
coordinates.  An opaque pointer to the tile will be returned,
or {\cf NULL} if no such file (or tile within the file) exists or can
be read.  The tile will not be purged from the cache until 
after {\cf release_tile()} is called on the tile pointer.  This is
thread-safe.
\apiend

\apiitem{void {\ce release_tile} (ImageCache::Tile *tile)}
After finishing with a tile, {\cf release_tile()} will allow it to 
once again be purged from the tile cache if required.
\apiend

\apiitem{const void * {\ce tile_pixels} (ImageCache::Tile *tile,
  TypeDesc \&format)}
For a tile retrived by {\cf get_tile()}, return a pointer to the
pixel data itself, and also store in {\cf format} the data type that
the pixels are internally stored in (which may be different than
the data type of the pixels in the disk file).  This method should
only be called on a tile that has been requested by 
{\cf get_tile()} but has not yet been released with {\cf release_tile()}.
\apiend

\apiitem{void {\ce invalidate} (ustring filename)}
Invalidate any loaded tiles or open file handles associated with
the filename, so that any subsequent queries will be forced to
re-open the file or re-load any tiles (even those that were
previously loaded and would ordinarily be reused).  A client
might do this if, for example, they are aware that an image
being held in the cache has been updated on disk.  This is safe
to do even if other procedures are currently holding 
reference-counted tile pointers from the named image, but those 
procedures will not get updated pixels until they release the 
tiles they are holding.
\apiend

\apiitem{void {\ce invalidate_all} (bool force=false)}
Invalidate all loaded tiles and open file handles, so that any
subsequent queries will be forced to re-open the file or re-load any
tiles (even those that were previously loaded and would ordinarily be
reused).  A client might do this if, for example, they are aware that an
image being held in the cache has been updated on disk.  This is safe to
do even if other procedures are currently holding reference-counted tile
pointers from the named image, but those procedures will not get updated
pixels until they release the tiles they are holding.  If force is true,
everything will be invalidated, no matter how wasteful it is, but if
force is false, in actuality files will only be invalidated if their
modification times have been changed since they were first opened.
\apiend

\subsection{Errors and statistics}
\label{sec:imagecache:api:geterror}
\label{sec:imagecache:api:getstats}

\apiitem{std::string {\ce geterror} ()}
If any other API routines return {\cf false}, indicating that an error
has occurred, this routine will retrieve the error and clear the error
status.  If no error has occurred since the last time {\cf geterror()}
was called, it will return an empty string.
\apiend

\apiitem{std::string {\ce getstats} (int level=1)}
Returns a big string containing useful statistics about the \ImageCache
operations, suitable for saving to a file or outputting to the terminal.
The {\cf level} indicates the amount of detail in the statistics,
with higher numbers (up to a maximum of 5) yielding more and more
esoteric information.
\apiend


\index{Image Cache|)}

\chapwidthend
