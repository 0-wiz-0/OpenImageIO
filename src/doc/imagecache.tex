\chapter{Cached Images: {\cf ImageCache}}
\label{chap:imagecache}
\index{Image Cache|(}

\def\ImageCache{{\cf ImageCache}\xspace}


\section{Image Cache Introduction and Theory of Operation}
\label{sec:imagecache:intro}

Coming soon.
FIXME

\section{ImageCache API}
\label{sec:imagecache:api}

\subsection{Creating and destroying an image cache}
\label{sec:imagecache:api:createdestroy}

\ImageCache is an abstract API described as a pure virtual class.  The
actual internal implementation is not exposed through the external API
of \product.  Because of this, you cannot construct or destroy the
concrete implementation, so two static methods of \ImageCache are
provided:

\apiitem{static ImageCache *ImageCache::{\ce create} (bool shared=true)}

Creates a new \ImageCache and returns a pointer to it.  If 
{\cf shared} is {\cf true}, {\cf create()} will return a pointer
to a shared \ImageCache (so that multiple parts of an application
that request an \ImageCache will all end up with the same one).
If {\cf shared} is {\cf false}, a completely unique \ImageCache
will be created and returned.

\apiend

\apiitem{static void ImageCache::{\ce destroy} (ImageCache *x)}
Destroys an allocated \ImageCache, including freeing all system
resources that it holds.

This is necessary to ensure that the memory is freed in a way that
matches the way it was allocated within the library.  Note that simply
using {\cf delete} on the pointer will not always work (at least,
not on some platforms in which a DSO/DLL can end up using a different
allocator than the main program).

It is safe to destroy even a shared \ImageCache, as the implementation
of {\cf destroy()} will recognize a shared one and only truly release
its resources if it has been requested to be destroyed as many times as
shared \ImageCache's were created.
\apiend

\subsection{Setting options and limits for the image cache}
\label{sec:imagecache:api:options}

The following member functions of \ImageCache allow you to set
(and in some cases retrieve) options that control the overall
behavior of the image cache:

\apiitem{bool {\ce attribute} (const std::string \&name, TypeDesc type,
  const void *val)}
\indexapi{attribute}

Sets an attribute (i.e., a property or option) of the \ImageCache.
The {\cf name} designates the name of the attribute, {\cf type}
describes the type of data, and {\cf val} is a pointer to memory 
containing the new value for the attribute.

If the \ImageCache recognizes a valid attribute name that matches the
type specified, the attribute will be set to the new value and {\cf
  attribute()} will return {\cf true}.  If {\cf name} is not recognized
as a valid attribute name, or if the types do not match (e.g., {\cf
  type} is {\cf TypeDesc::FLOAT} but the named attribute is a string),
the attribute will not be modified, and {\cf attribute()} will return
{\cf false}.

Here are examples:

\begin{code}
      ImageCache *ts; 
      ...
      int maxfiles = 50;
      ts->attribute ("max_open_files", TypeDesc::INT, &maxfiles);

      const char *path = "/my/path";
      ts->attribute ("searchpath", TypeDesc::STRING, &path);
\end{code}

Note that when passing a string, you need to pass a pointer to the {\cf
  char*}, not a pointer to the first character.  (Rationale: for an {\cf
  int} attribute, you pass the address of the {\cf int}.  So for a
string, which is a {\cf char*}, you need to pass the address of the
string, i.e., a {\cf char**}).

The complete list of attributes can be found at the end of this section.

\apiend

\apiitem{bool {\ce attribute} (const std::string \&name, int val) \\
bool {\ce attribute} (const std::string \&name, float val) \\
bool {\ce attribute} (const std::string \&name, double val) \\
bool {\ce attribute} (const std::string \&name, const char *val) \\
bool {\ce attribute} (const std::string \&name, const std::string \& val)}
Specialized versions of {\cf attribute()} in which the data type is
implied by the type of the argument.

For example, the following are equivalent to the example above for the
general (pointer) form of {\cf attribute()}:

\begin{code}
      ts->attribute ("max_open_files", 50);
      ts->attribute ("searchpath", "/my/path");
\end{code}

\apiend


\apiitem{bool {\ce getattribute} (const std::string \&name, TypeDesc type,
  void *val)}
\indexapi{getattribute}

Gets the current value of an attribute of the \ImageCache.
The {\cf name} designates the name of the attribute, {\cf type}
describes the type of data, and {\cf val} is a pointer to memory 
where the user would like the value placed.

If the \ImageCache recognizes a valid attribute name that matches the
type specified, the attribute value will be stored at address {\cf val}
and {\cf attribute()} will return {\cf true}.  If {\cf name} is not recognized
as a valid attribute name, or if the types do not match (e.g., {\cf
  type} is {\cf TypeDesc::FLOAT} but the named attribute is a string),
no data will be written to {\cf val}, and {\cf attribute()} will return
{\cf false}.

Here are examples:

\begin{code}
      ImageCache *ts; 
      ...
      int maxfiles;
      ts->getattribute ("max_open_files", TypeDesc::INT, &maxfiles);

      const char *path;
      ts->getattribute ("searchpath", TypeDesc::STRING, &path);
\end{code}

Note that when passing a string, you need to pass a pointer to the {\cf
  char*}, not a pointer to the first character.  Also, the {\cf char*}
will end up pointing to characters owned by the \ImageCache; the
caller does not need to ever free the memory that contains the
characters.

The complete list of attributes can be found at the end of this section.


\apiend

\apiitem{bool {\ce getattribute} (const std::string \&name, int \&val) \\
bool {\ce getattribute} (const std::string \&name, float \&val) \\
bool {\ce getattribute} (const std::string \&name, double \&val) \\
bool {\ce getattribute} (const std::string \&name, char **val) \\
bool {\ce getattribute} (const std::string \&name, std::string \& val)}
Specialized versions of {\cf getattribute()} in which the data type is
implied by the type of the argument.

For example, the following are equivalent to the example above for the
general (pointer) form of {\cf getattribute()}:

\begin{code}
      int maxfiles;
      ts->getattribute ("max_open_files", &maxfiles);
      const char *path;
      ts->getattribute ("searchpath", &path);
\end{code}

\apiend


\subsubsection*{Image cache attributes}

Recognized attributes include the following:

\apiitem{int max_open_files}
The maximum number of file handles that the image cache will
hold open simultaneously.  (Default = 100)
\apiend

\apiitem{float max_memory_MB}
The maximum amount of memory (measured in MB) that the image cache
will use for its ``tile cache.'' (Default: 50.0 MB)
\apiend

\apiitem{string searchpath}
The search path for images: a colon-separated list of
directories that will be searched in order for any image name
that is not specified as an absolute path. (Default: no search path.)
\apiend

\apiitem{int autotile}
This attributes controls how the image cache deals with images that
are not ``tiled'' (i.e., are stored as scanlines). 

If {\cf autotile} is set to 0 (the default), an untiled image will be
treated as if it were a single tile of the resolution of the whole
image.  This is simple and fast, but can lead to poor cache behavior if
you are simultaneously accessing many large untiled images.

If {\cf autotile} is nonzero (e.g., 64 is a good recommended value), any
untiled images will be read and cached as if they were constructed in
tiles of size {\cf autotile} $\times$ {\cf autotile}.  This leads to
slightly more expensive disk access if you are using only a few
images, but if you are using many untiled images, the caching be much
more efficient.
\apiend

\apiitem{int automip}
If {\cf automip} is set to 0 (the default), an untiled single-subimage
file will only be able to utilize that single subimage.

If {\cf autotile} is nonzero, any untiled, single-subimage
(un-MIP-mapped) images will have lower-resolution versions generated
on-demand if pixels are requested from the ``upper'' subimage levels
(which don't really exist in the file).  Essentially this makes the
\ImageCache pretend that the file is MIP-mapped even if it isn't.
\apiend


\newpage
\subsection{Getting information about images}
\label{sec:imagecache:api:getimageinfo}
\label{sec:imagecache:api:getimagespec}

\apiitem{bool {\ce get_image_info} (ustring filename, ustring dataname, \\
  \bigspc\bigspc TypeDesc datatype, void *data)}
Retrieves information about the image named by {\cf filename}.
The {\cf dataname} is a keyword indcating what information should
be retrieved, {\cf datatype} is the type of data expected, and
{\cf data} points to caller-owned memory where the results should be
placed.  It is up to the caller to ensure that {\cf data} contains
enough space to hold an item of the requested {\cf datatype}.

The return value is {\cf true} if {\cf get_image_info()} is able
to find the requested {\cf dataname} and it matched the requested
{\cf datatype}.  If the requested data was not found, or was not
of the right data type, {\cf get_image_info()} will return {\cf false}.

Supported {\cf dataname} values include:

\begin{description}
\item[\spc] \spc
\vspace{-12pt} \item[\rm \kw{resolution}] The resolution of the image file, which
is an array of 2 integers (described as {\cf TypeDesc(INT,2)}).

\item[\rm \kw{texturetype}] A string describing the type of texture
of the given file, which describes how the texture may be used (also
which texture API call is probably the right one for it).
This currently may return one of: \qkw{unknown}, \qkw{Plain Texture},
\qkw{Volume Texture}, \qkw{Shadow}, 
or \qkw{Environment}.

\item[\rm \kw{textureformat}] A string describing the format of the
given file, which describes the kind of texture stored in the file.
This currently may return one of: \qkw{unknown}, \qkw{Plain Texture},
\qkw{Volume Texture}, \qkw{Shadow}, \qkw{CubeFace Shadow}, \qkw{Volume
  Shadow}, \qkw{LatLong Environment}, or \qkw{CubeFace Environment}.
Note that there are several kinds of shadows and environment maps,
all accessible through the same API calls.

\item[\rm \kw{channels}] The number of color channels in the file 
(an integer).

\item[\rm \kw{viewingmatrix}] The viewing matrix, which is a
$4 \times 4$ matrix (an {\cf Imath::M44f}, described as {\cf
  TypeDesc(FLOAT,MATRIX)}).

\item[\rm \kw{projectionmatrix}] The projection matrix, which is a
$4 \times 4$ matrix (an {\cf Imath::M44f}, described as {\cf
  TypeDesc(FLOAT,MATRIX)}).

\item[Anything else] -- For all other data names, the
the metadata of the image file will be searched for an item that
matches both the name and data type.

\end{description}
\apiend

\apiitem{bool {\ce get_imagespec} (ustring filename, ImageSpec \&spec,
  int subimage=0)}

If the named image is found and able to be opened by an available
ImageIO plugin, and the designated subimage exists, this function copies
its image specification for that subimage into {\cf spec} and returns
{\cf true}.  Otherwise, if the file is not found, could not be opened,
is not of a format readable by any ImageIO plugin that could be find, or
the designated subimage did not exist in the file, the return value is
{\cf false} and {\cf spec} will not be modified.

\apiend

\newpage
\subsection{Getting pixels}
\label{sec:imagecache:api:getpixels}

\apiitem{bool {\ce get\_pixels} (ustring filename, int level, \\
         \bigspc int xmin, int xmax, int ymin, int ymax,
                            int zmin, int zmax, \\
         \bigspc TypeDesc format, void *result)}

Retrieve the rectangle of raw pixels spanning (xmin, ymin, zmin) through
(xmax, ymax, zmax) (inclusive, specified as integer pixel coordinates),
of the designated subimage {\cf level}, storing the pixel values
beginning at the address specified by result.  The pixel values will be
converted to the type specified by {\cf format}.  It is up to the caller
to ensure that result points to an area of memory big enough to
accommodate the requested rectangle (taking into consideration its
dimensions, number of channels, and data format).

\apiend

\subsection{Dealing with tiles}
\label{sec:imagecache:api:tiles}

\apiitem{ImageCache::Tile {\ce get\_tile} (ustring filename, int level, \\
\bigspc int x, int y, int z)}
\apiend

\apiitem{void {\ce release\_tile} (ImageCache::Tile *tile)}
\apiend

\apiitem{const void * {\ce tile_pixels} (ImageCache::Tile *tile, TypeDesc \&format)}
\apiend

\subsection{Errors and statistics}
\label{sec:imagecache:api:geterror}
\label{sec:imagecache:api:getstats}

\apiitem{std::string {\ce geterror} ()}
If any other API routines return {\cf false}, indicating that an error
has occurred, this routine will retrieve the error and clear the error
status.  If no error has occurred since the last time {\cf geterror()}
was called, it will return an empty string.
\apiend

\apiitem{std::string {\ce getstats} (int level=1)}
Returns a big string containing useful statistics about the \ImageCache
operations, suitable for saving to a file or outputting to the terminal.
The {\cf level} indicates the amount of detail in the statistics,
with higher numbers (up to a maximum of 5) yielding more and more
esoteric information.
\apiend


\index{Image Cache|)}

\chapwidthend
