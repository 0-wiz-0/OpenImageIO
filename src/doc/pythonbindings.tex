\chapter{Python Bindings}
\label{chap:pythonbindings}
\indexapi{Python|()}

\section{Overview}

\OpenImageIO provides Python language bindings for much of its
functionality.

\smallskip

You must ensure that the environment variable {\cf PYTHONPATH} includes
the {\cf python} subdirectory of the \OpenImageIO installation.

\smallskip

A Python program must import the {\cf OpenImageIO} package:
\begin{code}
    import OpenImageIO
\end{code}
\noindent In most of our examples below, we assume that for the sake
of brevity, we will alias the package name as follows:
\begin{code}
    import OpenImageIO as oiio
\end{code}

\section{TypeDesc}
\label{sec:pythontypedesc}

The \TypeDesc class that describes data types of pixels and metadata,
described in detail in Section~\ref{sec:TypeDesc}, is replicated for Python.

\apiitem{BASETYPE}
The {\cf BASETYPE} enum corresponds to the C++ {\cf TypeDesc::BASETYPE} and
contains the following values: \\
{\cf UNKNOWN NONE UINT8 INT8 UINT16 INT16 UINT32 INT32 UINT64 INT64 \\
HALF FLOAT DOUBLE STRING PTR} \\
These names are also exported to the {\cf OpenImageIO} namespace.
\apiend

\apiitem{AGGREGATE}
The {\cf AGGREGATE} enum corresponds to the C++ {\cf TypeDesc::AGGREGATE} and
contains the following values: \\
{\cf SCALAR VEC2 VEC3 VEC4 MATRIX44} \\
These names are also exported to the {\cf OpenImageIO} namespace.
\apiend

\apiitem{VECSEMANTICS}
The {\cf VECSEMANTICS} enum corresponds to the C++ {\cf TypeDesc::VECSEMANTICS} and
contains the following values: \\
{\cf NOXFORM COLOR POINT VECTOR NORMAL} \\
These names are also exported to the {\cf OpenImageIO} namespace.
\apiend

\apiitem{TypeDesc () \\
TypeDesc (basetype) \\
TypeDesc (basetype, aggregate) \\
TypeDesc (basetype, aggregate, vecsemantics) \\
TypeDesc (basetype, aggregate, vecsemantics, arraylen) \\
TypeDesc (str)}

Construct a {\cf TypeDesc} object.  

\noindent Examples:
\begin{code}
    import OpenImageIO as oiio

    # make a default (UNKNOWN) TypeDesc
    t = oiio.TypeDesc()

    # make a TypeDesc describing an unsigned 8 bit int
    t = oiio.TypeDesc(oiio.UINT8)

    # make a TypeDesc describing an array of 14 'half' values
    t = oiio.TypeDesc(oiio.HALF, oiio.SCALAR, oiio.NOXFORM, 14)

    # make a TypeDesc describing a float point
    t = oiio.TypeDesc(oiio.FLOAT, oiio.VEC3, oiio.POINT)

    # Some constructors from a string description
    t = oiio.TypeDesc("uint8")
    t = oiio.TypeDesc("half[14]")
    t = oiio.TypeDesc("point")     # equiv to FLOAT, VEC3, POINT
\end{code}
\apiend

\apiitem{TypeDesc.TypeFloat() \\
TypeDesc.TypeInt() \\
TypeDesc.TypeString() \\
TypeDesc.TypeColor() \\
TypeDesc.TypePoint() \\
TypeDesc.TypeVector() \\
TypeDesc.TypeNormal() \\
TypeDesc.TypeMatrix()}
Pre-constructed \TypeDesc objects for some common types.

\noindent Example:
\begin{code}
    t = oiio.TypeDesc.TypeFloat()
\end{code}
\apiend

\apiitem{string {\ce str} (TypeDesc)}
Returns a string that describes the \TypeDesc.

\noindent Example:
\begin{code}
    print str(oiio.TypeDesc(oiio.UINT16))

    > int16
\end{code}
\apiend

\apiitem{TypeDesc.{\ce basetype} \\
TypeDesc.{\ce aggregate} \\
TypeDesc.{\ce vecsemantics} \\
TypeDesc.{\ce arraylen}}
Access to the raw fields in the \TypeDesc.

\noindent Example:
\begin{code}
    t = oiio.TypeDesc(...)
    if t.basetype == oiio.FLOAT :
        print "It's made of floats"
\end{code}
\apiend

\apiitem{int TypeDesc.{\ce size} () \\
int TypeDesc.{\ce basesize} () \\
TypeDesc TypeDesc.{\ce elementtype} () \\
int TypeDesc.{\ce numelements} () \\
int TypeDesc.{\ce elementsize} ()}
The {\cf size()} is the size in bytes, of the type described.  The
{\cf basesize()} is the size in bytes of the {\cf basetype}.

The {\cf elementtype()} is the type of each array element, if it is an
array, or just the full type if it is not an array.  The {\cf elementsize()}
is the size, in bytes, of the {\cf elementtype} (thus, returning the same
value as {\cf size()} if the type is not an array).  The {\cf numelements()}
method returns {\cf arraylen} if it is an array, or {\cf 1} if it is not
an array.

\noindent Example:
\begin{code}
    t = oiio.TypeDesc("point[2]")
    print "size =", t.size()
    print "elementtype =", t.elementtype()
    print "elementsize =", t.elementsize()

    > size = 24
    > elementtype = point
    > elementsize = 12
\end{code}
\apiend

\apiitem{bool typedesc {\ce ==} typedesc \\
bool typedesc {\ce !=} typedesc \\
bool TypeDesc.{\ce equivalent} (typedesc) \\}
Test for equality or inequality.  The {\cf equivalent()} method is more
forgiving than {\cf ==}, in that it considers {\cf POINT}, {\cf VECTOR},
and {\cf NORMAL} vector semantics to not constitute a difference from one
another.

\noindent Example:
\begin{code}
    f = oiio.TypeDesc("float")
    p = oiio.TypeDesc("point")
    v = oiio.TypeDesc("vector")
    print "float==point?", (f == p)
    print "vector==point?", (v == p)
    print "float.equivalent(point)?", f.equivalent(p)
    print "vector.equivalent(point)?", v.equivalent(p)

    > float==point? False
    > vector==point? False
    > float.equivalent(point)? False
    > vector.equivalent(point)? True
\end{code}
\apiend


\section{ROI}
\label{sec:pythonroi}

The \ROI class that describes an image extent or region of interest,
explained in deail in Section~\ref{sec:ROI}, is replicated for Python.

\apiitem{{\ce ROI} () \\
{\ce ROI} (xbegin, xend, ybegin, yend) \\
{\ce ROI} (xbegin, xend, ybegin, yend, zbegin, zend) \\
{\ce ROI} (xbegin, xend, ybegin, yend, zbegin, zend, chbegin, chend)}
Construct an \ROI with the given bounds.  The constructor with no 
arguments makes an \ROI that is ``undefined.''

\noindent Example:
\begin{code}
    import OpenImageIO as oiio
    ...
    roi = oiio.ROI (0, 640, 0, 480, 0, 1, 0, 4)   # video res RGBA
\end{code}
\apiend

\apiitem{int ROI.{\ce xbegin} \\
int ROI.{\ce xend} \\
int ROI.{\ce ybegin} \\
int ROI.{\ce yend} \\
int ROI.{\ce zbegin} \\
int ROI.{\ce zend} \\
int ROI.{\ce chbegin} \\
int ROI.{\ce chend}}
The basic fields of the \ROI.
\apiend

\apiitem{ROI ROI.{\ce All}}
A pre-constructed undefined \ROI.
\apiend

\apiitem{bool ROI.{\ce defined}}
{\cf True} if the \ROI is defined, {\cf False} if the \ROI is undefined.
\apiend

\apiitem{int ROI.{\ce width} \\
int ROI.{\ce height} \\
int ROI.{\ce depth} \\
int ROI.{\ce nchannels}}
The number of pixels in each dimension, and the number of channels,
as described by the \ROI.
\apiend

\apiitem{int ROI.{\ce npixels}}
The total number of pixels in the region described by the \ROI.
\apiend


\apiitem{ROI {\ce get_roi} (imagespec) \\
ROI {\ce get_roi_full} (imagespec)}
Returns the \ROI corresponding to the pixel data window of the given
\ImageSpec, or the display/full window, respectively.

\noindent Example:
\begin{code}
    spec = oiio.ImageSpec(...)
    roi = oiio.get_roi(spec)
\end{code}
\apiend

\apiitem{{\ce set_roi} (imagespec, roi) \\
{\ce set_roi_full} (imagespec, roi)}
Alter the \ImageSpec's resolution and offset to match the passed \ROI.

\noindent Example:
\begin{code}
    # spec is an ImageSpec
    # The following sets the full (display) window to be equal to the
    # pixel data window:
    oiio.set_roi_full (spec, oiio.get_roi(spec))
\end{code}
\apiend


\section{ImageSpec}
\label{sec:pythonimagespec}

The \ImageSpec class that describes an image, explained in deail in
Section~\ref{sec:ImageSpec}, is replicated for Python.

\apiitem{{\ce ImageSpec} ()\\
{\ce ImageSpec} (basetype) \\
{\ce ImageSpec} (typedesc) \\
{\ce ImageSpec} (xres, yres, nchannels, basetype) \\
{\ce ImageSpec} (xres, yres, nchannels, typespec)}
Constructors of an \ImageSpec. These correspond directly to the constructors
in the C++ bindings.

\noindent Example:
\begin{code}
    import OpenImageIO as oiio
    ...

    # default ctr
    s = oiio.ImageSpec()

    # construct with known pixel type, unknown resolution
    s = oiio.ImageSpec(oiio.UINT8)

    # construct with known resolution, channels, pixel data type
    s = oiio.ImageSpec(640, 480, 4, oiio.HALF)
\end{code}
\apiend

\apiitem{ImageSpec.{\ce width}, ImageSpec.{\ce height}, ImageSpec.{\ce depth} \\
ImageSpec.{\ce x}, ImageSpec.{\ce y}, ImageSpec.{\ce z}}
Resolution and offset of the image data ({\cf int} values).

\noindent Example:
\begin{code}
    s = oiio.ImageSpec (...)
    print "Data window is ({},{})-({},{})".format (s.x, s.x+s.width-1,
                                                   s.y, s.y+s.height-1)
\end{code}
\apiend

\apiitem{ImageSpec.{\ce full_width}, ImageSpec.{\ce full_height}, ImageSpec.{\ce full_depth} \\
ImageSpec.{\ce full_x}, ImageSpec.{\ce full_y}, ImageSpec.{\ce full_z}}
Resolution and offset of the ``full'' display window ({\cf int} values).
\apiend

\apiitem{ImageSpec.{\ce tile_width}, ImageSpec.{\ce tile_height}, ImageSpec.{\ce tile_depth}}
For tiled images, the resolution of the tiles ({\cf int} values).  Will be
{\cf 0} for  untiled images.
\apiend

\apiitem{typedesc ImageSpec.{\ce format}}
A \TypeDesc describing the pixel data.
\apiend

\apiitem{int ImageSpec.{\ce nchannels}}
An {\cf int} giving the number of color channels in the image.
\apiend

\apiitem{ImageSpec.{\ce channelnames}}
A tuple of strings containing the names of each color channel.
\apiend

\apiitem{ImageSpec.{\ce channelformats}}
If all color channels have the same format, that will be {\cf ImageSpec.format},
and {\cf channelformats} will be {\cf None}.  However, if there are different
formats per channel, they will be stored in {\cf channelformats} as a tuple
of BASETYPE values, and {\cf format} will contain the ``widest'' of them.

\noindent Example:
\begin{code}
    if spec.channelformats == None:
        print "All color channels are", str(spec.format)
    else:
        print "Channel formats: "
        for i in range(len(spec.channelformats)):
            print "\t", str(oiio.TypeDesc(spec.channelformats[i]))
\end{code}
\apiend

\apiitem{ImageSpec.{\ce alpha_channel} \\
ImageSpec.{\ce z_channel}}
The channel index containing the alpha or depth channel, respectively, or
-1 if either one does not exist or cannot be identified.
\apiend

\apiitem{ImageSpec.{\ce deep}}
Hold {\cf True} if the image is a \emph{deep} (multiple samples per pixel)
image, of {\cf False} if it is an ordinary image.
\apiend

\apiitem{ImageSpec.{\ce quant_black} \\
ImageSpec.{\ce quant_white} \\
ImageSpec.{\ce quant_min} \\
ImageSpec.{\ce quant_max}}
The quantization parameters used when the \ImageSpec is used to specify
how to open a file for output (refer to
Section~\ref{sec:imageoutput:quantization} for
a more complete explanation of each of these parameters.)
\apiend

\apiitem{ImageSpec.{\ce set_format} (basetype) \\
ImageSpec.{\ce set_format} (typedesc)}
Given a {\cf BASETYPE} or a \TypeDesc, sets the {\cf format} field and
also sets all the {\cf quantize} fields to the defaults for the given
data format.

\noindent Example:
\begin{code}
    s = oiio.ImageSpec ()
    s.set_format (oiio.UINT8)
\end{code}
\apiend

\apiitem{ImageSpec.{\ce default_channel_names} ()}
Sets {\cf channel_names} to the default names given the value of
the {\cf nchannels} field.
\apiend

\apiitem{ImageSpec.{\ce format_from_quantize} (black, white, min, max)}
Given the quantization parameters, returns a \TypeDesc giving the best
data format that matches the quantization.
\apiend

\apiitem{ImageSpec.{\ce channel_bytes} () \\
ImageSpec.{\ce channel_bytes} (channel, native=False)}
Returns the size of a single channel value, in bytes (as an
{\cf int}).
(Analogous to the C++ member functions, see 
Section~\ref{sec:ImageSpecMemberFuncs} for details.)
\apiend

\apiitem{ImageSpec.{\ce pixel_bytes} () \\
ImageSpec.{\ce pixel_bytes} (native) \\
ImageSpec.{\ce pixel_bytes} (chbegin, chend) \\
ImageSpec.{\ce pixel_bytes} (chbegin, chend, native=False)}
Returns the size of a pixel, in bytes (as an {\cf int}).
(Analogous to the C++ member functions, see 
Section~\ref{sec:ImageSpecMemberFuncs} for details.)
\apiend

\apiitem{ImageSpec.{\ce scanline_bytes} (native=False) \\
ImageSpec.{\ce tile_bytes} (native=False) \\
ImageSpec.{\ce image_bytes} (native=False)}
Returns the size of a scanline, tile, or the full image, in bytes (as an
{\cf int}). (Analogous to the C++ member functions, see 
Section~\ref{sec:ImageSpecMemberFuncs} for details.)
\apiend

\apiitem{ImageSpec.{\ce tile_pixels} () \\
ImageSpec.{\ce image_pixels} ()}
Returns the number of pixels in a tile or the full image, respectively
(as an {\cf int}). (Analogous to the C++ member functions, see 
Section~\ref{sec:ImageSpecMemberFuncs} for details.)
\apiend

\apiitem{ImageSpec.{\ce attribute} (name, int) \\
ImageSpec.{\ce attribute} (name, float) \\
ImageSpec.{\ce attribute} (name, string) \\
ImageSpec.{\ce attribute} (name, typedesc, data) \\}
Sets a metadata value in the {\cf extra_attribs}.  If the metadata item
is a single {\cf int}, {\cf float}, or {\cf string}, you can pass it
directly. For other types, you must pass the \TypeDesc and then the
data (for aggregate types or arrays, pass multiple values as a tuple).

\noindent Example:
\begin{code}
    s = oiio.ImageSpec (...)
    s.attribute ("foo_str", "blah")
    s.attribute ("foo_int", 14)
    s.attribute ("foo_float", 3.14)
    s.attribute ("foo_vector", oiio.TypeDesc.TypeVector, (1, 0, 11))
    s.attribute ("foo_matrix", oiio.TypeDesc.TypeMatrix,
                 (1, 0, 0, 0, 0, 2, 0, 0, 0, 0, 1, 0, 1, 2, 3, 1))
\end{code}
\apiend

\apiitem{ImageSpec.{\ce get_attribute} (name) \\
ImageSpec.{\ce get_attribute} (name, typedesc)}
Retrieves a named metadata value from {\cf extra_attribs}.  The generic
{\cf get_attribute()} function returns it regardless of type, or {\cf None}
if the attribute does not exist.  The typed variety will only succeed
if the attribute is actually of that type specified.

\noindent Example:
\begin{code}
    foo = s.get_attribute ("foo")   # None if not found
    foo = s.get_attribute ("foo", oiio.FLOAT)  # None if not found AND float
\end{code}
\apiend

\apiitem{ImageSpec.{\ce get_int_attribute} (name, defaultval=0) \\
ImageSpec.{\ce get_float_attribute} (name, defaultval=0.0) \\
ImageSpec.{\ce get_string_attribute} (name, defaultval="")}
Retrieves a named metadata value from {\cf extra_attribs}, if it is
found and is of the given type; returns the default value (or a passed
value) if not found.

\noindent Example:
\begin{code}
    # If "foo" is not found, or if it's not an int, return 0
    foo = s.get_int_attribute ("foo")

    # If "foo" is not found, or if it's not a string, return "blah"
    foo = s.get_string_attribute ("foo", "blah")
\end{code}
\apiend

\apiitem{ImageSpec.{\ce extra_attribs}}
Direct access to the {\cf extra_attribs} named metadata, appropriate for
iterating over the entire list rather than searching for a particular named
value.

\vspace{-10pt}
\apiitem{len(extra_attribs)}
\vspace{10pt}
Returns the number of extra attributes.
\apiend
\vspace{-24pt}
\apiitem{extra_attribs[i].name}
\vspace{10pt}
The name of the indexed attribute.
\apiend
\vspace{-24pt}
\apiitem{extra_attribs[i].type}
\vspace{10pt}
The type of the indexed attribute, as a \TypeDesc.
\apiend
\vspace{-24pt}
\apiitem{extra_attribs[i].value}
\vspace{10pt}
The value of the indexed attribute.
\apiend

\noindent Example:
\begin{code}
    s = oiio.ImageSpec(...)
    ...
    print "extra_attribs size is", len(s.extra_attribs)
    for i in range(len(s.extra_attribs)) :
        print i, s.extra_attribs[i].name, s.extra_attribs[i].type, " :"
        print "\t", s.extra_attribs[i].value
    print
\end{code}
\apiend


\newpage
\subsection*{Example: Header info}

Here is an illustrative example of the use of \ImageSpec, a working Python
function that opens a file and prints all the relevant header
information:

\begin{tinycode}
#!/usr/bin/env python 
import OpenImageIO as oiio

# Print the contents of an ImageSpec
def print_imagespec (spec, subimage=0, mip=0) :
    if spec.depth <= 1 :
        print ("  resolution %dx%d%+d%+d" % (spec.width, spec.height, spec.x, spec.y))
    else :
        print ("  resolution %dx%d%x%d+d%+d%+d" % 
               (spec.width, spec.height, spec.depth, spec.x, spec.y, spec.z))
    if (spec.width != spec.full_width or spec.height != spec.full_height
        or spec.depth != spec.full_depth) :
        if spec.full_depth <= 1 :
            print ("  full res   %dx%d%+d%+d" % 
                   (spec.full_width, spec.full_height, spec.full_x, spec.full_y))
        else :
            print ("  full res   %dx%d%x%d+d%+d%+d" % 
                   (spec.full_width, spec.full_height, spec.full_depth,
                    spec.full_x, spec.full_y, spec.full_z))
    if spec.tile_width :
        print ("  tile size  %dx%dx%d" % 
               (spec.tile_width, spec.tile_height, spec.tile_depth))
    else :
        print "  untiled"
    if mip >= 1 :
        return
    print "  " + str(spec.nchannels), "channels:", spec.channelnames
    print "  format = ", str(spec.format)
    if spec.channelformats :
        print "  channelformats = ", spec.channelformats
    print "  alpha channel = ", spec.alpha_channel
    print "  z channel = ", spec.z_channel
    print "  deep = ", spec.deep
    for i in range(len(spec.extra_attribs)) :
        if type(spec.extra_attribs[i].value) == str :
            print " ", spec.extra_attribs[i].name, "= \"" + spec.extra_attribs[i].value + "\""
        else :
            print " ", spec.extra_attribs[i].name, "=", spec.extra_attribs[i].value


def poor_mans_iinfo (filename) :
    input = oiio.ImageInput.open (filename)
    if not input :
        print 'Could not open "' + filename + '"'
        print "\tError: ", oiio.geterror()
        return
    print 'Opened "' + filename + '" as a ' + input.format_name()
    sub = 0
    mip = 0
    while True :
        if sub > 0 or mip > 0 :
            print "Subimage", sub, "MIP level", mip, ":"
        print_imagespec (input.spec(), mip=mip)
        mip = mip + 1
        if input.seek_subimage (sub, mip) :
            continue    # proceed to next MIP level
        else :
            sub = sub + 1
            mip = 0
            if input.seek_subimage (sub, mip) :
                continue    # proceed to next subimage
        break  # no more MIP levels or subimages
    input.close ()
\end{tinycode}


\section{ImageInput}
\label{sec:pythonimageinput}

See Chapter~\ref{chap:imageinput} for detailed explanations of the
C++ \ImageInput class APIs. The Python APIs are very similar. The biggest
difference is that in C++, the various {\cf read_*} functions write the
pixel values into an already-allocated array that belongs to the caller,
whereas the Python versions allocate and return an array holding the pixel
values (or {\cf None} if the read failed).


\apiitem{ImageInput.{\ce open} (filename) \\
ImageInput.{\ce open} (filename, config_imagespec)}
Creates an \ImageInput object and opens the named file.  Returns the
open \ImageInput upon success, or {\cf None} if it failed to open the
file (after which, {\cf OpenImageIO.geterror()} will contain an error
message).  In the second form, the optional \ImageSpec argument 
{\cf config} contains attributes that may set certain options when opening
the file.

\noindent Example:
\begin{code}
    input = oiio.ImageInput.open ("tahoe.jpg")
    if input == None :
        print "Error:", oiio.geterror()
        return
\end{code}
\apiend

\apiitem{bool ImageInput.{\ce close} ()}
Closes an open image file, returning {\cf True} if successful, {\cf False}
otherwise.

\noindent Example:
\begin{code}
    input = oiio.ImageInput.open (filename)
    ...
    input.close ()
\end{code}
\apiend


\apiitem{str ImageInput.{\ce format_name} ()}
Returns the format name of the open file.

\noindent Example:
\begin{code}
    input = oiio.ImageInput.open (filename)
    if input :
        print filename, "was a", input.format_name(), "file."
        input.close ()
  
\end{code}
\apiend

\apiitem{ImageSpec ImageInput.{\ce spec} ()}
Returns the \ImageSpec corresponding to the currently open subimage and
MIP level of the file.

\noindent Example:
\begin{code}
    input = oiio.ImageInput.open (filename)
    spec = input.spec()
    print "resolution ", spec.width, "x", spec.height
\end{code}
\apiend

\apiitem{int ImageInput.{\ce current_subimage} () \\
int ImageInput.{\ce current_miplevel} ()}
Returns the current subimage and/or MIP level of the file.
\apiend

\apiitem{bool ImageInput.{\ce seek_subimage} (subimage, miplevel)}
Repositions the file pointer to the given subimage and MIP level within the
file (starting with {\cf 0}).  This function returns {\cf True} upon success,
{\cf False} upon failure (which may include the file not having the
specified subimage or MIP level).

\noindent Example:
\begin{code}
    input = oiio.ImageInput.open (filename)
    mip = 0
    while True :
        ok = input.seek_subimage (0, mip)
        if not ok :
            break;
        spec = input.spec()
        print "MIP level", mip, "is", spec.width, "x", spec.height
\end{code}
\apiend

\apiitem{array ImageInput.{\ce read_image} (type=OpenImageIO.UNKNOWN)}
Read the entire image and return the pixels as an array of
$\mathit{width} \times \mathit{height} \times \mathit{depth} \times \mathit{nchannels}$
values (or {\cf None} if an error occurred).  The array will be of the type
specified by the {\cf TypeDesc} or {\cf BASETYPE} argument, or  if the
argument is missing or is {\cf TypeDesc(oiio.UNKNOWN)}, it will choose an
appropriate data type that is the ``smallest'' type that can hold the actual
types in the file.

\noindent Example:
\begin{code}
    input = oiio.ImageInput.open (filename)
    spec = input.spec ()
    pixels = input.read_image (oiio.FLOAT)
    print "The first pixel is", pixels[0:spec.nchannels]
    print "The second pixel is", pixels[spec.nchannels:(2*spec.nchannels)]
    input.close ()
\end{code}
\apiend

\apiitem{array ImageInput.{\ce read_scanline} (y, z, type=OpenImageIO.UNKNOWN)}
Read scanline number {\cf y} from depth plane {\cf z} from the open file,
returning it as an array of $\mathit{width} \times \mathit{nchannels}$
values (or {\cf None} if an error occurred). The array will be of the type
specified by the {\cf TypeDesc} or {\cf BASETYPE} argument, or if the
argument is missing or is {\cf TypeDesc(oiio.UNKNOWN)}, it will choose an
appropriate data type.

\noindent Example:
\begin{code}
    input = oiio.ImageInput.open (filename)
    spec = input.spec ()
    if spec.tile_width == 0 :
        for y in range(spec.y, spec.y+spec.height) :
            pixels = input.read_scanline (y, spec.z, oiio.FLOAT)
            # process the scanline
    else :
        print "It's a tiled file"
    input.close ()
\end{code}
\apiend

\apiitem{array ImageInput.{\ce read_tile} (x, y, z, type=OpenImageIO.UNKNOWN)}
Read the tile whose upper left corner is pixel {\cf (x,y,z)} from the open
file, returning it as an array of
$\mathit{width} \times \mathit{height} \times \mathit{depth} \times \mathit{nchannels}$
values (or {\cf None} if an error occurred). The array will be of the type
specified by the {\cf TypeDesc} or {\cf BASETYPE} argument, or if the
argument is missing or {\cf TypeDesc(oiio.UNKNOWN)}, it will choose an
appropriate data type.

\noindent Example:
\begin{code}
    input = oiio.ImageInput.open (filename)
    spec = input.spec ()
    if spec.tile_width > 0 :
        for z in range(spec.z, spec.z+spec.depth, spec.tile_depth) :
            for y in range(spec.y, spec.y+spec.height, spec.tile_height) :
                for x in range(spec.x, spec.x+spec.width, spec.tile_width) :
                    pixels = input.read_tile (x, y, z, oiio.FLOAT)
                    # process the tile
    else :
        print "It's a scanline file"
    input.close ()
\end{code}
\apiend

\apiitem{array ImageInput.{\ce read_scanlines} (ybegin, yend, z, chbegin, chend, \\
\bigspc\bigspc\spc type=OpenImageIO.UNKNOWN) \\
array ImageInput.{\ce read_tiles} (xbegin, xend, ybegin, yend, zbegin, zend, \\
    \bigspc\bigspc\spc chbegin, chend, type=OpenImageIO.UNKNOWN)}
Similar to the C++ routines, these functions read multiple scanlines or 
tiles at once, which in some cases may be more efficient than reading
each scanline or tile separately.  Additionally, they allow you to read only
a subset of channels.

\noindent Example:
\begin{code}
    input = oiio.ImageInput.open (filename)
    spec = input.spec ()

    # Read the whole image, the equivalent of
    #     pixels = input.read_image (type)
    # but do it using read_scanlines or read_tiles:
    if spec.tile_width == 0 :
        pixels = input.read_scanlines (spec.y, spec.y+spec.height, 0,
                                       0, spec.nchannels, oiio.FLOAT)
    else :
        pixels = input.read_tiles (spec.x, spec.x+spec.width,
                                   spec.y, spec.y+spec.height,
                                   spec.z, spec.z+spec.depth,
                                   0, spec.nchannels, oiio.FLOAT)
\end{code}
\apiend

\apiitem{str ImageInput.{\ce geterror} ()}
Retrieves the error message from the latest failed operation on an
ImageInput.

\noindent Example:
\begin{code}
    input = oiio.ImageInput.open (filename)
    if not input :
        print "Open error:", oiio.geterror()
        # N.B. error on open must be retrieved with the global geterror(),
        # since there is no ImageInput object!
    else :
        pixels = input.read_image (oiio.FLOAT)
        if not pixels :
            print "Read_image error:", input.geterror()
        input.close ()
\end{code}
\apiend

\newpage
\subsection*{Example: Reading pixel values from a file to find min/max}

\begin{code}
#!/usr/bin/env python 
import OpenImageIO as oiio

def find_min_max (filename) :
    input = oiio.ImageInput.open (filename)
    if not input :
        print 'Could not open "' + filename + '"'
        print "\tError: ", oiio.geterror()
        return
    spec = input.spec()
    nchans = spec.nchannels
    pixels = input.read_image(oiio.FLOAT)
    if not pixels :
        print "Could not read:", input.geterror()
        return
    input.close()    # we're done with the file at this point
    minval = pixels[0:nchans]   # initialize to the first pixel value
    maxval = pixels[0:nchans]
    i = 0    # absolute index
    for z in range(spec.depth) :
        for y in range(spec.height) :
            for x in range(spec.width) :
                for c in range(nchans) :
                    if pixels[i+c] < minval[c] :
                        minval[c] = pixels[i+c]
                    if pixels[i+c] > maxval[c] :
                        maxval[c] = pixels[i+c]
                i = i + nchans   # advance the index
    print "Min values per channel were", minval
    print "Max values per channel were", maxval
\end{code}
\newpage


\section{ImageOutput}
\label{sec:pythonimageoutput}

See Chapter~\ref{chap:imageoutput} for detailed explanations of the
C++ \ImageOutput class APIs. The Python APIs are very similar.

\apiitem{ImageOutput ImageOutput.{\ce create} (fileformat, plugin_searchpath="")}

Create a new \ImageOutput capable of writing the named file format (which may
also be a file name, with the type deduced from the extension).  There
is an optional parameter giving an colon-separated search path for finding
\ImageOutput plugins.  The function returns an \ImageOutput object, or
{\cf None} upon error (in which case, {OpenImageIO.geterror()} may be used
to retrieve the error message).

\noindent Example:
\begin{code}
    import OpenImageIO as oiio
    output = oiio.ImageOutput.create ("myfile.tif")
    if not output :
        print "Error:", oiio.geterror()
\end{code}
\apiend

\apiitem{str ImageOutput.{\ce format_name} ()}
The file format name of a created \ImageOutput.

\noindent Example:
\begin{code}
    output = oiio.ImageOutput.create (filename)
    if output :
        print "Created output", filename, "as a", output.format_name()
\end{code}
\apiend

\apiitem{bool ImageOutput.{\ce supports} (feature)}
For a created \ImageOutput, returns {\cf True} if the file format supports
the named feature (such as \qkw{tiles}, \qkw{mipmap}, etc., see
Section~\ref{sec:supportsfeaturelist} for the full list), or {\cf False}
if this file format does not support the feature.

\noindent Example:
\begin{code}
    output = oiio.ImageOutput.create (filename)
    if output :
        print output.format_name(), "supports..."
        print "tiles?", output.supports("tiles")
        print "multi-image?", output.supports("multiimage")
        print "MIP maps?", output.supports("mipmap")
        print "per-channel formats?", output.supports("channelformats")
\end{code}
\apiend

\apiitem{bool ImageOutput.{\ce open} (filename, imagespec, mode)}
Opens the named output file, with an \ImageSpec describing the image to
be output.  The {\cf mode} may be one of {\cf OpenImageIO.Create},
{\cf OpenImageIO.AppendSubimage}, or {\cf OpenImageIO.AppendMIPLevel}.
See Section~\ref{sec:imageoutputopen} for details.  Returns {\cf True}
upon success, {\cf False} upon failure (error messages retrieved via
{\cf ImageOutput.geterror()}.)

\noindent Example:
\begin{code}
    output = oiio.ImageOutput.create (filename)
    if not output :
        print "Error:", oiio.geterror()
    spec = oiio.ImageSpec (640, 480, 3, oiio.UINT8)
    ok = output.open (filename, spec, oiio.Create)
    if not ok :
        print "Could not open", filename, ":", output.geterror()
\end{code}
\apiend

\apiitem{bool ImageOutput.{\ce open} (filename, (imagespec, ...))}
This variety of {\cf open()} is used specifically for multi-subimage files.
A \emph{tuple} of \ImageSpec objects is passed, one for each subimage
that will be written to the file.  After each subimage is written, then
a regular call to {\cf open(name, newspec, {\ce AppendSubimage})} moves
on to the next subimage.
\apiend

\apiitem{bool ImageOutput.{\ce close} ()}
Closes an open output.
\apiend

\apiitem{ImageSpec ImageOutput.{\ce spec} ()}
Retrieves the \ImageSpec of the currently-open output image.
\apiend

\apiitem{bool ImageOutput.{\ce write_image} (typedesc, pixels, xstride=AutoStride, \\
\bigspc\bigspc\spc ystride=AutoStride, zstride=AutoStride)}
Write the currently opened image all at once.  The {\cf pixels} parameter
should be an {\cf array} containing data elements of the type described by
the \TypeDesc.  Optional strides describe data layout in the array (the
default values of {\cf AutoStride} are used if the data are contiguous).
Returns {\cf True} upon success, {\cf False} upon failure.

\noindent Example:
\begin{code}
    # This example reads a scanline file, then converts it to tiled
    # and writes to the same name.

    input = oiio.ImageInput.open (filename)
    spec = input.spec ()
    pixels = input.read_image (oiio.FLOAT)
    input.close ()

    output = oiio.ImageOutput.create (filename)
    if output.supports("tiles") :
        spec.tile_width = 64
        spec.tile_height = 64
        output.open (filename, spec, oiio.Create)
        output.write_image (oiio.FLOAT, pixels)
        output.close ()
\end{code}
\apiend

\apiitem{bool ImageOutput.{\ce write_scanline} (y, z, typesdesc, pixels, \\
\bigspc\bigspc xstride=AutoStride) \\
bool ImageOutput.{\ce write_scanlines} (ybegin, yend, z, typesdesc, pixels, \\
\bigspc\bigspc xstride=AutoStride)}

Write one or many scanlines to the currently open file.

\noindent Example:
\begin{code}
    # Copy a TIFF image to JPEG by copying scanline by scanline.
    input = oiio.ImageInput.open ("in.tif")
    spec = input.spec ()
    output = oiio.ImageOutput.create ("out.jpg")
    output.open (filename, spec, oiio.Create)
    for z in range(spec.z, spec.z+spec.depth) :
        for y in range(spec.y, spec.y+spec.height) :
            pixels = input.read_scanline (y, z, oiio.FLOAT)
            output.write_scanline (y, z, oiio.FLOAT, pixels)
    output.close ()
    input.close ()

    # The same example, but copying a whole "plane" of scanlines at a time:
    ...
    for z in range(spec.z, spec.z+spec.depth) :
        pixels = input.read_scanlines (spec.y, spec.y+spec.height,
                                       z, oiio.FLOAT)
        output.write_scanlines (spec.y, spec.y+spec.height,
                                z, oiio.FLOAT, pixels)
    ...
\end{code}
\apiend

\apiitem{bool ImageOutput.{\ce write_tile} (x, y, z, typesdesc, pixels, \\
\bigspc\bigspc\spc xstride=AutoStride, ystride=AutoStride, \\
\bigspc\bigspc\spc zstride=AutoStride) \\
bool ImageOutput.{\ce write_tiles} (xbegin, xend, ybegin, yend, zbegin, zend, \\
\bigspc\bigspc\spc  xstride=AutoStride, ystride=AutoStride, \\
\bigspc\bigspc\spc  zstride=AutoStride)}

Write one or many tiles to the currently open file.

\noindent Example:
\begin{code}
    input = oiio.ImageInput.open (in_filename)
    spec = input.spec ()
    output = oiio.ImageOutput.create (out_filename)
    output.open (out_filename, spec, oiio.Create)
    for z in range(spec.z, spec.z+spec.depth, spec.tile_depth) :
        for y in range(spec.y, spec.y+spec.height, spec.tile_height) :
            for x in range(spec.x, spec.x+spec.width, spec.tile_width) :
                pixels = input.read_tile (x, y, z, oiio.FLOAT)
                output.read_tile (x, y, z, oiio.FLOAT, pixels)
    output.close ()
    input.close ()

    # The same example, but copying a whole row of of tiles at a time:
    ...
    for z in range(spec.z, spec.z+spec.depth, spec.tile_depth) :
        for y in range(spec.y, spec.y+spec.height, spec.tile_height) :
            pixels = input.read_tiles (spec.x, spec.x+spec.width,
                                       y, y+tile_width,
                                       z, z+tile_width, oiio.FLOAT)
            output.read_tile (spec.x, spec.x+spec.width,
                              y, y+tile_width, z, z+tile_width,
                              oiio.FLOAT, pixels)
    ...
\end{code}
\apiend

\apiitem{bool ImageOutput.{\ce copy_image} (imageinput)}
Copy the current image of the open input to the open output. (The reason
this may be preferred in some circumstances is that, if input and
output were the same kind of input file format, they may have a special
efficient technique to copy pixels unaltered, for example by avoiding the 
decompression/recompression round trip.)

\noindent Example:
\begin{code}
    input = oiio.ImageInput.open (in_filename)
    spec = input.spec ()
    output = oiio.ImageOutput.create (out_filename)
    output.open (filename, spec, oiio.Create)
    output.copy_image (input)
    output.close ()
    input.close ()
\end{code}
\apiend

\apiitem{str ImageOuput.{\ce geterror} ()}
Retrieves the error message from the latest failed operation on an open
file.

\noindent Example:
\begin{code}
    output = oiio.ImageOutput.create (filename)
    if not output :
        print "Create error:", oiio.geterror()
        # N.B. error on create must be retrieved with the global geterror(),
        # since there is no ImageOutput object!
    else :
        ok = output.open (filename, spec, oiio.Create)
        if not ok :
            print "Open error:", output.geterror()
        ok = output.write_image (oiio.FLOAT, pixels)
        if not ok :
            print "Write error:", output.geterror()
        output.close ()
\end{code}
\apiend



\section{ImageBuf}
\label{sec:pythonimagebuf}

Coming soon.

\section{ImageBufAlgo}
\label{sec:pythonimagebufalgo}

Coming soon.


\section{Miscellaneous Utilities}
\label{sec:pythonmiscapi}

In the main {\cf OpenImageIO} module, there are a number of values and
functions that are useful.  These correspond to the C++ API functions
explained in Section~\ref{sec:miscapi}, please refer there for details.

\apiitem{int {\ce openimageio_version}}
The \product version number, 10000 for each
major version, 100 for each minor version, 1 for each patch.  For
example, \product 1.2.3 would return a value of 10203.
\apiend

\apiitem{str {\ce geterror} ()}
Retrieves the latest global error.
\apiend

\apiitem{bool {\ce attribute} (name, typedesc, value) \\
bool {\ce attribute} (name, int_value) \\
bool {\ce attribute} (name, float_value) \\
bool {\ce attribute} (name, str_value)}
Sets a global attribute (see Section~\ref{sec:miscapi} for details),
returning {\cf True} upon success, or {\cf False} if it was not a
recognized attribute.  

\noindent Example:
\begin{code}
    oiio.attribute ("threads", 0)
\end{code}
\apiend


\index{Python|)}

\chapwidthend
