\chapter{Python Bindings}
\label{chap:pythonbindings}
\indexapi{Python|()}

\section{Overview}

\OpenImageIO provides Python language bindings for much of its
functionality.

\smallskip

You must ensure that the environment variable {\cf PYTHONPATH} includes
the {\cf python} subdirectory of the \OpenImageIO installation.

\smallskip

A Python program must import the {\cf OpenImageIO} package:
\begin{code}
    import OpenImageIO
\end{code}
\noindent In most of our examples below, we assume that for the sake
of brevity, we will alias the package name as follows:
\begin{code}
    import OpenImageIO as oiio
\end{code}

\section{TypeDesc}

The \TypeDesc class that describes data types of pixels and metadata,
described in detail in Section~\ref{sec:TypeDesc}, is replicated for Python.

\apiitem{BASETYPE}
The {\cf BASETYPE} enum corresponds to the C++ {\cf TypeDesc::BASETYPE} and
contains the following values: \\
{\cf UNKNOWN NONE UINT8 INT8 UINT16 INT16 UINT32 INT32 UINT64 INT64 \\
HALF FLOAT DOUBLE STRING PTR} \\
These names are also exported to the {\cf OpenImageIO} namespace.
\apiend

\apiitem{AGGREGATE}
The {\cf AGGREGATE} enum corresponds to the C++ {\cf TypeDesc::AGGREGATE} and
contains the following values: \\
{\cf SCALAR VEC2 VEC3 VEC4 MATRIX44} \\
These names are also exported to the {\cf OpenImageIO} namespace.
\apiend

\apiitem{VECSEMANTICS}
The {\cf VECSEMANTICS} enum corresponds to the C++ {\cf TypeDesc::VECSEMANTICS} and
contains the following values: \\
{\cf NOXFORM COLOR POINT VECTOR NORMAL} \\
These names are also exported to the {\cf OpenImageIO} namespace.
\apiend

\apiitem{TypeDesc () \\
TypeDesc (basetype) \\
TypeDesc (basetype, aggregate) \\
TypeDesc (basetype, aggregate, vecsemantics) \\
TypeDesc (basetype, aggregate, vecsemantics, arraylen) \\
TypeDesc (str)}

Construct a {\cf TypeDesc} object.  

\noindent Examples:
\begin{code}
    import OpenImageIO as oiio

    # make a default (UNKNOWN) TypeDesc
    t = oiio.TypeDesc()

    # make a TypeDesc describing an unsigned 8 bit int
    t = oiio.TypeDesc(oiio.UINT8)

    # make a TypeDesc describing an array of 14 'half' values
    t = oiio.TypeDesc(oiio.HALF, oiio.SCALAR, oiio.NOXFORM, 14)

    # make a TypeDesc describing a float point
    t = oiio.TypeDesc(oiio.FLOAT, oiio.VEC3, oiio.POINT)

    # Some constructors from a string description
    t = oiio.TypeDesc("uint8")
    t = oiio.TypeDesc("half[14]")
    t = oiio.TypeDesc("point")     # equiv to FLOAT, VEC3, POINT
\end{code}
\apiend

\apiitem{TypeDesc.TypeFloat() \\
TypeDesc.TypeInt() \\
TypeDesc.TypeString() \\
TypeDesc.TypeColor() \\
TypeDesc.TypePoint() \\
TypeDesc.TypeVector() \\
TypeDesc.TypeNormal() \\
TypeDesc.TypeMatrix()}
Pre-constructed \TypeDesc objects for some common types.

\noindent Example:
\begin{code}
    t = oiio.TypeDesc.TypeFloat()
\end{code}
\apiend

\apiitem{string {\ce str} (TypeDesc)}
Returns a string that describes the \TypeDesc.

\noindent Example:
\begin{code}
    print str(oiio.TypeDesc(oiio.UINT16))

    > int16
\end{code}
\apiend

\apiitem{TypeDesc.{\ce basetype} \\
TypeDesc.{\ce aggregate} \\
TypeDesc.{\ce vecsemantics} \\
TypeDesc.{\ce arraylen}}
Access to the raw fields in the \TypeDesc.

\noindent Example:
\begin{code}
    t = oiio.TypeDesc(...)
    if t.basetype == oiio.FLOAT :
        print "It's made of floats"
\end{code}
\apiend

\apiitem{int TypeDesc.{\ce size} () \\
int TypeDesc.{\ce basesize} () \\
TypeDesc TypeDesc.{\ce elementtype} () \\
int TypeDesc.{\ce numelements} () \\
int TypeDesc.{\ce elementsize} ()}
The {\cf size()} is the size in bytes, of the type described.  The
{\cf basesize()} is the size in bytes of the {\cf basetype}.

The {\cf elementtype()} is the type of each array element, if it is an
array, or just the full type if it is not an array.  The {\cf elementsize()}
is the size, in bytes, of the {\cf elementtype} (thus, returning the same
value as {\cf size()} if the type is not an array).  The {\cf numelements()}
method returns {\cf arraylen} if it is an array, or {\cf 1} if it is not
an array.

\noindent Example:
\begin{code}
    t = oiio.TypeDesc("point[2]")
    print "size =", t.size()
    print "elementtype =", t.elementtype()
    print "elementsize =", t.elementsize()

    > size = 24
    > elementtype = point
    > elementsize = 12
\end{code}
\apiend

\apiitem{bool typedesc {\ce ==} typedesc \\
bool typedesc {\ce !=} typedesc \\
bool TypeDesc.{\ce equivalent} (typedesc) \\}
Test for equality or inequality.  The {\cf equivalent()} method is more
forgiving than {\cf ==}, in that it considers {\cf POINT}, {\cf VECTOR},
and {\cf NORMAL} vector semantics to not constitute a difference from one
another.

\noindent Example:
\begin{code}
    f = oiio.TypeDesc("float")
    p = oiio.TypeDesc("point")
    v = oiio.TypeDesc("vector")
    print "float==point?", (f == p)
    print "vector==point?", (v == p)
    print "float.equivalent(point)?", f.equivalent(p)
    print "vector.equivalent(point)?", v.equivalent(p)

    > float==point? False
    > vector==point? False
    > float.equivalent(point)? False
    > vector.equivalent(point)? True
\end{code}
\apiend


\section{ImageSpec}

The \ImageSpec class that describes an image, explained in deail in
Section~\ref{sec:ImageSpec}, is replicated for Python.

\apiitem{{\ce ImageSpec} ()\\
{\ce ImageSpec} (basetype) \\
{\ce ImageSpec} (typedesc) \\
{\ce ImageSpec} (xres, yres, nchannels, basetype) \\
{\ce ImageSpec} (xres, yres, nchannels, typespec)}
Constructors of an \ImageSpec. These correspond directly to the constructors
in the C++ bindings.

\noindent Example:
\begin{code}
    import OpenImageIO as oiio
    ...

    # default ctr
    s = oiio.ImageSpec()

    # construct with known pixel type, unknown resolution
    s = oiio.ImageSpec(oiio.UINT8)

    # construct with known resolution, channels, pixel data type
    s = oiio.ImageSpec(640, 480, 4, oiio.HALF)
\end{code}
\apiend

\apiitem{ImageSpec.{\ce width}, ImageSpec.{\ce height}, ImageSpec.{\ce depth} \\
ImageSpec.{\ce x}, ImageSpec.{\ce y}, ImageSpec.{\ce z}}
Resolution and offset of the image data ({\cf int} values).

\noindent Example:
\begin{code}
    s = oiio.ImageSpec (...)
    print "Data window is ({},{})-({},{})".format (s.x, s.x+s.width-1,
                                                   s.y, s.y+s.height-1)
\end{code}
\apiend

\apiitem{ImageSpec.{\ce full_width}, ImageSpec.{\ce full_height}, ImageSpec.{\ce full_depth} \\
ImageSpec.{\ce full_x}, ImageSpec.{\ce full_y}, ImageSpec.{\ce full_z}}
Resolution and offset of the ``full'' display window ({\cf int} values).
\apiend

\apiitem{ImageSpec.{\ce tile_width}, ImageSpec.{\ce tile_height}, ImageSpec.{\ce tile_depth}}
For tiled images, the resolution of the tiles ({\cf int} values).  Will be
{\cf 0} for  untiled images.
\apiend

\apiitem{typedesc ImageSpec.{\ce format}}
A \TypeDesc describing the pixel data.
\apiend

\apiitem{int ImageSpec.{\ce nchannels}}
An {\cf int} giving the number of color channels in the image.
\apiend

\apiitem{ImageSpec.{\ce channelnames}}
A tuple of strings containing the names of each color channel.
\apiend

\apiitem{ImageSpec.{\ce channelformats}}
If all color channels have the same format, that will be {\cf ImageSpec.format},
and {\cf channelformats} will be {\cf None}.  However, if there are different
formats per channel, they will be stored in {\cf channelformats} as a tuple
of BASETYPE values, and {\cf format} will contain the ``widest'' of them.

\noindent Example:
\begin{code}
    if spec.channelformats == None:
        print "All color channels are", str(spec.format)
    else:
        print "Channel formats: "
        for i in range(len(spec.channelformats)):
            print "\t", str(oiio.TypeDesc(spec.channelformats[i]))
\end{code}
\apiend

\apiitem{ImageSpec.{\ce alpha_channel} \\
ImageSpec.{\ce z_channel}}
The channel index containing the alpha or depth channel, respectively, or
-1 if either one does not exist or cannot be identified.
\apiend

\apiitem{ImageSpec.{\ce deep}}
Hold {\cf True} if the image is a \emph{deep} (multiple samples per pixel)
image, of {\cf False} if it is an ordinary image.
\apiend

\apiitem{ImageSpec.{\ce quant_black} \\
ImageSpec.{\ce quant_white} \\
ImageSpec.{\ce quant_min} \\
ImageSpec.{\ce quant_max}}
The quantization parameters used when the \ImageSpec is used to specify
how to open a file for output (refer to
Section~\ref{sec:imageoutput:quantization} for
a more complete explanation of each of these parameters.)
\apiend

\apiitem{ImageSpec.{\ce set_format} (basetype) \\
ImageSpec.{\ce set_format} (typedesc)}
Given a {\cf BASETYPE} or a \TypeDesc, sets the {\cf format} field and
also sets all the {\cf quantize} fields to the defaults for the given
data format.

\noindent Example:
\begin{code}
    s = oiio.ImageSpec ()
    s.set_format (oiio.UINT8)
\end{code}
\apiend

\apiitem{ImageSpec.{\ce default_channel_names} ()}
Sets {\cf channel_names} to the default names given the value of
the {\cf nchannels} field.
\apiend

\apiitem{ImageSpec.{\ce format_from_quantize} (black, white, min, max)}
Given the quantization parameters, returns a \TypeDesc giving the best
data format that matches the quantization.
\apiend

\apiitem{ImageSpec.{\ce channel_bytes} () \\
ImageSpec.{\ce channel_bytes} (channel, native=False)}
Returns the size of a single channel value, in bytes (as an
{\cf int}).
(Analogous to the C++ member functions, see 
Section~\ref{sec:ImageSpecMemberFuncs} for details.)
\apiend

\apiitem{ImageSpec.{\ce pixel_bytes} () \\
ImageSpec.{\ce pixel_bytes} (native) \\
ImageSpec.{\ce pixel_bytes} (chbegin, chend) \\
ImageSpec.{\ce pixel_bytes} (chbegin, chend, native=False)}
Returns the size of a pixel, in bytes (as an {\cf int}).
(Analogous to the C++ member functions, see 
Section~\ref{sec:ImageSpecMemberFuncs} for details.)
\apiend

\apiitem{ImageSpec.{\ce scanline_bytes} (native=False) \\
ImageSpec.{\ce tile_bytes} (native=False) \\
ImageSpec.{\ce image_bytes} (native=False)}
Returns the size of a scanline, tile, or the full image, in bytes (as an
{\cf int}). (Analogous to the C++ member functions, see 
Section~\ref{sec:ImageSpecMemberFuncs} for details.)
\apiend

\apiitem{ImageSpec.{\ce tile_pixels} () \\
ImageSpec.{\ce image_pixels} ()}
Returns the number of pixels in a tile or the full image, respectively
(as an {\cf int}). (Analogous to the C++ member functions, see 
Section~\ref{sec:ImageSpecMemberFuncs} for details.)
\apiend

\apiitem{ImageSpec.{\ce attribute} (name, int) \\
ImageSpec.{\ce attribute} (name, float) \\
ImageSpec.{\ce attribute} (name, string) \\
ImageSpec.{\ce attribute} (name, typedesc, data) \\}
Sets a metadata value in the {\cf extra_attribs}.  If the metadata item
is a single {\cf int}, {\cf float}, or {\cf string}, you can pass it
directly. For other types, you must pass the \TypeDesc and then the
data (for aggregate types or arrays, pass multiple values as a tuple).

\noindent Example:
\begin{code}
    s = oiio.ImageSpec (...)
    s.attribute ("foo_str", "blah")
    s.attribute ("foo_int", 14)
    s.attribute ("foo_float", 3.14)
    s.attribute ("foo_vector", oiio.TypeDesc.TypeVector, (1, 0, 11))
    s.attribute ("foo_matrix", oiio.TypeDesc.TypeMatrix,
                 (1, 0, 0, 0, 0, 2, 0, 0, 0, 0, 1, 0, 1, 2, 3, 1))
\end{code}
\apiend

\apiitem{ImageSpec.{\ce get_attribute} (name) \\
ImageSpec.{\ce get_attribute} (name, typedesc)}
Retrieves a named metadata value from {\cf extra_attribs}.  The generic
{\cf get_attribute()} function returns it regardless of type, or {\cf None}
if the attribute does not exist.  The typed variety will only succeed
if the attribute is actually of that type specified.

\noindent Example:
\begin{code}
    foo = s.get_attribute ("foo")   # None if not found
    foo = s.get_attribute ("foo", oiio.FLOAT)  # None if not found AND float
\end{code}
\apiend

\apiitem{ImageSpec.{\ce get_int_attribute} (name, defaultval=0) \\
ImageSpec.{\ce get_float_attribute} (name, defaultval=0.0) \\
ImageSpec.{\ce get_string_attribute} (name, defaultval="")}
Retrieves a named metadata value from {\cf extra_attribs}, if it is
found and is of the given type; returns the default value (or a passed
value) if not found.

\noindent Example:
\begin{code}
    # If "foo" is not found, or if it's not an int, return 0
    foo = s.get_int_attribute ("foo")

    # If "foo" is not found, or if it's not a string, return "blah"
    foo = s.get_string_attribute ("foo", "blah")
\end{code}
\apiend

\apiitem{ImageSpec.{\ce extra_attribs}}
Direct access to the {\cf extra_attribs} named metadata, appropriate for
iterating over the entire list rather than searching for a particular named
value.

\vspace{-10pt}
\apiitem{len(extra_attribs)}
\vspace{10pt}
Returns the number of extra attributes.
\apiend
\vspace{-24pt}
\apiitem{extra_attribs[i].name}
\vspace{10pt}
The name of the indexed attribute.
\apiend
\vspace{-24pt}
\apiitem{extra_attribs[i].type}
\vspace{10pt}
The type of the indexed attribute, as a \TypeDesc.
\apiend
\vspace{-24pt}
\apiitem{extra_attribs[i].value}
\vspace{10pt}
The value of the indexed attribute.
\apiend

\noindent Example:
\begin{code}
    s = oiio.ImageSpec(...)
    ...
    print "extra_attribs size is", len(s.extra_attribs)
    for i in range(len(s.extra_attribs)) :
        print i, s.extra_attribs[i].name, s.extra_attribs[i].type, " :"
        print "\t", s.extra_attribs[i].value
    print
\end{code}
\apiend


\index{Python|)}

\chapwidthend
