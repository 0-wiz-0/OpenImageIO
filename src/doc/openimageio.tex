\documentclass[11pt,letterpaper]{book}
\setlength{\oddsidemargin}{0.5in}
\setlength{\topmargin}{0in}
\setlength{\evensidemargin}{0.3in}
\setlength{\textwidth}{5.75in}
\setlength{\textheight}{8.5in}
%\setlength{\oddsidemargin}{1.25in}
%\setlength{\evensidemargin}{0.5in}

% don't do this \usepackage{times}    % Better fonts than Computer Modern
\renewcommand{\sfdefault}{phv}
\renewcommand{\rmdefault}{ptm}
% don't replace tt -- old is better \renewcommand{\ttdefault}{pcr}
%\usepackage{apalike}
\usepackage{pslatex}
\usepackage{techref}
\usepackage{epsfig}
\usepackage{verbatim}
\usepackage{moreverb}
\usepackage{graphicx}
\usepackage{xspace}
\usepackage{multicol}
\usepackage{color}
\usepackage{html}
\usepackage{version}
\usepackage{makeidx}
%\usepackage{showidx}
\usepackage[chapter]{algorithm}
\floatname{algorithm}{Listing}

\usepackage{syntax}

\usepackage{fancyhdr}
\pagestyle{fancy}
\fancyhead[LE,RO]{\bfseries\thepage}
\fancyhead[LO]{\bfseries\rightmark}
\fancyhead[RE]{\bfseries\leftmark}
\fancyfoot[C]{\bfseries OpenImageIO Programmer's Documentation}
\renewcommand{\footrulewidth}{1pt}


\def\product{{\sffamily OpenImageIO}\xspace}
\def\OpenImageIO{{\sffamily OpenImageIO}\xspace}
\def\versionnumber{0.1}
\def\productver{\product\ {\sffamily \versionnumber}\xspace}
\def\producthome{{\codefont \$IMAGEIOHOME}\xspace}
\def\ivbinary{{\codefont iv}\xspace}
\def\maketx{{\codefont maketx}\xspace}
\def\ImageIOFormatSpec{{\codefont ImageSpec}\xspace}
\def\ImageSpec{{\codefont ImageSpec}\xspace}
\def\ImageInput{{\codefont ImageInput}\xspace}
\def\ImageOutput{{\codefont ImageInput}\xspace}
\def\ParamBaseType{{\codefont ParamBaseType}\xspace}
%\def\opencall{{\codefont open()}\xspace}
\def\writeimage{{\codefont write\_image()}\xspace}
\def\writescanline{{\codefont write\_scanline()}\xspace}
\def\writetile{{\codefont write\_tile()}\xspace}
\def\readimage{{\codefont read\_image()}\xspace}
\def\readscanline{{\codefont read\_scanline()}\xspace}
\def\readtile{{\codefont read\_tile()}\xspace}
\def\AutoStride{{\codefont AutoStride}\xspace}


\title{ 
{\Huge{\bf \product}
%\textregistered\ 
{\bf\sffamily \versionnumber} \medskip \\ \huge Programmer Documentation
\\ \large (in progress) 
} \bigskip }
\author{Editor: Larry Gritz \\
% email should go here
 \bigskip \\
}
\date{{\large 
%Editor: Larry Gritz \\[2ex]
Date: 20 July, 2008
%\\ (with corrections, 11 July, 2008)
}}


%%%%%%%%%%%%%%%%%%%%%%%%%%%%%%%%%%%%%%%%%%%%%%%%%%%%%%%%%%%%%%%%%%%%%%%%%%
% Larry's favorite LaTeX macros for making technical books.  These have
% been refined for years, starting with SIGGRAPH course notes in the
% '90's, further refined for _Advanced RenderMan_.
%%%%%%%%%%%%%%%%%%%%%%%%%%%%%%%%%%%%%%%%%%%%%%%%%%%%%%%%%%%%%%%%%%%%%%%%%%

%
% Define typesetting commands for filenames and code
%
% Just like Advanced RenderMan -- all code in courier, keywords in text
%    courier but not bold.
\def\codefont{\ttfamily}	% font to use for code
\def\ce{\codefont\bfseries}	% emphasize something in code


%
% Define typesetting commands for filenames and code
%
\def\cf{\codefont}		% abbreviation for \codefont
\def\fn{\codefont}		% in-line filenames & unix commands
\def\kw{\codefont}      	% in-line keyword

\newcommand{\var}[1]{{\kw \emph{#1}}}  % variable
\newcommand{\qkw}[1]{{\kw "#1"}}       % quoted keyword
\newcommand{\qkws}[1]{{\small \kw "#1"}}       % quoted keyword, small
\newcommand{\qkwf}[1]{{\footnotesize \kw "#1"}}       % quoted keyword, tiny



% Define some environments for easy typesetting of small amounts of
% code.  These are mostly just wrappers around verbatim, but the
% different varieties also change font sizes.
\newenvironment{code}{\small \verbatimtab}{\endverbatimtab}
\newenvironment{smallcode}{\small \renewcommand{\baselinestretch}{0.8} \verbatimtab}{\endverbatimtab \renewcommand{\baselinestretch}{1}}
\newenvironment{tinycode}{\footnotesize \renewcommand{\baselinestretch}{0.75} \verbatimtab}{\endverbatimtab \renewcommand{\baselinestretch}{1}}

\begin{htmlonly}
\renewenvironment{code}{\begin{verbatim}}{\end{verbatim}}
\newenvironment{smallcode}{\begin{verbatim}}{\end{verbatim}}
\newenvironment{tinycode}{\begin{verbatim}}{\end{verbatim}}
\end{htmlonly}

\newcommand{\includedcode}[1]{{\small \verbatimtabinput{#1}}}
\newcommand{\smallincludedcode}[1]{{\small \renewcommand{\baselinestretch}{0.8} \verbatimtabinput{#1} \renewcommand{\baselinestretch}{1}}}
\newcommand{\tinyincludedcode}[1]{{\footnotesize \renewcommand{\baselinestretch}{0.75} \verbatimtabinput{#1} \renewcommand{\baselinestretch}{1}}}





% Also create a hyphenation list, essentially just to guarantee that
% type names aren't hyphenated
%\hyphenation{Attribute}

% Handy for parameter lists
\def\pl{\emph{parameterlist}\xspace}
\def\epl{\emph{...parameterlist...}\xspace}
\hyphenation{parameterlist}




%begin{latexonly}
\newenvironment{apilist}{\begin{list}{}{\medskip \item[]}}{\end{list}}
\newcommand{\apiitem}[1]{\vspace{12pt} \noindent {\bf\tt #1} \vspace{-10pt}\begin{apilist}\nopagebreak[4]}
\newcommand{\apiend}{\end{apilist}\medskip\pagebreak[2]}
\def\bigspc{\makebox[72pt]{}}
\def\spc{\makebox[24pt]{}}
\def\halfspc{\makebox[12pt]{}}
\def\neghalfspc{\hspace{-12pt}}
\def\negspc{\hspace{-24pt}}
\def\chapwidthbegin{}
\def\chapwidthend{}
%end{latexonly}


\begin{htmlonly}
\newcommand{\apiitem}[1]{\medskip \noindent {\bf #1} \begin{quote}}
\newcommand{\apiend}{\end{quote}}
\def\halfspc{\begin{rawhtml} &nbsp; &nbsp; \end{rawhtml}}
\def\spc{\halfspc\halfspc}
\pagecolor[named]{White}
\def\chapwidthbegin{\begin{rawhtml}<p><table cellspacing=1><tr><td width=550>\end{rawhtml}}
\def\chapwidthend{\begin{rawhtml}</td></tr></table>\end{rawhtml}}
\end{htmlonly}


\newcommand{\apibinding}[3]{\apiitem{#1\\[1ex]#2\\[1ex]#3}}

\newcommand{\CPPBINDING}[1]{\par {\small C++ BINDING:}\par {\spc \codefont #1}}
\newcommand{\PARAMETERS}{\par {\small PARAMETERS:} \par}
\newcommand{\EXAMPLE}{\par {\small EXAMPLE:} \par}
\newcommand{\EXAMPLES}{\par {\small EXAMPLES:} \par}
\newcommand{\SEEALSO}{\par \hspace{-20pt} See Also: \par}



% The \begin{algorithm} \end{algorithm} macros (in algorithm.sty) are
% great for code that can fit all on one page.  But when it can't, use
% these macros.  The first parameter is the caption, the second is the
% label name.
\newcommand{\longalgorithmbegin}[2]{\noindent\hrulefill \\
  \refstepcounter{algorithm}
  \noindent {\bf Listing \arabic{chapter}.\arabic{algorithm}}: #1 \label{#2} \\
  \addcontentsline{loa}{algorithm}{\numberline {\arabic{algorithm}} #1}
  \noindent\hrulefill
}
\newcommand{\longalgorithmend}{\noindent\hrulefill \\}


\def\NEW{\marginpar[\medskip\hfill~\fbox{\sffamily \Huge NEW!}~]{\medskip~\fbox{\sffamily \Huge NEW!}~}}
\newcommand{\NEWdown}[1]{\marginpar[\vspace{#1}\hfill\fbox{\sffamily \Huge NEW!}]{\vspace{#1}\fbox{\sffamily \Huge NEW!}}}
\def\DEPRECATED{\marginpar[\medskip\hfill~\fbox{\sffamily \Large Deprecated}]{\medskip~\fbox{\sffamily \Large Deprecated}}}
\newcommand{\DEPRECATEDdown}[1]{\marginpar{\vspace{#1}\fbox{\sffamily \Large Deprecated}}}
\def\CHANGED{\marginpar[\medskip\hfill~\fbox{\sffamily \huge CHANGED!}~]{\medskip~\fbox{\sffamily \huge CHANGED!}~}}
\def\ENHANCED{\marginpar[\medskip\hfill~\fbox{\sffamily \huge ENHANCED}~]{\medskip~\fbox{\sffamily \huge ENHANCED}~}}


\newcommand{\indexapi}[1]{\index{#1@\tt#1\rm}}



\newenvironment{annotate}{\medskip\sffamily\em\noindent}{\medskip}
%\newenvironment{annotate}{\begin{comment}}{\end{comment}}


\makeindex

\begin{document}
\frontmatter

\maketitle

%\include{speccopyr}

\setcounter{tocdepth}{1}
\tableofcontents

\mainmatter

%\include{intro}

\part{The ImageIO Library}

\chapter{Image I/O API}
\label{chap:imageioapi}
\index{Image I/O API|(}



\section{Data Type Descriptions: {\cf TypeDesc}}
\label{sec:dataformats}
\label{sec:TypeDesc}
\index{data formats}

There are two kinds of data that are important to \product:

\begin{itemize}
\item \emph{Internal data} is in the memory of the computer, used by an
  application program.
\item \emph{Native file data} is what is stored in an image file itself
  (i.e., on the ``other side'' of the abstraction layer that \product
  provides).
\end{itemize}

Both internal and file data is stored in a particular \emph{data format}
that describes the numerical encoding of the values.  \product
understands several types of data encodings, and there is 
a special type, \TypeDesc, that allows their enumeration.
A \TypeDesc describes a base data format type, aggregation into simple
vector and matrix types, and an array length (if
it's an array).

\TypeDesc supports the following base data format types, given by the
enumerated type {\cf BASETYPE}:

\medskip

\begin{tabular}{l p{4.75in}}
{\cf UINT8} &  8-bit integer values ranging from
  0..255, corresponding to the C/C++ {\cf unsigned char}. \\
{\cf INT8} &  8-bit integer values ranging from
  -128..127, corresponding to the C/C++ {\cf char}. \\
{\cf UINT16} &  16-bit integer values ranging
  from 0..65535, corresponding to the C/C++ {\cf unsigned short}. \\
{\cf INT16} &  16-bit integer values ranging
  from -32768..32767, corresponding to the C/C++ {\cf short}. \\
{\cf UINT} &  32-bit integer values,
  corresponding to the C/C++ {\cf unsigned int}. \\
{\cf INT} &  signed 32-bit integer values, corresponding
  to the C/C++ {\cf int}. \\
{\cf UINT64} &  64-bit integer values,
  corresponding to the C/C++ {\cf unsigned long long} (on most architectures). \\
{\cf INT64} &  signed 64-bit integer values, corresponding
  to the C/C++ {\cf long long} (on most architectures). \\
{\cf FLOAT} &  32-bit IEEE floating point values,
  corresponding to the C/C++ {\cf float}. \\
{\cf DOUBLE} &  64-bit IEEE floating point values,
  corresponding to the C/C++ {\cf double}. \\
{\cf HALF} &  16-bit floating point values in the format
  supported by OpenEXR and OpenGL.
\end{tabular}
\medskip

\noindent A \TypeDesc can be constructed using just this information, either as
a single scalar value, or an array of scalar values:

\apiitem{{\ce TypeDesc} (BASETYPE btype) \\
{\ce TypeDesc} (BASETYPE btype, int arraylength)}
Construct a type description of a single scalar value of the given base
type, or an array of such scalars if an array length is supplied.  For
example, {\cf TypeDesc(UINT8)} describes an unsigned 8-bit integer,
and {\cf TypeDesc(FLOAT,7)} describes an array of 7 32-bit float values.
Note also that a non-array \TypeDesc may be implicitly constructed from
just the {\cf BASETYPE}, so it's okay to pass a {\cf BASETYPE}
to any function parameter that takes a full \TypeDesc.
\apiend


\medskip
\noindent In addition, \TypeDesc supports certain aggregate types, described
by the enumerated type {\cf AGGREGATE}:

\medskip
\begin{tabular}{l p{4.75in}}
{\cf SCALAR} & a single scalar value (such as a raw {\cf int}
  or {\cf float} in C).  This is the default. \\
{\cf VEC2} & two values representing a 2D vector. \\
{\cf VEC3} & three values representing a 3D vector. \\
{\cf VEC4} & four values representing a 4D vector. \\
{\cf MATRIX44} & sixteen values representing a $4 \times 4$ matrix.
\end{tabular}
\medskip

\noindent And optionally, several vector transformation
semantics, described by the enumerated type {\cf VECSEMANTICS}:

\medskip
\begin{tabular}{p{1in} p{4.25in}}
{\cf NOXFORM} & indicates that the item is not a spatial quantity that
  undergoes any particular transformation. \\
{\cf COLOR} & indicates that the item is a ``color,'' not a spatial
  quantity (and of course therefore does not undergo a transformation). \\
{\cf POINT} &  indicates that the item represents a
  spatial position and should be transformed by a $4 \times 4$ matrix
  as if it had a 4th component of 1. \\
{\cf VECTOR} &  indicates that the item represents a
  spatial direction and should be transformed by a $4 \times 4$ matrix
  as if it had a 4th component of 0. \\
{\cf NORMAL} &  indicates that the item represents a
  surface normal and should be transformed like a vector, but using the
  inverse-transpose of a $4 \times 4$ matrix.
\end{tabular}
\medskip

\noindent These can be combined to fully describe a complex type:

\apiitem{{\ce TypeDesc} (BASETYPE btype, AGGREGATE agg, VECSEMANTICS
xform=NOXFORM)  \\
{\ce TypeDesc} (BASETYPE btype, AGGREGATE agg, int arraylen) \\
{\ce TypeDesc} (BASETYPE btype, AGGREGATE agg, VECSEMANTICS xform, int arraylen)
}
Construct a type description of an aggregate (or array of aggregates),
with optional vector transformation semantics.  For example, 
{\cf TypeDesc(HALF,COLOR)} describes an aggregate of 3 16-bit floats
comprising a color, and {\cf TypeDesc(FLOAT,VEC3,POINT)} describes 
an aggregate of 3 32-bit floats comprising a 3D position.

Note that aggregates and arrays are different.  A {\cf
  TypeDesc(FLOAT,3)} is an array of three floats, a {\cf
  TypeDesc(FLOAT,COLOR)} is a single 3-channel color comprised of
floats, and {\cf TypeDesc(FLOAT,3,COLOR)} is an array of 3 color values,
each of which is comprised of 3 floats.
\apiend

\bigskip

Of these, the only ones commonly used to store pixel values in image files
are scalars of {\cf UINT8}, {\cf UINT16}, {\cf FLOAT}, and {\cf HALF}
(the last only used by OpenEXR, to the best of our knowledge).

Note that the \TypeDesc (which is also used for applications other
than images) can describe many types not used by
\product.  Please ignore this extra complexity; only the above simple types are understood by
\product as pixel storage data types, though a few others, including
{\cf STRING} and {\cf MATRIX44} aggregates, are occasionally used for
\emph{metadata} for certain image file formats (see
Sections~\ref{sec:imageoutput:metadata}, \ref{sec:imageinput:metadata},
and the documentation of individual ImageIO plugins for details).

\section{Image Specification: {\cf ImageSpec}}
\label{sec:ImageSpec}
\indexapi{ImageSpec}

An \ImageSpec is a structure that describes the complete
format specification of a single image.  It contains:

\begin{itemize}
\item The image resolution (number of pixels).
\item The origin, if its upper left corner is not located beginning at
  pixel (0,0).
\item The full size and offset of an abstract ``full'' or ``display''
  image, useful for describing cropping or overscan.
\item Whether the image is organized into \emph{tiles}, and if so, the
  tile size.
\item The \emph{native data format} of the pixel values (e.g., float, 8-bit
  integer, etc.).
\item The number of color channels in the image (e.g., 3 for RGB
  images), names of the channels, and whether any particular channels
  represent \emph{alpha} and \emph{depth}.
\item Any presumed gamma correction or hints about color space of
  the pixel values.
\item Quantization parameters describing how floating point values
  should be converted to integers (in cases where users pass real values
  but integer values are stored in the file).  This is used only when
  writing images, not when reading them.
\item A user-extensible (and format-extensible) list of any other
  arbitrarily-named and -typed data that may help describe the image or
  its disk representation.
\end{itemize}

\subsection{\ImageSpec Data Members}

The \ImageSpec contains data fields for the values that are
required to describe nearly any image, and an extensible list of
arbitrary attributes that can hold metadata that may be user-defined or
specific to individual file formats.  Here are the hard-coded data
fields:

\apiitem{int width, height, depth \\
int x, y, z}

{\cf width, height, depth} are the size of the data of this image, i.e.,
the number of pixels in each dimension.  A {\cf depth} greater than 1
indicates a 3D ``volumetric'' image.

{\cf x, y, z} indicate the \emph{origin} of the pixel data of the image.
These default to (0,0,0), but setting them differently may indicate that
this image is offset from the usual origin.

Therefore the pixel data are defined over pixel coordinates
[{\cf x} ... {\cf x+width-1}] horizontally, 
[{\cf y} ... {\cf y+height-1}] vertically, 
and [{\cf z} ... {\cf z+depth-1}] in depth.
\apiend

\apiitem{int full_width, full_height, full_depth \\
int full_x, full_y, full_z}

These fields define a ``full'' or ``display'' image window over the
region [{\cf full_x} ... {\cf full_x+full_width-1}] horizontally, 
[{\cf full_y} ... {\cf full_y+full_height-1}] vertically, 
and [{\cf full_z} ... {\cf full_z+full_depth-1}] in depth.

Having the full display window different from the pixel data window can
be helpful in cases where you want to indicate that your image is a
\emph{crop window} of a larger image (if the pixel data window is a
subset of the full display window), or that the pixels include
\emph{overscan} (if the pixel data is a superset of the full display
window), or may simply indicate how different non-overlapping images
piece together.
\apiend

\apiitem{int tile_width, tile_height, tile_depth}
If nonzero, indicates that the image is stored on disk organized into
rectangular \emph{tiles} of the given dimension.  The default of 
(0,0,0) indicates that the image is stored in scanline order, rather
than as tiles.
\apiend

\apiitem{TypeDesc format}
Indicates the native format of the pixel data values themselves, as a 
\TypeDesc (see \ref{sec:TypeDesc}).  Typical values would be
{\cf TypeDesc::UINT8} for 8-bit unsigned values, {\cf TypeDesc::FLOAT} for 32-bit
floating-point values, etc.

\noindent NOTE: Currently, the implementation of OpenImageIO requires
all channels to have the same data format.
\apiend

\apiitem{int nchannels}
The number of \emph{channels} (color values) present in each pixel of
the image.  For example, an RGB image has 3 channels.
\apiend

\apiitem{std::vector<std::string> channelnames}
The names of each channel, in order.  Typically this will be \qkw{R},
\qkw{G},\qkw{B}, \qkw{A} (alpha), \qkw{Z} (depth), or other arbitrary
names.
\apiend

\apiitem{int alpha_channel}
The index of the channel that represents \emph{alpha} (pixel coverage
and/or transparency).  It defaults to -1 if no alpha channel is present,
or if it is not known which channel represents alpha.
\apiend

\apiitem{int z_channel}
The index of the channel that respresents \emph{z} or \emph{depth} (from
the camera).  It defaults to -1 if no depth channel is present, or if it
is not know which channel represents depth.
\apiend

\apiitem{LinearitySpec linearity}
Describes the mapping of pixel values to real-world units.  
{\cf LinearitySpec} is
an enumerated type that may take on the following values:
\begin{itemize}
\item[] 
\item {\cf Linear} (the default) indicates that pixel values map
  linearly.
\item {\cf GammaCorrected} indicates that the color pixel values have
  already been gamma corrected, using the exponent given by the {\cf
    gamma} field.  (It is still assumed that non-color values, such as
  alpha and depth, are linear.)
\item {\cf sRGB} indicates that color values are encoded using the sRGB
  mapping.  (It is still assumed that non-color values are linear.)
\item {\cf AdobeRGB} indictes that the values are encoded in the
Adobe RGB color space. (It is still assumed that non-color values are linear.)
\item {\cf Rec709} indicates that color values are encoded using the 
  Rec709 mapping.  (It is still assumed that non-color values are linear.)
\item {\cf KodakLog} indicates that color values are encoded using the 
  Kodak logaithmic mapping.  (It is still assumed that non-color values are linear.)
\end{itemize}
\apiend

\apiitem{float gamma}
The gamma exponent, if the pixel values in the image have already been
gamma corrected (indicated by {\cf linearity} having a value of {\cf
GammaCorrected}).  The default of 1.0 indicates that no gamma
correction has been applied.
\apiend

\apiitem{int quant_black, quant_white, quant_min, quant_max;\\
  float quant_dither}
Describes the \emph{quantization}, or mapping between real
(floating-point) values and the stored integer values.
Please refer to Section~\ref{sec:imageoutput:quantization} for
a more complete explanation of each of these parameters.
\apiend

\apiitem{ParamValueList extra_attribs}
A list of arbitrarily-named and arbitrarily-typed additional attributes
of the image, for any metadata not described by the hard-coded fields
described above.  This list may be manipulated with the {\cf
attribute()} and {\cf find_attribute()} methods.
\apiend

\subsection{\ImageSpec member functions}

\noindent \ImageSpec contains the following methods that
manipulate format specs or compute useful information about images given
their format spec:

\apiitem{{\ce ImageSpec} (int xres, int yres, int nchans, TypeDesc fmt = UINT8)}
Constructs an \ImageSpec with the given $x$ and $y$ resolution, number
of channels, and pixel data format.

All other fields are set to the obvious defaults -- the image is an
ordinary 2D image (not a volume), the image is not offset or a crop of a
bigger image, the image is scanline-oriented (not tiled), channel names
are ``R'', ``G'', ``B,'' and ``A'' (up to and including 4 channels,
beyond that they are named ``channel \emph{n}''), the fourth channel (if
it exists) is assumed to be alpha, values are assumed to be linear, and
quantization (if \emph{fmt} describes an integer type) is done in
such a way that the maximum positive integer range maps to (0.0, 1.0).
\apiend

\apiitem{void {\ce set_format} (TypeDesc fmt)}
Sets the format as described, and also sets all quantization parameters
to the default for that data type (as explained in 
Section~\ref{sec:imageoutput:quantization}).
\apiend

\apiitem{void {\ce default_channel_names} ()}
Sets the {\cf channelnames} to reasonable defaults for the number of
channels.  Specifically, channel names are set to ``R'', ``G'', ``B,''
and ``A'' (up to and including 4 channels, beyond that they are named
``channel\emph{n}''.
\apiend

\apiitem{static TypeDesc \\
{\ce format_from_quantize} (int quant_black, int quant_white,\\
\bigspc \bigspc                          int quant_min, int quant_max)}
Utility function that, given quantization parameters, returns a data
type that may be used without unacceptable loss of significant bits.
% FIXME - elaborate?
\apiend

\apiitem{size_t {\ce channel_bytes} () const}
Returns the number of bytes comprising each channel of each pixel (i.e.,
the size of a single value of the type described by the {\cf format} field).
\apiend

\apiitem{size_t {\ce pixel_bytes} () const}
Returns the number of bytes comprising each pixel (i.e. the number of
channels multiplied by the channel size).
\apiend

\apiitem{imagesize_t {\ce scanline_bytes} () const}
Returns the number of bytes comprising each scanline (i.e. {\cf width}
pixels).  
This will return {\cf std::numeric_limits<imagesize_t>::max()} in the event
of an overflow where it's not representable in an {\cf imagesize_t}.
\apiend

\apiitem{imagesize_t {\ce tile_pixels} () const}
Returns the number of tiles comprising an image tile (if it's a tiled image).
This will return {\cf std::numeric_limits<imagesize_t>::max()} in the event
of an overflow where it's not representable in an {\cf imagesize_t}.
\apiend

\apiitem{imagesize_t {\ce tile_bytes} () const}
Returns the number of bytes comprising an image tile (if it's a tiled image).
This will return {\cf std::numeric_limits<imagesize_t>::max()} in the event
of an overflow where it's not representable in an {\cf imagesize_t}.
\apiend

\apiitem{imagesize_t {\ce image_pixels} () const}
Returns the number of pixels comprising an entire image image of these dimensions.
This will return {\cf std::numeric_limits<imagesize_t>::max()} in the event
of an overflow where it's not representable in an {\cf imagesize_t}.
\apiend

\apiitem{imagesize_t {\ce image_bytes} () const}
Returns the number of bytes comprising an entire image of these dimensions.
This will return {\cf std::numeric_limits<imagesize_t>::max()} in the event
of an overflow where it's not representable in an {\cf imagesize_t}.
\apiend

\apiitem{bool {\ce size_t_safe} () const}
Return {\cf true} if an image described by this spec can the sizes
(in pixels or bytes) of its scanlines, tiles, and the entire image can
be represented by a {\cf size_t} on that platform.  If this returns
{\cf false}, the client application should be very careful allocating
storage!
\apiend

% FIXME - document auto_stride() ?

\apiitem{void {\ce attribute} (const std::string \&name, TypeDesc type, \\
\bigspc const void *value)}
Add a metadata attribute to {\cf extra_attribs}, with the given name and
data type.  The {\cf value} pointer specifies
the address of the data to be copied.
\apiend

\apiitem{void {\ce attribute} (const std::string \&name, unsigned int value)\\
    void {\ce attribute} (const std::string \&name, int value)\\
    void {\ce attribute} (const std::string \&name, float value)\\
    void {\ce attribute} (const std::string \&name, const char *value)\\
    void {\ce attribute} (const std::string \&name, const std::string \&value)}
Shortcuts for passing attributes comprised of a single integer,
floating-point value, or string.
\apiend

\apiitem{ImageIOParameter * {\ce find_attribute} (const std::string \&name,\\
\bigspc\bigspc\spc                           TypeDesc searchtype=UNKNOWN,\\
\bigspc\bigspc\spc                           bool casesensitive=false)\\
const ImageIOParameter * {\ce find_attribute} (const std::string \&name,\\
\bigspc\bigspc\spc                           TypeDesc searchtype=UNKNOWN,\\
\bigspc\bigspc\spc                           bool casesensitive=false) const
\\
}

Searches {\cf extra_attribs} for an attribute matching {\cf name},
returning a pointer to the attribute record, or NULL if there was no
match.  If {\cf searchtype} is {\cf TypeDesc::UNKNOWN}, the search will be made
regardless of the data type, whereas other values of {\cf searchtype}
will reject a matching name if the data type does not also match.  The
name comparison will be exact if {\cf casesensitive} is true, otherwise
in a case-insensitive manner if {\cf caseinsensitive} is false.
\apiend

\apiitem{int {\ce get_int_attribute} (const std::string \&name, int
  defaultval=0) const}
Gets an integer metadata attribute (silently converting to {\cf int}
even if if the data is really int8, uint8, int16, uint16, or uint32),
and simply substituting the supplied default value if no such metadata
exists.  This is a convenience function for when you know you are just
looking for a simple integer value.
\apiend

\apiitem{float {\ce get_float_attribute} (const std::string \&name,\\
\bigspc\bigspc float defaultval=0) const}
Gets a float metadata attribute (silently converting to {\cf float} even
if the data is really half or double), simply substituting the supplied
default value if no such metadata exists.  This is a convenience
function for when you know you are just looking for a simple float value.
\apiend

\apiitem{std::string {\ce get_string_attribute} (const std::string \&name, \\
\bigspc\bigspc const std::string \&defaultval=std::string()) const}
Gets a string metadata attribute, simply substituting the supplied
default value if no such metadata exists.  This is a convenience
function for when you know you are just looking for a simple string value.
\apiend


\apiitem{std::string {\ce metadata_val} (const ImageIOParamaeter \&p,
  bool human=true) const}
For a given parameter (in this \ImageSpec's {\cf extra_attribs} field),
format the value nicely as a string.  If {\cf human} is true, use
especially human-readable explanations (units, or decoding of
values) for certain known metadata.
\apiend


\index{Image I/O API|)}

\chapwidthend

\chapter{ImageOutput: Writing Images}
\label{chap:imageoutput}
\index{Image I/O API|(}
\indexapi{ImageOutput}


\section{Image Output Made Simple}
\label{sec:imageoutput:simple}

Here is the simplest sequence required to write the pixels of a 2D image
to a file:

\begin{code}
        #include "imageio.h"
        using namespace OpenImageIO;
        ...

        const char *filename = "foo.jpg";
        const int xres = 640, yres = 480;
        const int channels = 3;  // RGB
        unsigned char pixels[xres*yres*channels];

        ImageOutput *out = ImageOutput::create (filename);
        if (! out)
            return;
        ImageSpec spec (xres, yres, channels, TypeDesc::UINT8);
        out->open (filename, spec);
        out->write_image (TypeDesc::UINT8, pixels);
        out->close ();
        delete out;
\end{code}

\noindent This little bit of code does a surprising amount of useful work:  

\begin{itemize}
\item Search for an ImageIO plugin that is capable of writing the file
  (\qkw{foo.jpg}), deducing the format from the file extension.  When it
  finds such a plugin, it creates a subclass instance of \ImageOutput
  that writes the right kind of file format.
  \begin{code}
        ImageOutput *out = ImageOutput::create (filename);
  \end{code}
\item Open the file, write the correct headers, and in all other
  important ways prepare a file with the given dimensions ($640 \times
  480$), number of color channels (3), and data format (unsigned 8-bit
  integer).
  \begin{code}
        ImageSpec spec (xres, yres, channels, TypeDesc::UINT8);
        out->open (filename, spec);
  \end{code}
\item Write the entire image, hiding all details of the encoding of
  image data in the file, whether the file is scanline- or tile-based,
  or what is the native format of data in the file (in this case, our
  in-memory data is unsigned 8-bit and we've requested the same format
  for disk storage, but if they had been different, {\kw write_image()}
  would do all the conversions for us).
  \begin{code}
        out->write_image (TypeDesc::UINT8, &pixels);
  \end{code}
\item Close the file, destroy and free the \ImageOutput we had created,
  and perform all other cleanup and release of any resources needed by
  the plugin.
  \begin{code}
        out->close ();
        delete out;
  \end{code}
\end{itemize}



\section{Advanced Image Output}
\label{sec:imageoutput:advanced}

Let's walk through many of the most common things you might want to do,
but that are more complex than the simple example above.

\subsection{Writing individual scanlines, tiles, and rectangles}
\label{sec:imageoutput:scanlinestiles}

The simple example of Section~\ref{sec:imageoutput:simple} wrote an
entire image with one call.  But sometimes you are generating output a
little at a time and do not wish to retain the entire image in memory
until it is time to write the file.  \product allows you to write images
one scanline at a time, one tile at a time, or by individual rectangles.

\subsubsection{Writing individual scanlines}

Individual scanlines may be written using the \writescanline API
call:

\begin{code}
        ...
        unsigned char scanline[xres*channels];
        out->open (filename, spec);
        int z = 0;   // Always zero for 2D images
        for (int y = 0;  y < yres;  ++y) {
            ... generate data in scanline[0..xres*channels-1] ...
            out->write_scanline (y, z, TypeDesc::UINT8, scanline);
        }
        out->close ();
        ...
\end{code}

The first two arguments to \writescanline specify which scanline is
being written by its vertical ($y$) scanline number (beginning with 0)
and, for volume images, its slice ($z$) number (the slice number should
be 0 for 2D non-volume images).  This is followed by a \TypeDesc
describing the data you are supplying, and a pointer to the pixel data
itself.  Additional optional arguments describe the data stride, which
can be ignored for contiguous data (use of strides is explained in
Section~\ref{sec:imageoutput:strides}).

All \ImageOutput implementations will accept scanlines in strict order
(starting with scanline 0, then 1, up to {\kw yres-1}, without skipping
any).  See Section~\ref{sec:imageoutput:randomrewrite} for details
on out-of-order or repeated scanlines.

The full description of the \writescanline function may be found
in Section~\ref{sec:imageoutput:reference}.

\subsubsection{Writing individual tiles}

Not all image formats (and therefore not all \ImageOutput
implementations) support tiled images.  If the format does not support
tiles, then \writetile will fail.  An application using \product
should gracefully handle the case that tiled output is not available for
the chosen format.

Once you {\kw create()} an \ImageOutput, you can ask if it is capable
of writing a tiled image by using the {\kw supports("tiles")} query:

\begin{code}
        ...
        ImageOutput *out = ImageOutput::create (filename);
        if (! out->supports ("tiles")) {
            // Tiles are not supported
        }
\end{code}

Assuming that the \ImageOutput supports tiled images, you need to
specifically request a tiled image when you {\kw open()} the file.  This
is done by setting the tile size in the \ImageSpec passed
to {\kw open()}.  If the tile dimensions are not set, they will default
to zero, which indicates that scanline output should be used rather than
tiled output.

\begin{code}
        int tilesize = 64;
        ImageSpec spec (xres, yres, channels, TypeDesc::UINT8);
        spec.tile_width = tilesize;
        spec.tile_height = tilesize;
        out->open (filename, spec);
        ...
\end{code}

In this example, we have used square tiles (the same number of pixels
horizontally and vertically), but this is not a requirement of \product.
However, it is possible that some image formats may only support square
tiles, or only certain tile sizes (such as restricting tile sizes to
powers of two).  Such restrictions should be documented by each
individual plugin.

\begin{code}
        unsigned char tile[tilesize*tilesize*channels];
        int z = 0;   // Always zero for 2D images
        for (int y = 0;  y < yres;  y += tilesize) {
            for (int x = 0;  x < xres;  x += tilesize) {
                ... generate data in tile[] ..
                out->write_tile (x, y, z, TypeDesc::UINT8, tile);
            }
        }
        out->close ();
        ...
\end{code}

The first three arguments to \writetile specify which tile is
being written by the pixel coordinates of any pixel contained in the
tile: $x$ (column), $y$ (scanline), and $z$ (slice, which should always
be 0 for 2D non-volume images).  This is followed by a \TypeDesc
describing the data you are supplying, and a pointer to the tile's pixel
data itself, which should be ordered by increasing slice, increasing
scanline within each slice, and increasing column within each scanline.
Additional optional arguments describe the data stride, which can be
ignored for contiguous data (use of strides is explained in
Section~\ref{sec:imageoutput:strides}).

All \ImageOutput implementations that support tiles will accept tiles in
strict order of increasing $y$ rows, and within each row, increasing $x$
column, without missing any tiles.  See
Section~\ref{sec:imageoutput:randomrewrite} for details on out-of-order
or repeated tiles.

The full description of the \writetile function may be found
in Section~\ref{sec:imageoutput:reference}.

\subsubsection{Writing arbitrary rectangles}

Some \ImageOutput implementations --- such as those implementing an
interactive image display, but probably not any that are outputting
directly to a file --- may allow you to send arbitrary rectangular pixel
regions.  Once you {\kw create()} an \ImageOutput, you can ask if it is
capable of accepting arbitrary rectangles by using the {\kw
supports("rectangles")} query:

\begin{code}
        ...
        ImageOutput *out = ImageOutput::create (filename);
        if (! out->supports ("rectangles")) {
            // Rectangles are not supported
        }
\end{code}

If rectangular regions are supported, they may be sent using
the {\kw write_rectangle()} API call:

\begin{code}
        unsigned int rect[...];
        ... generate data in rect[] ..
        out->write_rectangle (xmin, xmax, ymin, ymax, zmin, zmax, TypeDesc::UINT8, rect);
        ...
\end{code}

The first six arguments to {\kw write_rectangle()} specify the region of
pixels that is being transmitted by supplying the minimum and maximum
pixel indices in $x$ (column), $y$ (scanline), and $z$ (slice, always 0
for 2D non-volume images).  The total number of pixels being transmitted
is therefore:
\begin{code}
        (xmax-xmin+1) * (ymax-ymin+1) * (zmax-zmin+1)
\end{code}
\noindent This is followed by a \TypeDesc describing the data you
are supplying, and a pointer to the rectangle's pixel data itself, which
should be ordered by increasing slice, increasing scanline within each
slice, and increasing column within each scanline.  Additional optional
arguments describe the data stride, which can be ignored for contiguous
data (use of strides is explained in
Section~\ref{sec:imageoutput:strides}).


\subsection{Converting data formats}
\label{sec:imageoutput:convertingformats}

The code examples of the previous sections all assumed that your
internal pixel data is stored as unsigned 8-bit integers (i.e., 0-255
range).  But \product is significantly more flexible.  

You may request that the output image be stored in any of several
formats.  This is done by setting the {\kw format} field of the
\ImageSpec prior to calling {\kw open}.  You can do this upon
construction of the \ImageSpec, as in the following example
that requests a spec that stores data as 16-bit unsigned integers:
\begin{code}
        ImageSpec spec (xres, yres, channels, TypeDesc::UINT16);
\end{code}

\noindent Or, for an \ImageSpec that has already been
constructed, you may reset its format using the {\kw set_format()}
method (which also resets the various quantization fields of the
spec to the defaults for the data format you have specified).  

\begin{code}
        ImageSpec spec (...);
        spec.set_format (TypeDesc::UINT16);
\end{code}

Note that resetting the format must be done \emph{before} passing the
spec to {\kw open()}, or it will have no effect on the file.

Individual file formats, and therefore \ImageOutput implementations, may
only support a subset of the formats understood by the \product library.
Each \ImageOutput plugin implementation should document which data
formats it supports.  An individual \ImageOutput implementation may
choose to simply fail to {\kw open()}, though the recommended behavior
is for {\kw open()} to succeed but in fact choose a data format
supported by the file format that best preserves the precision and range
of the originally-requested data format.

It is not required that the pixel data passed to \writeimage,
\writescanline, \writetile, or {\kw write_rectangle()} actually be in
the same data format as that requested as the native format of the file.
You can fully mix and match data you pass to the various {\kw write}
routines and \product will automatically convert from the internal
format to the native file format.  For example, the following code will
open a TIFF file that stores pixel data as 16-bit unsigned integers
(values ranging from 0 to 65535), compute internal pixel values as
floating-point values, with \writeimage performing the conversion
automatically:

\begin{code}
        ImageOutput *out = ImageOutput::create ("myfile.tif");
        ImageSpec spec (xres, yres, channels, TypeDesc::UINT16);
        out->open (filename, spec);
        ...
        float pixels [xres*yres*channels];
        ...
        out->write_image (TypeDesc::FLOAT, pixels);
\end{code}

\noindent Note that \writescanline, \writetile, and {\cf
  write_rectangle} have a parameter that works in a corresponding
manner.

Please refer to Section~\ref{sec:imageoutput:quantization} for more
information on how values are translated among the supported data
formats by default, and how to change the formulas by specifying
quantization in the \ImageSpec.


\subsection{Data Strides}
\label{sec:imageoutput:strides}

In the preceeding examples, we have assumed that the block of data being
passed to the {\cf write} functions are \emph{contiguous}, that is:

\begin{itemize}
\item each pixel in memory consists of a number of data values equal to
  the declared number of channels that are being written to the file;
\item successive column pixels within a row directly follow each other in
  memory, with the first channel of pixel $x$ immediately following
  last channel of pixel $x-1$ of the same row;
\item for whole images, tiles or rectangles, the data for each row
  immediately follows the previous one in memory (the first pixel of row
  $y$ immediately follows the last column of row $y-1$);
\item for 3D volumetric images, the first pixel of slice $z$ immediately
  follows the last pixel of of slice $z-1$.
\end{itemize}

Please note that this implies that data passed to
\writetile be contiguous in the shape of a single tile (not just an
offset into a whole image worth of pixels), and that data passed to {\cf
  write_rectangle()} be contiguous in the dimensions of the rectangle.

The \writescanline function takes an optional {\cf xstride} argument,
and the \writeimage, \writetile, and {\cf write_rectangle} functions
take optional {\cf xstride}, {\cf ystride}, and {\cf zstride} values
that describe the distance, in \emph{bytes}, between successive pixel
columns, rows, and slices, respectively, of the data you are passing.
For any of these values that are not supplied, or are given as the
special constant {\cf AutoStride}, contiguity will be assumed.

By passing different stride values, you can achieve some surprisingly
flexible functionality.  A few representative examples follow:

\begin{itemize}
\item Flip an image vertically upon writing, by using \emph{negative}
  $y$ stride:
  \begin{code}
        unsigned char pixels[xres*yres*channels];
        int scanlinesize = xres * channels * sizeof(pixels[0]);
        ...
        out->write_image (TypeDesc::UINT8,
                          (char *)pixels+(yres-1)*scanlinesize, // offset to last
                          AutoStride,                  // default x stride
                          -scanlinesize,               // special y stride
                          AutoStride);                 // default z stride
  \end{code}
\item Write a tile that is embedded within a whole image of pixel data,
  rather than having a one-tile-only memory layout:
  \begin{code}
        unsigned char pixels[xres*yres*channels];
        int pixelsize = channels * sizeof(pixels[0]);
        int scanlinesize = xres * pixelsize;
        ...
        out->write_tile (x, y, 0, TypeDesc::UINT8,
                         (char *)pixels + y*scanlinesize + x*pixelsize,
                         pixelsize,
                         scanlinesize);
  \end{code}
\item Write only a subset of channels to disk.  In this example, our
  internal data layout consists of 4 channels, but we write just 
  channel 3 to disk as a one-channel image:
  \begin{code}
        // In-memory representation is 4 channel
        const int xres = 640, yres = 480;
        const int channels = 4;  // RGBA
        const int channelsize = sizeof(unsigned char);
        unsigned char pixels[xres*yres*channels];

        // File representation is 1 channel
        ImageOutput *out = ImageOutput::create (filename);
        ImageSpec spec (xres, yres, 1, TypeDesc::UINT8);
        out->open (filename, spec);

        // Use strides to write out a one-channel "slice" of the image
        out->write_image (TypeDesc::UINT8,
                          (char *)pixels+3*channelsize, // offset to chan 3
                          channels*channelsize,         // 4 channel x stride
                          AutoStride,                   // default y stride
                          AutoStride);                  // default z stride
        ...
  \end{code}
\end{itemize}

Please consult Section~\ref{sec:imageoutput:reference} for detailed
descriptions of the stride parameters to each {\cf write} function.


\subsection{Writing a crop window or overscan region}
\label{sec:imageoutput:cropwindows}
\index{crop windows} \index{overscan}

% FIXME -- Marcos suggests adding a figure here to illustrate
% the w/h/d, xyz, full

The \ImageSpec fields {\cf width}, {\cf height}, and {\cf depth}
describe the dimensions of the actual pixel data.

At times, it may be useful to also describe an abstract \emph{full} or
\emph{display} image window, whose position and size may not correspond
exactly to the data pixels.  For example, a pixel data window that is a
subset of the full display window might indicate a \emph{crop window}; a
pixel data window that is a superset of the full display window might
indicate \emph{overscan} regions (pixels defined outside the eventual
viewport).

The \ImageSpec fields {\cf full_width}, {\cf full_height}, and
{\cf full_depth} describe the dimensions of the full display
window, and {\cf full_x}, {\cf full_y}, {\cf full_z} describe its
origin (upper left corner).  The fields {\cf x}, {\cf y}, {\cf z}
describe the origin (upper left corner)
of the pixel data.

These fields collectively describe an abstract full display image
ranging from [{\cf full_x} ... {\cf full_x+full_width-1}] horizontally,
[{\cf full_y} ... {\cf full_y+full_height-1}] vertically,
and [{\cf full_z} ... {\cf full_z+full_depth-1}] in depth (if it is
a 3D volume), and actual pixel data over the pixel coordinate range 
[{\cf x} ... {\cf x+width-1}] horizontally,
[{\cf y} ... {\cf y+height-1}] vertically,
and [{\cf z} ... {\cf z+depth-1}] in depth (if it is a volume).

Not all image file formats have a way to describe display windows.  An
\ImageOutput implementation that cannot express display windows will
always write out the {\cf width} $\times$ {\cf height} pixel data, may
upon writing lose information about offsets or crop windows.

Here is a code example that opens an image file that will contain a $32
\times 32$ pixel crop window within an abstract $640 \times 480$ full
size image.  Notice that the pixel indices (column, scanline, slice)
passed to the {\cf write} functions are the coordinates relative to
the full image, not relative to the crop widow, but the data pointer
passed to the {\cf write} functions should point to the beginning of
the actual pixel data being passed (not the the hypothetical start of
the full data, if it was all present).

\begin{code}
        int fullwidth = 640, fulllength = 480; // Full display image size
        int cropwidth = 16, croplength = 16;  // Crop window size
        int xorigin = 32, yorigin = 128;      // Crop window position
        unsigned char pixels [cropwidth * croplength * channels]; // Crop size!
        ...
        ImageOutput *out = ImageOutput::create (filename);
        ImageSpec spec (cropwidth, croplength, channels, TypeDesc::UINT8);
        spec.full_x = 0;
        spec.full_y = 0;
        spec.full_width = fullwidth;
        spec.full_length = fulllength;
        spec.x = xorigin;
        spec.y = yorigin;
        out->open (filename, spec);
        ...
        int z = 0;   // Always zero for 2D images
        for (int y = yorigin;  y < yorigin+croplength;  ++y) {
            out->write_scanline (y, z, TypeDesc::UINT8,
                                 (y-yorigin)*cropwidth*channels);
        }
        out->close ();
\end{code}


\subsection{Writing metadata}
\label{sec:imageoutput:metadata}

The \ImageSpec passed to {\cf open()} can specify all the common
required properties that describe an image: data format, dimensions,
number of channels, tiling.  However, there may be a variety of
additional \emph{metadata}\footnote{\emph{Metadata} refers to data about
data, in this case, data about the image that goes beyond the pixel
values and description thereof.} that should be carried along with the
image or saved in the file.  

The remainder of this section explains how to store additional metadata
in the \ImageSpec.  It is up to the \ImageOutput to store these
in the file, if indeed the file format is able to accept the data.
Individual \ImageOutput implementations should document which metadata
they respect.

\subsubsection{Channel names}

In addition to specifying the number of color channels, it is also
possible to name those channels.  Only a few \ImageOutput
implementations have a way of saving this in the file, but some do, so
you may as well do it if you have information about what the channels
represent.

By convention, channel names for red, green, blue, and alpha (or a main
image) should be named \qkw{R}, \qkw{G}, \qkw{B}, and \qkw{A},
respectively.  Beyond this guideline, however, you can use any names you
want.

The \ImageSpec has a vector of strings called {\cf
  channelnames}.  Upon construction, it starts out with reasonable
default values.  If you use it
at all, you should make sure that it contains the same number of strings
as the number of color channels in your image.  Here is an example:

\begin{code}
        int channels = 4;
        ImageSpec spec (width, length, channels, TypeDesc::UINT8);
        spec.channelnames.clear ();
        spec.channelnames.push_back ("R");
        spec.channelnames.push_back ("G");
        spec.channelnames.push_back ("B");
        spec.channelnames.push_back ("A");
\end{code}

Here is another example in which custom channel names are used to 
label the channels in an 8-channel image containing beauty pass
RGB, per-channel opacity, and texture $s,t$ coordinates for each pixel.

\begin{code}
        int channels = 8;
        ImageSpec spec (width, length, channels, TypeDesc::UINT8);
        spec.channelnames.clear ();
        spec.channelnames.push_back ("R");
        spec.channelnames.push_back ("G");
        spec.channelnames.push_back ("B");
        spec.channelnames.push_back ("opacityR");
        spec.channelnames.push_back ("opacityG");
        spec.channelnames.push_back ("opacityB");
        spec.channelnames.push_back ("texture_s");
        spec.channelnames.push_back ("texture_t");
\end{code}

The main advantage to naming color channels is that if you are saving to
a file format that supports channel names, then any application that
uses \product to read the image back has the option to retain those
names and use them for helpful purposes.  For example, the {\cf iv}
image viewer will display the channel names when viewing individual
channels or displaying numeric pixel values in ``pixel view'' mode.


\subsubsection{Specially-designated channels}

The \ImageSpec contains two fields, {\cf alpha_channel} and {\cf
  z_channel}, which can be used to designate which channel indices are
used for alpha and $z$ depth, if any.  Upon construction, these are both
set to {\cf -1}, indicating that it is not known which channels 
are alpha or depth.  Here is an example of setting up a 5-channel output
that represents RGBAZ:

\begin{code}
        int channels = 5;
        ImageSpec spec (width, length, channels, format);
        spec.channelnames.push_back ("R");
        spec.channelnames.push_back ("G");
        spec.channelnames.push_back ("B");
        spec.channelnames.push_back ("A");
        spec.channelnames.push_back ("Z");
        spec.alpha_channel = 3;
        spec.z_channel = 4;
\end{code}

There are two advantages to designating the alpha and depth channels in
this manner:  
\begin{itemize}
\item Some file formats may require that these channels be stored in a
  particular order, with a particular precision, or the \ImageOutput may
  in some other way need to know about these special channels.
\item Certain operations that make sense for colors should not apply to
  alpha or $z$.  For example, if your call to {\cf write} reduces
  precision (e.g., converts from {\cf float} to integer pixels) it will
  typically add random \emph{dither} to eliminate banding artifacts
  in the quantization.  But for a variety of reasons, you want to add
  dither only to color channels and not to alpha.  So setting {\cf
    alpha_channel} will cause {\cf write} to not dither that channel.
\end{itemize}

\subsubsection{Linearity hints}

We certainly hope that you are using only modern file formats that
support high precision and extended range pixels (such as OpenEXR) and
keeping all your images in a linear color space.  But you may have to
work with file formats that dictate the use of nonlinear color values.
This is prevalent in formats that store pixels only as 8-bit values,
since 256 values are not enough to linearly represent colors without
banding artifacts in the dim values.

Since this can (and probably will) happen, the \ImageSpec has
fields that allow you to explain what color space your image pixels are
in.  Each individual \ImageOutput should document how it uses this (or
not).

The \ImageSpec field {\cf linearity} can take on any of the
following values:
\begin{description}
\item[\halfspc \rm \kw{ImageSpec::UnknownLinearity}] the default,
  indicates that you have made no claim about the color space of your
  pixel data.
\item[\halfspc \rm \kw{ImageSpec::Linear}] indicates that the pixel
  values you are passing repesent linear values.
\item[\halfspc \rm \kw{ImageSpec::GammaCorrected}] indicates that the
  color pixel values (but not alpha or $z$) that you are passing have
  already been gamma corrected (raised to the power $1/\gamma$), and
  that the gamma exponent may be found in the {\cf gamma} field of the
  \ImageSpec.
\item[\halfspc \rm \kw{ImageSpec::sRGB}] indicates that the color pixel
  values that you are passing are already in sRGB color space.
\item[\halfspc \rm \kw{ImageSpec::AdobeRGB}] indicates that the color pixel
  values that you are passing are already in Adobe RGB color space.
\item[\halfspc \rm \kw{ImageSpec::Rec709}] indicates that the color pixel
  values that you are passing are already in Rec709 color space.
\item[\halfspc \rm \kw{ImageSpec::KodakLog}] indicates that the color pixel
  values that you are passing are already in Kodak logarithmic color space.
\end{description}

\noindent Here is a simple example of setting up the \ImageSpec
when you know that the pixel values you are writing are linear:

\begin{code}
        ImageSpec spec (width, length, channels, format);
        spec.linearity = ImageSpec::Linear;
        ...
\end{code}

If a particular \ImageOutput implementation is required (by the rules of
the file format it writes) to have pixels in a particular color space,
then it will convert the color values of your image to the right color
space if it is not already in that space.  For example, JPEG images
must be in sRGB space, so if you declare your pixels to be {\kw Linear},
the JPEG \ImageOutput will convert to sRGB.

If you leave the linearity set to the default of {\cf UnknownLinearity},
the values will not be transformed, since the plugin can't be sure that
it's not in the correct space to begin with.  

The linearity only describes color channels.  An \ImageOutput plugin
will assume that alpha or depth ($z$) channels (designated by the {\cf
  alpha_channel} and {\cf z_channel} fields, respectively) always
represent linear values and should never be transformed.


\subsubsection{Arbitrary metadata}

For all other metadata that you wish to save in the file, you can attach
the data to the \ImageSpec using the {\cf attribute()} methods.
These come in polymorphic varieties that allow you to attach an
attribute name and a value consisting of a single {\cf int}, {\cf
  unsigned int}, {\cf float}, {\cf char*}, or {\cf std::string}, as
shown in the following examples:

\begin{code}
        ImageSpec spec (...);
        ...

        unsigned int u = 1;
        spec.attribute ("Orientation", u);

        float x = 72.0;
        spec.attribute ("dotsize", f);

        std::string s = "Fabulous image writer 1.0";
        spec.attribute ("Software", s);
\end{code}

These are convenience routines for metadata that consist of a single
value of one of these common types.  For other data types, or more
complex arrangements, you can use the more general form of {\cf
  attribute()}, which takes arguments giving the name, type (as a
\TypeDesc), number of values (1 for a single value, $>1$ for an
  array), and then a pointer to the data values.  For example,

\begin{code}
        ImageSpec spec (...);

        // Attach a 4x4 matrix to describe the camera coordinates
        float mymatrix[16] = { ... };
        spec.attribute ("worldtocamera", TypeDesc::TypeMatrix, &mymatrix);

        // Attach an array of two floats giving the CIE neutral color
        float neutral[2] = { ... };
        spec.attribute ("adoptedNeutral", TypeDesc(TypeDesc::FLOAT, 2), &neutral);
\end{code}

In general, most image file formats (and therefore most \ImageOutput
implementations) are aware of only a small number of name/value pairs
that they predefine and will recognize.  Some file formats (OpenEXR,
notably) do accept arbitrary user data and save it in the image file.
If an \ImageOutput does not recognize your metadata and does not support
arbitrary metadata, that metadatum will be silently ignored and will not
be saved with the file.

Each individual \ImageOutput implementation should document the names,
types, and meanings of all metadata attributes that they understand.


\subsection{Controlling quantization}
\label{sec:imageoutput:quantization}

It is possible that your internal data format (that in which you compute
pixel values that you pass to the {\cf write} functions) is of greater
precision or range than the native data format of the output file.  This
can occur either because you specified a lower-precision data format in
the \ImageSpec that you passed to {\cf open()}, or else that the
image file format dictates a particular data format that does not match
your internal format.  For example, you may compute {\cf float} pixels
and pass those to {\cf write_image()}, but if you are writing a
JPEG/JFIF file, the values must be stored in the file as 8-bit unsigned
integers.

The conversion from floating-point formats to integer formats (or from
higher to lower integer, which is done by first converting to float) is
controlled by five fields within the \ImageSpec: {\cf
  quant_black}, {\cf quant_white}, {\cf quant_min}, {\cf quant_max},
and {\cf quant_dither}.
Float 0.0 maps to the integer value given by {\cf quant_black}, and
float 1.0 maps to the integer value given by {\cf quant_white}.  Then,
for color channels only (not alpha or depth), a random amount is added
in the range ({\cf -quant_dither..quant_dither}), in order to reduce
banding artifacts.  The result is then clamped to lie within the range of
{\cf quant_min} and {\cf quant_max}, inclusive.  Finally, this result is
truncated its integer value for final output.  Here is the code that
implements this transformation ({\cf T} is the final output integer
type):

\begin{code}
        float value = quant_black * (1 - input) + quant_white * input;
        if (it's a color channel)
            value += quant_dither * (2 * random() - 1);
        T output = (T) clamp ((int)(value + 0.5), quant_min, quant_max);
\end{code}

The values of the quantization parameters are set in one of three ways:
(1) upon construction of the \ImageSpec, they are set to the
default quantization values for the given data format; (2) upon call to
{\cf ImageSpec::set_format()}, the quantization values are set
to the defaults for the given data format; (3) or, after being first set
up in this manner, you may manually change the quantization parameters
in the \ImageSpec, if you want something other than the default
quantization.

\noindent Default quantization for each integer type is as follows:\\

\smallskip
\begin{tabular}{|l|r|r|r|r|r|}
\hline
{\bf Data Format} & {\bf black} & {\bf white} & {\bf min} & {\bf max} & {\bf
  dither} \\
\hline
{\cf UINT8}  & 0 &        255 &     0 & 255 & 0.5 \\
{\cf INT8}   & 0 &        127 &  -128 & 127 & 0.5 \\
{\cf UINT16} & 0 &      65535 &     0 & 65535 & 0.5 \\
{\cf INT16}  & 0 &      32767 & -32768 & 32767 & 0.5 \\
{\cf UINT}   & 0 & 4294967295 & 0 & 4294967295 & 0.5 \\
{\cf INT}    & 0 & 2147483647 & -2147483648 & 2147483647 & 0.5 \\
\hline
{\cf FLOAT} & & & & & \\
{\cf HALF} & 0 & 1 & N/A & N/A & 0 \\
{\cf DOUBLE} & & & & & \\
\hline
\end{tabular} \\
\smallskip

\noindent Note that the default is to use the entire positive range
of each integer type to represent the floating-point (0..1) range.
Floating-point types do not attempt to remap values, do not add dither,
and do not clamp (except to their full floating-point range).

The default will almost always be what you want.  But just as an
example, here's how you would specify a quantization for a 16-bit file
in which 1.0 maps to 16383 (14 bits of positive range) rather than
filling the full 16 bit:

\begin{code}
        ImageSpec spec (width, length, channels, TypeDesc::UINT16);
        spec.quant_black  = 0;
        spec.quant_white  = 16383;
        spec.quant_min    = 0;
        spec.quant_max    = 16383;
        spec.quant_dither = 0.5;
\end{code}


\subsection{Random access and repeated transmission of pixels}
\label{sec:imageoutput:randomrewrite}

All \ImageOutput implementations that support scanlines and tiles should write pixels in strict
order of increasing $z$ slice, increasing $y$ scanlines/rows within each
slice, and increasing $x$ column within each row.  It is generally not
safe to skip scanlines or tiles, or transmit them out of order, unless
the plugin specifically advertises that it supports random access or
rewrites, which may be queried using:

\begin{code}
        ImageOutput *out = ImageOutput::create (filename);
        if (out->supports ("random_access"))
            ...
\end{code}

\noindent Similarly, you should assume the plugin will not correctly
handle repeated transmissions of a scanline or tile that has already
been sent, unless it advertises that it supports rewrites, which may be
queried using:

\begin{code}
        if (out->supports ("rewrite"))
            ...
\end{code}


\subsection{Multi-image files and MIP-maps}
\label{sec:imageoutput:multiimage}
\label{sec:imageoutput:mipmap}

Some image file formats support multiple discrete subimages to be stored
in one file, and/or multiple resolutions (MIP-map levels).  Given a
created \ImageOutput, you can query whether multiple images may be
stored in the file:

\begin{code}
        ImageOutput *out = ImageOutput::create (filename);
        if (out->supports ("multiimage"))
            ...
        if (out->supports ("mipmap"))
            ...
\end{code}

If you are working with an \ImageOutput that supports multiple images
or MIP-map levels,
it is easy to write these images.  All you have to do is, after writing
all the pixels of one image but before calling {\cf close()}, call {\cf
  open()} again for the next subimage or MIP level and passing the
appropriate value for the optional third
\emph{mode} argument.  (See
Section~\ref{sec:imageoutput:reference} for the full technical
description of the arguments to {\cf open()}.)  The {\cf close()}
routine is called just once, after all subimages and MIP levels are completed.

Below is pseudocode for writing a MIP-map (a multi-resolution image
used for texture mapping):

\begin{code}
        const char *filename = "foo.tif";
        const int xres = 512, yres = 512;
        const int channels = 3;  // RGB
        unsigned char *pixels = new unsigned char [xres*yres*channels];

        // Create the ImageOutput
        ImageOutput *out = ImageOutput::create (filename);

        // Be sure we can support either mipmaps or subimages
        if (! out->supports ("mipmap") && ! out->supports ("multiimage")) {
            std::cerr << "Cannot write a MIP-map\n";
            delete out;
            return;
        }
        // Set up spec for the highest resolution
        ImageSpec spec (xres, yres, channels, TypeDesc::UINT8);

        // Use Create mode for the first level.
        ImageOutput::OpenMode appendmode = ImageOutput::Create;

        // Write images, halving every time, until we're down to
        // 1 pixel in either dimension
        while (spec.width >= 1 && spec.height >= 1) {
            out->open (filename, spec, mode);
            out->write_image (TypeDesc::UINT8, pixels);
            // Assume halve() resamples the image to half resolution
            halve (pixels, spec.width, spec.height);
            // Don't forget to change spec for the next iteration
            spec.width /= 2;
            spec.height /= 2;
            // For subsequent levels, change the append mode argument to
            // open().  If the format doesn't support MIPmaps directly,
            // try to emulate it with subimages.
            if (out->supports("mipmap"))
                appendmode = ImageOutput::AppendMIPLevel;
            else
                appendmode = ImageOutput::AppendSubimage;
        }
        out->close ();
        delete out;
\end{code}

In this example, we have used \writeimage, but of course \writescanline,
\writetile, and {\cf write_rectangle()} work as you would expect, on the
current subimage.

\subsection{Copying an entire image}
\label{sec:imageoutput:copyimage}

Suppose you want to copy an image, perhaps with alterations to the 
metadata but not to the pixels.  You could open an \ImageInput and
perform a {\cf read_image()}, and open another \ImageOutput and
call {\cf write_image()} to output the pixels from the input image.
However, for compressed images, this may be inefficient due to the
unnecessary decompression and subsequent re-compression.  In addition,
if the compression is \emph{lossy}, the output image may not contain
pixel values identical to the original input.

A special {\cf copy_image} method of \ImageOutput is available that
attempts to copy an image from an open \ImageInput (of the same
format) to the output as efficiently as possible with without altering
pixel values, if at all possible.

Not all format plugins will provide an implementation of {\cf
  copy_image} (in fact, most will not), but the default implemenatation
simply copies pixels one scanline or tile at a time (with
decompression/recompression) so it's still safe to call.  Furthermore,
even a provided {\cf copy_image} is expected to fall back on the default
implementation if the input and output are not able to do an efficient
copy.  Nevertheless, this method is recommended
for copying images so that maximal advantage will be taken in cases
where savings can be had.

The following is an example use of {\cf copy_image} to transfer pixels
without alteration while modifying the image description metadata:

\begin{code}
    // Open the input file
    const char *input = "input.jpg";
    ImageInput *in = ImageInput::create (input);
    ImageSpec in_spec;
    in->open (input, in_spec);

    // Make an output spec, identical to the input except for metadata
    ImageSpec out_spec = in_spec;
    out_spec.attribute ("ImageDescription", "My Title");

    // Create the output file and copy the image
    const char *output = "output.jpg";
    ImageOutput *out = ImageOutput::create (output);
    out->open (output, out_spec);
    out->copy_image (in);

    // Clean up
    out->close ();
    delete out;
    in->close ();
    delete in;
\end{code}


\subsection{Custom search paths for plugins}
\label{sec:imageoutput:searchpaths}

When you call {\cf ImageOutput::create()}, the \product library will try
to find a plugin that is able to write the format implied by your
filename.  These plugins are alternately known as DLL's on Windows (with
the {\cf .dll} extension), DSO's on Linux (with the {\cf .so}
extension), and dynamic libraries on Mac OS X (with the {\cf .dylib}
extension).  

\product will look for matching plugins according to
\emph{search paths}, which are strings giving a list of directories to
search, with each directory separated by a colon (`{\cf :}').  Within
a search path, any
substrings of the form {\cf \$\{FOO\}} will be replaced
by the value of environment variable {\cf FOO}.  For
example, the searchpath \qkw{\$\{HOME\}/plugins:/shared/plugins}
will first check the directory \qkw{/home/tom/plugins} (assuming the
user's home directory is {\cf /home/tom}), and if not
found there, will then check the directory \qkw{/shared/plugins}.

The first search path it will check is that stored in the environment
variable {\cf IMAGEIO_LIBRARY_PATH}.  It will check each directory in
turn, in the order that they are listed in the variable.  If no adequate
plugin is found in any of the directories listed in this environment
variable, then it will check the custom searchpath passed as the
optional second argument to {\cf ImageOutput::create()}, searching in
the order that the directories are listed.  Here is an example:

\begin{code} 
        char *mysearch = "/usr/myapp/lib:${HOME}/plugins";
        ImageOutput *out = ImageOutput::create (filename, mysearch);
        ...
\end{code} % $


\subsection{Error checking}
\label{sec:imageoutput:errors}

Nearly every \ImageOutput API function returns a {\cf bool} indicating
whether the operation succeeded ({\cf true}) or failed ({\cf false}).
In the case of a failure, the \ImageOutput will have saved an error
message describing in more detail what went wrong, and the latest
error message is accessible using the \ImageOutput method 
{\cf geterror()}, which returns the message as a {\cf std::string}.

The exception to this rule is {\cf ImageOutput::create}, which returns
{\cf NULL} if it could not create an appropriate \ImageOutput.  And in
this case, since no \ImageOutput exists for which you can call its {\cf
  geterror()} function, there exists a global {\cf geterror()}
function (in the {\cf OpenImageIO} namespace) that retrieves the latest
error message resulting from a call to {\cf create}.

Here is another version of the simple image writing code from
Section~\ref{sec:imageoutput:simple}, but this time it is fully 
elaborated with error checking and reporting:

\begin{code}
        #include "imageio.h"
        using namespace OpenImageIO;
        ...

        const char *filename = "foo.jpg";
        const int xres = 640, yres = 480;
        const int channels = 3;  // RGB
        unsigned char pixels[xres*yres*channels];

        ImageOutput *out = ImageOutput::create (filename);
        if (! out) {
            std::cerr << "Could not create an ImageOutput for " 
                      << filename << ", error = " 
                      << OpenImageIO::geterror() << "\n";
            return;
        }
        ImageSpec spec (xres, yres, channels, TypeDesc::UINT8);

        if (! out->open (filename, spec)) {
            std::cerr << "Could not open " << filename 
                      << ", error = " << out->geterror() << "\n";
            delete out;
            return;
        }

        if (! out->write_image (TypeDesc::UINT8, pixels)) {
            std::cerr << "Could not write pixels to " << filename 
                      << ", error = " << out->geterror() << "\n";
            delete out;
            return;
        }

        if (! out->close ()) {
            std::cerr << "Error closing " << filename 
                      << ", error = " << out->geterror() << "\n";
            delete out;
            return;
        }

        delete out;
\end{code}



\section{\ImageOutput Class Reference}
\label{sec:imageoutput:reference}

\apiitem{static ImageOutput * {\ce create} (const std::string \&filename, \\
\bigspc\bigspc\spc const std::string \&plugin_searchpath="")}

Create an \ImageOutput that can be used to write an image file.  The
type of image file (and hence, the particular subclass of \ImageOutput
returned, and the plugin that contains its methods) is inferred from the
extension of the file name.  The {\kw plugin_searchpath} parameter is a
colon-separated list of directories to search for \product plugin
DSO/DLL's.

\apiend

\apiitem{const char * {\ce format_name} ()}
Returns the canonical name of the format that this \ImageOutput
instance is capable of writing.
\apiend

\apiitem{bool {\ce supports} (const std::string \&feature)}
\label{sec:supportsfeaturelist}
Given the name of a \emph{feature}, tells if this \ImageOutput 
instance supports that feature.  The following features are recognized
by this query:
\begin{description}
\item[\spc] \spc 
\item[\rm \qkw{tiles}] Is this plugin able to write tiled images?
\item[\rm \qkw{rectangles}] Can this plugin accept arbitrary rectangular
  pixel regions (via {\kw write_rectangle()})?  False indicates that
  pixels must be transmitted via \writescanline (if
  scanline-oriented) or \writetile (if tile-oriented, and only if
  {\kw supports("tiles")} returns true).
\item[\rm \qkw{random_access}] May tiles or scanlines be written in any
  order?  False indicates that they must be in successive order.
\item[\rm \qkw{multiimage}] Does this format support multiple subimages
  within a single file?
\item[\rm \qkw{mipmap}] Does this format support resolutions per
  image/subimage (MIP-map levels)?
\item[\rm \qkw{volumes}] Does this format support ``3D'' pixel arrays
  (a.k.a.\ volume images)?
\item[\rm \qkw{rewrite}] Does this plugin allow the same scanline or
  tile to be sent more than once?  Generally this is true for plugins
  that implement some sort of interactive display, rather than a saved
  image file.
\item[\rm \qkw{empty}] Does this plugin support passing a NULL data
  pointer to the various {\kw write} routines to indicate that the
  entire data block is composed of pixels with value zero.  Plugins
  that support this achieve a speedup when passing blank scanlines or
  tiles (since no actual data needs to be transmitted or converted).
\end{description}

\noindent This list of queries may be extended in future releases.
Since this can be done simply by recognizing new query strings, and does
not require any new API entry points, addition of support for new
queries does not break ``link compatibility'' with previously-compiled
plugins.
\apiend

\apiitem{bool {\ce open} (const std::string \&name, const ImageSpec \&newspec,\\
\bigspc  OpenMode mode=Create)}

Open the file with given {\kw name}, with resolution, and other format
data as given in {\kw newspec}.  This function returns {\kw true} for
success, {\kw false} for failure.  Note that it is legal to call 
{\kw open()} multiple times on the same file without a call to
{\kw close()}, if it supports multiimage and {\kw mode} is 
{\kw AppendSubimage}, or if it supports MIP-maps and {\kw mode} is 
{\kw AppendMIPlevel} -- this is interpreted as appending a subimage, or
a MIP level to the current subimage, respectively.

\apiend

\apiitem{const ImageSpec \& {\ce spec} ()}
Returns the spec internally associated with this currently open
\ImageOutput.
\apiend

\apiitem{bool {\ce close} ()}
Closes the currently open file associated with this \ImageOutput
and frees any memory or resources associated with it.
\apiend

\apiitem{bool {\ce write_scanline} (int y, int z, TypeDesc format,
     const void *data, \\
\bigspc stride_t xstride=AutoStride)}

Write a full scanline that includes pixels $(*,y,z)$.  For 2D non-volume
images, $z$ is ignored.  The {\kw xstride} value gives the distance
between successive pixels (in bytes).  Strides set to the special value
{\kw AutoStride} imply contiguous data, i.e., \\ \spc {\kw xstride} $=$
{\kw spec.nchannels*format.size()} \\ This method
automatically converts the data from the specified {\kw format} to the
actual output format of the file.  Return {\kw true} for success, {\kw
  false} for failure.  It is a failure to call \writescanline with an
out-of-order scanline if this format driver does not support random
access.

\apiend

\apiitem{bool {\ce write_tile} (int x, int y, int z, TypeDesc format,
                             const void *data, \\ \bigspc stride_t xstride=AutoStride,
                             stride_t ystride=AutoStride, \\ \bigspc stride_t zstride=AutoStride)}

Write the tile with $(x,y,z)$ as the upper left corner.  For 2D
non-volume images, $z$ is ignored.  The three stride values give the
distance (in bytes) between successive pixels, scanlines, and volumetric
slices, respectively.  Strides set to the special value {\kw AutoStride}
imply contiguous data, i.e., \\
\spc {\kw xstride} $=$ {\kw spec.nchannels*format.size()} \\
\spc {\kw ystride} $=$ {\kw xstride*spec.tile_width} \\
\spc {\kw zstride} $=$ {\kw ystride*spec.tile_height} \\
This method automatically converts the
data from the specified {\kw format} to the actual output format of the
file.  Return {\kw true} for success, {\kw false} for failure.  It is a
failure to call \writetile with an out-of-order tile if this
format driver does not support random access.

\apiend

\apiitem{bool {\ce write_rectangle} ({\small int xmin, int xmax, int ymin, int ymax,
                                  int zmin, int zmax,} \\ \bigspc TypeDesc format,
                                  const void *data, \\ \bigspc stride_t xstride=AutoStride,
                                  stride_t ystride=AutoStride, \\
                                  \bigspc stride_t zstride=AutoStride)}

Write pixels whose $x$ coords range over {\kw xmin}...{\kw xmax}
(inclusive), $y$ coords over {\kw ymin}...{\kw ymax}, and $z$ coords
over {\kw zmin}...{\kw zmax}.  The three stride values give the distance
(in bytes) between successive pixels, scanlines, and volumetric slices,
respectively.  Strides set to the special value {\kw AutoStride} imply
contiguous data, i.e.,\\
\spc {\kw xstride} $=$ {\kw spec.nchannels*format.size()} \\
\spc {\kw ystride} $=$ {\kw xstride*(xmax-xmin+1)} \\
\spc {\kw zstride} $=$ {\kw ystride*(ymax-ymin+1)}\\
This method automatically converts the data from the specified 
{\kw format} to the actual output format of the fil.  Return {\kw true}
for success, {\kw false} for failure.  It is a failure to call 
{\kw write_rectangle} for a format plugin that does not return true for
{\kw supports("rectangles")}.

\apiend

\apiitem{bool {\ce write_image} (TypeDesc format, const void *data, \\
                              \bigspc stride_t xstride=AutoStride, stride_t ystride=AutoStride,
                              \\ \bigspc stride_t zstride=AutoStride, \\
                              \bigspc ProgressCallback progress_callback=NULL,\\
                              \bigspc void *progress_callback_data=NULL)}

Write the entire image of {\kw spec.width} $\times$ {\kw spec.height}
$\times$ {\kw spec.depth}
pixels, with the given strides and in the desired format.
Strides set to the special value {\kw AutoStride} imply contiguous data,
i.e., \\
\spc {\kw xstride} $=$ {\kw spec.nchannels * format.size()} \\
\spc {\kw ystride} $=$ {\kw xstride * spec.width} \\
\spc {\kw zstride} $=$ {\kw ystride * spec.height}\\
The function will internally either call \writescanline or 
\writetile, depending on whether the file is scanline- or
tile-oriented.

Because this may be an expensive operation, a progress callback may be passed.
Periodically, it will be called as follows:
\begin{code}
        progress_callback (progress_callback_data, float done)
\end{code}
\noindent where \emph{done} gives the portion of the image 
(between 0.0 and 1.0) that has been written thus far.

\apiend

\apiitem{bool {\ce copy_image} (ImageInput *in)}

Read the current subimage of {\cf in}, and write it as the next subimage
of {\cf *this}, in a way that is efficient and does not alter pixel
values, if at all possible.  Both {\cf in} and {\cf this} must be a
properly-opened \ImageInput and \ImageOutput, respectively, and their
current images must match in size and number of channels.  Return {\cf true}
if it works ok, {\cf false} if for some reason the operation wasn't possible.

If a particular \ImageOutput implementation does not supply a
{\cf copy_image} method, it will inherit the default implementation,
which is to simply read scanlines or tiles from {\cf in} and write
them to {\cf *this}.  However, some format implementations may have a
special technique for directly copying raw pixel data from the
input to the output, when both input and output are the same
file type and the same data format.  This can be more efficient 
than {\cf in->read_image} followed by {\cf out->write_image}, and avoids any
unintended pixel alterations, especially for formats that use
lossy compression.
\apiend

\apiitem{int {\ce send_to_output} (const char *format, ...)}
General message passing between client and image output server.
This is currently undefined and is reserved for future use.
\apiend

\apiitem{int {\ce send_to_client} (const char *format, ...)}
General message passing between client and image output server.
This is currently undefined and is reserved for future use.
\apiend

\apiitem{std::string {\ce geterror} ()}
Returns the current error string describing what went wrong if
any of the public methods returned {\kw false} indicating an error.
(Hopefully the implementation plugin called {\kw error()} with a
helpful error message.)
\apiend



\index{Image I/O API|)}

\chapwidthend

\chapter{Image I/O: Reading Images}
\label{chap:imageinput}
\index{Image I/O API|(}


\section{Image Input Made Simple}
\label{sec:imageinput:simple}

Here is the simplest sequence required to open an image file, find
out its resolution, and read the pixels (converting them into
8-bit values in memory, even if that's not the way they're stored in the file):

\begin{code}
        #include <OpenImageIO/imageio.h>
        OIIO_NAMESPACE_USING
        ...

        const char *filename = "foo.jpg";
        int xres, yres, channels;
        unsigned char *pixels;

        ImageInput *in = ImageInput::create (filename);
        if (! in)
            return;
        ImageSpec spec;
        in->open (filename, spec);
        xres = spec.width;
        yres = spec.height;
        channels = spec.nchannels;
        pixels = new unsigned char [xres*yres*channels];
        in->read_image (TypeDesc::UINT8, pixels);
        in->close ();
        delete in;
\end{code}

\noindent Here is a breakdown of what work this code is doing:

\begin{itemize}
\item Search for an ImageIO plugin that is capable of reading the file
  (\qkw{foo.jpg}), first by trying to deduce the correct plugin from the
  file extension, but if that fails, by opening every ImageIO plugin it
  can find until one will open the file without error.  When it finds
  the right plugin, it creates a subclass instance of \ImageInput that
  reads the right kind of file format.
  \begin{code}
        ImageInput *in = ImageInput::create (filename);
  \end{code}
\item Open the file, read the header, and put all relevant metadata
  about the file in a specification structure.
  \begin{code}
        ImageSpec spec;
        in->open (filename, spec);
  \end{code}
\item The specification contains vital information such as the
  dimensions of the image, number of color channels, and data type of
  the pixel values.  This is enough to allow us to allocate enough space
  for the image.
  \begin{code}
        xres = spec.width;
        yres = spec.height;
        channels = spec.nchannels;
        pixels = new unsigned char [xres*yres*channels];
  \end{code}
  Note that in this example, we don't care what data format is used for
  the pixel data in the file --- we allocate enough space for unsigned
  8-bit integer pixel values, and will rely on \product's ability to
  convert to our requested format from the native data format of the
  file.
\item Read the entire image, hiding all details of the encoding of image
  data in the file, whether the file is scanline- or tile-based, or what
  is the native format of the data in the file (in this case, we request
  that it be automatically converted to unsigned 8-bit integers).
  \begin{code}
        in->read_image (TypeDesc::UINT8, pixels);
  \end{code}
\item Close the file, destroy and free the \ImageInput we had created,
  and perform all other cleanup and release of any resources used by
  the plugin.
  \begin{code}
        in->close ();
        delete in;
  \end{code}
\end{itemize}



\section{Advanced Image Input}
\label{sec:advancedimageinput}

Let's walk through some of the most common things you might want to do,
but that are more complex than the simple example above.


\subsection{Reading individual scanlines and tiles}
\label{sec:imageinput:scanlinestiles}

The simple example of Section~\ref{sec:imageinput:simple} read an
entire image with one call.  But sometimes you want to read a large
image a
little at a time and do not wish to retain the entire image in memory
as you process it.  \product allows you to read images
one scanline at a time or one tile at a time.

Examining the \ImageSpec reveals whether the file is scanline or
tile-oriented: a scanline image will have {\cf spec.tile_width} 
and {\cf spec.tile_height} set to 0, whereas a tiled images will
have nonzero values for the tile dimensions.


\subsubsection{Reading scanlines}

Individual scanlines may be read using the \readscanline API
call:

\begin{code}
        ...
        in->open (filename, spec);
        if (spec.tile_width == 0) {
            unsigned char *scanline = new unsigned char [spec.width*spec.channels];
            for (int y = 0;  y < yres;  ++y) {
                in->read_scanline (y, 0, TypeDesc::UINT8, scanline);
                ... process data in scanline[0..width*channels-1] ...
            }
            delete [] scanline;
        } else {
            ... handle tiles, or reject the file ...
        }
        in->close ();
        ...
\end{code}

The first two arguments to \readscanline specify which scanline
is being read by its vertical ($y$) scanline number (beginning with 0)
and, for volume images, its slice ($z$) number (the slice number should
be 0 for 2D non-volume images).  This is followed by a \TypeDesc
describing the data type of the pixel buffer you are supplying, and a
pointer to the pixel buffer itself.  Additional optional arguments
describe the data stride, which can be ignored for contiguous data (use
of strides is explained in Section~\ref{sec:imageinput:strides}).

Nearly all \ImageInput implementations will be most efficient reading
scanlines in strict order (starting with scanline 0, then 1, up to {\kw
  yres-1}, without skipping any).  An \ImageInput is required to accept
\readscanline requests in arbitrary order, but depending on the file
format and reader implementation, out-of-order scanline reads may be
inefficient.

There is also a {\cf read_scanlines()} function that operates similarly,
except that it takes a {\cf ybegin} and {\cf yend} that specify a range,
reading all scanlines {\cf ybegin} $\le y <$ {\cf yend}.  For most image
format readers, this is implemented as a loop over individual scanlines,
but some image format readers may be able to read a contiguous block of
scanlines more efficiently than reading each one individually.

The full descriptions of the \readscanline and {\cf read_scanlines()}
functions may be found in Section~\ref{sec:imageinput:reference}.

\subsubsection{Reading tiles}

Once you {\kw open()} an image file, you can find out if it is a tiled
image (and the tile size) by examining the \ImageSpec's {\cf
  tile_width}, {\cf tile_height}, and {\cf tile_depth} fields.
If they are zero, it's a scanline image and you should read pixels
using \readscanline, not \readtile.

\begin{code}
        ...
        in->open (filename, spec);
        if (spec.tile_width == 0) {
            ... read by scanline ...
        } else {
            // Tiles
            int tilesize = spec.tile_width * spec.tile_height;
            unsigned char *tile = new unsigned char [tilesize * spec.channels];
            for (int y = 0;  y < yres;  y += spec.tile_height) {
                for (int x = 0;  x < xres;  x += spec.tile_width) {
                    in->read_tile (x, y, 0, TypeDesc::UINT8, tile);
                    ... process the pixels in tile[] ..
                }
            }
            delete [] tile;
        }
        in->close ();
        ...
\end{code}

The first three arguments to \readtile specify which tile is
being read by the pixel coordinates of any pixel contained in the
tile: $x$ (column), $y$ (scanline), and $z$ (slice, which should always
be 0 for 2D non-volume images).  This is followed by a \TypeDesc
describing the data format of the pixel buffer you are supplying, and a
pointer to the pixel buffer.  Pixel data will be written to your buffer
in order of increasing slice, increasing
scanline within each slice, and increasing column within each scanline.
Additional optional arguments describe the data stride, which can be
ignored for contiguous data (use of strides is explained in
Section~\ref{sec:imageinput:strides}).

All \ImageInput implementations are required to support reading tiles in
arbitrary order (i.e., not in strict order of increasing $y$ rows, and
within each row, increasing $x$ column, without missing any tiles).

The full description of the \readtile function may be found
in Section~\ref{sec:imageinput:reference}.


\subsection{Converting formats}
\label{sec:imageinput:convertingformat}

The code examples of the previous sections all assumed that your
internal pixel data is stored as unsigned 8-bit integers (i.e., 0-255
range).  But \product is significantly more flexible.  

You may request that the pixels be stored in any of several formats.
This is done merely by passing the {\cf read} function the data type
of your pixel buffer, as one of the enumerated type \TypeDesc.

%FIXME
%Individual file formats, and therefore \ImageInput implementations, may
%only support a subset of the formats understood by the \product library.
%Each \ImageInput plugin implementation should document which data
%formats it supports.  An individual \ImageInput implementation may
%choose to simply fail open {\kw open()}, though the recommended behavior
%is for {\kw open()} to succeed but in fact choose a data format
%supported by the file format that best preserves the precision and range
%of the originally-requested data format.

It is not required that the pixel data buffer passed to \readimage,
\readscanline, or \readtile actually be in the same data format as the
data in the file being read.  \product will automatically convert from
native data type of the file to the internal data format of your choice.
For example, the following code will open a TIFF and read pixels into
your internal buffer represented as {\cf float} values.  This will work
regardless of whether the TIFF file itself is using 8-bit, 16-bit, or
float values.

\begin{code}
        ImageInput *in = ImageInput::create ("myfile.tif");
        ImageSpec spec;
        in->open (filename, spec);
        ...
        int numpixels = spec.width * spec.height;
        float pixels = new float [numpixels * channels];
        ...
        in->read_image (TypeDesc::FLOAT, pixels);
\end{code}

\noindent Note that \readscanline and \readtile have a parameter that
works in a corresponding manner.

You can, of course, find out the native type of the file simply by
examining {\cf spec.format}.  If you wish, you may then allocate a
buffer big enough for an image of that type and request the native type
when reading, therefore eliminating any translation among types and
seeing the actual numerical values in the file.

%FIXME
%Please refer to Section~\ref{sec:imageinput:quantization} for more
%information on how values are translated among the supported data
%formats by default, and how to change the formulas by specifying
%quantization in the \ImageSpec.


\subsection{Data Strides}
\label{sec:imageinput:strides}

In the preceeding examples, we have assumed that the buffer passed to
the {\cf read} functions (i.e., the place where you want your pixels
to be stored) is \emph{contiguous}, that is:

\begin{itemize}
\item each pixel in memory consists of a number of data values equal to
  the number of channels in the file;
\item successive column pixels within a row directly follow each other in
  memory, with the first channel of pixel $x$ immediately following
  last channel of pixel $x-1$ of the same row;
\item for whole images or tiles, the data for each row
  immediately follows the previous one in memory (the first pixel of row
  $y$ immediately follows the last column of row $y-1$);
\item for 3D volumetric images, the first pixel of slice $z$ immediately
  follows the last pixel of of slice $z-1$.
\end{itemize}

Please note that this implies that \readtile will write pixel data into
your buffer so that it is contiguous in the shape of a single tile, not
just an offset into a whole image worth of pixels.

The \readscanline function takes an optional {\cf xstride} argument, and
the \readimage and \readtile functions take optional {\cf xstride}, 
{\cf ystride}, and {\cf zstride} values that describe the distance, in
\emph{bytes}, between successive pixel columns, rows, and slices,
respectively, of your pixel buffer.  For any of these values that are
not supplied, or are given as the special constant {\cf AutoStride},
contiguity will be assumed.

By passing different stride values, you can achieve some surprisingly
flexible functionality.  A few representative examples follow:

\begin{itemize}
\item Flip an image vertically upon reading, by using \emph{negative}
  $y$ stride:
  \begin{code}
        unsigned char pixels[spec.width * spec.height * spec.nchannels];
        int scanlinesize = spec.width * spec.nchannels * sizeof(pixels[0]);
        ...
        in->read_image (TypeDesc::UINT8,
                        (char *)pixels+(yres-1)*scanlinesize, // offset to last
                        AutoStride,                  // default x stride
                        -scanlinesize,               // special y stride
                        AutoStride);                 // default z stride
  \end{code}
\item Read a tile into its spot in a buffer whose layout matches
  a whole image of pixel data,
  rather than having a one-tile-only memory layout:
  \begin{code}
        unsigned char pixels[spec.width * spec.height * spec.nchannels];
        int pixelsize = spec.nchannels * sizeof(pixels[0]);
        int scanlinesize = xpec.width * pixelsize;
        ...
        in->read_tile (x, y, 0, TypeDesc::UINT8,
                       (char *)pixels + y*scanlinesize + x*pixelsize,
                       pixelsize,
                       scanlinesize);
  \end{code}
\end{itemize}

Please consult Section~\ref{sec:imageinput:reference} for detailed
descriptions of the stride parameters to each {\cf read} function.


\subsection{Reading metadata}
\label{sec:imageinput:metadata}

The \ImageSpec that is filled in by {\cf ImageInput::open()}
specifies all the common properties that describe an image: data format,
dimensions, number of channels, tiling.  However, there may be a variety
of additional \emph{metadata} that are present in the image file and
could be queried by your application.

The remainder of this section explains how to query additional metadata
in the \ImageSpec.  It is up to the \ImageInput to read these
from the file, if indeed the file format is able to carry additional
data.  Individual \ImageInput implementations should document which
metadata they read.

\subsubsection{Channel names}

In addition to specifying the number of color channels, the
\ImageSpec also stores the names of those channels in its {\cf
  channelnames} field, which is a {\cf vector<std::string>}.  Its length
should always be equal to the number of channels (it's the
responsibility of the \ImageInput to ensure this).

Only a few file formats (and thus \ImageInput implementations) have a
way of specifying custom channel names, so most of the time you will see
that the channel names follow the default convention of being named
\qkw{R}, \qkw{G}, \qkw{B}, and \qkw{A}, for red, green, blue, and alpha,
respectively.

Here is example code that prints the names of the channels in an image:

\begin{code}
        ImageInput *in = ImageInput::create (filename);
        ImageSpec spec;
        in->open (filename, spec);
        for (int i = 0;  i < spec.nchannels;  ++i)
            std::cout << "Channel " << i << " is " 
                      << spec.channelnames[i] << "\n";
\end{code}

\subsubsection{Specially-designated channels}

The \ImageSpec contains two fields, {\cf alpha_channel} and {\cf
  z_channel}, which designate which channel numbers represent alpha and
$z$ depth, if any.  If either is set to {\cf -1}, it indicates that it
is not known which channel is used for that data.

If you are doing something special with alpha or depth, it is probably
safer to respect the {\cf alpha_channel} and {\cf z_channel}
designations (if not set to {\cf -1}) rather than merely assuming that,
for example, channel 3 is always the alpha channel.

\subsubsection{Arbitrary metadata}

All other metadata found in the file will be stored in the
\ImageSpec's {\cf extra_attribs} field, which is a 
\ParamValueList, which is itself essentially a vector of
\ParamValue instances.  Each \ParamValue
stores one meta-datum consisting of a name, type (specified by 
a \TypeDesc), number of values, and data pointer.

If you know the name of a specific piece of metadata you want to use,
you can find it using the {\cf ImageSpec::find_attribute()}
method, which returns a pointer to the matching \ParamValue,
or {\cf NULL} if no match was found.  An optional \TypeDesc
argument can narrow the search to only parameters that match the
specified type as well as the name.  Below is an
example that looks for orientation information, expecting it to consist 
of a single integer:

\begin{code}
        ImageInput *in = ImageInput::create (filename);
        ImageSpec spec;
        in->open (filename, spec);
        ...
        ParamValue *p = spec.find_attribute ("Orientation", TypeDesc::INT);
        if (p) {
            int orientation = * (int *) p->data();
        } else {
            std::cout << "No integer orientation in the file\n";
        }
\end{code}

By convention, \ImageInput plugins will save all integer metadata as
32-bit integers ({\cf TypeDesc::INT} or {\cf TypeDesc::UINT}), even if the file format
dictates that a particular item is stored in the file as a 8- or 16-bit
integer.  This is just to keep client applications from having to deal
with all the types.  Since there is relatively little metadata compared
to pixel data, there's no real memory waste of promoting all integer
types to int32 metadata.  Floating-point metadata and string metadata
may also exist, of course.

It is also possible to step through all the metadata, item by item.
This can be accomplished using the technique of the following example:

\begin{code}
        for (size_t i = 0;  i < spec.extra_attribs.size();  ++i) {
            const ParamValue &p (spec.extra_attribs[i]);
            printf ("    \%s: ", p.name.c_str());
            if (p.type() == TypeDesc::STRING)
                printf ("\"\%s\"", *(const char **)p.data());
            else if (p.type() == TypeDesc::FLOAT)
                printf ("\%g", *(const float *)p.data());
            else if (p.type() == TypeDesc::INT)
                printf ("\%d", *(const int *)p.data());
            else if (p.type() == TypeDesc::UINT)
                printf ("\%u", *(const unsigned int *)p.data());
            else
                printf ("<unknown data type>");
            printf ("\n");
        }
\end{code}

Each individual \ImageInput implementation should document the names,
types, and meanings of all metadata attributes that they understand.

\subsubsection{Color space hints}

We certainly hope that you are using only modern file formats that
support high precision and extended range pixels (such as OpenEXR) and
keeping all your images in a linear color space.  But you may have to
work with file formats that dictate the use of nonlinear color values.
This is prevalent in formats that store pixels only as 8-bit values,
since 256 values are not enough to linearly represent colors without
banding artifacts in the dim values.

The {\cf ImageSpec::extra_attribs} field may store metadata that reveals
the color space the image file in the \qkw{oiio:ColorSpace}
attribute, which may take on any of the following values:

\begin{description}
\item[\halfspc \rm \qkw{Linear}] indicates that the
  color pixel values are known to be linear.
\item[\halfspc \rm \qkw{GammaCorrected}] indicates
  that the color pixel values (but not alpha or $z$) have
  already been gamma corrected (raised to the power $1/\gamma$), and
  that the gamma exponent may be found in the \qkw{oiio:Gamma} metadata.
\item[\halfspc \rm \qkw{sRGB}] indicates that the
  color pixel values are in sRGB color space.
\item[\halfspc \rm \qkw{AdobeRGB}] indicates that the
  color pixel values are in Adobe RGB color space.
\item[\halfspc \rm \qkw{Rec709}] indicates that the
  color pixel values are in Rec709 color space.
\item[\halfspc \rm \qkw{KodakLog}] indicates that the
  color pixel values are in Kodak logarithmic color space.
\end{description}

The \ImageInput sets the \qkw{oiio:ColorSpace} metadata in a
purely advisory capacity --- the {\cf read} will not convert pixel
values among color spaces.  Many image file formats only support
nonlinear color spaces (for example, JPEG/JFIF dictates use of sRGB).
So your application should intelligently deal with gamma-corrected and
sRGB input, at the very least.

The color space hints only describe color channels.  You should assume that
alpha or depth ($z$) channels (designated by the {\cf alpha_channel} and
{\cf z_channel} fields, respectively) always represent linear values and
should never be transformed by your application.


%\subsection{Controlling quantization and encoding}
%\label{sec:imageinput:quantization}
%
%FIXME


%\subsection{Random access and repeated transmission of pixels}
%\label{sec:imageinput:randomrepeated}
%
%FIXME


\subsection{Multi-image files and MIP-maps}
\label{sec:imageinput:multiimage}
\label{sec:imageinput:mipmap}

Some image file formats support multiple discrete subimages to be stored
in one file, and/or miltiple resolutions for each image to form a
MIPmap.  When you {\cf open()} an \ImageInput, it will by default point
to the first (i.e., number 0) subimage in the file, and the highest
resolution (level 0) MIP-map level.  You can switch to viewing another
subimage or MIP-map level using the {\cf seek_subimage()} function:

\begin{code}
        ImageInput *in = ImageInput::create (filename);
        ImageSpec spec;
        in->open (filename, spec);
        ...
        int subimage = 1;
        int miplevel = 0;
        if (in->seek_subimage (subimage, miplevel, spec)) {
            ...
        } else {
            ... no such subimage/miplevel ...
        }
\end{code}

The {\cf seek_subimage()} function takes three arguments: the index of
the subimage to switch to (starting with 0), the MIPmap level (starting
with 0 for the highest-resolution level), and a reference to an
\ImageSpec, into which will be stored the spec of the new
subimage/miplevel.  The {\cf seek_subimage()} function returns {\cf
  true} upon success, and {\cf false} if no such subimage or MIP level
existed.  It is legal to visit subimages and MIP levels out of order;
the \ImageInput is responsible for making it work properly.  It is also
possible to find out which subimage and MIP level is currently being
viewed, using the {\cf current_subimage()} and {\cf current_miplevel()}
functions, which return the index of the current subimage and MIP
levels, respectively.

Below is pseudocode for reading all the levels of a MIP-map (a
multi-resolution image used for texture mapping) that shows how to read
multi-image files:

\begin{code}
        ImageInput *in = ImageInput::create (filename);
        ImageSpec spec;
        in->open (filename, spec);

        int num_miplevels = 0;
        while (in->seek_subimage (0, num_miplevels, spec)) {
            // Note: spec has the format of the current subimage/miplevel
            int npixels = spec.width * spec.height;
            int nchannels = spec.nchannels;
            unsigned char *pixels = new unsigned char [npixels * nchannels];
            in->read_image (TypeDesc::UINT8, pixels);

            ... do whatever you want with this level, in pixels ...

            delete [] pixels;
            ++num_miplevels;
        }
        // Note: we break out of the while loop when seek_subimage fails
        // to find a next MIP level.

        in->close ();
        delete in;
\end{code}

In this example, we have used \readimage, but of course \readscanline
and \readtile work as you would expect, on the current subimage and MIP
level.


\subsection{Per-channel formats}
\label{sec:imageinput:channelformats}

Some image formats allow separate per-channel data formats (for example,
{\cf half} data for colors and {\cf float} data for depth).  If you want
to read the pixels in their true native per-channel formats,
the following steps are necessary:

\begin{enumerate}
\item Check the \ImageSpec's {\cf channelformats} vector.  If non-empty,
  the channels in the file do not all have the same format.
\item When calling {\cf read_scanline}, {\cf read_scanlines},
  {\cf read_tile}, {\cf read_tiles}, or {\cf read_image}, 
  pass a format of {\cf TypeDesc::UNKNOWN} to indicate that
  you would like the raw data in native per-channel format of the file
  written to your {\cf data} buffer.
\end{enumerate}

For example, the following code fragment will read a 5-channel image
to an OpenEXR file, consisting of R/G/B/A channels in {\cf half} and
a Z channel in {\cf float}:

\begin{code}
        ImageSpec spec;
        ImageInput *in = ImageInput::create (filename);
        in->open (filename, spec);

        // Allocate enough space
        unsigned char *pixels = new unsigned char [spec.image_bytes(true)];

        in->read_image (TypeDesc::UNKNOWN, /* use native channel formats */
                        pixels);           /* data buffer */

        if (spec.channelformats.size() > 0) {
            ... the buffer contains packed data in the native 
                per-channel formats ...
        } else {
            ... the buffer contains all data per spec.format ...
        }
\end{code}


\subsection{Custom search paths for plugins}
\label{sec:imageinput:searchpaths}

Please see Section~\ref{sec:imageoutput:searchpaths} for discussion
about search paths for finding plugins that implement \ImageOutput.

In a similar fashion, calls to {\cf ImageOutput::create()}
will search for plugins in each directory listed in the environment
variable {\cf OIIO_LIBRARY_PATH}, in the order that they are listed.
If no adequate plugin is found, then it will check the custom searchpath
passed as the optional second argument to {\cf ImageInput::create()}.
Here is an example:

\begin{code}
        char *mysearch = "/usr/myapp/lib:${HOME}/plugins";
        ImageInput *in = ImageInput::create (filename, mysearch);
        ...
\end{code} %$


\subsection{Error checking}
\label{sec:imageinput:errors}
\index{error checking}

Nearly every \ImageInput API function returns a {\cf bool} indicating
whether the operation succeeded ({\cf true}) or failed ({\cf false}).
In the case of a failure, the \ImageInput will have saved an error
message describing in more detail what went wrong, and the latest
error message is accessible using the \ImageInput method 
{\cf geterror()}, which returns the message as a {\cf std::string}.

The exception to this rule is {\cf ImageInput::create}, which returns
{\cf NULL} if it could not create an appropriate \ImageInput.  And in
this case, since no \ImageInput exists for which you can call its {\cf
  geterror()} function, there exists a global {\cf geterror()}
function (in the {\cf OpenImageIO} namespace) that retrieves the latest
error message resulting from a call to {\cf create}.

Here is another version of the simple image reading code from
Section~\ref{sec:imageinput:simple}, but this time it is fully
elaborated with error checking and reporting:

\begin{code}
        #include <OpenImageIO/imageio.h>
        OIIO_NAMESPACE_USING
        ...

        const char *filename = "foo.jpg";
        int xres, yres, channels;
        unsigned char *pixels;

        ImageInput *in = ImageInput::create (filename);
        if (! in) {
            std::cerr << "Could not create an ImageInput for " 
                      << filename << ", error = " 
                      << OpenImageIO::geterror() << "\n";
            return;
        }

        ImageSpec spec;
        if (! in->open (filename, spec)) {
            std::cerr << "Could not open " << filename 
                      << ", error = " << in->geterror() << "\n";
            delete in;
            return;
        }
        xres = spec.width;
        yres = spec.height;
        channels = spec.nchannels;
        pixels = new unsigned char [xres*yres*channels];

        if (! in->read_image (TypeDesc::UINT8, pixels)) {
            std::cerr << "Could not read pixels from " << filename 
                      << ", error = " << in->geterror() << "\n";
            delete in;
            return;
        }

        if (! in->close ()) {
            std::cerr << "Error closing " << filename 
                      << ", error = " << in->geterror() << "\n";
            delete in;
            return;
        }
        delete in;
\end{code}


\newpage
\section{\ImageInput Class Reference}
\label{sec:imageinput:reference}

\apiitem{ImageInput * {\ce create} (const std::string \&filename, \\
\bigspc\bigspc   const std::string \&plugin_searchpath="")}
Create and return an \ImageInput implementation that is able
to read the given file.  The {\kw plugin_searchpath} parameter is a
colon-separated list of directories to search for \product plugin
DSO/DLL's (not a searchpath for the image itself!).  This will
actually just try every ImageIO plugin it can locate, until it
finds one that's able to open the file without error.  This just
creates the \ImageInput, it does not open the file.
\apiend

\apiitem{const char * {\ce format_name} (void) const}
Return the name of the format implemented by this class.
\apiend

\apiitem{bool {\ce open} (const std::string \&name, ImageSpec \&newspec)}
Opens the file with given name and seek to the first subimage in the
file.  Various file attributes are put in
{\kw newspec} and a copy is also saved internally to the
\ImageInput (retrievable via {\kw spec()}.  From examining
{\kw newspec} or {\kw spec()}, you can discern the resolution, if it's
tiled, number of channels, native data format, and other metadata about
the image.  Return {\kw true} if the file was found and opened okay,
otherwise {\kw false}.
\apiend

\apiitem{bool {\ce open} (const std::string \&name, ImageSpec \&newspec,\\
\bigspc  const ImageSpec \&config)}

Opens the file with given name, similarly to {\cf open(name, newspec)}.
However, in this version, any non-default fields of {\cf config},
including metadata, will be taken to be configuration requests,
preferences, or hints.  The default implementation of 
{\cf open (name, newspec, config)} will simply ignore {\cf config} and
calls the usual {\cf open (name, newspec)}.  But a plugin may choose to
implement this version of {\cf open} and respond in some way to the
configuration requests.  Supported configuration requests should be
documented by each plugin.
\apiend

\apiitem {const ImageSpec \& {\ce spec} (void) const}
Returns a reference to the image format specification of the
current subimage.  Note that the contents of the spec are
invalid before {\kw open()} or after {\kw close()}.
\apiend

\apiitem{bool {\ce close} ()}
Closes an open image.
\apiend


\apiitem{int {\ce current_subimage} (void) const}
Returns the index of the subimage that is currently being read.
The first subimage (or the only subimage, if there is just one) is
number 0.
\apiend


\apiitem{bool {\ce seek_subimage} (int subimage, int miplevel, ImageSpec \&newspec)}

Seek to the given subimage and MIP-map level within the open image file.
The first subimage in the file has index 0, and for each subimage, the
highest-resolution MIP level has index 0.  Return {\kw true} on success,
{\kw false} on failure (including that there is not a subimage or MIP
level with those indices).  The new subimage's vital statistics are put
in {\kw newspec} (and also saved internally in a way that can be
retrieved via {\kw spec()}).  The \ImageInput is expected to give the
appearance of random access to subimages and MIP levels --- in other
words, if it can't randomly seek to the given subimage or MIP level, it
should transparently close, reopen, and sequentially read through prior
subimages and levels.

\apiend

\apiitem{bool {\ce read_scanline} (int y, int z, TypeDesc format, void *data,\\
  \bigspc\spc\spc                      stride_t xstride=AutoStride)}

Read the scanline that includes pixels $(*,y,z)$ into {\kw data}
($z=0$ for non-volume images),
converting if necessary from the native data format of the file into the
{\kw format} specified.
If {\cf format} is {\cf TypeDesc::UNKNOWN}, the data will be preserved 
in its native format (including per-channel formats, if applicable).
The {\kw xstride}
value gives the data spacing of adjacent pixels (in bytes).  Strides set
to the special value {\kw AutoStride} imply contiguous data, i.e., \\
  \spc {\kw xstride} $=$ {\kw spec.nchannels * spec.pixel_size()} \\
The \ImageInput is expected to give the appearance of random access
--- in other words, if it can't randomly seek to the given scanline, it
should transparently close, reopen, and sequentially read through prior
scanlines.  The base \ImageInput class has a default implementation
that calls {\kw read_native_scanline()} and then does appropriate format
conversion, so there's no reason for each format plugin to override this
method.
\apiend

\apiitem{bool {\ce read_scanline} (int y, int z, float *data)}
This simplified version of {\kw read_scanline()} reads to contiguous 
float pixels.
\apiend

\apiitem{bool {\ce read_scanlines} (int ybegin, int yend, int z,\\
  \bigspc TypeDesc format, void *data,\\
  \bigspc stride_t xstride=AutoStride, stride_t ystride=AutoStride) \\
bool {\ce read_scanlines} (int ybegin, int yend, int z,\\
  \bigspc int firstchan, int nchans, TypeDesc format, void *data,\\
  \bigspc                      stride_t xstride=AutoStride, stride_t ystride=AutoStride)}

Read all the scanlines that include pixels $(*,y,z)$, where
$\mathit{ybegin} \le y < \mathit{yend}$, into {\kw data}.  This is 
essentially identical to \readscanline, except that can read more than
one scanline at a time, which may be more efficient for certain image
format readers.

The version that specifies a channel range will read only
channels $[${\cf firstchan},{\cf firstchan+nchans}$)$ into the buffer.
\apiend


\apiitem{bool {\ce read_tile} (int x, int y, int z, TypeDesc format,
                            void *data, \\ \bigspc stride_t xstride=AutoStride,
                            stride_t ystride=AutoStride, \\ \bigspc stride_t
                            zstride=AutoStride)}
Read the tile whose upper-left origin is $(x,y,z)$ into {\kw data}
($z=0$ for non-volume images),
converting if necessary from the native data format of the file into the 
{\kw format} specified.
If {\cf format} is {\cf TypeDesc::UNKNOWN}, the data will be preserved 
in its native format (including per-channel formats, if applicable).
The stride values
give the data spacing of adjacent pixels, scanlines, and volumetric
slices, respectively (measured in bytes).  Strides set to the special
value of {\kw AutoStride} imply contiguous data, i.e., \\
\spc {\kw xstride} $=$ {\kw spec.nchannels * spec.pixel_size()} \\
\spc {\kw ystride} $=$ {\kw xstride * spec.tile_width} \\
\spc {\kw zstride} $=$ {\kw ystride * spec.tile_height} \\
The \ImageInput is expected to give the appearance of random access
--- in other words, if it can't randomly seek to the given tile, it
should transparently close, reopen, and sequentially read through prior
tiles.  The base \ImageInput class has a default implementation
that calls {\cf read_native_tile()} and then does appropriate format conversion,
so there's no reason for each format plugin to override this method.

This function returns {\cf true} if it successfully reads the tile,
otherwise {\cf false} for a failure.
The call will fail if the image is not tiled, or if $(x,y,z)$ is not
actually a tile boundary.
\apiend


\apiitem{bool {\ce read_tile} (int x, int y, int z, float *data)}
Simple version of {\kw read_tile} that reads to contiguous float pixels.
\apiend


\apiitem{bool {\ce read_tiles} (int xbegin, int xend, int ybegin, int
  yend, \\ \bigspc int zbegin, int zend, TypeDesc format,
                            void *data, \\ \bigspc stride_t xstride=AutoStride,
                            stride_t ystride=AutoStride, \\ \bigspc stride_t
                            zstride=AutoStride) \\
bool {\ce read_tiles} (int xbegin, int xend, int ybegin, int yend, \\
 \bigspc int zbegin, int zend, int firstchan, int nchans,\\
 \bigspc TypeDesc format, void *data, \\ 
 \bigspc stride_t xstride=AutoStride, stride_t ystride=AutoStride, \\
 \bigspc stride_t zstride=AutoStride)}
Read the tiles bounded by {\kw xbegin} $\le x <$ {\kw xend},
{\kw ybegin} $\le y <$ {\kw yend}, {\kw zbegin} $\le z <$ {\kw zend}
into {\kw data}
converting if necessary from the file's native data format into
the specified buffer {\kw format}.
If {\cf format} is {\cf TypeDesc::UNKNOWN}, the data will be preserved 
in its native format (including per-channel formats, if applicable).
The stride values
give the data spacing of adjacent pixels, scanlines, and volumetric
slices, respectively (measured in bytes).  Strides set to the special
value of {\kw AutoStride} imply contiguous data, i.e., \\
\spc {\kw xstride} $=$ {\kw spec.nchannels * spec.pixel_size()} \\
\spc {\kw ystride} $=$ {\kw xstride * spec.tile_width} \\
\spc {\kw zstride} $=$ {\kw ystride * spec.tile_height} \\
The \ImageInput is expected to give the appearance of random access
--- in other words, if it can't randomly seek to the given tile, it
should transparently close, reopen, and sequentially read through prior
tiles.  The base \ImageInput class has a default implementation
that calls {\cf read_native_tiles()} and then does appropriate format conversion,
so there's no reason for each format plugin to override this method.

This function returns {\cf true} if it successfully reads the tiles,
otherwise {\cf false} for a failure.
The call will fail if the image is not tiled, or if the pixel ranges
do not fall along tile (or image) boundaries, or if it is not a valid
tile range.

The version that specifies a channel range will read only
channels $[${\cf firstchan},{\cf firstchan+nchans}$)$ into the buffer.
\apiend


\apiitem{bool {\ce read_image} (TypeDesc format, void *data, \\
                             \bigspc stride_t xstride=AutoStride,
                             stride_t ystride=AutoStride, \\
                             \bigspc stride_t zstride=AutoStride, \\
                             \bigspc ProgressCallback progress_callback=NULL,\\
                             \bigspc void *progress_callback_data=NULL)}

Read the entire image of {\kw spec.width * spec.height * spec.depth}
pixels into data (which must already be sized large enough for
the entire image) with the given strides, converting into the desired
data format.  
If {\cf format} is {\cf TypeDesc::UNKNOWN}, the data will be preserved 
in its native format (including per-channel formats, if applicable).
This function will automatically handle either tiles or scanlines in
the file.

Strides set to the special value of {\kw AutoStride} imply contiguous
data, i.e., \\
\spc {\kw xstride} $=$ {\kw spec.nchannels * pixel_size()} \\
\spc {\kw ystride} $=$ {\kw xstride * spec.width} \\
\spc {\kw zstride} $=$ {\kw ystride * spec.height} \\
The function will internally either call {\kw read_scanlines} or 
{\kw read_tiles}, depending on whether the file is scanline- or
tile-oriented.

Because this may be an expensive operation, a progres callback may be passed.
Periodically, it will be called as follows:\\
\begin{code}
    progress_callback (progress_callback_data, float done)
\end{code}
\noindent where \emph{done} gives the portion of the image 
(between 0.0 and 1.0) that has been read thus far.
\apiend

\apiitem{bool {\ce read_image} (float *data)}
Simple version of {\kw read_image()} reads to contiguous float pixels.
\apiend

\apiitem{bool {\ce read_native_scanline} (int y, int z, void *data)}
The {\kw read_native_scanline()} function is just like {\kw
  read_scanline()}, except that it keeps the data in the native format
of the disk file and always reads into contiguous memory (no strides).
It's up to the user to have enough space allocated and know what to do
with the data.  IT IS EXPECTED THAT EACH FORMAT PLUGIN WILL OVERRIDE
THIS METHOD.
\apiend

\apiitem{bool {\ce read_native_scanlines} (int ybegin, int yend, int z, void *data)}
The {\kw read_native_scanlines()} function is just like 
{\cf read_native_scanline}, except that it reads
a range of scanlines rather than only one scanline.  It is not necessary
for format plugins to override this method --- a default implementation
in the \ImageInput base class simply calls {\cf read_native_scanline}
for each scanline in the range.  But format plugins may optionally
override this method if there is a way to achieve higher performance by
reading multiple scanlines at once.
\apiend

\apiitem{bool {\ce read_native_scanlines} (int ybegin, int yend, int z,
\\ \bigspc  int firstchan, int nchans, void *data)}
A variant of {\cf read_native_scanlines} that reads only a subset of 
channels \\ $[${\cf firstchan},{\cf firstchan+nchans}$)$.  
If a format reader subclass does
not override this method, the default implementation will simply
call the all-channel version of {\cf read_native_scanlines} into a
temporary buffer and copy the subset of channels.
\apiend

\apiitem{bool {\ce read_native_tile} (int x, int y, int z, void *data)}
The {\kw read_native_tile()} function is just like {\kw read_tile()}, 
except that it keeps the data in the native format of the disk file and
always read into contiguous memory (no strides).  It's up to the user to
have enough space allocated and know what to do with the data.  IT IS
EXPECTED THAT EACH FORMAT PLUGIN WILL OVERRIDE THIS METHOD IF IT
SUPPORTS TILED IMAGES.
\apiend

\apiitem{bool {\ce read_native_tiles} (int xbegin, int xend, int ybegin,
  int yend, \\ \bigspc int zbegin, int zend, void *data)}
The {\kw read_native_tiles()} function is just like {\kw read_tiles()}, 
except that it keeps the data in the native format of the disk file and
always read into contiguous memory (no strides).  
If a format reader does not override this method, the default
implementation it will simply be a loop calling read_native_tile
for each tile in the block.
\apiend

\apiitem{bool {\ce read_native_tiles} (int xbegin, int xend, int ybegin,
  int yend, \\ \bigspc int zbegin, int zend, int firstchan, int nchans, void *data)}
A variant of {\kw read_native_tiles()} that reads only a subset of 
channels \\ $[${\cf firstchan},{\cf firstchan+nchans}$)$.  
If a format reader subclass does
not override this method, the default implementation will simply
call the all-channel version of {\cf read_native_tiles} into a
temporary buffer and copy the subset of channels.
\apiend

\apiitem{int {\ce send_to_input} (const char *format, ...)}
General message passing between client and image input server.
This is currently undefined and is reserved for future use.
\apiend

\apiitem{int {\ce send_to_client} (const char *format, ...)}
General message passing between client and image input server.
This is currently undefined and is reserved for future use.
\apiend

\apiitem{std::string {\ce geterror} () const}
\index{error checking}
Returns the current error string describing what went wrong if
any of the public methods returned {\kw false} indicating an error.
(Hopefully the implementation plugin called {\kw error()} with a
helpful error message.)
\apiend



\index{Image I/O API|)}

\chapwidthend


\chapter{Writing ImageIO Plugins}

\chapter{Bundled ImageIO Plugins}
\section{TIFF}
\section{JPEG}
\section{OpenEXR}
\section{HDR/RGBE}


\part{Image Utilities}

\chapter{The {\kw iv} Image Viewer}
\chapter{Getting Image information With {\kw iinfo}}
\chapter{Converting Image Formats With {\kw iconvert}}

\part{Appendices}
\begin{appendix}

\chapter{Building OpenImageIO}

%\include{header}
\chapter{Glossary}

\begin{description}

\item[Channel] One of several data values persent in each pixel.
  Examples include red, green, blue, alpha, etc.  The data in one
  channel of a pixel may be represented by a single number, whereas the
  pixel as a whole requires one number for each channel.

\item[Client] A client (as in ``client application'') is a program or
  library that uses \product or any of its constituent libraries.

\item[Data format] The type of numerical representation used to
  store a piece of data.  Examples include 8-bit unsigned integers,
  32-bit floating-point numbers, etc.

\item[Image File Format] The specification and data layout of an
  image on disk.  For example, TIFF, JPEG/JFIF, OpenEXR, etc.

\item[Metadata] Data about data.  As used in \product, this means
  Information about an image, beyond describing the values of the pixels
  themselves.  Examples include the name of the artist that created the
  image, the date that an image was scanned, the camera settings used
  when a photograph was taken, etc.

\item[Native data format] The \emph{data format} used in the disk file
  representing an image.  Note that with \product, this may be different
  than the data format used by an application to store the image
  in the computer's RAM.

\item[Pixel] One pixel element of an image, consisting of one number
  describing each \emph{channel} of data at a particular location in an
  image.

\item[Scanline] A single horizontal row of pixels of an image.  See also
  \emph{tile}.

\item[Scanline Image] An image whose data layout on disk is organized by
  breaking the image up into horizontal scanlines, typically with the
  ability to read or write an entire scanline at once.  See also
  \emph{tiled image}.

\item[Tile] A rectangular region of pixels of an image.  A rectangular
  tile is more spatially coherent than a scanline that stretches across
  the entire image --- that is, a pixel's neighbors are most likely in
  the same tile, whereas a pixel in a scanline image will typically have
  most of its immediate neighbors on different scanlines (requiring
  additional scanline reads in order to access them).

\item[Tiled Image] An image whose data layout on disk is organized by
  breaking the image up into rectangular regions of pixels called
  \emph{tiles}.  All the pixels in a tile can be read or written at
  once, and individual tiles may be read or written separately from
  other tiles.

\item[Volume Image] A 3-D set of pixels that has not only horizontal and
  vertical dimensions, but also a "depth" dimension.

\end{description}

\chapwidthend

\end{appendix}

\backmatter

%\bibliographystyle{alpha}	%% Select for [GH95]
\bibliographystyle{apalike}    %% Select for (Gritz and Hahn, 1995)
%\addcontentsline{toc}{chapter}{Bibliography}
%\bibliography{bmrtbib}

\addcontentsline{toc}{chapter}{Index}
\printindex

\end{document}


% Canonical figure
%\begin{figure}[ht]
%\noindent
%\includegraphics[width=5in]{Figures/bredow/foo} 
%\caption{Caption
%\label{fig:foo}}
%\end{figure}
