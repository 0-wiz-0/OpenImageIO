\documentclass[11pt,letterpaper]{book}
\setlength{\oddsidemargin}{0.5in}
\setlength{\topmargin}{0in}
\setlength{\evensidemargin}{0.3in}
\setlength{\textwidth}{5.75in}
\setlength{\textheight}{8.5in}
%\setlength{\oddsidemargin}{1.25in}
%\setlength{\evensidemargin}{0.5in}

% don't do this \usepackage{times}    % Better fonts than Computer Modern
%\usepackage{times}
\renewcommand{\sfdefault}{phv}
\renewcommand{\rmdefault}{ptm}
% don't replace tt -- old is better \renewcommand{\ttdefault}{pcr}
%\usepackage{apalike}
%\usepackage{pslatex}
\usepackage{techref}
\usepackage{epsfig}
\usepackage{verbatim}
\usepackage{moreverb}
\usepackage{graphicx}
\usepackage{xspace}
\usepackage{multicol}
\usepackage{color}
\usepackage{html}
%\usepackage{version}
\usepackage{makeidx}
%\usepackage{showidx}
\usepackage[chapter]{algorithm}
\floatname{algorithm}{Listing}

\usepackage{syntax}

\usepackage{fancyhdr}
\pagestyle{fancy}
\fancyhead[LE,RO]{\bfseries\thepage}
\fancyhead[LO]{\bfseries\rightmark}
\fancyhead[RE]{\bfseries\leftmark}
\fancyfoot[C]{\bfseries OpenImageIO Programmer's Documentation}
\renewcommand{\footrulewidth}{1pt}


\def\product{{\sffamily OpenImageIO}\xspace}
\def\OpenImageIO{{\sffamily OpenImageIO}\xspace}
\def\versionnumber{1.1}
\def\productver{\product\ {\sffamily \versionnumber}\xspace}
\def\producthome{{\codefont \$IMAGEIOHOME}\xspace}
\def\ivbinary{{\codefont iv}\xspace}
\def\maketx{{\codefont maketx}\xspace}
\def\iconvert{{\codefont iconvert}\xspace}
\def\idiff{{\codefont idiff}\xspace}
\def\iinfo{{\codefont iinfo}\xspace}
\def\igrep{{\codefont igrep}\xspace}
\def\oiiotool{{\codefont oiiotool}\xspace}
\def\ImageSpec{{\codefont ImageSpec}\xspace}
\def\ImageInput{{\codefont ImageInput}\xspace}
\def\ImageOutput{{\codefont ImageOutput}\xspace}
\def\ParamBaseType{{\codefont ParamBaseType}\xspace}
\def\TypeDesc{{\codefont TypeDesc}\xspace}
\def\ParamValue{{\codefont ParamValue}\xspace}
\def\ParamValueList{{\codefont ParamValueList}\xspace}
\def\ImageBuf{{\codefont ImageBuf}\xspace}
\def\ImageCache{{\codefont ImageCache}\xspace}
\def\TextureSystem{{\codefont TextureSystem}\xspace}
%\def\opencall{{\codefont open()}\xspace}
\def\writeimage{{\codefont write\_image()}\xspace}
\def\writescanline{{\codefont write\_scanline()}\xspace}
\def\writetile{{\codefont write\_tile()}\xspace}
\def\readimage{{\codefont read\_image()}\xspace}
\def\readscanline{{\codefont read\_scanline()}\xspace}
\def\readtile{{\codefont read\_tile()}\xspace}
\def\AutoStride{{\codefont AutoStride}\xspace}


\title{ 
{\Huge{\bf \product}
%\textregistered\ 
{\bf\sffamily \versionnumber} \medskip \\ \huge Programmer Documentation
\\ \large (in progress) 
} \bigskip }
\author{Editor: Larry Gritz \\
\emph{lg@openimageio.org}
 \bigskip \\
}
\date{{\large 
%Editor: Larry Gritz \\[2ex]
Date: 9 Apr, 2012
% \\ (with corrections, 11 Nov 2011)
}}


%%%%%%%%%%%%%%%%%%%%%%%%%%%%%%%%%%%%%%%%%%%%%%%%%%%%%%%%%%%%%%%%%%%%%%%%%%
% Larry's favorite LaTeX macros for making technical books.  These have
% been refined for years, starting with SIGGRAPH course notes in the
% '90's, further refined for _Advanced RenderMan_.
%%%%%%%%%%%%%%%%%%%%%%%%%%%%%%%%%%%%%%%%%%%%%%%%%%%%%%%%%%%%%%%%%%%%%%%%%%

%
% Define typesetting commands for filenames and code
%
% Just like Advanced RenderMan -- all code in courier, keywords in text
%    courier but not bold.
\def\codefont{\ttfamily}	% font to use for code
\def\ce{\codefont\bfseries}	% emphasize something in code


%
% Define typesetting commands for filenames and code
%
\def\cf{\codefont}		% abbreviation for \codefont
\def\fn{\codefont}		% in-line filenames & unix commands
\def\kw{\codefont}      	% in-line keyword

\newcommand{\var}[1]{{\kw \emph{#1}}}  % variable
\newcommand{\qkw}[1]{{\kw "#1"}}       % quoted keyword
\newcommand{\qkws}[1]{{\small \kw "#1"}}       % quoted keyword, small
\newcommand{\qkwf}[1]{{\footnotesize \kw "#1"}}       % quoted keyword, tiny



% Define some environments for easy typesetting of small amounts of
% code.  These are mostly just wrappers around verbatim, but the
% different varieties also change font sizes.
\newenvironment{code}{\small \verbatimtab}{\endverbatimtab}
\newenvironment{smallcode}{\small \renewcommand{\baselinestretch}{0.8} \verbatimtab}{\endverbatimtab \renewcommand{\baselinestretch}{1}}
\newenvironment{tinycode}{\footnotesize \renewcommand{\baselinestretch}{0.75} \verbatimtab}{\endverbatimtab \renewcommand{\baselinestretch}{1}}

\begin{htmlonly}
\renewenvironment{code}{\begin{verbatim}}{\end{verbatim}}
\newenvironment{smallcode}{\begin{verbatim}}{\end{verbatim}}
\newenvironment{tinycode}{\begin{verbatim}}{\end{verbatim}}
\end{htmlonly}

\newcommand{\includedcode}[1]{{\small \verbatimtabinput{#1}}}
\newcommand{\smallincludedcode}[1]{{\small \renewcommand{\baselinestretch}{0.8} \verbatimtabinput{#1} \renewcommand{\baselinestretch}{1}}}
\newcommand{\tinyincludedcode}[1]{{\footnotesize \renewcommand{\baselinestretch}{0.75} \verbatimtabinput{#1} \renewcommand{\baselinestretch}{1}}}





% Also create a hyphenation list, essentially just to guarantee that
% type names aren't hyphenated
%\hyphenation{Attribute}

% Handy for parameter lists
\def\pl{\emph{parameterlist}\xspace}
\def\epl{\emph{...parameterlist...}\xspace}
\hyphenation{parameterlist}




%begin{latexonly}
\newenvironment{apilist}{\begin{list}{}{\medskip \item[]}}{\end{list}}
\newcommand{\apiitem}[1]{\vspace{12pt} \noindent {\bf\tt #1} \vspace{-10pt}\begin{apilist}\nopagebreak[4]}
\newcommand{\apiend}{\end{apilist}\medskip\pagebreak[2]}
\def\bigspc{\makebox[72pt]{}}
\def\spc{\makebox[24pt]{}}
\def\halfspc{\makebox[12pt]{}}
\def\neghalfspc{\hspace{-12pt}}
\def\negspc{\hspace{-24pt}}
\def\chapwidthbegin{}
\def\chapwidthend{}
%end{latexonly}


\begin{htmlonly}
\newcommand{\apiitem}[1]{\medskip \noindent {\bf #1} \begin{quote}}
\newcommand{\apiend}{\end{quote}}
\def\halfspc{\begin{rawhtml} &nbsp; &nbsp; \end{rawhtml}}
\def\spc{\halfspc\halfspc}
\pagecolor[named]{White}
\def\chapwidthbegin{\begin{rawhtml}<p><table cellspacing=1><tr><td width=550>\end{rawhtml}}
\def\chapwidthend{\begin{rawhtml}</td></tr></table>\end{rawhtml}}
\end{htmlonly}


\newcommand{\apibinding}[3]{\apiitem{#1\\[1ex]#2\\[1ex]#3}}

\newcommand{\CPPBINDING}[1]{\par {\small C++ BINDING:}\par {\spc \codefont #1}}
\newcommand{\PARAMETERS}{\par {\small PARAMETERS:} \par}
\newcommand{\EXAMPLE}{\par {\small EXAMPLE:} \par}
\newcommand{\EXAMPLES}{\par {\small EXAMPLES:} \par}
\newcommand{\SEEALSO}{\par \hspace{-20pt} See Also: \par}



% The \begin{algorithm} \end{algorithm} macros (in algorithm.sty) are
% great for code that can fit all on one page.  But when it can't, use
% these macros.  The first parameter is the caption, the second is the
% label name.
\newcommand{\longalgorithmbegin}[2]{\noindent\hrulefill \\
  \refstepcounter{algorithm}
  \noindent {\bf Listing \arabic{chapter}.\arabic{algorithm}}: #1 \label{#2} \\
  \addcontentsline{loa}{algorithm}{\numberline {\arabic{algorithm}} #1}
  \noindent\hrulefill
}
\newcommand{\longalgorithmend}{\noindent\hrulefill \\}


\def\NEW{\marginpar[\medskip\hfill~\fbox{\sffamily \Huge NEW!}~]{\medskip~\fbox{\sffamily \Huge NEW!}~}}
\newcommand{\NEWdown}[1]{\marginpar[\vspace{#1}\hfill\fbox{\sffamily \Huge NEW!}]{\vspace{#1}\fbox{\sffamily \Huge NEW!}}}
\def\DEPRECATED{\marginpar[\medskip\hfill~\fbox{\sffamily \Large Deprecated}]{\medskip~\fbox{\sffamily \Large Deprecated}}}
\newcommand{\DEPRECATEDdown}[1]{\marginpar{\vspace{#1}\fbox{\sffamily \Large Deprecated}}}
\def\CHANGED{\marginpar[\medskip\hfill~\fbox{\sffamily \huge CHANGED!}~]{\medskip~\fbox{\sffamily \huge CHANGED!}~}}
\def\ENHANCED{\marginpar[\medskip\hfill~\fbox{\sffamily \huge ENHANCED}~]{\medskip~\fbox{\sffamily \huge ENHANCED}~}}


\newcommand{\indexapi}[1]{\index{#1@\tt#1\rm}}



\newenvironment{annotate}{\medskip\sffamily\em\noindent}{\medskip}
%\newenvironment{annotate}{\begin{comment}}{\end{comment}}


\makeindex

\begin{document}
\frontmatter

\maketitle

%\include{speccopyr}

\vspace*{2in}

\begin{centering}
\emph{I kinda like ``Oy-e-oh'' with a bit of a groaning Yiddish accent, as in\\
``OIIO, did you really write yet another file I/O library?''} \\
\end{centering}
\medskip
\begin{centering}
\center Dan Wexler \\
\end{centering}




\setcounter{tocdepth}{1}
\tableofcontents

\mainmatter

\chapter{Introduction}
\label{chap:oiiointro}



Welcome to \product!

\bigskip

\section{Overview}

\product provides simple but powerful \ImageInput and \ImageOutput APIs
that abstract the reading and writing of 2D image file formats.  They
don't support every possible way of encoding images in memory, but for a
reasonable and common set of desired functionality, they provide an
exceptionally easy way for an application using the APIs support a wide
--- and extensible --- selection of image formats without knowing the
details of any of these formats.

Concrete instances of these APIs, each of which implements the ability
to read and/or write a different image file format, are stored as
plugins (i.e., dynamic libraries, DLL's, or DSO's) that are loaded at
runtime.  The \product distribution contains such plugins for several
popular formats.  Any user may create conforming plugins that implement
reading and writing capabilities for other image formats, and any
application that uses \product would be able to use those plugins.

The library also implements the helper class {\kw ImageBuf}, which is a
handy way to store and manipulate images in memory.  {\kw ImageBuf}
itself uses \ImageInput and \ImageOutput for its file I/O, and therefore
is also agnostic as to image file formats.

The {\kw ImageCache} class transparently manages a cache so that it can
access truly vast amounts of image data (thousands of image files
totaling tens of GB) very efficiently using only a tiny amount (tens of
megabytes at most) of runtime memory.  Additionally, a {\kw
  TextureSystem} class provides filtered MIP-map texture lookups, atop
the nice caching behavior of {\kw ImageCache}.

Finally, the \product distribution contains several utility programs
that operate on images, each of which is built atop \ImageInput and
\ImageOutput, and therefore may read or write any image file type for
which an appropriate plugin is found at runtime.  Paramount among these
utilities is {\fn iv}, a really fantastic and powerful image viewing
application.  Additionally, there are programs for converting images
among different formats, comparing image data between two images, 
and examining image metadata.

All of this is released as ``open source'' software using the very
permissive BSD license.  So you should feel free to use any or all of
\product in your own software, whether it is private or public, open
source or proprietary, free or commercial.  You may also modify it on
your own.  You are also encouraged to contribute to the continued
development of \product and to share any improvements that you make on
your own, though you are by no means required to do so.

\section{Simplifying Assumptions}

\product is not the only image library in the world.  Certainly there
are many fine libraries that implement a single image format (including
the excellent {\fn libtiff}, {\fn jpeg-6b}, and {\fn OpenEXR} that
\product itself relies on).  Many libraries attempt to present a uniform
API for reading and writing multiple image file formats.  Most of these
support a fixed set of image formats, though a few of these
also attempt to provide an extensible set by using the plugin approach.

But in our experience, these libraries are all flawed in one or more
ways: (1) They either support only a few formats, or many formats but
with the majority of them somehow incomplete or incorrect.  (2) Their
APIs are not sufficiently expressive as to handle all the image features
we need (such as tiled images, which is critical for our texture
library).  (3) Their APIs are \emph{too complete}, trying to handle
every possible permutation of image format features, and as a result
are horribly complicated.

The third sin is the most severe, and is almost always the main problem
at the end of the day.  Even among the many open source image libraries
that rely on extensible plugins, we have not found one that is both
sufficiently flexible and has APIs anywhere near as simple to understand
and use as those of \product.

Good design is usually a matter of deciding what \emph{not} to do, and
\product is no exception.  We achieve power and elegance only by
making simplifying assumptions.  Among them:

\begin{itemize}
  \item \product only deals with ordinary 2D images, and to a limited
    extent 3D volumes, and image files that contain multiple (but
    finite) independent images within them.  \product {\bf~ does not deal
      with motion picture files.}  At least, not currently.

  \item Pixel data are 8- 16- or 32-bit int (signed or unsigned), 16-
    32- or 64-bit float.  NOTHING ELSE.  No $<8$ bit images, or pixels
    boundaries that aren't byte boundaries.  Files with $<8$ bits will
    appear to the client as 8-bit unsigned grayscale images.

  \item Only fully elaborated, non-compressed data are accepted
    and returned by the API.  Compression or special encodings are
    handled entirely within an \product plugin.

  \item Color space is grayscale or RGB.  Non-spectral data, such as
    XYZ, CMYK, or YUV, are converted to RGB upon reading.\

  \item All color channels have the same data format.  Upon read, an
    \ImageInput ought to convert all channels to the one with the highest
    precision in the file.

  \item All image channels in a subimage are sampled at the same
    resolution.  For file formats that allow some channels to be
    subsampled, they will be automatically up-sampled to the highest
    resolution channel in the subimage.

  \item Color information is always in the order R, G, B, and the alpha
    channel, if any, always follows RGB, and z channel (if any) always
    follows alpha.  So if a file actually stores ABGR, the plugin is
    expected to rearrange it as RGBA.

\end{itemize}

It's important to remember that these restrictions apply to data passed
through the APIs, not to the files themselves.  It's perfectly fine to
have an \product plugin that supports YUV data, or 4 bits per channel, or
any other exotic feature.  You could even write a movie-reading
\ImageInput (despite \product's claims of not supporting movies) and
make it look to the client like it's just a series of images within the
file.  It's just that all the nonconforming details are handled entirely
within the \product plugin and are not exposed through the main \product
APIs.


\subsection*{Historical Origins}

\product is the evolution of concepts and tools I've been working on 
for two decades.

In the 1980's, every program I wrote that output images would have a
simple, custom format and viewer.  I soon graduated to using a standard
image file format (TIFF) with my own library implementation.  Then I
switched to Sam Leffler's stable and complete {\fn libtiff}.

In the mid-to-late-1990's, I worked at Pixar as one of the main
implementors of PhotoRealistic RenderMan, which had \emph{display
  drivers} that consisted of an API for opening files and outputting
pixels, and a set of DSO/DLL plugins that each implement image output
for each of a dozen or so different file format.  The plugins all
responded to the same API, so the renderer itself did not need to know
how to the details of the image file formats, and users could (in
theory, but rarely in practice) extend the set of output image formats
the renderer could use by writing their own plugins.

This was the seed of a good idea, but PRMan's display driver plugin API
was abstruse and hard to use.  So when I started Exluna in 2000, Matt
Pharr, Craig Kolb, and I designed a new API for image output for our own
renderer, Entropy.  This API, called ``ExDisplay,'' was C++, and much
simpler, clearer, and easier to use than PRMan's display drivers.

NVIDIA's Gelato (circa 2002), whose early work was done by myself, Dan
Wexler, Jonathan Rice, and Eric Enderton, had an API
called ``ImageIO.''  ImageIO was 
\emph{much} more powerful and descriptive than ExDisplay, and had an
API for \emph{reading} as well as writing images.  Gelato was not only
``format agnostic'' for its image output, but also for its
image input (textures, image viewer, and other image utilities).
We released the API specification and headers (though not the
library implementation) using the BSD open source license, firmly
repudiating any notion that the API should be specific to NVIDIA or
Gelato.

For Gelato 3.0 (circa 2007), we refined ImageIO again (by this time,
Philip Nemec was also a major influence, in addition to Dan, Eric, and
myself\footnote{Gelato as a whole had many other contributors; those
  I've named here are the ones I recall contributing to the design or
  implementation of the ImageIO APIs}).  This revision was not a major
overhaul but more of a fine tuning.  Our ideas were clearly approaching
stability.  But, alas, the Gelato project was canceled before Gelato 3.0
was released, and despite our prodding, NVIDIA executives would not open
source the full ImageIO code and related tools.

After I left NVIDIA, I was determined to recreate this work once
again -- and ONLY once more -- and release it as open source from the
start.  Thus, \product was born.  I started with the existing Gelato
ImageIO specification and headers (which were BSD licensed all along),
and made some more refinements since I had to rewrite the entire
implementation from scratch anyway.  I think the additional changes are
all improvements.  This is the software you have in your hands today.


\subsection*{Acknowledgments}

\begin{comment}
The direct precursor to \product was Gelato's ImageIO, which was
co-designed and implemented by Larry Gritz, Dan Wexler, Jonathan Rice, Eric
Enderton, and Philip Nemec.

Big thanks to our bosses at NVIDIA for allowing us to share the API spec
and headers under the BSD license.  And thanks to their inability to
open source their own implementation in a timely manner, I was forced to
create this clearly superior descendant.
\end{comment}

\product incorporates or depends upon several other open source
packages:

\begin{itemize}
\item {\cf libtiff} \copyright 1988-1997 Sam Leffler and 1991-1997 Silicon
Graphics, Inc. \\ {\cf http://libtiff.org}
\item {\cf jpeg-6b} \copyright 1991-1998, Thomas G. Lane.  {\cf http://www.ijg.org}
\item OpenEXR, Ilmbase, and Half \copyright 2006, Industrial Light \& Magic.\\
{\cf http://www.openexr.com}
\item {\cf zlib} \copyright 1995-2005 Jean-loup Gailly and Mark Adler. 
{\cf http://www.zlib.net}
\item {\cf libpng} \copyright 1998-2008 Glenn Randers-Pehrson, et al.  
{\cf http://www.libpng.org}
\item The {\cf atomic<>} template in {\cf src/include/tbb} is from
Intel's Thread Building Blocks, \copyright 2005--2008 Intel Corporation.
{\cf http://www.threadingbuildingblocks.org/}
\item The SHA-1 implemenation we use is public domain by
Dominik Reichl \\ {\cf http://www.dominik-reichl.de/}
\item Boost {\cf http://www.boost.org}
\item GLEW \copyright 2002-2007 Milan Ikits, et al. 
{\cf http://glew.sourceforge.net}
\end{itemize}

These other packages are all distributed under licenses that allow them
to be used by and distributed with \product.

\chapwidthend


\part{The OpenImageIO Library}

\chapter{Image I/O API}
\label{chap:imageioapi}
\index{Image I/O API|(}



\section{Data Type Descriptions: {\cf TypeDesc}}
\label{sec:dataformats}
\label{sec:TypeDesc}
\index{data formats}

There are two kinds of data that are important to \product:

\begin{itemize}
\item \emph{Internal data} is in the memory of the computer, used by an
  application program.
\item \emph{Native file data} is what is stored in an image file itself
  (i.e., on the ``other side'' of the abstraction layer that \product
  provides).
\end{itemize}

Both internal and file data is stored in a particular \emph{data format}
that describes the numerical encoding of the values.  \product
understands several types of data encodings, and there is 
a special type, \TypeDesc, that allows their enumeration.
A \TypeDesc describes a base data format type, aggregation into simple
vector and matrix types, and an array length (if
it's an array).

\TypeDesc supports the following base data format types, given by the
enumerated type {\cf BASETYPE}:

\medskip

\begin{tabular}{l p{4.75in}}
{\cf UINT8} &  8-bit integer values ranging from
  0..255, corresponding to the C/C++ {\cf unsigned char}. \\
{\cf INT8} &  8-bit integer values ranging from
  -128..127, corresponding to the C/C++ {\cf char}. \\
{\cf UINT16} &  16-bit integer values ranging
  from 0..65535, corresponding to the C/C++ {\cf unsigned short}. \\
{\cf INT16} &  16-bit integer values ranging
  from -32768..32767, corresponding to the C/C++ {\cf short}. \\
{\cf UINT} &  32-bit integer values,
  corresponding to the C/C++ {\cf unsigned int}. \\
{\cf INT} &  signed 32-bit integer values, corresponding
  to the C/C++ {\cf int}. \\
{\cf UINT64} &  64-bit integer values,
  corresponding to the C/C++ {\cf unsigned long long} (on most architectures). \\
{\cf INT64} &  signed 64-bit integer values, corresponding
  to the C/C++ {\cf long long} (on most architectures). \\
{\cf FLOAT} &  32-bit IEEE floating point values,
  corresponding to the C/C++ {\cf float}. \\
{\cf DOUBLE} &  64-bit IEEE floating point values,
  corresponding to the C/C++ {\cf double}. \\
{\cf HALF} &  16-bit floating point values in the format
  supported by OpenEXR and OpenGL.
\end{tabular}
\medskip

\noindent A \TypeDesc can be constructed using just this information, either as
a single scalar value, or an array of scalar values:

\apiitem{{\ce TypeDesc} (BASETYPE btype) \\
{\ce TypeDesc} (BASETYPE btype, int arraylength)}
Construct a type description of a single scalar value of the given base
type, or an array of such scalars if an array length is supplied.  For
example, {\cf TypeDesc(UINT8)} describes an unsigned 8-bit integer,
and {\cf TypeDesc(FLOAT,7)} describes an array of 7 32-bit float values.
Note also that a non-array \TypeDesc may be implicitly constructed from
just the {\cf BASETYPE}, so it's okay to pass a {\cf BASETYPE}
to any function parameter that takes a full \TypeDesc.
\apiend


\medskip
\noindent In addition, \TypeDesc supports certain aggregate types, described
by the enumerated type {\cf AGGREGATE}:

\medskip
\begin{tabular}{l p{4.75in}}
{\cf SCALAR} & a single scalar value (such as a raw {\cf int}
  or {\cf float} in C).  This is the default. \\
{\cf VEC2} & two values representing a 2D vector. \\
{\cf VEC3} & three values representing a 3D vector. \\
{\cf VEC4} & four values representing a 4D vector. \\
{\cf MATRIX44} & sixteen values representing a $4 \times 4$ matrix.
\end{tabular}
\medskip

\noindent And optionally, several vector transformation
semantics, described by the enumerated type {\cf VECSEMANTICS}:

\medskip
\begin{tabular}{p{1in} p{4.25in}}
{\cf NOXFORM} & indicates that the item is not a spatial quantity that
  undergoes any particular transformation. \\
{\cf COLOR} & indicates that the item is a ``color,'' not a spatial
  quantity (and of course therefore does not undergo a transformation). \\
{\cf POINT} &  indicates that the item represents a
  spatial position and should be transformed by a $4 \times 4$ matrix
  as if it had a 4th component of 1. \\
{\cf VECTOR} &  indicates that the item represents a
  spatial direction and should be transformed by a $4 \times 4$ matrix
  as if it had a 4th component of 0. \\
{\cf NORMAL} &  indicates that the item represents a
  surface normal and should be transformed like a vector, but using the
  inverse-transpose of a $4 \times 4$ matrix.
\end{tabular}
\medskip

\noindent These can be combined to fully describe a complex type:

\apiitem{{\ce TypeDesc} (BASETYPE btype, AGGREGATE agg, VECSEMANTICS
xform=NOXFORM)  \\
{\ce TypeDesc} (BASETYPE btype, AGGREGATE agg, int arraylen) \\
{\ce TypeDesc} (BASETYPE btype, AGGREGATE agg, VECSEMANTICS xform, int arraylen)
}
Construct a type description of an aggregate (or array of aggregates),
with optional vector transformation semantics.  For example, 
{\cf TypeDesc(HALF,COLOR)} describes an aggregate of 3 16-bit floats
comprising a color, and {\cf TypeDesc(FLOAT,VEC3,POINT)} describes 
an aggregate of 3 32-bit floats comprising a 3D position.

Note that aggregates and arrays are different.  A {\cf
  TypeDesc(FLOAT,3)} is an array of three floats, a {\cf
  TypeDesc(FLOAT,COLOR)} is a single 3-channel color comprised of
floats, and {\cf TypeDesc(FLOAT,3,COLOR)} is an array of 3 color values,
each of which is comprised of 3 floats.
\apiend

\bigskip

Of these, the only ones commonly used to store pixel values in image files
are scalars of {\cf UINT8}, {\cf UINT16}, {\cf FLOAT}, and {\cf HALF}
(the last only used by OpenEXR, to the best of our knowledge).

Note that the \TypeDesc (which is also used for applications other
than images) can describe many types not used by
\product.  Please ignore this extra complexity; only the above simple types are understood by
\product as pixel storage data types, though a few others, including
{\cf STRING} and {\cf MATRIX44} aggregates, are occasionally used for
\emph{metadata} for certain image file formats (see
Sections~\ref{sec:imageoutput:metadata}, \ref{sec:imageinput:metadata},
and the documentation of individual ImageIO plugins for details).

\section{Image Specification: {\cf ImageSpec}}
\label{sec:ImageSpec}
\indexapi{ImageSpec}

An \ImageSpec is a structure that describes the complete
format specification of a single image.  It contains:

\begin{itemize}
\item The image resolution (number of pixels).
\item The origin, if its upper left corner is not located beginning at
  pixel (0,0).
\item The full size and offset of an abstract ``full'' or ``display''
  image, useful for describing cropping or overscan.
\item Whether the image is organized into \emph{tiles}, and if so, the
  tile size.
\item The \emph{native data format} of the pixel values (e.g., float, 8-bit
  integer, etc.).
\item The number of color channels in the image (e.g., 3 for RGB
  images), names of the channels, and whether any particular channels
  represent \emph{alpha} and \emph{depth}.
\item Any presumed gamma correction or hints about color space of
  the pixel values.
\item Quantization parameters describing how floating point values
  should be converted to integers (in cases where users pass real values
  but integer values are stored in the file).  This is used only when
  writing images, not when reading them.
\item A user-extensible (and format-extensible) list of any other
  arbitrarily-named and -typed data that may help describe the image or
  its disk representation.
\end{itemize}

\subsection{\ImageSpec Data Members}

The \ImageSpec contains data fields for the values that are
required to describe nearly any image, and an extensible list of
arbitrary attributes that can hold metadata that may be user-defined or
specific to individual file formats.  Here are the hard-coded data
fields:

\apiitem{int width, height, depth \\
int x, y, z}

{\cf width, height, depth} are the size of the data of this image, i.e.,
the number of pixels in each dimension.  A {\cf depth} greater than 1
indicates a 3D ``volumetric'' image.

{\cf x, y, z} indicate the \emph{origin} of the pixel data of the image.
These default to (0,0,0), but setting them differently may indicate that
this image is offset from the usual origin.

Therefore the pixel data are defined over pixel coordinates
[{\cf x} ... {\cf x+width-1}] horizontally, 
[{\cf y} ... {\cf y+height-1}] vertically, 
and [{\cf z} ... {\cf z+depth-1}] in depth.
\apiend

\apiitem{int full_width, full_height, full_depth \\
int full_x, full_y, full_z}

These fields define a ``full'' or ``display'' image window over the
region [{\cf full_x} ... {\cf full_x+full_width-1}] horizontally, 
[{\cf full_y} ... {\cf full_y+full_height-1}] vertically, 
and [{\cf full_z} ... {\cf full_z+full_depth-1}] in depth.

Having the full display window different from the pixel data window can
be helpful in cases where you want to indicate that your image is a
\emph{crop window} of a larger image (if the pixel data window is a
subset of the full display window), or that the pixels include
\emph{overscan} (if the pixel data is a superset of the full display
window), or may simply indicate how different non-overlapping images
piece together.
\apiend

\apiitem{int tile_width, tile_height, tile_depth}
If nonzero, indicates that the image is stored on disk organized into
rectangular \emph{tiles} of the given dimension.  The default of 
(0,0,0) indicates that the image is stored in scanline order, rather
than as tiles.
\apiend

\apiitem{TypeDesc format}
Indicates the native format of the pixel data values themselves, as a 
\TypeDesc (see \ref{sec:TypeDesc}).  Typical values would be
{\cf TypeDesc::UINT8} for 8-bit unsigned values, {\cf TypeDesc::FLOAT} for 32-bit
floating-point values, etc.

\noindent NOTE: Currently, the implementation of OpenImageIO requires
all channels to have the same data format.
\apiend

\apiitem{int nchannels}
The number of \emph{channels} (color values) present in each pixel of
the image.  For example, an RGB image has 3 channels.
\apiend

\apiitem{std::vector<std::string> channelnames}
The names of each channel, in order.  Typically this will be \qkw{R},
\qkw{G},\qkw{B}, \qkw{A} (alpha), \qkw{Z} (depth), or other arbitrary
names.
\apiend

\apiitem{int alpha_channel}
The index of the channel that represents \emph{alpha} (pixel coverage
and/or transparency).  It defaults to -1 if no alpha channel is present,
or if it is not known which channel represents alpha.
\apiend

\apiitem{int z_channel}
The index of the channel that respresents \emph{z} or \emph{depth} (from
the camera).  It defaults to -1 if no depth channel is present, or if it
is not know which channel represents depth.
\apiend

\apiitem{LinearitySpec linearity}
Describes the mapping of pixel values to real-world units.  
{\cf LinearitySpec} is
an enumerated type that may take on the following values:
\begin{itemize}
\item[] 
\item {\cf Linear} (the default) indicates that pixel values map
  linearly.
\item {\cf GammaCorrected} indicates that the color pixel values have
  already been gamma corrected, using the exponent given by the {\cf
    gamma} field.  (It is still assumed that non-color values, such as
  alpha and depth, are linear.)
\item {\cf sRGB} indicates that color values are encoded using the sRGB
  mapping.  (It is still assumed that non-color values are linear.)
\item {\cf AdobeRGB} indictes that the values are encoded in the
Adobe RGB color space. (It is still assumed that non-color values are linear.)
\item {\cf Rec709} indicates that color values are encoded using the 
  Rec709 mapping.  (It is still assumed that non-color values are linear.)
\item {\cf KodakLog} indicates that color values are encoded using the 
  Kodak logaithmic mapping.  (It is still assumed that non-color values are linear.)
\end{itemize}
\apiend

\apiitem{float gamma}
The gamma exponent, if the pixel values in the image have already been
gamma corrected (indicated by {\cf linearity} having a value of {\cf
GammaCorrected}).  The default of 1.0 indicates that no gamma
correction has been applied.
\apiend

\apiitem{int quant_black, quant_white, quant_min, quant_max;\\
  float quant_dither}
Describes the \emph{quantization}, or mapping between real
(floating-point) values and the stored integer values.
Please refer to Section~\ref{sec:imageoutput:quantization} for
a more complete explanation of each of these parameters.
\apiend

\apiitem{ParamValueList extra_attribs}
A list of arbitrarily-named and arbitrarily-typed additional attributes
of the image, for any metadata not described by the hard-coded fields
described above.  This list may be manipulated with the {\cf
attribute()} and {\cf find_attribute()} methods.
\apiend

\subsection{\ImageSpec member functions}

\noindent \ImageSpec contains the following methods that
manipulate format specs or compute useful information about images given
their format spec:

\apiitem{{\ce ImageSpec} (int xres, int yres, int nchans, TypeDesc fmt = UINT8)}
Constructs an \ImageSpec with the given $x$ and $y$ resolution, number
of channels, and pixel data format.

All other fields are set to the obvious defaults -- the image is an
ordinary 2D image (not a volume), the image is not offset or a crop of a
bigger image, the image is scanline-oriented (not tiled), channel names
are ``R'', ``G'', ``B,'' and ``A'' (up to and including 4 channels,
beyond that they are named ``channel \emph{n}''), the fourth channel (if
it exists) is assumed to be alpha, values are assumed to be linear, and
quantization (if \emph{fmt} describes an integer type) is done in
such a way that the maximum positive integer range maps to (0.0, 1.0).
\apiend

\apiitem{void {\ce set_format} (TypeDesc fmt)}
Sets the format as described, and also sets all quantization parameters
to the default for that data type (as explained in 
Section~\ref{sec:imageoutput:quantization}).
\apiend

\apiitem{void {\ce default_channel_names} ()}
Sets the {\cf channelnames} to reasonable defaults for the number of
channels.  Specifically, channel names are set to ``R'', ``G'', ``B,''
and ``A'' (up to and including 4 channels, beyond that they are named
``channel\emph{n}''.
\apiend

\apiitem{static TypeDesc \\
{\ce format_from_quantize} (int quant_black, int quant_white,\\
\bigspc \bigspc                          int quant_min, int quant_max)}
Utility function that, given quantization parameters, returns a data
type that may be used without unacceptable loss of significant bits.
% FIXME - elaborate?
\apiend

\apiitem{size_t {\ce channel_bytes} () const}
Returns the number of bytes comprising each channel of each pixel (i.e.,
the size of a single value of the type described by the {\cf format} field).
\apiend

\apiitem{size_t {\ce pixel_bytes} () const}
Returns the number of bytes comprising each pixel (i.e. the number of
channels multiplied by the channel size).
\apiend

\apiitem{imagesize_t {\ce scanline_bytes} () const}
Returns the number of bytes comprising each scanline (i.e. {\cf width}
pixels).  
This will return {\cf std::numeric_limits<imagesize_t>::max()} in the event
of an overflow where it's not representable in an {\cf imagesize_t}.
\apiend

\apiitem{imagesize_t {\ce tile_pixels} () const}
Returns the number of tiles comprising an image tile (if it's a tiled image).
This will return {\cf std::numeric_limits<imagesize_t>::max()} in the event
of an overflow where it's not representable in an {\cf imagesize_t}.
\apiend

\apiitem{imagesize_t {\ce tile_bytes} () const}
Returns the number of bytes comprising an image tile (if it's a tiled image).
This will return {\cf std::numeric_limits<imagesize_t>::max()} in the event
of an overflow where it's not representable in an {\cf imagesize_t}.
\apiend

\apiitem{imagesize_t {\ce image_pixels} () const}
Returns the number of pixels comprising an entire image image of these dimensions.
This will return {\cf std::numeric_limits<imagesize_t>::max()} in the event
of an overflow where it's not representable in an {\cf imagesize_t}.
\apiend

\apiitem{imagesize_t {\ce image_bytes} () const}
Returns the number of bytes comprising an entire image of these dimensions.
This will return {\cf std::numeric_limits<imagesize_t>::max()} in the event
of an overflow where it's not representable in an {\cf imagesize_t}.
\apiend

\apiitem{bool {\ce size_t_safe} () const}
Return {\cf true} if an image described by this spec can the sizes
(in pixels or bytes) of its scanlines, tiles, and the entire image can
be represented by a {\cf size_t} on that platform.  If this returns
{\cf false}, the client application should be very careful allocating
storage!
\apiend

% FIXME - document auto_stride() ?

\apiitem{void {\ce attribute} (const std::string \&name, TypeDesc type, \\
\bigspc const void *value)}
Add a metadata attribute to {\cf extra_attribs}, with the given name and
data type.  The {\cf value} pointer specifies
the address of the data to be copied.
\apiend

\apiitem{void {\ce attribute} (const std::string \&name, unsigned int value)\\
    void {\ce attribute} (const std::string \&name, int value)\\
    void {\ce attribute} (const std::string \&name, float value)\\
    void {\ce attribute} (const std::string \&name, const char *value)\\
    void {\ce attribute} (const std::string \&name, const std::string \&value)}
Shortcuts for passing attributes comprised of a single integer,
floating-point value, or string.
\apiend

\apiitem{ImageIOParameter * {\ce find_attribute} (const std::string \&name,\\
\bigspc\bigspc\spc                           TypeDesc searchtype=UNKNOWN,\\
\bigspc\bigspc\spc                           bool casesensitive=false)\\
const ImageIOParameter * {\ce find_attribute} (const std::string \&name,\\
\bigspc\bigspc\spc                           TypeDesc searchtype=UNKNOWN,\\
\bigspc\bigspc\spc                           bool casesensitive=false) const
\\
}

Searches {\cf extra_attribs} for an attribute matching {\cf name},
returning a pointer to the attribute record, or NULL if there was no
match.  If {\cf searchtype} is {\cf TypeDesc::UNKNOWN}, the search will be made
regardless of the data type, whereas other values of {\cf searchtype}
will reject a matching name if the data type does not also match.  The
name comparison will be exact if {\cf casesensitive} is true, otherwise
in a case-insensitive manner if {\cf caseinsensitive} is false.
\apiend

\apiitem{int {\ce get_int_attribute} (const std::string \&name, int
  defaultval=0) const}
Gets an integer metadata attribute (silently converting to {\cf int}
even if if the data is really int8, uint8, int16, uint16, or uint32),
and simply substituting the supplied default value if no such metadata
exists.  This is a convenience function for when you know you are just
looking for a simple integer value.
\apiend

\apiitem{float {\ce get_float_attribute} (const std::string \&name,\\
\bigspc\bigspc float defaultval=0) const}
Gets a float metadata attribute (silently converting to {\cf float} even
if the data is really half or double), simply substituting the supplied
default value if no such metadata exists.  This is a convenience
function for when you know you are just looking for a simple float value.
\apiend

\apiitem{std::string {\ce get_string_attribute} (const std::string \&name, \\
\bigspc\bigspc const std::string \&defaultval=std::string()) const}
Gets a string metadata attribute, simply substituting the supplied
default value if no such metadata exists.  This is a convenience
function for when you know you are just looking for a simple string value.
\apiend


\apiitem{std::string {\ce metadata_val} (const ImageIOParamaeter \&p,
  bool human=true) const}
For a given parameter (in this \ImageSpec's {\cf extra_attribs} field),
format the value nicely as a string.  If {\cf human} is true, use
especially human-readable explanations (units, or decoding of
values) for certain known metadata.
\apiend


\index{Image I/O API|)}

\chapwidthend

\chapter{ImageOutput: Writing Images}
\label{chap:imageoutput}
\index{Image I/O API|(}
\indexapi{ImageOutput}


\section{Image Output Made Simple}
\label{sec:imageoutput:simple}

Here is the simplest sequence required to write the pixels of a 2D image
to a file:

\begin{code}
        #include "imageio.h"
        using namespace OpenImageIO;
        ...

        const char *filename = "foo.jpg";
        const int xres = 640, yres = 480;
        const int channels = 3;  // RGB
        unsigned char pixels[xres*yres*channels];

        ImageOutput *out = ImageOutput::create (filename);
        if (! out)
            return;
        ImageSpec spec (xres, yres, channels, TypeDesc::UINT8);
        out->open (filename, spec);
        out->write_image (TypeDesc::UINT8, pixels);
        out->close ();
        delete out;
\end{code}

\noindent This little bit of code does a surprising amount of useful work:  

\begin{itemize}
\item Search for an ImageIO plugin that is capable of writing the file
  (\qkw{foo.jpg}), deducing the format from the file extension.  When it
  finds such a plugin, it creates a subclass instance of \ImageOutput
  that writes the right kind of file format.
  \begin{code}
        ImageOutput *out = ImageOutput::create (filename);
  \end{code}
\item Open the file, write the correct headers, and in all other
  important ways prepare a file with the given dimensions ($640 \times
  480$), number of color channels (3), and data format (unsigned 8-bit
  integer).
  \begin{code}
        ImageSpec spec (xres, yres, channels, TypeDesc::UINT8);
        out->open (filename, spec);
  \end{code}
\item Write the entire image, hiding all details of the encoding of
  image data in the file, whether the file is scanline- or tile-based,
  or what is the native format of data in the file (in this case, our
  in-memory data is unsigned 8-bit and we've requested the same format
  for disk storage, but if they had been different, {\kw write_image()}
  would do all the conversions for us).
  \begin{code}
        out->write_image (TypeDesc::UINT8, &pixels);
  \end{code}
\item Close the file, destroy and free the \ImageOutput we had created,
  and perform all other cleanup and release of any resources needed by
  the plugin.
  \begin{code}
        out->close ();
        delete out;
  \end{code}
\end{itemize}



\section{Advanced Image Output}
\label{sec:imageoutput:advanced}

Let's walk through many of the most common things you might want to do,
but that are more complex than the simple example above.

\subsection{Writing individual scanlines, tiles, and rectangles}
\label{sec:imageoutput:scanlinestiles}

The simple example of Section~\ref{sec:imageoutput:simple} wrote an
entire image with one call.  But sometimes you are generating output a
little at a time and do not wish to retain the entire image in memory
until it is time to write the file.  \product allows you to write images
one scanline at a time, one tile at a time, or by individual rectangles.

\subsubsection{Writing individual scanlines}

Individual scanlines may be written using the \writescanline API
call:

\begin{code}
        ...
        unsigned char scanline[xres*channels];
        out->open (filename, spec);
        int z = 0;   // Always zero for 2D images
        for (int y = 0;  y < yres;  ++y) {
            ... generate data in scanline[0..xres*channels-1] ...
            out->write_scanline (y, z, TypeDesc::UINT8, scanline);
        }
        out->close ();
        ...
\end{code}

The first two arguments to \writescanline specify which scanline is
being written by its vertical ($y$) scanline number (beginning with 0)
and, for volume images, its slice ($z$) number (the slice number should
be 0 for 2D non-volume images).  This is followed by a \TypeDesc
describing the data you are supplying, and a pointer to the pixel data
itself.  Additional optional arguments describe the data stride, which
can be ignored for contiguous data (use of strides is explained in
Section~\ref{sec:imageoutput:strides}).

All \ImageOutput implementations will accept scanlines in strict order
(starting with scanline 0, then 1, up to {\kw yres-1}, without skipping
any).  See Section~\ref{sec:imageoutput:randomrewrite} for details
on out-of-order or repeated scanlines.

The full description of the \writescanline function may be found
in Section~\ref{sec:imageoutput:reference}.

\subsubsection{Writing individual tiles}

Not all image formats (and therefore not all \ImageOutput
implementations) support tiled images.  If the format does not support
tiles, then \writetile will fail.  An application using \product
should gracefully handle the case that tiled output is not available for
the chosen format.

Once you {\kw create()} an \ImageOutput, you can ask if it is capable
of writing a tiled image by using the {\kw supports("tiles")} query:

\begin{code}
        ...
        ImageOutput *out = ImageOutput::create (filename);
        if (! out->supports ("tiles")) {
            // Tiles are not supported
        }
\end{code}

Assuming that the \ImageOutput supports tiled images, you need to
specifically request a tiled image when you {\kw open()} the file.  This
is done by setting the tile size in the \ImageSpec passed
to {\kw open()}.  If the tile dimensions are not set, they will default
to zero, which indicates that scanline output should be used rather than
tiled output.

\begin{code}
        int tilesize = 64;
        ImageSpec spec (xres, yres, channels, TypeDesc::UINT8);
        spec.tile_width = tilesize;
        spec.tile_height = tilesize;
        out->open (filename, spec);
        ...
\end{code}

In this example, we have used square tiles (the same number of pixels
horizontally and vertically), but this is not a requirement of \product.
However, it is possible that some image formats may only support square
tiles, or only certain tile sizes (such as restricting tile sizes to
powers of two).  Such restrictions should be documented by each
individual plugin.

\begin{code}
        unsigned char tile[tilesize*tilesize*channels];
        int z = 0;   // Always zero for 2D images
        for (int y = 0;  y < yres;  y += tilesize) {
            for (int x = 0;  x < xres;  x += tilesize) {
                ... generate data in tile[] ..
                out->write_tile (x, y, z, TypeDesc::UINT8, tile);
            }
        }
        out->close ();
        ...
\end{code}

The first three arguments to \writetile specify which tile is
being written by the pixel coordinates of any pixel contained in the
tile: $x$ (column), $y$ (scanline), and $z$ (slice, which should always
be 0 for 2D non-volume images).  This is followed by a \TypeDesc
describing the data you are supplying, and a pointer to the tile's pixel
data itself, which should be ordered by increasing slice, increasing
scanline within each slice, and increasing column within each scanline.
Additional optional arguments describe the data stride, which can be
ignored for contiguous data (use of strides is explained in
Section~\ref{sec:imageoutput:strides}).

All \ImageOutput implementations that support tiles will accept tiles in
strict order of increasing $y$ rows, and within each row, increasing $x$
column, without missing any tiles.  See
Section~\ref{sec:imageoutput:randomrewrite} for details on out-of-order
or repeated tiles.

The full description of the \writetile function may be found
in Section~\ref{sec:imageoutput:reference}.

\subsubsection{Writing arbitrary rectangles}

Some \ImageOutput implementations --- such as those implementing an
interactive image display, but probably not any that are outputting
directly to a file --- may allow you to send arbitrary rectangular pixel
regions.  Once you {\kw create()} an \ImageOutput, you can ask if it is
capable of accepting arbitrary rectangles by using the {\kw
supports("rectangles")} query:

\begin{code}
        ...
        ImageOutput *out = ImageOutput::create (filename);
        if (! out->supports ("rectangles")) {
            // Rectangles are not supported
        }
\end{code}

If rectangular regions are supported, they may be sent using
the {\kw write_rectangle()} API call:

\begin{code}
        unsigned int rect[...];
        ... generate data in rect[] ..
        out->write_rectangle (xmin, xmax, ymin, ymax, zmin, zmax, TypeDesc::UINT8, rect);
        ...
\end{code}

The first six arguments to {\kw write_rectangle()} specify the region of
pixels that is being transmitted by supplying the minimum and maximum
pixel indices in $x$ (column), $y$ (scanline), and $z$ (slice, always 0
for 2D non-volume images).  The total number of pixels being transmitted
is therefore:
\begin{code}
        (xmax-xmin+1) * (ymax-ymin+1) * (zmax-zmin+1)
\end{code}
\noindent This is followed by a \TypeDesc describing the data you
are supplying, and a pointer to the rectangle's pixel data itself, which
should be ordered by increasing slice, increasing scanline within each
slice, and increasing column within each scanline.  Additional optional
arguments describe the data stride, which can be ignored for contiguous
data (use of strides is explained in
Section~\ref{sec:imageoutput:strides}).


\subsection{Converting data formats}
\label{sec:imageoutput:convertingformats}

The code examples of the previous sections all assumed that your
internal pixel data is stored as unsigned 8-bit integers (i.e., 0-255
range).  But \product is significantly more flexible.  

You may request that the output image be stored in any of several
formats.  This is done by setting the {\kw format} field of the
\ImageSpec prior to calling {\kw open}.  You can do this upon
construction of the \ImageSpec, as in the following example
that requests a spec that stores data as 16-bit unsigned integers:
\begin{code}
        ImageSpec spec (xres, yres, channels, TypeDesc::UINT16);
\end{code}

\noindent Or, for an \ImageSpec that has already been
constructed, you may reset its format using the {\kw set_format()}
method (which also resets the various quantization fields of the
spec to the defaults for the data format you have specified).  

\begin{code}
        ImageSpec spec (...);
        spec.set_format (TypeDesc::UINT16);
\end{code}

Note that resetting the format must be done \emph{before} passing the
spec to {\kw open()}, or it will have no effect on the file.

Individual file formats, and therefore \ImageOutput implementations, may
only support a subset of the formats understood by the \product library.
Each \ImageOutput plugin implementation should document which data
formats it supports.  An individual \ImageOutput implementation may
choose to simply fail to {\kw open()}, though the recommended behavior
is for {\kw open()} to succeed but in fact choose a data format
supported by the file format that best preserves the precision and range
of the originally-requested data format.

It is not required that the pixel data passed to \writeimage,
\writescanline, \writetile, or {\kw write_rectangle()} actually be in
the same data format as that requested as the native format of the file.
You can fully mix and match data you pass to the various {\kw write}
routines and \product will automatically convert from the internal
format to the native file format.  For example, the following code will
open a TIFF file that stores pixel data as 16-bit unsigned integers
(values ranging from 0 to 65535), compute internal pixel values as
floating-point values, with \writeimage performing the conversion
automatically:

\begin{code}
        ImageOutput *out = ImageOutput::create ("myfile.tif");
        ImageSpec spec (xres, yres, channels, TypeDesc::UINT16);
        out->open (filename, spec);
        ...
        float pixels [xres*yres*channels];
        ...
        out->write_image (TypeDesc::FLOAT, pixels);
\end{code}

\noindent Note that \writescanline, \writetile, and {\cf
  write_rectangle} have a parameter that works in a corresponding
manner.

Please refer to Section~\ref{sec:imageoutput:quantization} for more
information on how values are translated among the supported data
formats by default, and how to change the formulas by specifying
quantization in the \ImageSpec.


\subsection{Data Strides}
\label{sec:imageoutput:strides}

In the preceeding examples, we have assumed that the block of data being
passed to the {\cf write} functions are \emph{contiguous}, that is:

\begin{itemize}
\item each pixel in memory consists of a number of data values equal to
  the declared number of channels that are being written to the file;
\item successive column pixels within a row directly follow each other in
  memory, with the first channel of pixel $x$ immediately following
  last channel of pixel $x-1$ of the same row;
\item for whole images, tiles or rectangles, the data for each row
  immediately follows the previous one in memory (the first pixel of row
  $y$ immediately follows the last column of row $y-1$);
\item for 3D volumetric images, the first pixel of slice $z$ immediately
  follows the last pixel of of slice $z-1$.
\end{itemize}

Please note that this implies that data passed to
\writetile be contiguous in the shape of a single tile (not just an
offset into a whole image worth of pixels), and that data passed to {\cf
  write_rectangle()} be contiguous in the dimensions of the rectangle.

The \writescanline function takes an optional {\cf xstride} argument,
and the \writeimage, \writetile, and {\cf write_rectangle} functions
take optional {\cf xstride}, {\cf ystride}, and {\cf zstride} values
that describe the distance, in \emph{bytes}, between successive pixel
columns, rows, and slices, respectively, of the data you are passing.
For any of these values that are not supplied, or are given as the
special constant {\cf AutoStride}, contiguity will be assumed.

By passing different stride values, you can achieve some surprisingly
flexible functionality.  A few representative examples follow:

\begin{itemize}
\item Flip an image vertically upon writing, by using \emph{negative}
  $y$ stride:
  \begin{code}
        unsigned char pixels[xres*yres*channels];
        int scanlinesize = xres * channels * sizeof(pixels[0]);
        ...
        out->write_image (TypeDesc::UINT8,
                          (char *)pixels+(yres-1)*scanlinesize, // offset to last
                          AutoStride,                  // default x stride
                          -scanlinesize,               // special y stride
                          AutoStride);                 // default z stride
  \end{code}
\item Write a tile that is embedded within a whole image of pixel data,
  rather than having a one-tile-only memory layout:
  \begin{code}
        unsigned char pixels[xres*yres*channels];
        int pixelsize = channels * sizeof(pixels[0]);
        int scanlinesize = xres * pixelsize;
        ...
        out->write_tile (x, y, 0, TypeDesc::UINT8,
                         (char *)pixels + y*scanlinesize + x*pixelsize,
                         pixelsize,
                         scanlinesize);
  \end{code}
\item Write only a subset of channels to disk.  In this example, our
  internal data layout consists of 4 channels, but we write just 
  channel 3 to disk as a one-channel image:
  \begin{code}
        // In-memory representation is 4 channel
        const int xres = 640, yres = 480;
        const int channels = 4;  // RGBA
        const int channelsize = sizeof(unsigned char);
        unsigned char pixels[xres*yres*channels];

        // File representation is 1 channel
        ImageOutput *out = ImageOutput::create (filename);
        ImageSpec spec (xres, yres, 1, TypeDesc::UINT8);
        out->open (filename, spec);

        // Use strides to write out a one-channel "slice" of the image
        out->write_image (TypeDesc::UINT8,
                          (char *)pixels+3*channelsize, // offset to chan 3
                          channels*channelsize,         // 4 channel x stride
                          AutoStride,                   // default y stride
                          AutoStride);                  // default z stride
        ...
  \end{code}
\end{itemize}

Please consult Section~\ref{sec:imageoutput:reference} for detailed
descriptions of the stride parameters to each {\cf write} function.


\subsection{Writing a crop window or overscan region}
\label{sec:imageoutput:cropwindows}
\index{crop windows} \index{overscan}

% FIXME -- Marcos suggests adding a figure here to illustrate
% the w/h/d, xyz, full

The \ImageSpec fields {\cf width}, {\cf height}, and {\cf depth}
describe the dimensions of the actual pixel data.

At times, it may be useful to also describe an abstract \emph{full} or
\emph{display} image window, whose position and size may not correspond
exactly to the data pixels.  For example, a pixel data window that is a
subset of the full display window might indicate a \emph{crop window}; a
pixel data window that is a superset of the full display window might
indicate \emph{overscan} regions (pixels defined outside the eventual
viewport).

The \ImageSpec fields {\cf full_width}, {\cf full_height}, and
{\cf full_depth} describe the dimensions of the full display
window, and {\cf full_x}, {\cf full_y}, {\cf full_z} describe its
origin (upper left corner).  The fields {\cf x}, {\cf y}, {\cf z}
describe the origin (upper left corner)
of the pixel data.

These fields collectively describe an abstract full display image
ranging from [{\cf full_x} ... {\cf full_x+full_width-1}] horizontally,
[{\cf full_y} ... {\cf full_y+full_height-1}] vertically,
and [{\cf full_z} ... {\cf full_z+full_depth-1}] in depth (if it is
a 3D volume), and actual pixel data over the pixel coordinate range 
[{\cf x} ... {\cf x+width-1}] horizontally,
[{\cf y} ... {\cf y+height-1}] vertically,
and [{\cf z} ... {\cf z+depth-1}] in depth (if it is a volume).

Not all image file formats have a way to describe display windows.  An
\ImageOutput implementation that cannot express display windows will
always write out the {\cf width} $\times$ {\cf height} pixel data, may
upon writing lose information about offsets or crop windows.

Here is a code example that opens an image file that will contain a $32
\times 32$ pixel crop window within an abstract $640 \times 480$ full
size image.  Notice that the pixel indices (column, scanline, slice)
passed to the {\cf write} functions are the coordinates relative to
the full image, not relative to the crop widow, but the data pointer
passed to the {\cf write} functions should point to the beginning of
the actual pixel data being passed (not the the hypothetical start of
the full data, if it was all present).

\begin{code}
        int fullwidth = 640, fulllength = 480; // Full display image size
        int cropwidth = 16, croplength = 16;  // Crop window size
        int xorigin = 32, yorigin = 128;      // Crop window position
        unsigned char pixels [cropwidth * croplength * channels]; // Crop size!
        ...
        ImageOutput *out = ImageOutput::create (filename);
        ImageSpec spec (cropwidth, croplength, channels, TypeDesc::UINT8);
        spec.full_x = 0;
        spec.full_y = 0;
        spec.full_width = fullwidth;
        spec.full_length = fulllength;
        spec.x = xorigin;
        spec.y = yorigin;
        out->open (filename, spec);
        ...
        int z = 0;   // Always zero for 2D images
        for (int y = yorigin;  y < yorigin+croplength;  ++y) {
            out->write_scanline (y, z, TypeDesc::UINT8,
                                 (y-yorigin)*cropwidth*channels);
        }
        out->close ();
\end{code}


\subsection{Writing metadata}
\label{sec:imageoutput:metadata}

The \ImageSpec passed to {\cf open()} can specify all the common
required properties that describe an image: data format, dimensions,
number of channels, tiling.  However, there may be a variety of
additional \emph{metadata}\footnote{\emph{Metadata} refers to data about
data, in this case, data about the image that goes beyond the pixel
values and description thereof.} that should be carried along with the
image or saved in the file.  

The remainder of this section explains how to store additional metadata
in the \ImageSpec.  It is up to the \ImageOutput to store these
in the file, if indeed the file format is able to accept the data.
Individual \ImageOutput implementations should document which metadata
they respect.

\subsubsection{Channel names}

In addition to specifying the number of color channels, it is also
possible to name those channels.  Only a few \ImageOutput
implementations have a way of saving this in the file, but some do, so
you may as well do it if you have information about what the channels
represent.

By convention, channel names for red, green, blue, and alpha (or a main
image) should be named \qkw{R}, \qkw{G}, \qkw{B}, and \qkw{A},
respectively.  Beyond this guideline, however, you can use any names you
want.

The \ImageSpec has a vector of strings called {\cf
  channelnames}.  Upon construction, it starts out with reasonable
default values.  If you use it
at all, you should make sure that it contains the same number of strings
as the number of color channels in your image.  Here is an example:

\begin{code}
        int channels = 4;
        ImageSpec spec (width, length, channels, TypeDesc::UINT8);
        spec.channelnames.clear ();
        spec.channelnames.push_back ("R");
        spec.channelnames.push_back ("G");
        spec.channelnames.push_back ("B");
        spec.channelnames.push_back ("A");
\end{code}

Here is another example in which custom channel names are used to 
label the channels in an 8-channel image containing beauty pass
RGB, per-channel opacity, and texture $s,t$ coordinates for each pixel.

\begin{code}
        int channels = 8;
        ImageSpec spec (width, length, channels, TypeDesc::UINT8);
        spec.channelnames.clear ();
        spec.channelnames.push_back ("R");
        spec.channelnames.push_back ("G");
        spec.channelnames.push_back ("B");
        spec.channelnames.push_back ("opacityR");
        spec.channelnames.push_back ("opacityG");
        spec.channelnames.push_back ("opacityB");
        spec.channelnames.push_back ("texture_s");
        spec.channelnames.push_back ("texture_t");
\end{code}

The main advantage to naming color channels is that if you are saving to
a file format that supports channel names, then any application that
uses \product to read the image back has the option to retain those
names and use them for helpful purposes.  For example, the {\cf iv}
image viewer will display the channel names when viewing individual
channels or displaying numeric pixel values in ``pixel view'' mode.


\subsubsection{Specially-designated channels}

The \ImageSpec contains two fields, {\cf alpha_channel} and {\cf
  z_channel}, which can be used to designate which channel indices are
used for alpha and $z$ depth, if any.  Upon construction, these are both
set to {\cf -1}, indicating that it is not known which channels 
are alpha or depth.  Here is an example of setting up a 5-channel output
that represents RGBAZ:

\begin{code}
        int channels = 5;
        ImageSpec spec (width, length, channels, format);
        spec.channelnames.push_back ("R");
        spec.channelnames.push_back ("G");
        spec.channelnames.push_back ("B");
        spec.channelnames.push_back ("A");
        spec.channelnames.push_back ("Z");
        spec.alpha_channel = 3;
        spec.z_channel = 4;
\end{code}

There are two advantages to designating the alpha and depth channels in
this manner:  
\begin{itemize}
\item Some file formats may require that these channels be stored in a
  particular order, with a particular precision, or the \ImageOutput may
  in some other way need to know about these special channels.
\item Certain operations that make sense for colors should not apply to
  alpha or $z$.  For example, if your call to {\cf write} reduces
  precision (e.g., converts from {\cf float} to integer pixels) it will
  typically add random \emph{dither} to eliminate banding artifacts
  in the quantization.  But for a variety of reasons, you want to add
  dither only to color channels and not to alpha.  So setting {\cf
    alpha_channel} will cause {\cf write} to not dither that channel.
\end{itemize}

\subsubsection{Linearity hints}

We certainly hope that you are using only modern file formats that
support high precision and extended range pixels (such as OpenEXR) and
keeping all your images in a linear color space.  But you may have to
work with file formats that dictate the use of nonlinear color values.
This is prevalent in formats that store pixels only as 8-bit values,
since 256 values are not enough to linearly represent colors without
banding artifacts in the dim values.

Since this can (and probably will) happen, the \ImageSpec has
fields that allow you to explain what color space your image pixels are
in.  Each individual \ImageOutput should document how it uses this (or
not).

The \ImageSpec field {\cf linearity} can take on any of the
following values:
\begin{description}
\item[\halfspc \rm \kw{ImageSpec::UnknownLinearity}] the default,
  indicates that you have made no claim about the color space of your
  pixel data.
\item[\halfspc \rm \kw{ImageSpec::Linear}] indicates that the pixel
  values you are passing repesent linear values.
\item[\halfspc \rm \kw{ImageSpec::GammaCorrected}] indicates that the
  color pixel values (but not alpha or $z$) that you are passing have
  already been gamma corrected (raised to the power $1/\gamma$), and
  that the gamma exponent may be found in the {\cf gamma} field of the
  \ImageSpec.
\item[\halfspc \rm \kw{ImageSpec::sRGB}] indicates that the color pixel
  values that you are passing are already in sRGB color space.
\item[\halfspc \rm \kw{ImageSpec::AdobeRGB}] indicates that the color pixel
  values that you are passing are already in Adobe RGB color space.
\item[\halfspc \rm \kw{ImageSpec::Rec709}] indicates that the color pixel
  values that you are passing are already in Rec709 color space.
\item[\halfspc \rm \kw{ImageSpec::KodakLog}] indicates that the color pixel
  values that you are passing are already in Kodak logarithmic color space.
\end{description}

\noindent Here is a simple example of setting up the \ImageSpec
when you know that the pixel values you are writing are linear:

\begin{code}
        ImageSpec spec (width, length, channels, format);
        spec.linearity = ImageSpec::Linear;
        ...
\end{code}

If a particular \ImageOutput implementation is required (by the rules of
the file format it writes) to have pixels in a particular color space,
then it will convert the color values of your image to the right color
space if it is not already in that space.  For example, JPEG images
must be in sRGB space, so if you declare your pixels to be {\kw Linear},
the JPEG \ImageOutput will convert to sRGB.

If you leave the linearity set to the default of {\cf UnknownLinearity},
the values will not be transformed, since the plugin can't be sure that
it's not in the correct space to begin with.  

The linearity only describes color channels.  An \ImageOutput plugin
will assume that alpha or depth ($z$) channels (designated by the {\cf
  alpha_channel} and {\cf z_channel} fields, respectively) always
represent linear values and should never be transformed.


\subsubsection{Arbitrary metadata}

For all other metadata that you wish to save in the file, you can attach
the data to the \ImageSpec using the {\cf attribute()} methods.
These come in polymorphic varieties that allow you to attach an
attribute name and a value consisting of a single {\cf int}, {\cf
  unsigned int}, {\cf float}, {\cf char*}, or {\cf std::string}, as
shown in the following examples:

\begin{code}
        ImageSpec spec (...);
        ...

        unsigned int u = 1;
        spec.attribute ("Orientation", u);

        float x = 72.0;
        spec.attribute ("dotsize", f);

        std::string s = "Fabulous image writer 1.0";
        spec.attribute ("Software", s);
\end{code}

These are convenience routines for metadata that consist of a single
value of one of these common types.  For other data types, or more
complex arrangements, you can use the more general form of {\cf
  attribute()}, which takes arguments giving the name, type (as a
\TypeDesc), number of values (1 for a single value, $>1$ for an
  array), and then a pointer to the data values.  For example,

\begin{code}
        ImageSpec spec (...);

        // Attach a 4x4 matrix to describe the camera coordinates
        float mymatrix[16] = { ... };
        spec.attribute ("worldtocamera", TypeDesc::TypeMatrix, &mymatrix);

        // Attach an array of two floats giving the CIE neutral color
        float neutral[2] = { ... };
        spec.attribute ("adoptedNeutral", TypeDesc(TypeDesc::FLOAT, 2), &neutral);
\end{code}

In general, most image file formats (and therefore most \ImageOutput
implementations) are aware of only a small number of name/value pairs
that they predefine and will recognize.  Some file formats (OpenEXR,
notably) do accept arbitrary user data and save it in the image file.
If an \ImageOutput does not recognize your metadata and does not support
arbitrary metadata, that metadatum will be silently ignored and will not
be saved with the file.

Each individual \ImageOutput implementation should document the names,
types, and meanings of all metadata attributes that they understand.


\subsection{Controlling quantization}
\label{sec:imageoutput:quantization}

It is possible that your internal data format (that in which you compute
pixel values that you pass to the {\cf write} functions) is of greater
precision or range than the native data format of the output file.  This
can occur either because you specified a lower-precision data format in
the \ImageSpec that you passed to {\cf open()}, or else that the
image file format dictates a particular data format that does not match
your internal format.  For example, you may compute {\cf float} pixels
and pass those to {\cf write_image()}, but if you are writing a
JPEG/JFIF file, the values must be stored in the file as 8-bit unsigned
integers.

The conversion from floating-point formats to integer formats (or from
higher to lower integer, which is done by first converting to float) is
controlled by five fields within the \ImageSpec: {\cf
  quant_black}, {\cf quant_white}, {\cf quant_min}, {\cf quant_max},
and {\cf quant_dither}.
Float 0.0 maps to the integer value given by {\cf quant_black}, and
float 1.0 maps to the integer value given by {\cf quant_white}.  Then,
for color channels only (not alpha or depth), a random amount is added
in the range ({\cf -quant_dither..quant_dither}), in order to reduce
banding artifacts.  The result is then clamped to lie within the range of
{\cf quant_min} and {\cf quant_max}, inclusive.  Finally, this result is
truncated its integer value for final output.  Here is the code that
implements this transformation ({\cf T} is the final output integer
type):

\begin{code}
        float value = quant_black * (1 - input) + quant_white * input;
        if (it's a color channel)
            value += quant_dither * (2 * random() - 1);
        T output = (T) clamp ((int)(value + 0.5), quant_min, quant_max);
\end{code}

The values of the quantization parameters are set in one of three ways:
(1) upon construction of the \ImageSpec, they are set to the
default quantization values for the given data format; (2) upon call to
{\cf ImageSpec::set_format()}, the quantization values are set
to the defaults for the given data format; (3) or, after being first set
up in this manner, you may manually change the quantization parameters
in the \ImageSpec, if you want something other than the default
quantization.

\noindent Default quantization for each integer type is as follows:\\

\smallskip
\begin{tabular}{|l|r|r|r|r|r|}
\hline
{\bf Data Format} & {\bf black} & {\bf white} & {\bf min} & {\bf max} & {\bf
  dither} \\
\hline
{\cf UINT8}  & 0 &        255 &     0 & 255 & 0.5 \\
{\cf INT8}   & 0 &        127 &  -128 & 127 & 0.5 \\
{\cf UINT16} & 0 &      65535 &     0 & 65535 & 0.5 \\
{\cf INT16}  & 0 &      32767 & -32768 & 32767 & 0.5 \\
{\cf UINT}   & 0 & 4294967295 & 0 & 4294967295 & 0.5 \\
{\cf INT}    & 0 & 2147483647 & -2147483648 & 2147483647 & 0.5 \\
\hline
{\cf FLOAT} & & & & & \\
{\cf HALF} & 0 & 1 & N/A & N/A & 0 \\
{\cf DOUBLE} & & & & & \\
\hline
\end{tabular} \\
\smallskip

\noindent Note that the default is to use the entire positive range
of each integer type to represent the floating-point (0..1) range.
Floating-point types do not attempt to remap values, do not add dither,
and do not clamp (except to their full floating-point range).

The default will almost always be what you want.  But just as an
example, here's how you would specify a quantization for a 16-bit file
in which 1.0 maps to 16383 (14 bits of positive range) rather than
filling the full 16 bit:

\begin{code}
        ImageSpec spec (width, length, channels, TypeDesc::UINT16);
        spec.quant_black  = 0;
        spec.quant_white  = 16383;
        spec.quant_min    = 0;
        spec.quant_max    = 16383;
        spec.quant_dither = 0.5;
\end{code}


\subsection{Random access and repeated transmission of pixels}
\label{sec:imageoutput:randomrewrite}

All \ImageOutput implementations that support scanlines and tiles should write pixels in strict
order of increasing $z$ slice, increasing $y$ scanlines/rows within each
slice, and increasing $x$ column within each row.  It is generally not
safe to skip scanlines or tiles, or transmit them out of order, unless
the plugin specifically advertises that it supports random access or
rewrites, which may be queried using:

\begin{code}
        ImageOutput *out = ImageOutput::create (filename);
        if (out->supports ("random_access"))
            ...
\end{code}

\noindent Similarly, you should assume the plugin will not correctly
handle repeated transmissions of a scanline or tile that has already
been sent, unless it advertises that it supports rewrites, which may be
queried using:

\begin{code}
        if (out->supports ("rewrite"))
            ...
\end{code}


\subsection{Multi-image files and MIP-maps}
\label{sec:imageoutput:multiimage}
\label{sec:imageoutput:mipmap}

Some image file formats support multiple discrete subimages to be stored
in one file, and/or multiple resolutions (MIP-map levels).  Given a
created \ImageOutput, you can query whether multiple images may be
stored in the file:

\begin{code}
        ImageOutput *out = ImageOutput::create (filename);
        if (out->supports ("multiimage"))
            ...
        if (out->supports ("mipmap"))
            ...
\end{code}

If you are working with an \ImageOutput that supports multiple images
or MIP-map levels,
it is easy to write these images.  All you have to do is, after writing
all the pixels of one image but before calling {\cf close()}, call {\cf
  open()} again for the next subimage or MIP level and passing the
appropriate value for the optional third
\emph{mode} argument.  (See
Section~\ref{sec:imageoutput:reference} for the full technical
description of the arguments to {\cf open()}.)  The {\cf close()}
routine is called just once, after all subimages and MIP levels are completed.

Below is pseudocode for writing a MIP-map (a multi-resolution image
used for texture mapping):

\begin{code}
        const char *filename = "foo.tif";
        const int xres = 512, yres = 512;
        const int channels = 3;  // RGB
        unsigned char *pixels = new unsigned char [xres*yres*channels];

        // Create the ImageOutput
        ImageOutput *out = ImageOutput::create (filename);

        // Be sure we can support either mipmaps or subimages
        if (! out->supports ("mipmap") && ! out->supports ("multiimage")) {
            std::cerr << "Cannot write a MIP-map\n";
            delete out;
            return;
        }
        // Set up spec for the highest resolution
        ImageSpec spec (xres, yres, channels, TypeDesc::UINT8);

        // Use Create mode for the first level.
        ImageOutput::OpenMode appendmode = ImageOutput::Create;

        // Write images, halving every time, until we're down to
        // 1 pixel in either dimension
        while (spec.width >= 1 && spec.height >= 1) {
            out->open (filename, spec, mode);
            out->write_image (TypeDesc::UINT8, pixels);
            // Assume halve() resamples the image to half resolution
            halve (pixels, spec.width, spec.height);
            // Don't forget to change spec for the next iteration
            spec.width /= 2;
            spec.height /= 2;
            // For subsequent levels, change the append mode argument to
            // open().  If the format doesn't support MIPmaps directly,
            // try to emulate it with subimages.
            if (out->supports("mipmap"))
                appendmode = ImageOutput::AppendMIPLevel;
            else
                appendmode = ImageOutput::AppendSubimage;
        }
        out->close ();
        delete out;
\end{code}

In this example, we have used \writeimage, but of course \writescanline,
\writetile, and {\cf write_rectangle()} work as you would expect, on the
current subimage.

\subsection{Copying an entire image}
\label{sec:imageoutput:copyimage}

Suppose you want to copy an image, perhaps with alterations to the 
metadata but not to the pixels.  You could open an \ImageInput and
perform a {\cf read_image()}, and open another \ImageOutput and
call {\cf write_image()} to output the pixels from the input image.
However, for compressed images, this may be inefficient due to the
unnecessary decompression and subsequent re-compression.  In addition,
if the compression is \emph{lossy}, the output image may not contain
pixel values identical to the original input.

A special {\cf copy_image} method of \ImageOutput is available that
attempts to copy an image from an open \ImageInput (of the same
format) to the output as efficiently as possible with without altering
pixel values, if at all possible.

Not all format plugins will provide an implementation of {\cf
  copy_image} (in fact, most will not), but the default implemenatation
simply copies pixels one scanline or tile at a time (with
decompression/recompression) so it's still safe to call.  Furthermore,
even a provided {\cf copy_image} is expected to fall back on the default
implementation if the input and output are not able to do an efficient
copy.  Nevertheless, this method is recommended
for copying images so that maximal advantage will be taken in cases
where savings can be had.

The following is an example use of {\cf copy_image} to transfer pixels
without alteration while modifying the image description metadata:

\begin{code}
    // Open the input file
    const char *input = "input.jpg";
    ImageInput *in = ImageInput::create (input);
    ImageSpec in_spec;
    in->open (input, in_spec);

    // Make an output spec, identical to the input except for metadata
    ImageSpec out_spec = in_spec;
    out_spec.attribute ("ImageDescription", "My Title");

    // Create the output file and copy the image
    const char *output = "output.jpg";
    ImageOutput *out = ImageOutput::create (output);
    out->open (output, out_spec);
    out->copy_image (in);

    // Clean up
    out->close ();
    delete out;
    in->close ();
    delete in;
\end{code}


\subsection{Custom search paths for plugins}
\label{sec:imageoutput:searchpaths}

When you call {\cf ImageOutput::create()}, the \product library will try
to find a plugin that is able to write the format implied by your
filename.  These plugins are alternately known as DLL's on Windows (with
the {\cf .dll} extension), DSO's on Linux (with the {\cf .so}
extension), and dynamic libraries on Mac OS X (with the {\cf .dylib}
extension).  

\product will look for matching plugins according to
\emph{search paths}, which are strings giving a list of directories to
search, with each directory separated by a colon (`{\cf :}').  Within
a search path, any
substrings of the form {\cf \$\{FOO\}} will be replaced
by the value of environment variable {\cf FOO}.  For
example, the searchpath \qkw{\$\{HOME\}/plugins:/shared/plugins}
will first check the directory \qkw{/home/tom/plugins} (assuming the
user's home directory is {\cf /home/tom}), and if not
found there, will then check the directory \qkw{/shared/plugins}.

The first search path it will check is that stored in the environment
variable {\cf IMAGEIO_LIBRARY_PATH}.  It will check each directory in
turn, in the order that they are listed in the variable.  If no adequate
plugin is found in any of the directories listed in this environment
variable, then it will check the custom searchpath passed as the
optional second argument to {\cf ImageOutput::create()}, searching in
the order that the directories are listed.  Here is an example:

\begin{code} 
        char *mysearch = "/usr/myapp/lib:${HOME}/plugins";
        ImageOutput *out = ImageOutput::create (filename, mysearch);
        ...
\end{code} % $


\subsection{Error checking}
\label{sec:imageoutput:errors}

Nearly every \ImageOutput API function returns a {\cf bool} indicating
whether the operation succeeded ({\cf true}) or failed ({\cf false}).
In the case of a failure, the \ImageOutput will have saved an error
message describing in more detail what went wrong, and the latest
error message is accessible using the \ImageOutput method 
{\cf geterror()}, which returns the message as a {\cf std::string}.

The exception to this rule is {\cf ImageOutput::create}, which returns
{\cf NULL} if it could not create an appropriate \ImageOutput.  And in
this case, since no \ImageOutput exists for which you can call its {\cf
  geterror()} function, there exists a global {\cf geterror()}
function (in the {\cf OpenImageIO} namespace) that retrieves the latest
error message resulting from a call to {\cf create}.

Here is another version of the simple image writing code from
Section~\ref{sec:imageoutput:simple}, but this time it is fully 
elaborated with error checking and reporting:

\begin{code}
        #include "imageio.h"
        using namespace OpenImageIO;
        ...

        const char *filename = "foo.jpg";
        const int xres = 640, yres = 480;
        const int channels = 3;  // RGB
        unsigned char pixels[xres*yres*channels];

        ImageOutput *out = ImageOutput::create (filename);
        if (! out) {
            std::cerr << "Could not create an ImageOutput for " 
                      << filename << ", error = " 
                      << OpenImageIO::geterror() << "\n";
            return;
        }
        ImageSpec spec (xres, yres, channels, TypeDesc::UINT8);

        if (! out->open (filename, spec)) {
            std::cerr << "Could not open " << filename 
                      << ", error = " << out->geterror() << "\n";
            delete out;
            return;
        }

        if (! out->write_image (TypeDesc::UINT8, pixels)) {
            std::cerr << "Could not write pixels to " << filename 
                      << ", error = " << out->geterror() << "\n";
            delete out;
            return;
        }

        if (! out->close ()) {
            std::cerr << "Error closing " << filename 
                      << ", error = " << out->geterror() << "\n";
            delete out;
            return;
        }

        delete out;
\end{code}



\section{\ImageOutput Class Reference}
\label{sec:imageoutput:reference}

\apiitem{static ImageOutput * {\ce create} (const std::string \&filename, \\
\bigspc\bigspc\spc const std::string \&plugin_searchpath="")}

Create an \ImageOutput that can be used to write an image file.  The
type of image file (and hence, the particular subclass of \ImageOutput
returned, and the plugin that contains its methods) is inferred from the
extension of the file name.  The {\kw plugin_searchpath} parameter is a
colon-separated list of directories to search for \product plugin
DSO/DLL's.

\apiend

\apiitem{const char * {\ce format_name} ()}
Returns the canonical name of the format that this \ImageOutput
instance is capable of writing.
\apiend

\apiitem{bool {\ce supports} (const std::string \&feature)}
\label{sec:supportsfeaturelist}
Given the name of a \emph{feature}, tells if this \ImageOutput 
instance supports that feature.  The following features are recognized
by this query:
\begin{description}
\item[\spc] \spc 
\item[\rm \qkw{tiles}] Is this plugin able to write tiled images?
\item[\rm \qkw{rectangles}] Can this plugin accept arbitrary rectangular
  pixel regions (via {\kw write_rectangle()})?  False indicates that
  pixels must be transmitted via \writescanline (if
  scanline-oriented) or \writetile (if tile-oriented, and only if
  {\kw supports("tiles")} returns true).
\item[\rm \qkw{random_access}] May tiles or scanlines be written in any
  order?  False indicates that they must be in successive order.
\item[\rm \qkw{multiimage}] Does this format support multiple subimages
  within a single file?
\item[\rm \qkw{mipmap}] Does this format support resolutions per
  image/subimage (MIP-map levels)?
\item[\rm \qkw{volumes}] Does this format support ``3D'' pixel arrays
  (a.k.a.\ volume images)?
\item[\rm \qkw{rewrite}] Does this plugin allow the same scanline or
  tile to be sent more than once?  Generally this is true for plugins
  that implement some sort of interactive display, rather than a saved
  image file.
\item[\rm \qkw{empty}] Does this plugin support passing a NULL data
  pointer to the various {\kw write} routines to indicate that the
  entire data block is composed of pixels with value zero.  Plugins
  that support this achieve a speedup when passing blank scanlines or
  tiles (since no actual data needs to be transmitted or converted).
\end{description}

\noindent This list of queries may be extended in future releases.
Since this can be done simply by recognizing new query strings, and does
not require any new API entry points, addition of support for new
queries does not break ``link compatibility'' with previously-compiled
plugins.
\apiend

\apiitem{bool {\ce open} (const std::string \&name, const ImageSpec \&newspec,\\
\bigspc  OpenMode mode=Create)}

Open the file with given {\kw name}, with resolution, and other format
data as given in {\kw newspec}.  This function returns {\kw true} for
success, {\kw false} for failure.  Note that it is legal to call 
{\kw open()} multiple times on the same file without a call to
{\kw close()}, if it supports multiimage and {\kw mode} is 
{\kw AppendSubimage}, or if it supports MIP-maps and {\kw mode} is 
{\kw AppendMIPlevel} -- this is interpreted as appending a subimage, or
a MIP level to the current subimage, respectively.

\apiend

\apiitem{const ImageSpec \& {\ce spec} ()}
Returns the spec internally associated with this currently open
\ImageOutput.
\apiend

\apiitem{bool {\ce close} ()}
Closes the currently open file associated with this \ImageOutput
and frees any memory or resources associated with it.
\apiend

\apiitem{bool {\ce write_scanline} (int y, int z, TypeDesc format,
     const void *data, \\
\bigspc stride_t xstride=AutoStride)}

Write a full scanline that includes pixels $(*,y,z)$.  For 2D non-volume
images, $z$ is ignored.  The {\kw xstride} value gives the distance
between successive pixels (in bytes).  Strides set to the special value
{\kw AutoStride} imply contiguous data, i.e., \\ \spc {\kw xstride} $=$
{\kw spec.nchannels*format.size()} \\ This method
automatically converts the data from the specified {\kw format} to the
actual output format of the file.  Return {\kw true} for success, {\kw
  false} for failure.  It is a failure to call \writescanline with an
out-of-order scanline if this format driver does not support random
access.

\apiend

\apiitem{bool {\ce write_tile} (int x, int y, int z, TypeDesc format,
                             const void *data, \\ \bigspc stride_t xstride=AutoStride,
                             stride_t ystride=AutoStride, \\ \bigspc stride_t zstride=AutoStride)}

Write the tile with $(x,y,z)$ as the upper left corner.  For 2D
non-volume images, $z$ is ignored.  The three stride values give the
distance (in bytes) between successive pixels, scanlines, and volumetric
slices, respectively.  Strides set to the special value {\kw AutoStride}
imply contiguous data, i.e., \\
\spc {\kw xstride} $=$ {\kw spec.nchannels*format.size()} \\
\spc {\kw ystride} $=$ {\kw xstride*spec.tile_width} \\
\spc {\kw zstride} $=$ {\kw ystride*spec.tile_height} \\
This method automatically converts the
data from the specified {\kw format} to the actual output format of the
file.  Return {\kw true} for success, {\kw false} for failure.  It is a
failure to call \writetile with an out-of-order tile if this
format driver does not support random access.

\apiend

\apiitem{bool {\ce write_rectangle} ({\small int xmin, int xmax, int ymin, int ymax,
                                  int zmin, int zmax,} \\ \bigspc TypeDesc format,
                                  const void *data, \\ \bigspc stride_t xstride=AutoStride,
                                  stride_t ystride=AutoStride, \\
                                  \bigspc stride_t zstride=AutoStride)}

Write pixels whose $x$ coords range over {\kw xmin}...{\kw xmax}
(inclusive), $y$ coords over {\kw ymin}...{\kw ymax}, and $z$ coords
over {\kw zmin}...{\kw zmax}.  The three stride values give the distance
(in bytes) between successive pixels, scanlines, and volumetric slices,
respectively.  Strides set to the special value {\kw AutoStride} imply
contiguous data, i.e.,\\
\spc {\kw xstride} $=$ {\kw spec.nchannels*format.size()} \\
\spc {\kw ystride} $=$ {\kw xstride*(xmax-xmin+1)} \\
\spc {\kw zstride} $=$ {\kw ystride*(ymax-ymin+1)}\\
This method automatically converts the data from the specified 
{\kw format} to the actual output format of the fil.  Return {\kw true}
for success, {\kw false} for failure.  It is a failure to call 
{\kw write_rectangle} for a format plugin that does not return true for
{\kw supports("rectangles")}.

\apiend

\apiitem{bool {\ce write_image} (TypeDesc format, const void *data, \\
                              \bigspc stride_t xstride=AutoStride, stride_t ystride=AutoStride,
                              \\ \bigspc stride_t zstride=AutoStride, \\
                              \bigspc ProgressCallback progress_callback=NULL,\\
                              \bigspc void *progress_callback_data=NULL)}

Write the entire image of {\kw spec.width} $\times$ {\kw spec.height}
$\times$ {\kw spec.depth}
pixels, with the given strides and in the desired format.
Strides set to the special value {\kw AutoStride} imply contiguous data,
i.e., \\
\spc {\kw xstride} $=$ {\kw spec.nchannels * format.size()} \\
\spc {\kw ystride} $=$ {\kw xstride * spec.width} \\
\spc {\kw zstride} $=$ {\kw ystride * spec.height}\\
The function will internally either call \writescanline or 
\writetile, depending on whether the file is scanline- or
tile-oriented.

Because this may be an expensive operation, a progress callback may be passed.
Periodically, it will be called as follows:
\begin{code}
        progress_callback (progress_callback_data, float done)
\end{code}
\noindent where \emph{done} gives the portion of the image 
(between 0.0 and 1.0) that has been written thus far.

\apiend

\apiitem{bool {\ce copy_image} (ImageInput *in)}

Read the current subimage of {\cf in}, and write it as the next subimage
of {\cf *this}, in a way that is efficient and does not alter pixel
values, if at all possible.  Both {\cf in} and {\cf this} must be a
properly-opened \ImageInput and \ImageOutput, respectively, and their
current images must match in size and number of channels.  Return {\cf true}
if it works ok, {\cf false} if for some reason the operation wasn't possible.

If a particular \ImageOutput implementation does not supply a
{\cf copy_image} method, it will inherit the default implementation,
which is to simply read scanlines or tiles from {\cf in} and write
them to {\cf *this}.  However, some format implementations may have a
special technique for directly copying raw pixel data from the
input to the output, when both input and output are the same
file type and the same data format.  This can be more efficient 
than {\cf in->read_image} followed by {\cf out->write_image}, and avoids any
unintended pixel alterations, especially for formats that use
lossy compression.
\apiend

\apiitem{int {\ce send_to_output} (const char *format, ...)}
General message passing between client and image output server.
This is currently undefined and is reserved for future use.
\apiend

\apiitem{int {\ce send_to_client} (const char *format, ...)}
General message passing between client and image output server.
This is currently undefined and is reserved for future use.
\apiend

\apiitem{std::string {\ce geterror} ()}
Returns the current error string describing what went wrong if
any of the public methods returned {\kw false} indicating an error.
(Hopefully the implementation plugin called {\kw error()} with a
helpful error message.)
\apiend



\index{Image I/O API|)}

\chapwidthend

\chapter{Image I/O: Reading Images}
\label{chap:imageinput}
\index{Image I/O API|(}


\section{Image Input Made Simple}
\label{sec:imageinput:simple}

Here is the simplest sequence required to open an image file, find
out its resolution, and read the pixels (converting them into
8-bit values in memory, even if that's not the way they're stored in the file):

\begin{code}
        #include <OpenImageIO/imageio.h>
        OIIO_NAMESPACE_USING
        ...

        const char *filename = "foo.jpg";
        int xres, yres, channels;
        unsigned char *pixels;

        ImageInput *in = ImageInput::create (filename);
        if (! in)
            return;
        ImageSpec spec;
        in->open (filename, spec);
        xres = spec.width;
        yres = spec.height;
        channels = spec.nchannels;
        pixels = new unsigned char [xres*yres*channels];
        in->read_image (TypeDesc::UINT8, pixels);
        in->close ();
        delete in;
\end{code}

\noindent Here is a breakdown of what work this code is doing:

\begin{itemize}
\item Search for an ImageIO plugin that is capable of reading the file
  (\qkw{foo.jpg}), first by trying to deduce the correct plugin from the
  file extension, but if that fails, by opening every ImageIO plugin it
  can find until one will open the file without error.  When it finds
  the right plugin, it creates a subclass instance of \ImageInput that
  reads the right kind of file format.
  \begin{code}
        ImageInput *in = ImageInput::create (filename);
  \end{code}
\item Open the file, read the header, and put all relevant metadata
  about the file in a specification structure.
  \begin{code}
        ImageSpec spec;
        in->open (filename, spec);
  \end{code}
\item The specification contains vital information such as the
  dimensions of the image, number of color channels, and data type of
  the pixel values.  This is enough to allow us to allocate enough space
  for the image.
  \begin{code}
        xres = spec.width;
        yres = spec.height;
        channels = spec.nchannels;
        pixels = new unsigned char [xres*yres*channels];
  \end{code}
  Note that in this example, we don't care what data format is used for
  the pixel data in the file --- we allocate enough space for unsigned
  8-bit integer pixel values, and will rely on \product's ability to
  convert to our requested format from the native data format of the
  file.
\item Read the entire image, hiding all details of the encoding of image
  data in the file, whether the file is scanline- or tile-based, or what
  is the native format of the data in the file (in this case, we request
  that it be automatically converted to unsigned 8-bit integers).
  \begin{code}
        in->read_image (TypeDesc::UINT8, pixels);
  \end{code}
\item Close the file, destroy and free the \ImageInput we had created,
  and perform all other cleanup and release of any resources used by
  the plugin.
  \begin{code}
        in->close ();
        delete in;
  \end{code}
\end{itemize}



\section{Advanced Image Input}
\label{sec:advancedimageinput}

Let's walk through some of the most common things you might want to do,
but that are more complex than the simple example above.


\subsection{Reading individual scanlines and tiles}
\label{sec:imageinput:scanlinestiles}

The simple example of Section~\ref{sec:imageinput:simple} read an
entire image with one call.  But sometimes you want to read a large
image a
little at a time and do not wish to retain the entire image in memory
as you process it.  \product allows you to read images
one scanline at a time or one tile at a time.

Examining the \ImageSpec reveals whether the file is scanline or
tile-oriented: a scanline image will have {\cf spec.tile_width} 
and {\cf spec.tile_height} set to 0, whereas a tiled images will
have nonzero values for the tile dimensions.


\subsubsection{Reading scanlines}

Individual scanlines may be read using the \readscanline API
call:

\begin{code}
        ...
        in->open (filename, spec);
        if (spec.tile_width == 0) {
            unsigned char *scanline = new unsigned char [spec.width*spec.channels];
            for (int y = 0;  y < yres;  ++y) {
                in->read_scanline (y, 0, TypeDesc::UINT8, scanline);
                ... process data in scanline[0..width*channels-1] ...
            }
            delete [] scanline;
        } else {
            ... handle tiles, or reject the file ...
        }
        in->close ();
        ...
\end{code}

The first two arguments to \readscanline specify which scanline
is being read by its vertical ($y$) scanline number (beginning with 0)
and, for volume images, its slice ($z$) number (the slice number should
be 0 for 2D non-volume images).  This is followed by a \TypeDesc
describing the data type of the pixel buffer you are supplying, and a
pointer to the pixel buffer itself.  Additional optional arguments
describe the data stride, which can be ignored for contiguous data (use
of strides is explained in Section~\ref{sec:imageinput:strides}).

Nearly all \ImageInput implementations will be most efficient reading
scanlines in strict order (starting with scanline 0, then 1, up to {\kw
  yres-1}, without skipping any).  An \ImageInput is required to accept
\readscanline requests in arbitrary order, but depending on the file
format and reader implementation, out-of-order scanline reads may be
inefficient.

There is also a {\cf read_scanlines()} function that operates similarly,
except that it takes a {\cf ybegin} and {\cf yend} that specify a range,
reading all scanlines {\cf ybegin} $\le y <$ {\cf yend}.  For most image
format readers, this is implemented as a loop over individual scanlines,
but some image format readers may be able to read a contiguous block of
scanlines more efficiently than reading each one individually.

The full descriptions of the \readscanline and {\cf read_scanlines()}
functions may be found in Section~\ref{sec:imageinput:reference}.

\subsubsection{Reading tiles}

Once you {\kw open()} an image file, you can find out if it is a tiled
image (and the tile size) by examining the \ImageSpec's {\cf
  tile_width}, {\cf tile_height}, and {\cf tile_depth} fields.
If they are zero, it's a scanline image and you should read pixels
using \readscanline, not \readtile.

\begin{code}
        ...
        in->open (filename, spec);
        if (spec.tile_width == 0) {
            ... read by scanline ...
        } else {
            // Tiles
            int tilesize = spec.tile_width * spec.tile_height;
            unsigned char *tile = new unsigned char [tilesize * spec.channels];
            for (int y = 0;  y < yres;  y += spec.tile_height) {
                for (int x = 0;  x < xres;  x += spec.tile_width) {
                    in->read_tile (x, y, 0, TypeDesc::UINT8, tile);
                    ... process the pixels in tile[] ..
                }
            }
            delete [] tile;
        }
        in->close ();
        ...
\end{code}

The first three arguments to \readtile specify which tile is
being read by the pixel coordinates of any pixel contained in the
tile: $x$ (column), $y$ (scanline), and $z$ (slice, which should always
be 0 for 2D non-volume images).  This is followed by a \TypeDesc
describing the data format of the pixel buffer you are supplying, and a
pointer to the pixel buffer.  Pixel data will be written to your buffer
in order of increasing slice, increasing
scanline within each slice, and increasing column within each scanline.
Additional optional arguments describe the data stride, which can be
ignored for contiguous data (use of strides is explained in
Section~\ref{sec:imageinput:strides}).

All \ImageInput implementations are required to support reading tiles in
arbitrary order (i.e., not in strict order of increasing $y$ rows, and
within each row, increasing $x$ column, without missing any tiles).

The full description of the \readtile function may be found
in Section~\ref{sec:imageinput:reference}.


\subsection{Converting formats}
\label{sec:imageinput:convertingformat}

The code examples of the previous sections all assumed that your
internal pixel data is stored as unsigned 8-bit integers (i.e., 0-255
range).  But \product is significantly more flexible.  

You may request that the pixels be stored in any of several formats.
This is done merely by passing the {\cf read} function the data type
of your pixel buffer, as one of the enumerated type \TypeDesc.

%FIXME
%Individual file formats, and therefore \ImageInput implementations, may
%only support a subset of the formats understood by the \product library.
%Each \ImageInput plugin implementation should document which data
%formats it supports.  An individual \ImageInput implementation may
%choose to simply fail open {\kw open()}, though the recommended behavior
%is for {\kw open()} to succeed but in fact choose a data format
%supported by the file format that best preserves the precision and range
%of the originally-requested data format.

It is not required that the pixel data buffer passed to \readimage,
\readscanline, or \readtile actually be in the same data format as the
data in the file being read.  \product will automatically convert from
native data type of the file to the internal data format of your choice.
For example, the following code will open a TIFF and read pixels into
your internal buffer represented as {\cf float} values.  This will work
regardless of whether the TIFF file itself is using 8-bit, 16-bit, or
float values.

\begin{code}
        ImageInput *in = ImageInput::create ("myfile.tif");
        ImageSpec spec;
        in->open (filename, spec);
        ...
        int numpixels = spec.width * spec.height;
        float pixels = new float [numpixels * channels];
        ...
        in->read_image (TypeDesc::FLOAT, pixels);
\end{code}

\noindent Note that \readscanline and \readtile have a parameter that
works in a corresponding manner.

You can, of course, find out the native type of the file simply by
examining {\cf spec.format}.  If you wish, you may then allocate a
buffer big enough for an image of that type and request the native type
when reading, therefore eliminating any translation among types and
seeing the actual numerical values in the file.

%FIXME
%Please refer to Section~\ref{sec:imageinput:quantization} for more
%information on how values are translated among the supported data
%formats by default, and how to change the formulas by specifying
%quantization in the \ImageSpec.


\subsection{Data Strides}
\label{sec:imageinput:strides}

In the preceeding examples, we have assumed that the buffer passed to
the {\cf read} functions (i.e., the place where you want your pixels
to be stored) is \emph{contiguous}, that is:

\begin{itemize}
\item each pixel in memory consists of a number of data values equal to
  the number of channels in the file;
\item successive column pixels within a row directly follow each other in
  memory, with the first channel of pixel $x$ immediately following
  last channel of pixel $x-1$ of the same row;
\item for whole images or tiles, the data for each row
  immediately follows the previous one in memory (the first pixel of row
  $y$ immediately follows the last column of row $y-1$);
\item for 3D volumetric images, the first pixel of slice $z$ immediately
  follows the last pixel of of slice $z-1$.
\end{itemize}

Please note that this implies that \readtile will write pixel data into
your buffer so that it is contiguous in the shape of a single tile, not
just an offset into a whole image worth of pixels.

The \readscanline function takes an optional {\cf xstride} argument, and
the \readimage and \readtile functions take optional {\cf xstride}, 
{\cf ystride}, and {\cf zstride} values that describe the distance, in
\emph{bytes}, between successive pixel columns, rows, and slices,
respectively, of your pixel buffer.  For any of these values that are
not supplied, or are given as the special constant {\cf AutoStride},
contiguity will be assumed.

By passing different stride values, you can achieve some surprisingly
flexible functionality.  A few representative examples follow:

\begin{itemize}
\item Flip an image vertically upon reading, by using \emph{negative}
  $y$ stride:
  \begin{code}
        unsigned char pixels[spec.width * spec.height * spec.nchannels];
        int scanlinesize = spec.width * spec.nchannels * sizeof(pixels[0]);
        ...
        in->read_image (TypeDesc::UINT8,
                        (char *)pixels+(yres-1)*scanlinesize, // offset to last
                        AutoStride,                  // default x stride
                        -scanlinesize,               // special y stride
                        AutoStride);                 // default z stride
  \end{code}
\item Read a tile into its spot in a buffer whose layout matches
  a whole image of pixel data,
  rather than having a one-tile-only memory layout:
  \begin{code}
        unsigned char pixels[spec.width * spec.height * spec.nchannels];
        int pixelsize = spec.nchannels * sizeof(pixels[0]);
        int scanlinesize = xpec.width * pixelsize;
        ...
        in->read_tile (x, y, 0, TypeDesc::UINT8,
                       (char *)pixels + y*scanlinesize + x*pixelsize,
                       pixelsize,
                       scanlinesize);
  \end{code}
\end{itemize}

Please consult Section~\ref{sec:imageinput:reference} for detailed
descriptions of the stride parameters to each {\cf read} function.


\subsection{Reading metadata}
\label{sec:imageinput:metadata}

The \ImageSpec that is filled in by {\cf ImageInput::open()}
specifies all the common properties that describe an image: data format,
dimensions, number of channels, tiling.  However, there may be a variety
of additional \emph{metadata} that are present in the image file and
could be queried by your application.

The remainder of this section explains how to query additional metadata
in the \ImageSpec.  It is up to the \ImageInput to read these
from the file, if indeed the file format is able to carry additional
data.  Individual \ImageInput implementations should document which
metadata they read.

\subsubsection{Channel names}

In addition to specifying the number of color channels, the
\ImageSpec also stores the names of those channels in its {\cf
  channelnames} field, which is a {\cf vector<std::string>}.  Its length
should always be equal to the number of channels (it's the
responsibility of the \ImageInput to ensure this).

Only a few file formats (and thus \ImageInput implementations) have a
way of specifying custom channel names, so most of the time you will see
that the channel names follow the default convention of being named
\qkw{R}, \qkw{G}, \qkw{B}, and \qkw{A}, for red, green, blue, and alpha,
respectively.

Here is example code that prints the names of the channels in an image:

\begin{code}
        ImageInput *in = ImageInput::create (filename);
        ImageSpec spec;
        in->open (filename, spec);
        for (int i = 0;  i < spec.nchannels;  ++i)
            std::cout << "Channel " << i << " is " 
                      << spec.channelnames[i] << "\n";
\end{code}

\subsubsection{Specially-designated channels}

The \ImageSpec contains two fields, {\cf alpha_channel} and {\cf
  z_channel}, which designate which channel numbers represent alpha and
$z$ depth, if any.  If either is set to {\cf -1}, it indicates that it
is not known which channel is used for that data.

If you are doing something special with alpha or depth, it is probably
safer to respect the {\cf alpha_channel} and {\cf z_channel}
designations (if not set to {\cf -1}) rather than merely assuming that,
for example, channel 3 is always the alpha channel.

\subsubsection{Arbitrary metadata}

All other metadata found in the file will be stored in the
\ImageSpec's {\cf extra_attribs} field, which is a 
\ParamValueList, which is itself essentially a vector of
\ParamValue instances.  Each \ParamValue
stores one meta-datum consisting of a name, type (specified by 
a \TypeDesc), number of values, and data pointer.

If you know the name of a specific piece of metadata you want to use,
you can find it using the {\cf ImageSpec::find_attribute()}
method, which returns a pointer to the matching \ParamValue,
or {\cf NULL} if no match was found.  An optional \TypeDesc
argument can narrow the search to only parameters that match the
specified type as well as the name.  Below is an
example that looks for orientation information, expecting it to consist 
of a single integer:

\begin{code}
        ImageInput *in = ImageInput::create (filename);
        ImageSpec spec;
        in->open (filename, spec);
        ...
        ParamValue *p = spec.find_attribute ("Orientation", TypeDesc::INT);
        if (p) {
            int orientation = * (int *) p->data();
        } else {
            std::cout << "No integer orientation in the file\n";
        }
\end{code}

By convention, \ImageInput plugins will save all integer metadata as
32-bit integers ({\cf TypeDesc::INT} or {\cf TypeDesc::UINT}), even if the file format
dictates that a particular item is stored in the file as a 8- or 16-bit
integer.  This is just to keep client applications from having to deal
with all the types.  Since there is relatively little metadata compared
to pixel data, there's no real memory waste of promoting all integer
types to int32 metadata.  Floating-point metadata and string metadata
may also exist, of course.

It is also possible to step through all the metadata, item by item.
This can be accomplished using the technique of the following example:

\begin{code}
        for (size_t i = 0;  i < spec.extra_attribs.size();  ++i) {
            const ParamValue &p (spec.extra_attribs[i]);
            printf ("    \%s: ", p.name.c_str());
            if (p.type() == TypeDesc::STRING)
                printf ("\"\%s\"", *(const char **)p.data());
            else if (p.type() == TypeDesc::FLOAT)
                printf ("\%g", *(const float *)p.data());
            else if (p.type() == TypeDesc::INT)
                printf ("\%d", *(const int *)p.data());
            else if (p.type() == TypeDesc::UINT)
                printf ("\%u", *(const unsigned int *)p.data());
            else
                printf ("<unknown data type>");
            printf ("\n");
        }
\end{code}

Each individual \ImageInput implementation should document the names,
types, and meanings of all metadata attributes that they understand.

\subsubsection{Color space hints}

We certainly hope that you are using only modern file formats that
support high precision and extended range pixels (such as OpenEXR) and
keeping all your images in a linear color space.  But you may have to
work with file formats that dictate the use of nonlinear color values.
This is prevalent in formats that store pixels only as 8-bit values,
since 256 values are not enough to linearly represent colors without
banding artifacts in the dim values.

The {\cf ImageSpec::extra_attribs} field may store metadata that reveals
the color space the image file in the \qkw{oiio:ColorSpace}
attribute, which may take on any of the following values:

\begin{description}
\item[\halfspc \rm \qkw{Linear}] indicates that the
  color pixel values are known to be linear.
\item[\halfspc \rm \qkw{GammaCorrected}] indicates
  that the color pixel values (but not alpha or $z$) have
  already been gamma corrected (raised to the power $1/\gamma$), and
  that the gamma exponent may be found in the \qkw{oiio:Gamma} metadata.
\item[\halfspc \rm \qkw{sRGB}] indicates that the
  color pixel values are in sRGB color space.
\item[\halfspc \rm \qkw{AdobeRGB}] indicates that the
  color pixel values are in Adobe RGB color space.
\item[\halfspc \rm \qkw{Rec709}] indicates that the
  color pixel values are in Rec709 color space.
\item[\halfspc \rm \qkw{KodakLog}] indicates that the
  color pixel values are in Kodak logarithmic color space.
\end{description}

The \ImageInput sets the \qkw{oiio:ColorSpace} metadata in a
purely advisory capacity --- the {\cf read} will not convert pixel
values among color spaces.  Many image file formats only support
nonlinear color spaces (for example, JPEG/JFIF dictates use of sRGB).
So your application should intelligently deal with gamma-corrected and
sRGB input, at the very least.

The color space hints only describe color channels.  You should assume that
alpha or depth ($z$) channels (designated by the {\cf alpha_channel} and
{\cf z_channel} fields, respectively) always represent linear values and
should never be transformed by your application.


%\subsection{Controlling quantization and encoding}
%\label{sec:imageinput:quantization}
%
%FIXME


%\subsection{Random access and repeated transmission of pixels}
%\label{sec:imageinput:randomrepeated}
%
%FIXME


\subsection{Multi-image files and MIP-maps}
\label{sec:imageinput:multiimage}
\label{sec:imageinput:mipmap}

Some image file formats support multiple discrete subimages to be stored
in one file, and/or miltiple resolutions for each image to form a
MIPmap.  When you {\cf open()} an \ImageInput, it will by default point
to the first (i.e., number 0) subimage in the file, and the highest
resolution (level 0) MIP-map level.  You can switch to viewing another
subimage or MIP-map level using the {\cf seek_subimage()} function:

\begin{code}
        ImageInput *in = ImageInput::create (filename);
        ImageSpec spec;
        in->open (filename, spec);
        ...
        int subimage = 1;
        int miplevel = 0;
        if (in->seek_subimage (subimage, miplevel, spec)) {
            ...
        } else {
            ... no such subimage/miplevel ...
        }
\end{code}

The {\cf seek_subimage()} function takes three arguments: the index of
the subimage to switch to (starting with 0), the MIPmap level (starting
with 0 for the highest-resolution level), and a reference to an
\ImageSpec, into which will be stored the spec of the new
subimage/miplevel.  The {\cf seek_subimage()} function returns {\cf
  true} upon success, and {\cf false} if no such subimage or MIP level
existed.  It is legal to visit subimages and MIP levels out of order;
the \ImageInput is responsible for making it work properly.  It is also
possible to find out which subimage and MIP level is currently being
viewed, using the {\cf current_subimage()} and {\cf current_miplevel()}
functions, which return the index of the current subimage and MIP
levels, respectively.

Below is pseudocode for reading all the levels of a MIP-map (a
multi-resolution image used for texture mapping) that shows how to read
multi-image files:

\begin{code}
        ImageInput *in = ImageInput::create (filename);
        ImageSpec spec;
        in->open (filename, spec);

        int num_miplevels = 0;
        while (in->seek_subimage (0, num_miplevels, spec)) {
            // Note: spec has the format of the current subimage/miplevel
            int npixels = spec.width * spec.height;
            int nchannels = spec.nchannels;
            unsigned char *pixels = new unsigned char [npixels * nchannels];
            in->read_image (TypeDesc::UINT8, pixels);

            ... do whatever you want with this level, in pixels ...

            delete [] pixels;
            ++num_miplevels;
        }
        // Note: we break out of the while loop when seek_subimage fails
        // to find a next MIP level.

        in->close ();
        delete in;
\end{code}

In this example, we have used \readimage, but of course \readscanline
and \readtile work as you would expect, on the current subimage and MIP
level.


\subsection{Per-channel formats}
\label{sec:imageinput:channelformats}

Some image formats allow separate per-channel data formats (for example,
{\cf half} data for colors and {\cf float} data for depth).  If you want
to read the pixels in their true native per-channel formats,
the following steps are necessary:

\begin{enumerate}
\item Check the \ImageSpec's {\cf channelformats} vector.  If non-empty,
  the channels in the file do not all have the same format.
\item When calling {\cf read_scanline}, {\cf read_scanlines},
  {\cf read_tile}, {\cf read_tiles}, or {\cf read_image}, 
  pass a format of {\cf TypeDesc::UNKNOWN} to indicate that
  you would like the raw data in native per-channel format of the file
  written to your {\cf data} buffer.
\end{enumerate}

For example, the following code fragment will read a 5-channel image
to an OpenEXR file, consisting of R/G/B/A channels in {\cf half} and
a Z channel in {\cf float}:

\begin{code}
        ImageSpec spec;
        ImageInput *in = ImageInput::create (filename);
        in->open (filename, spec);

        // Allocate enough space
        unsigned char *pixels = new unsigned char [spec.image_bytes(true)];

        in->read_image (TypeDesc::UNKNOWN, /* use native channel formats */
                        pixels);           /* data buffer */

        if (spec.channelformats.size() > 0) {
            ... the buffer contains packed data in the native 
                per-channel formats ...
        } else {
            ... the buffer contains all data per spec.format ...
        }
\end{code}


\subsection{Custom search paths for plugins}
\label{sec:imageinput:searchpaths}

Please see Section~\ref{sec:imageoutput:searchpaths} for discussion
about search paths for finding plugins that implement \ImageOutput.

In a similar fashion, calls to {\cf ImageOutput::create()}
will search for plugins in each directory listed in the environment
variable {\cf OIIO_LIBRARY_PATH}, in the order that they are listed.
If no adequate plugin is found, then it will check the custom searchpath
passed as the optional second argument to {\cf ImageInput::create()}.
Here is an example:

\begin{code}
        char *mysearch = "/usr/myapp/lib:${HOME}/plugins";
        ImageInput *in = ImageInput::create (filename, mysearch);
        ...
\end{code} %$


\subsection{Error checking}
\label{sec:imageinput:errors}
\index{error checking}

Nearly every \ImageInput API function returns a {\cf bool} indicating
whether the operation succeeded ({\cf true}) or failed ({\cf false}).
In the case of a failure, the \ImageInput will have saved an error
message describing in more detail what went wrong, and the latest
error message is accessible using the \ImageInput method 
{\cf geterror()}, which returns the message as a {\cf std::string}.

The exception to this rule is {\cf ImageInput::create}, which returns
{\cf NULL} if it could not create an appropriate \ImageInput.  And in
this case, since no \ImageInput exists for which you can call its {\cf
  geterror()} function, there exists a global {\cf geterror()}
function (in the {\cf OpenImageIO} namespace) that retrieves the latest
error message resulting from a call to {\cf create}.

Here is another version of the simple image reading code from
Section~\ref{sec:imageinput:simple}, but this time it is fully
elaborated with error checking and reporting:

\begin{code}
        #include <OpenImageIO/imageio.h>
        OIIO_NAMESPACE_USING
        ...

        const char *filename = "foo.jpg";
        int xres, yres, channels;
        unsigned char *pixels;

        ImageInput *in = ImageInput::create (filename);
        if (! in) {
            std::cerr << "Could not create an ImageInput for " 
                      << filename << ", error = " 
                      << OpenImageIO::geterror() << "\n";
            return;
        }

        ImageSpec spec;
        if (! in->open (filename, spec)) {
            std::cerr << "Could not open " << filename 
                      << ", error = " << in->geterror() << "\n";
            delete in;
            return;
        }
        xres = spec.width;
        yres = spec.height;
        channels = spec.nchannels;
        pixels = new unsigned char [xres*yres*channels];

        if (! in->read_image (TypeDesc::UINT8, pixels)) {
            std::cerr << "Could not read pixels from " << filename 
                      << ", error = " << in->geterror() << "\n";
            delete in;
            return;
        }

        if (! in->close ()) {
            std::cerr << "Error closing " << filename 
                      << ", error = " << in->geterror() << "\n";
            delete in;
            return;
        }
        delete in;
\end{code}


\newpage
\section{\ImageInput Class Reference}
\label{sec:imageinput:reference}

\apiitem{ImageInput * {\ce create} (const std::string \&filename, \\
\bigspc\bigspc   const std::string \&plugin_searchpath="")}
Create and return an \ImageInput implementation that is able
to read the given file.  The {\kw plugin_searchpath} parameter is a
colon-separated list of directories to search for \product plugin
DSO/DLL's (not a searchpath for the image itself!).  This will
actually just try every ImageIO plugin it can locate, until it
finds one that's able to open the file without error.  This just
creates the \ImageInput, it does not open the file.
\apiend

\apiitem{const char * {\ce format_name} (void) const}
Return the name of the format implemented by this class.
\apiend

\apiitem{bool {\ce open} (const std::string \&name, ImageSpec \&newspec)}
Opens the file with given name and seek to the first subimage in the
file.  Various file attributes are put in
{\kw newspec} and a copy is also saved internally to the
\ImageInput (retrievable via {\kw spec()}.  From examining
{\kw newspec} or {\kw spec()}, you can discern the resolution, if it's
tiled, number of channels, native data format, and other metadata about
the image.  Return {\kw true} if the file was found and opened okay,
otherwise {\kw false}.
\apiend

\apiitem{bool {\ce open} (const std::string \&name, ImageSpec \&newspec,\\
\bigspc  const ImageSpec \&config)}

Opens the file with given name, similarly to {\cf open(name, newspec)}.
However, in this version, any non-default fields of {\cf config},
including metadata, will be taken to be configuration requests,
preferences, or hints.  The default implementation of 
{\cf open (name, newspec, config)} will simply ignore {\cf config} and
calls the usual {\cf open (name, newspec)}.  But a plugin may choose to
implement this version of {\cf open} and respond in some way to the
configuration requests.  Supported configuration requests should be
documented by each plugin.
\apiend

\apiitem {const ImageSpec \& {\ce spec} (void) const}
Returns a reference to the image format specification of the
current subimage.  Note that the contents of the spec are
invalid before {\kw open()} or after {\kw close()}.
\apiend

\apiitem{bool {\ce close} ()}
Closes an open image.
\apiend


\apiitem{int {\ce current_subimage} (void) const}
Returns the index of the subimage that is currently being read.
The first subimage (or the only subimage, if there is just one) is
number 0.
\apiend


\apiitem{bool {\ce seek_subimage} (int subimage, int miplevel, ImageSpec \&newspec)}

Seek to the given subimage and MIP-map level within the open image file.
The first subimage in the file has index 0, and for each subimage, the
highest-resolution MIP level has index 0.  Return {\kw true} on success,
{\kw false} on failure (including that there is not a subimage or MIP
level with those indices).  The new subimage's vital statistics are put
in {\kw newspec} (and also saved internally in a way that can be
retrieved via {\kw spec()}).  The \ImageInput is expected to give the
appearance of random access to subimages and MIP levels --- in other
words, if it can't randomly seek to the given subimage or MIP level, it
should transparently close, reopen, and sequentially read through prior
subimages and levels.

\apiend

\apiitem{bool {\ce read_scanline} (int y, int z, TypeDesc format, void *data,\\
  \bigspc\spc\spc                      stride_t xstride=AutoStride)}

Read the scanline that includes pixels $(*,y,z)$ into {\kw data}
($z=0$ for non-volume images),
converting if necessary from the native data format of the file into the
{\kw format} specified.
If {\cf format} is {\cf TypeDesc::UNKNOWN}, the data will be preserved 
in its native format (including per-channel formats, if applicable).
The {\kw xstride}
value gives the data spacing of adjacent pixels (in bytes).  Strides set
to the special value {\kw AutoStride} imply contiguous data, i.e., \\
  \spc {\kw xstride} $=$ {\kw spec.nchannels * spec.pixel_size()} \\
The \ImageInput is expected to give the appearance of random access
--- in other words, if it can't randomly seek to the given scanline, it
should transparently close, reopen, and sequentially read through prior
scanlines.  The base \ImageInput class has a default implementation
that calls {\kw read_native_scanline()} and then does appropriate format
conversion, so there's no reason for each format plugin to override this
method.
\apiend

\apiitem{bool {\ce read_scanline} (int y, int z, float *data)}
This simplified version of {\kw read_scanline()} reads to contiguous 
float pixels.
\apiend

\apiitem{bool {\ce read_scanlines} (int ybegin, int yend, int z,\\
  \bigspc TypeDesc format, void *data,\\
  \bigspc stride_t xstride=AutoStride, stride_t ystride=AutoStride) \\
bool {\ce read_scanlines} (int ybegin, int yend, int z,\\
  \bigspc int firstchan, int nchans, TypeDesc format, void *data,\\
  \bigspc                      stride_t xstride=AutoStride, stride_t ystride=AutoStride)}

Read all the scanlines that include pixels $(*,y,z)$, where
$\mathit{ybegin} \le y < \mathit{yend}$, into {\kw data}.  This is 
essentially identical to \readscanline, except that can read more than
one scanline at a time, which may be more efficient for certain image
format readers.

The version that specifies a channel range will read only
channels $[${\cf firstchan},{\cf firstchan+nchans}$)$ into the buffer.
\apiend


\apiitem{bool {\ce read_tile} (int x, int y, int z, TypeDesc format,
                            void *data, \\ \bigspc stride_t xstride=AutoStride,
                            stride_t ystride=AutoStride, \\ \bigspc stride_t
                            zstride=AutoStride)}
Read the tile whose upper-left origin is $(x,y,z)$ into {\kw data}
($z=0$ for non-volume images),
converting if necessary from the native data format of the file into the 
{\kw format} specified.
If {\cf format} is {\cf TypeDesc::UNKNOWN}, the data will be preserved 
in its native format (including per-channel formats, if applicable).
The stride values
give the data spacing of adjacent pixels, scanlines, and volumetric
slices, respectively (measured in bytes).  Strides set to the special
value of {\kw AutoStride} imply contiguous data, i.e., \\
\spc {\kw xstride} $=$ {\kw spec.nchannels * spec.pixel_size()} \\
\spc {\kw ystride} $=$ {\kw xstride * spec.tile_width} \\
\spc {\kw zstride} $=$ {\kw ystride * spec.tile_height} \\
The \ImageInput is expected to give the appearance of random access
--- in other words, if it can't randomly seek to the given tile, it
should transparently close, reopen, and sequentially read through prior
tiles.  The base \ImageInput class has a default implementation
that calls {\cf read_native_tile()} and then does appropriate format conversion,
so there's no reason for each format plugin to override this method.

This function returns {\cf true} if it successfully reads the tile,
otherwise {\cf false} for a failure.
The call will fail if the image is not tiled, or if $(x,y,z)$ is not
actually a tile boundary.
\apiend


\apiitem{bool {\ce read_tile} (int x, int y, int z, float *data)}
Simple version of {\kw read_tile} that reads to contiguous float pixels.
\apiend


\apiitem{bool {\ce read_tiles} (int xbegin, int xend, int ybegin, int
  yend, \\ \bigspc int zbegin, int zend, TypeDesc format,
                            void *data, \\ \bigspc stride_t xstride=AutoStride,
                            stride_t ystride=AutoStride, \\ \bigspc stride_t
                            zstride=AutoStride) \\
bool {\ce read_tiles} (int xbegin, int xend, int ybegin, int yend, \\
 \bigspc int zbegin, int zend, int firstchan, int nchans,\\
 \bigspc TypeDesc format, void *data, \\ 
 \bigspc stride_t xstride=AutoStride, stride_t ystride=AutoStride, \\
 \bigspc stride_t zstride=AutoStride)}
Read the tiles bounded by {\kw xbegin} $\le x <$ {\kw xend},
{\kw ybegin} $\le y <$ {\kw yend}, {\kw zbegin} $\le z <$ {\kw zend}
into {\kw data}
converting if necessary from the file's native data format into
the specified buffer {\kw format}.
If {\cf format} is {\cf TypeDesc::UNKNOWN}, the data will be preserved 
in its native format (including per-channel formats, if applicable).
The stride values
give the data spacing of adjacent pixels, scanlines, and volumetric
slices, respectively (measured in bytes).  Strides set to the special
value of {\kw AutoStride} imply contiguous data, i.e., \\
\spc {\kw xstride} $=$ {\kw spec.nchannels * spec.pixel_size()} \\
\spc {\kw ystride} $=$ {\kw xstride * spec.tile_width} \\
\spc {\kw zstride} $=$ {\kw ystride * spec.tile_height} \\
The \ImageInput is expected to give the appearance of random access
--- in other words, if it can't randomly seek to the given tile, it
should transparently close, reopen, and sequentially read through prior
tiles.  The base \ImageInput class has a default implementation
that calls {\cf read_native_tiles()} and then does appropriate format conversion,
so there's no reason for each format plugin to override this method.

This function returns {\cf true} if it successfully reads the tiles,
otherwise {\cf false} for a failure.
The call will fail if the image is not tiled, or if the pixel ranges
do not fall along tile (or image) boundaries, or if it is not a valid
tile range.

The version that specifies a channel range will read only
channels $[${\cf firstchan},{\cf firstchan+nchans}$)$ into the buffer.
\apiend


\apiitem{bool {\ce read_image} (TypeDesc format, void *data, \\
                             \bigspc stride_t xstride=AutoStride,
                             stride_t ystride=AutoStride, \\
                             \bigspc stride_t zstride=AutoStride, \\
                             \bigspc ProgressCallback progress_callback=NULL,\\
                             \bigspc void *progress_callback_data=NULL)}

Read the entire image of {\kw spec.width * spec.height * spec.depth}
pixels into data (which must already be sized large enough for
the entire image) with the given strides, converting into the desired
data format.  
If {\cf format} is {\cf TypeDesc::UNKNOWN}, the data will be preserved 
in its native format (including per-channel formats, if applicable).
This function will automatically handle either tiles or scanlines in
the file.

Strides set to the special value of {\kw AutoStride} imply contiguous
data, i.e., \\
\spc {\kw xstride} $=$ {\kw spec.nchannels * pixel_size()} \\
\spc {\kw ystride} $=$ {\kw xstride * spec.width} \\
\spc {\kw zstride} $=$ {\kw ystride * spec.height} \\
The function will internally either call {\kw read_scanlines} or 
{\kw read_tiles}, depending on whether the file is scanline- or
tile-oriented.

Because this may be an expensive operation, a progres callback may be passed.
Periodically, it will be called as follows:\\
\begin{code}
    progress_callback (progress_callback_data, float done)
\end{code}
\noindent where \emph{done} gives the portion of the image 
(between 0.0 and 1.0) that has been read thus far.
\apiend

\apiitem{bool {\ce read_image} (float *data)}
Simple version of {\kw read_image()} reads to contiguous float pixels.
\apiend

\apiitem{bool {\ce read_native_scanline} (int y, int z, void *data)}
The {\kw read_native_scanline()} function is just like {\kw
  read_scanline()}, except that it keeps the data in the native format
of the disk file and always reads into contiguous memory (no strides).
It's up to the user to have enough space allocated and know what to do
with the data.  IT IS EXPECTED THAT EACH FORMAT PLUGIN WILL OVERRIDE
THIS METHOD.
\apiend

\apiitem{bool {\ce read_native_scanlines} (int ybegin, int yend, int z, void *data)}
The {\kw read_native_scanlines()} function is just like 
{\cf read_native_scanline}, except that it reads
a range of scanlines rather than only one scanline.  It is not necessary
for format plugins to override this method --- a default implementation
in the \ImageInput base class simply calls {\cf read_native_scanline}
for each scanline in the range.  But format plugins may optionally
override this method if there is a way to achieve higher performance by
reading multiple scanlines at once.
\apiend

\apiitem{bool {\ce read_native_scanlines} (int ybegin, int yend, int z,
\\ \bigspc  int firstchan, int nchans, void *data)}
A variant of {\cf read_native_scanlines} that reads only a subset of 
channels \\ $[${\cf firstchan},{\cf firstchan+nchans}$)$.  
If a format reader subclass does
not override this method, the default implementation will simply
call the all-channel version of {\cf read_native_scanlines} into a
temporary buffer and copy the subset of channels.
\apiend

\apiitem{bool {\ce read_native_tile} (int x, int y, int z, void *data)}
The {\kw read_native_tile()} function is just like {\kw read_tile()}, 
except that it keeps the data in the native format of the disk file and
always read into contiguous memory (no strides).  It's up to the user to
have enough space allocated and know what to do with the data.  IT IS
EXPECTED THAT EACH FORMAT PLUGIN WILL OVERRIDE THIS METHOD IF IT
SUPPORTS TILED IMAGES.
\apiend

\apiitem{bool {\ce read_native_tiles} (int xbegin, int xend, int ybegin,
  int yend, \\ \bigspc int zbegin, int zend, void *data)}
The {\kw read_native_tiles()} function is just like {\kw read_tiles()}, 
except that it keeps the data in the native format of the disk file and
always read into contiguous memory (no strides).  
If a format reader does not override this method, the default
implementation it will simply be a loop calling read_native_tile
for each tile in the block.
\apiend

\apiitem{bool {\ce read_native_tiles} (int xbegin, int xend, int ybegin,
  int yend, \\ \bigspc int zbegin, int zend, int firstchan, int nchans, void *data)}
A variant of {\kw read_native_tiles()} that reads only a subset of 
channels \\ $[${\cf firstchan},{\cf firstchan+nchans}$)$.  
If a format reader subclass does
not override this method, the default implementation will simply
call the all-channel version of {\cf read_native_tiles} into a
temporary buffer and copy the subset of channels.
\apiend

\apiitem{int {\ce send_to_input} (const char *format, ...)}
General message passing between client and image input server.
This is currently undefined and is reserved for future use.
\apiend

\apiitem{int {\ce send_to_client} (const char *format, ...)}
General message passing between client and image input server.
This is currently undefined and is reserved for future use.
\apiend

\apiitem{std::string {\ce geterror} () const}
\index{error checking}
Returns the current error string describing what went wrong if
any of the public methods returned {\kw false} indicating an error.
(Hopefully the implementation plugin called {\kw error()} with a
helpful error message.)
\apiend



\index{Image I/O API|)}

\chapwidthend

\chapter{Writing ImageIO Plugins}
\label{chap:writingplugins}



\section{Plugin Introduction}
\label{sec:pluginintro}

As explained in Chapters~\ref{chap:imageinput} and
\ref{chap:imageoutput}, the ImageIO library does not know how to read or
write any particular image formats, but rather relies on plugins located
and loaded dynamically at run-time.  This set of plugins, and therefore
the set of image file formats that \product or its clients can read and
write, is extensible without needing to modify \product itself.  

This chapter explains how to write your own \product plugins.  We will
first explain separately how to write image file readers and writers,
then tie up the loose ends of how to build the plugins themselves.

\section{Image Readers}
\label{sec:pluginreaders}

A plugin that reads a particular image file format must implement a
\emph{subclass} of \ImageInput (described in
Chapter~\ref{chap:imageinput}).  This is actually very straightforward
and consists of the following steps, which we will illustrate with a
real-world example of writing a JPEG/JFIF plug-in.

\begin{enumerate}
\item Read the base class definition from {\fn imageio.h}.  It may also
  be helpful to enclose the contents of your plugin in the same
  namespace that the \product library uses:

  \begin{code}
    #include <OpenImageIO/imageio.h>
    OIIO_PLUGIN_NAMESPACE_BEGIN

    ... everything else ...

    OIIO_PLUGIN_NAMESPACE_END
  \end{code}

\item Declare three public items:

  \begin{enumerate}
    \item An integer called \emph{name}{\cf _imageio_version} that identifies
      the version of the ImageIO protocol implemented by the plugin,
      defined in {\fn imageio.h} as the constant {\cf OIIO_PLUGIN_VERSION}.
      This allows the library to be sure it is not loading a plugin
      that was compiled against an incompatible version of \product.
    \item A function named \emph{name}{\cf _input_imageio_create} that
      takes no arguments and returns a new instance of your \ImageInput
      subclass.  (Note that \emph{name} is the name of your format,
      and must match the name of the plugin itself.)
    \item An array of {\cf char *} called \emph{name}{\cf _input_extensions}
      that contains the list of file extensions that are likely to indicate
      a file of the right format.  The list is terminated by a {\cf NULL}
      pointer.
  \end{enumerate}

  All of these items must be inside an `{\cf extern "C"}' block in order
  to avoid name mangling by the C++ compiler, and we provide handy
  macros {\cf OIIO_PLUGIN_EXPORTS_BEGIN} and {\cf OIIO_PLUGIN_EXPORTS_END}
  to make this easy.  Depending on your
  compiler, you may need to use special commands to dictate that the
  symbols will be exported in the DSO; we provide a special {\cf
  OIIO_EXPORT} macro for this purpose, defined in {\fn export.h}.

  Putting this all together, we get the following for our JPEG example:

  \begin{code}
    OIIO_PLUGIN_EXPORTS_BEGIN
        OIIO_EXPORT int jpeg_imageio_version = OIIO_PLUGIN_VERSION;
        OIIO_EXPORT JpgInput *jpeg_input_imageio_create () {
            return new JpgInput;
        }
        OIIO_EXPORT const char *jpeg_input_extensions[] = {
            "jpg", "jpe", "jpeg", NULL
        };
    OIIO_PLUGIN_EXPORTS_END
  \end{code}

\item The definition and implementation of an \ImageInput subclass for
  this file format.  It must publicly inherit \ImageInput, and must
  overload the following methods which are ``pure virtual'' in the
  \ImageInput base class:

  \begin{enumerate}
    \item {\cf format_name()} should return the name of the format, which
      ought to match the name of the plugin and by convention is
      strictly lower-case and contains no whitespace.
    \item {\cf open()} should open the file and return true, or should
      return false if unable to do so (including if the file was found
      but turned out not to be in the format that your plugin is trying
      to implement).
    \item {\cf close()} should close the file, if open.
    \item {\cf read_native_scanline} should read a single scanline from
      the file into the address provided, uncompressing it but
      keeping it in its native data format without any translation.
    \item The virtual destructor, which should {\cf close()} if the file
      is still open, addition to performing any other tear-down activities.
  \end{enumerate}
  
  Additionally, your \ImageInput subclass may optionally choose to
  overload any of the following methods, which are defined in the
  \ImageInput base class and only need to be overloaded if the default
  behavior is not appropriate for your plugin:

  \begin{enumerate}
    \item[(f)] {\cf supports()}, only if your format supports any of
      the optional features described in
      Section~\ref{sec:inputsupportsfeaturelist}.
    \item[(g)] {\cf valid_file()}, if your format has a way to
      determine if a file is of the given format in a way that is less
      expensive than a full {\cf open()}.
    \item[(h)] {\cf seek_subimage()}, only if your format supports
      reading multiple subimages within a single file.
    \item[(i)] {\cf read_native_scanlines()}, only if your format has a speed
      advantage when reading multiple scanlines at once.  If you do not
      supply this function, the default implementation will simply call
      {\cf read_scanline()} for each scanline in the range.
    \item[(j)] {\cf read_native_tile()}, only if your format supports
      reading tiled images.
    \item[(k)] {\cf read_native_tiles()}, only if your format supports
      reading tiled images and there is a speed advantage when reading
      multiple tiles at once.  If you do not supply this function, the
      default implementation will simply call {\cf read_native_tile()} for each
      tile in the range.
    \item[(l)] ``Channel subset'' versions of {\cf read_native_scanlines()}
      and/or {\cf read_native_tiles()}, only if your format has a more
      efficient means of reading a subset of channels.  If you do not
      supply these methods, the default implementation will simply use
      {\cf read_native_scanlines()} or {\cf read_native_tiles()} to read
      into a temporary all-channel buffer and then copy the channel
      subset into the user's buffer.
    \item[(m)] {\cf read_native_deep_scanlines()} and/or 
      {\cf read_native_deep_tiles()}, only if your format supports
      ``deep'' data images.
  \end{enumerate}

  Here is how the class definition looks for our JPEG example.  Note
  that the JPEG/JFIF file format does not support multiple subimages
  or tiled images.

  \begin{code}
    class JpgInput : public ImageInput {
     public:
        JpgInput () { init(); }
        virtual ~JpgInput () { close(); }
        virtual const char * format_name (void) const { return "jpeg"; }
        virtual bool open (const std::string &name, ImageSpec &spec);
        virtual bool read_native_scanline (int y, int z, void *data);
        virtual bool close ();
     private:
        FILE *m_fd;
        bool m_first_scanline;
        struct jpeg_decompress_struct m_cinfo;
        struct jpeg_error_mgr m_jerr;

        void init () { m_fd = NULL; }
    };
  \end{code}
\end{enumerate}

Your subclass implementation of {\cf open()}, {\cf close()}, and {\cf
  read_native_scanline()} are the heart of an \ImageInput
implementation.  (Also {\cf read_native_tile()} and {\cf
  seek_subimage()}, for those image formats that support them.)

The remainder of this section simply lists the full implementation of
our JPEG reader, which relies heavily on the open source {\fn jpeg-6b}
library to perform the actual JPEG decoding.

\includedcode{../jpeg.imageio/jpeginput.cpp}




\section{Image Writers}
\label{sec:pluginwriters}

A plugin that writes a particular image file format must implement a
\emph{subclass} of \ImageOutput (described in
Chapter~\ref{chap:imageoutput}).  This is actually very straightforward
and consists of the following steps, which we will illustrate with a
real-world example of writing a JPEG/JFIF plug-in.

\begin{enumerate}
\item Read the base class definition from {\fn imageio.h}, just as
  with an image reader (see Section~\ref{sec:pluginreaders}).

\item Declare three public items:

  \begin{enumerate}
    \item An integer called \emph{name}{\cf _imageio_version} that identifies
      the version of the ImageIO protocol implemented by the plugin,
      defined in {\fn imageio.h} as the constant {\cf OIIO_PLUGIN_VERSION}.
      This allows the library to be sure it is not loading a plugin
      that was compiled against an incompatible version of \product.
      Note that if your plugin has both a reader and writer and they
      are compiled as separate modules (C++ source files), you don't
      want to declare this in \emph{both} modules; either one is fine.
    \item A function named \emph{name}{\cf _output_imageio_create} that
      takes no arguments and returns a new instance of your \ImageOutput
      subclass.  (Note that \emph{name} is the name of your format,
      and must match the name of the plugin itself.)
    \item An array of {\cf char *} called \emph{name}{\cf _output_extensions}
      that contains the list of file extensions that are likely to indicate
      a file of the right format.  The list is terminated by a {\cf NULL}
      pointer.
  \end{enumerate}

  All of these items must be inside an `{\cf extern "C"}' block in order
  to avoid name mangling by the C++ compiler, and we provide handy
  macros {\cf OIIO_PLUGIN_EXPORTS_BEGIN} and {\cf OIIO_PLUGIN_EXPORTS_END}
  to mamke this easy.  Depending on your
  compiler, you may need to use special commands to dictate that the
  symbols will be exported in the DSO; we provide a special {\cf
  OIIO_EXPORT} macro for this purpose, defined in {\fn export.h}.

  Putting this all together, we get the following for our JPEG example:

  \begin{code}
    OIIO_PLUGIN_EXPORTS_BEGIN
        OIIO_EXPORT int jpeg_imageio_version = OIIO_PLUGIN_VERSION;
        OIIO_EXPORT JpgOutput *jpeg_output_imageio_create () {
            return new JpgOutput;
        }
        OIIO_EXPORT const char *jpeg_input_extensions[] = {
            "jpg", "jpe", "jpeg", NULL
        };
    OIIO_PLUGIN_EXPORTS_END
  \end{code}

\item The definition and implementation of an \ImageOutput subclass for
  this file format.  It must publicly inherit \ImageOutput, and must
  overload the following methods which are ``pure virtual'' in the
  \ImageOutput base class:

  \begin{enumerate}
    \item {\cf format_name()} should return the name of the format, which
      ought to match the name of the plugin and by convention is
      strictly lower-case and contains no whitespace.
    \item {\cf supports()} should return {\cf true} if its argument
      names a feature supported by your format plugin, {\cf false} if it
      names a feature not supported by your plugin.  See
      Section~\ref{sec:supportsfeaturelist} for the list of feature
      names.
    \item {\cf open()} should open the file and return true, or should
      return false if unable to do so (including if the file was found
      but turned out not to be in the format that your plugin is trying
      to implement).
    \item {\cf close()} should close the file, if open.
    \item {\cf write_scanline} should write a single scanline to
      the file, translating from internal to native data format and
      handling strides properly.
    \item The virtual destructor, which should {\cf close()} if the file
      is still open, addition to performing any other tear-down activities.
  \end{enumerate}
  
  Additionally, your \ImageOutput subclass may optionally choose to
  overload any of the following methods, which are defined in the
  \ImageOutput base class and only need to be overloaded if the default
  behavior is not appropriate for your plugin:

  \begin{enumerate}
    \item[(g)] {\cf write_scanlines()}, only if your format supports
      writing scanlines and you can get a performance improvement when
      outputting multiple scanlines at once.  If you don't supply
      {\cf write_scanlines()}, the default implementation will simply
      call {\cf write_scanline()} separately for each scanline in the
      range.
    \item[(h)] {\cf write_tile()}, only if your format supports
      writing tiled images.
    \item[(i)] {\cf write_tiles()}, only if your format supports
      writing tiled images and you can get a performance improvement
      when outputting multiple tiles at once.  If you don't supply
      {\cf write_tiles()}, the default implementation will simply
      call {\cf write_tile()} separately for each tile in the range.
    \item[(j)] {\cf write_rectangle()}, only if your format supports
      writing arbitrary rectangles.
    \item[(k)] {\cf write_image()}, only if you have a more clever
      method of doing so than the default implementation that calls
      {\cf write_scanline()} or {\cf write_tile()} repeatedly.
    \item[(l)] {\cf write_deep_scanlines()} and/or 
      {\cf write_deep_tiles()}, only if your format supports
      ``deep'' data images.
  \end{enumerate}

  It is not strictly required, but certainly appreciated, if a file format
  does not support tiles, to nonetheless accept an \ImageSpec that specifies
  tile sizes by allocating a full-image buffer in {\cf open()}, providing an
  implementation of {\cf write_tile()} that copies the tile of data to the
  right spots in the buffer, and having {\cf close()} then call 
  {\cf write_scanlines} to process the buffer now that the image has been
  fully sent.

  Here is how the class definition looks for our JPEG example.  Note
  that the JPEG/JFIF file format does not support multiple subimages
  or tiled images.

  \begin{code}
    class JpgOutput : public ImageOutput {
     public:
        JpgOutput () { init(); }
        virtual ~JpgOutput () { close(); }
        virtual const char * format_name (void) const { return "jpeg"; }
        virtual bool supports (const std::string &property) const { return false; }
        virtual bool open (const std::string &name, const ImageSpec &spec,
                           bool append=false);
        virtual bool write_scanline (int y, int z, TypeDesc format,
                                     const void *data, stride_t xstride);
        bool close ();
     private:
        FILE *m_fd;
        std::vector<unsigned char> m_scratch;
        struct jpeg_compress_struct m_cinfo;
        struct jpeg_error_mgr m_jerr;

        void init () { m_fd = NULL; }
    };
  \end{code}
\end{enumerate}

Your subclass implementation of {\cf open()}, {\cf close()}, and {\cf
  write_scanline()} are the heart of an \ImageOutput implementation.
(Also {\cf write_tile()}, for those image formats that support tiled
output.)

An \ImageOutput implementation must properly handle all data formats and
strides passed to {\cf write_scanline()} or {\cf write_tile()}, unlike
an \ImageInput implementation, which only needs to read scanlines or
tiles in their native format and then have the super-class handle the
translation.  But don't worry, all the heavy lifting can be accomplished
with the following helper functions provided as protected member
functions of \ImageOutput that convert a scanline, tile, or rectangular
array of values from one format to the native format(s) of the file.

\apiitem{const void * {\ce to_native_scanline} (TypeDesc format, const void *data, \\
                \bigspc stride_t xstride, std::vector<unsigned char> \&scratch, \\
                                    \bigspc unsigned int dither=0,
                                    int yorigin=0, int zorigin=0)}

Convert a full scanline of pixels (pointed to by \emph{data}) with the
given \emph{format} and strides into contiguous pixels in the native
format (described by the \ImageSpec returned by the {\cf spec()} member
function).  The location of the newly converted data is returned, which
may either be the original \emph{data} itself if no data conversion was
necessary and the requested layout was contiguous (thereby avoiding
unnecessary memory copies), or may point into memory allocated within
the \emph{scratch} vector passed by the user.  In either case, the
caller doesn't need to worry about thread safety or freeing any
allocated memory (other than eventually destroying the scratch vector).
\apiend

\apiitem{const void * {\ce to_native_tile} (TypeDesc format, const void *data,\\
                            \bigspc stride_t xstride, stride_t ystride, stride_t zstride,\\
                            \bigspc std::vector<unsigned char> \&scratch,
                            unsigned int dither=0, \\ \bigspc int xorigin=0,
                             int yorigin=0, int zorigin=0)}

Convert a full tile of pixels (pointed to by \emph{data}) with the given
\emph{format} and strides into contiguous pixels in the native format
(described by the \ImageSpec returned by the {\cf spec()} member
function).  The location of the newly converted data is returned, which
may either be the original \emph{data} itself if no data conversion was
necessary and the requested layout was contiguous (thereby avoiding
unnecessary memory copies), or may point into memory allocated within
the \emph{scratch} vector passed by the user.  In either case, the
caller doesn't need to worry about thread safety or freeing any
allocated memory (other than eventually destroying the scratch vector).

\apiend

\apiitem{const void * {\ce to_native_rectangle} (int xbegin, int xend, \\
                                    \bigspc int ybegin, int yend,
                                     int zbegin, int zend, \\ \bigspc
                                     TypeDesc format, const void
                                     *data, \\ \bigspc
                                     stride_t xstride, stride_t ystride,
                                     stride_t zstride, \\ \bigspc
                                     std::vector<unsigned char> \&scratch,
                            unsigned int dither=0, \\ \bigspc int xorigin=0,
                             int yorigin=0, int zorigin=0)}

Convert a rectangle of pixels (pointed to by \emph{data}) with the given
\emph{format}, dimensions, and strides into contiguous pixels in the
native format (described by the \ImageSpec returned by the {\cf spec()}
member function).  The location of the newly converted data is returned,
which may either be the original \emph{data} itself if no data
conversion was necessary and the requested layout was contiguous
(thereby avoiding unnecessary memory copies), or may point into memory
allocated within the \emph{scratch} vector passed by the user.  In
either case, the caller doesn't need to worry about thread safety or
freeing any allocated memory (other than eventually destroying the
scratch vector).

\apiend

For {\cf float} to 8 bit integer conversions only, if {\cf dither} parameter
is nonzero, random dither will be added to reduce quantization banding
artifacts; in this case, the specific nonzero {\cf dither} value is used as
a seed for the hash function that produces the per-pixel dither amounts, and
the optional {\cf origin} parameters help it to align the pixels to the
right position in the dither pattern.


\bigskip
\bigskip

\noindent
The remainder of this section simply lists the full implementation of
our JPEG writer, which relies heavily on the open source {\fn jpeg-6b}
library to perform the actual JPEG encoding.

\includedcode{../jpeg.imageio/jpegoutput.cpp}


\section{Tips and Conventions}
\label{sec:plugintipsconventions}

\product's main goal is to hide all the pesky details of individual file
formats from the client application.  This inevitably leads to various
mismatches between a file format's true capabilities and requests that
may be made through the \product APIs.  This section outlines
conventions, tips, and rules of thumb that we recommend for image file
support.

\subsection*{Readers}
\begin{itemize}
\item If the file format stores images in a non-spectral color space
  (for example, YUV), the reader should automatically convert to RGB to
  pass through the OIIO APIs.  In such a case, the reader should signal
  the file's true color space via a \qkw{Foo:colorspace} attribute in
  the \ImageSpec.
\item ``Palette'' images should be automatically converted by the reader
  to RGB.
\item If the file supports thumbnail images in its header, the reader
  should store the thumbnail dimensions in attributes
  \qkw{thumbnail_width}, \qkw{thumbnail_height}, and
  \qkw{thumbnail_nchannels} (all of which should be {\cf int}), and the
  thumbnail pixels themselves in \qkw{thumbnail_image} as an array of
  channel values (the array length is the total number of channel
  samples in the thumbnail).
\end{itemize}

\subsection*{Writers}

The overall rule of thumb is: try to always ``succeed'' at writing the
file, outputting the closest approximation of the user's data as
possible.  But it is permissible to fail the {\cf open()} call if it is
clearly nonsensical or there is no possible way to output a decent
approximation of the user's data.  Some tips:

\begin{itemize}
\item If the client application requests a data format not directly
  supported by the file type, silently write the supported data format
  that will result in the least precision or range loss.
\item It is customary to fail a call to {\cf open()} if the \ImageSpec
  requested a number of color channels plainly not supported by the
  file format.  As an exception to this rule, it is permissible for a
  file format that does not support alpha channels to silently drop
  the fourth (alpha) channel of a 4-channel output request.
\item If the app requests a \qkw{Compression} not supported by the file
  format, you may choose as a default any lossless compression
  supported.  Do not use a lossy compression unless you are fairly 
  certain that the app wanted a lossy compression.
\item If the file format is able to store images in a non-spectral color
  space (for example, YUV), the writer may accept a \qkw{Foo:colorspace}
  attribute in the \ImageSpec as a request to automatically convert and
  store the data in that format (but it will always be passed as RGB
  through the OIIO APIs).
\item If the file format can support thumbnail images in its header, and
  the \ImageSpec contain attributes \qkw{thumbnail_width},
  \qkw{thumbnail_height}, \qkw{thumbnail_nchannels}, and
  \qkw{thumbnail_image}, the writer should attempt to store the
  thumbnail if possible.
\end{itemize}



\section{Building ImageIO Plugins}
\label{sec:buildingplugins}

FIXME -- spell out how to compile and link plugins on each of the major
platforms.


\chapwidthend

\chapter{Bundled ImageIO Plugins}
\label{chap:bundledplugins}
\index{Plugins!bundled|(}

This chapter lists all the image format plugins that are bundled with
\product.  For each plugin, we delineate any limitations, custom
attributes, etc.  The plugins are listed alphabetically by format name.


\vspace{.25in}

\section{BMP}
\label{sec:bundledplugins:bmp}
\index{BMP}

BMP is a bitmap image file format used mostly on Windows systems.
BMP files use the file extension {\cf .bmp}.

BMP is not a nice format for high-quality or high-performance images.
It only supports unsigned integer 1-, 2-, 4-, and 8- bits per channel; only
grayscale, RGB, and RGBA; does not support MIPmaps, multiimage, or
tiles.

%\subsubsection*{Attributes}
\vspace{.125in}
\noindent\begin{tabular}{p{1.5in}|p{0.5in}|p{3.25in}}
\ImageSpec Attribute & Type & BMP header data or explanation \\
\hline
\qkw{XResolution} & float & hres \\
\qkw{YResolution} & float & vres \\
\qkw{ResolutionUnit} & string & always \qkw{m} (pixels per meter)
\end{tabular}



\vspace{.25in}

\section{Cineon}
\label{sec:bundledplugins:cineon}
\index{Cineon}

Cineon is an image file format developed by Kodak that is commonly
used for scanned motion picture film and digital intermediates.
Cineon files use the file extension {\cf .cin}.

%FIXME

\vspace{.25in}



\section{DDS}
\label{sec:bundledplugins:dds}
\index{DDS}

DDS (Direct Draw Surface) is an image file format designed by Microsoft
for use in Direct3D graphics.  DDS files use the extension {\cf .dds}.

DDS is an awful format, with several compression modes that are all so
lossy as to be completely useless for high-end graphics.  Nevertheless,
they are widely used in games and graphics hardware directly supports
these compression modes.  Alas.

\product currently only supports reading DDS files, not writing them.

%\subsubsection*{Attributes}
\vspace{.125in}

\noindent\begin{tabular}{p{1.5in}|p{0.5in}|p{3.5in}}
\ImageSpec Attribute & Type & DDS header data or explanation \\
\hline
\qkw{compression} & string & compression type \\
\qkw{oiio:BitsPerSample} & int & bits per sample \\
\qkw{textureformat} & string & Set correctly to one of \qkws{Plain
  Texture}, \qkws{Volume Texture}, or \qkws{CubeFace Environment}. \\
\qkw{texturetype} & string & Set correctly to one of \qkws{Plain
  Texture}, \qkws{Volume Texture}, or \qkws{Environment}. \\
\qkw{dds:CubeMapSides} & string & For environment maps, which cube
  faces are present (e.g., \qkw{+x -x +y -y} if $x$ \& $y$ faces are
  present, but not $z$). \\
\end{tabular}

%\subsubsection*{Limitations}
%\begin{itemize}
%\item blah
%\end{itemize}


\vspace{.25in}

\section{DPX}
\label{sec:bundledplugins:dpx}
\index{DPX}


DPX (Digital Picture Exchange) is an image file format used for 
motion picture film scanning, output, and digital intermediates.
DPX files use the file extension {\cf .dpx}.

%\subsubsection*{Attributes}
\vspace{.125in}

\noindent\begin{tabular}{p{1.8in}|p{0.65in}|p{2.75in}}
OIIO Attribute & Type & DPX header data or explanation \\
\hline
\qkw{ImageDescription} & string & Description of image element \\
\qkw{Copyright} & string & Copyright statement \\
\qkw{Software} & string & Creator \\
\qkw{DocumentName} & string & Project name \\
\qkw{DateTime} & string & Creation date/time \\
\qkw{Orientation} & int & the orientation of the DPX image data (see
  \ref{metadata:orientation}) \\
\qkw{compression} & string & The compression type \\
\qkw{PixelAspectRatio} & float & pixel aspect ratio \\
\qkw{oiio:BitsPerSample} & int & the true bits per sample of the DPX file. \\
\qkw{oiio:Endian} & string & When writing, force a particular endianness
                             for the output file (\qkw{little} or \qkw{big}) \\
\qkw{smpte:TimeCode} & int[2] & SMPTE time code (vecsemantics will be
                                marked as TIMECODE) \\
\qkw{smpte:KeyCode} & int[7] & SMPTE key code (vecsemantics will be
                                marked as KEYCODE) \\

\end{tabular} 

\noindent\begin{tabular}{p{1.8in}|p{0.65in}|p{2.75in}}
OIIO Attribute & Type & DPX header data or explanation \\
\hline
\qkw{dpx:Transfer} & string & Transfer characteristic \\
\qkw{dpx:Colorimetric} & string & Colorimetric specification \\
\qkw{dpx:ImageDescriptor} & string & ImageDescriptor \\
\qkw{dpx:Packing} & string & Image packing method \\
\qkw{dpx:TimeCode} & int & SMPTE time code \\
\qkw{dpx:UserBits} & int & SMPTE user bits \\
\qkw{dpx:SourceDateTime} & string & source time and date \\
\qkw{dpx:FilmEdgeCode} & string & FilmEdgeCode \\
\qkw{dpx:Signal} & string & Signal (\qkw{Undefined}, \qkw{NTSC},
  \qkw{PAL}, etc.) \\
\qkw{dpx:UserData} & UCHAR[*] & User data (stored in an array
  whose length is whatever it was in the DPX file) \\
\qkw{dpx:EncryptKey} & int & Encryption key (-1 is not encrypted) \\
\qkw{dpx:DittoKey} & int & Ditto (0 = same as previous frame, 1 =
  new) \\
\qkw{dpx:LowData} & int & reference low data code value \\
\qkw{dpx:LowQuantity} & float & reference low quantity \\
\qkw{dpx:HighData} & int & reference high data code value \\
\qkw{dpx:HighQuantity} & float & reference high quantity \\
\qkw{dpx:XScannedSize} & float & X scanned size \\
\qkw{dpx:YScannedSize} & float & Y scanned size \\
\qkw{dpx:FramePosition} & int & frame position in sequence \\
\qkw{dpx:SequenceLength} & int & sequence length (frames) \\
\qkw{dpx:HeldCount} & int & held count (1 = default) \\
\qkw{dpx:FrameRate} & float & frame rate of original (frames/s) \\
\qkw{dpx:ShutterAngle} & float & shutter angle of camera (deg) \\
\qkw{dpx:Version} & string & version of header format \\
\qkw{dpx:Format} & string & format (e.g., \qkw{Academy}) \\
\qkw{dpx:FrameId} & string & frame identification \\
\qkw{dpx:SlateInfo} & string & slate information \\
\qkws{dpx:SourceImageFileName} & string & source image filename \\
\qkw{dpx:InputDevice} & string & input device name \\
\qkwf{dpx:InputDeviceSerialNumber} & string & input device serial number \\
\qkw{dpx:Interlace} & int & interlace (0 = noninterlace, 1 = 2:1 interlace)\\
\qkw{dpx:FieldNumber} & int & field number \\
\qkws{dpx:HorizontalSampleRate} & float & horizontal sampling rate (Hz) \\
\qkws{dpx:VerticalSampleRate} & float & vertical sampling rate (Hz) \\
\qkws{dpx:TemporalFrameRate} & float & temporal sampling rate (Hz) \\
\qkw{dpx:TimeOffset} & float & time offset from sync to first
pixel (ms) \\
\qkw{dpx:BlackLevel} & float & black level code value \\
\qkw{dpx:BlackGain} & float & black gain \\
\qkw{dpx:BreakPoint} & float & breakpoint \\
\qkw{dpx:WhiteLevel} & float & reference white level code value \\
\qkw{dpx:IntegrationTimes} & float & integration time (s) \\
\qkw{dpx:EndOfLinePadding} & int & Padded bytes at the end of each line \\
\qkw{dpx:EndOfImagePadding} & int & Padded bytes at the end of each image \\
\end{tabular}

%\subsubsection*{Limitations}
%\begin{itemize}
%\item blah
%\end{itemize}


\vspace{.25in}

\section{Field3D}
\label{sec:bundledplugins:field3d}
\index{Field3D}

Field3d is an open-source volume data file format.  Field3d files
commonly use the extension {\cf .f3d}.
The official Field3D site is:
\url{http://sites.google.com/site/field3d/}
Currently, \product only reads Field3d files, and does not write them.

Fields are comprised of multiple \emph{layers} (which appear to \product
as subimages).  Each layer/subimage may have a different name,
resolution, and coordinate mapping.  Layers may be scalar (1 channel) or
vector (3 channel) fields, and the data may be {\cf half}, {\cf float},
or {\cf double}.

\product always reports Field3D files as tiled.  If the Field3d file has
a ``block size'', the block size will be reported as the tile size.
Otherwise, the tile size will be the size of the entire volume.

%\subsubsection*{Attributes}
\vspace{.125in}

\noindent\begin{tabular}{p{1.6in}|p{0.6in}|p{3.0in}}
\ImageSpec Attribute & Type & Field3d header data or explanation \\
\hline
\qkw{ImageDescription} & string & unique layer name \\
\qkw{oiio:subimagename} & string & unique layer name \\
\qkw{field3d:partition} & string & the partition name \\
\qkw{field3d:layer} & string & the layer (a.k.a.\ attribute) name \\
\qkw{field3d:fieldtype} & string & field type, one of:
   \qkw{dense}, \qkw{sparse}, or \qkw{MAC} \\
\qkw{field3d:mapping} & string & the coordinate mapping type \\
\qkws{field3d:localtoworld} & matrix of doubles & if a
  matrixMapping, the local-to-world transformation matrix \\
\qkw{worldtocamera} & matrix & if a matrixMapping, the
  world-to-local coordinate mapping \\
\end{tabular}

\vspace{10pt}

The ``unique layer name'' is generally the partition name + ``:'' +
attribute name (example: \qkw{defaultfield:density}), with the following
exceptions: (1) if the partition and attribute names are identical, just
one is used rather than it being pointlessly concatenated (e.g.,
\qkw{density}, not \qkw{density:density}); (2) if there are mutiple
partitions + attribute combinations with identical names in the same
file, ``.\emph{number}'' will be added after the partition name for 
subsequent layers (e.g., \qkw{default:density}, \qkw{default.2:density},
\qkw{default.3:density}).

\vspace{.25in}

\section{FITS}
\label{sec:bundledplugins:fits}
\index{FITS}

FITS (Flexible Image Transport System) is an image file format used
for scientific applications, particularly professional astronomy.
FITS files use the file extension {\cf .fits}.
Official FITS specs and other info may be found at:
\url{http://fits.gsfc.nasa.gov/} 

\product supports multiple images in FITS files, and supports the
following pixel data types: UINT8, UINT16, UINT32, FLOAT, DOUBLE.

FITS files can store various kinds of arbitrary data arrays, but
\product's support of FITS is mostly limited using FITS for image
storage.  Currently, \product only supports 2D FITS data (images), not
3D (volume) data, nor 1-D or higher-dimensional arrays.

%\subsubsection*{Attributes}
\vspace{.125in}

\noindent\begin{tabular}{p{1.5in}|p{0.5in}|p{3.5in}}
\ImageSpec Attribute & Type & FITS header data or explanation \\
\hline
\qkw{Orientation} & int & derived from FITS ``ORIENTAT'' field. \\
\qkw{DateTime} & string & derived from the FITS ``DATE'' field. \\
\qkw{Comment} & string & FITS ``COMMENT'' (*) \\
\qkw{History} & string & FITS ``HISTORY'' (*) \\
\qkw{Hierarch} & string & FITS ``HIERARCH'' (*) \\[1.5ex]
\emph{other} & & all other FITS keywords will be added to the \ImageSpec
    as arbitrary named metadata.
\end{tabular}

\noindent (*) Note: If the file contains multiple COMMENT, HISTORY, or HIERARCH
  fields, their text will be appended to form a single attribute (of
  each) in \product's \ImageSpec.

\vspace{.25in}

\section{GIF}
\label{sec:bundledplugins:gif}
\index{GIF}

GIF (Graphics Interchange Format) is an image file format developed by 
CompuServe in 1987.  Nowadays it is widely used to display basic animations
despite its technical limitations.

%\subsubsection*{Attributes}
\vspace{.125in}

\noindent\begin{tabular}{p{1.5in}|p{0.5in}|p{3.25in}}
\ImageSpec Attribute & Type & GIF header data or explanation \\
\hline
\qkw{gif:DelayMs} & int & Delay between frames in miliseconds. \\
\qkw{gif:Interlacing} & int & Specifies if image is interlaced (0 or 1). \\
\qkw{gif:LoopCount} & int & Number of times the animation should be played 
(0--65535, 0 stands for infinity). \\
\qkw{ImageDescription} & string & The GIF comment field.
\end{tabular}

\subsubsection*{Limitations}

\begin{itemize}
\item GIF only supports 3-channel (RGB) images and at most 8 bits per 
channel.
\item Each subimage can include its own palette or use global palette.
Palettes contain up to 256 colors of which one can be used as background 
color. It is then emulated with additional Alpha channel by \product's reader.
\end{itemize}

\vspace{.25in}

\section{HDR/RGBE}
\label{sec:bundledplugins:hdr}
\index{HDR} \index{RGBE}

HDR (High Dynamic Range), also known as RGBE (rgb with extended range),
is a simple format developed for the Radiance renderer to store high
dynamic range images.  HDR/RGBE files commonly use the file extensions
{\cf .hdr}.  The format is described in this section of the Radiance
documentation: \url{http://radsite.lbl.gov/radiance/refer/filefmts.pdf}

RGBE does not support tiles, multiple subimages, mipmapping, true half
or float pixel values, or arbitrary metadata.  Only RGB (3 channel)
files are supported.

RGBE became important because it was developed at a time when no
standard file formats supported high dynamic range, and is still used
for many legacy applications and to distribute HDR environment maps.
But newer formats with native HDR support, such as OpenEXR, are vastly
superior and should be preferred except when legacy file access is
required.

%\subsubsection*{Attributes}
\vspace{.125in}

\noindent\begin{tabular}{p{1.5in}|p{0.5in}|p{3.25in}}
\ImageSpec Attribute & Type & RGBE header data or explanation \\
\hline
\qkw{Orientation} & int & encodes the orientation (see
  \ref{metadata:orientation}) \\
{\cf ImageSpec.gamma} & float & the gamma correction specified in the
  RGBE header.
\end{tabular}


\vspace{.25in}

\section{ICO}
\label{sec:bundledplugins:ico}
\index{ICO}

ICO is an image file format used for small images (usually icons) on
Windows.  ICO files use the file extension {\cf .ico}.

%\subsubsection*{Attributes}
\vspace{.125in}

\noindent\begin{tabular}{p{1.5in}|p{0.5in}|p{3.25in}}
\ImageSpec Attribute & Type & ICO header data or explanation \\
\hline
\qkw{oiio:BitsPerSample} & int & the true bits per sample in the ICO file. \\
\qkw{ico:PNG} & int & if nonzero, will cause the ICO to be written
  out using PNG format.
\end{tabular}

\subsubsection*{Limitations}

\begin{itemize}
\item ICO only supports UINT8 and UINT16 formats; all output images will
  be silently converted to one of these.
\item ICO only supports \emph{small} images, up to $256 \times 256$.
  Requests to write larger images will fail their {\cf open()} call.
\end{itemize}


\vspace{.25in}

\vspace{.25in}

\section{IFF}
\label{sec:bundledplugins:iff}
\index{IFF}

IFF files are used by Autodesk Maya and use the file extension {\cf .iff}.

%\subsubsection*{Attributes}
\vspace{.125in}

\noindent\begin{tabular}{p{1.8in}|p{0.65in}|p{2.75in}}
OIIO Attribute & Type & DPX header data or explanation \\
\hline
\qkw{Artist} & string & The IFF ``author'' \\
\qkw{DateTime} & string & Creation date/time \\
\qkw{compression} & string & The compression type \\
\qkw{oiio:BitsPerSample} & int & the true bits per sample of the IFF file. \\
\end{tabular} 



%\subsubsection*{Limitations}
%\begin{itemize}
%\item blah
%\end{itemize}


\vspace{.25in}

\section{JPEG}
\label{sec:bundledplugins:jpeg}
\index{JPEG}

JPEG (Joint Photographic Experts Group), or more properly the JFIF file
format containing JPEG-compressed pixel data, is one of the most popular
file formats on the Internet, with applications, and from digital
cameras, scanners, and other image acquisition devices.  JPEG/JFIF files
usually have the file extension {\cf .jpg}, {\cf .jpe}, {\cf .jpeg},
{\cf .jif}, {\cf .jfif}, or {\cf .jfi}.  The JFIF file format is
described by \url{http://www.w3.org/Graphics/JPEG/jfif3.pdf}.

Although we strive to support JPEG/JFIF because it is so widely used, we
acknowledge that it is a poor format for high-end work: it supports only
1- and 3-channel images, has no support for alpha channels, no support
for high dynamic range or even 16 bit integer pixel data, by convention
stores sRGB data and is ill-suited to linear color spaces, and does not
support multiple subimages or MIPmap levels.  There are newer formats
also blessed by the Joint Photographic Experts Group that attempt to
address some of these issues, such as JPEG-2000, but these do not have
anywhere near the acceptance of the original JPEG/JFIF format.

%\subsubsection*{Attributes}
\vspace{.125in}

\noindent\begin{tabular}{p{1.5in}|p{0.5in}|p{3.25in}}
\ImageSpec Attribute & Type & JPEG header data or explanation \\
\hline
\qkw{ImageDescription} & string & the JPEG Comment field \\
\qkw{Orientation} & int & the image orientation \\[2ex]
\qkw{XResolution}, \qkw{YResolution},
\qkw{ResolutionUnit} & & The resolution and units from the Exif header \\[2ex]
\qkw{CompressionQuality} & int & Quality of compression (1-100) \\[2ex]
\qkw{ICCProfile} & uint8[] & The ICC color profile \\[2ex]
\qkw{jpeg:subsampling} & string & Describes the chroma subsampling,
    e.g., \qkw{4:2:0} (the default), \qkw{4:4:4}, \qkw{4:2:2},
    \qkw{4:2:1}. \\[2ex]
& & \\
Exif, IPTC, XMP, GPS & & Extensive Exif, IPTC, XMP, and GPS data are supported by the
  reader/writer, and you should assume that nearly everything described
  Appendix~\ref{chap:stdmetadata} is properly translated when using
  JPEG files.
\end{tabular}

\subsubsection*{Limitations}
\begin{itemize}
\item JPEG/JFIF only supports 1- (grayscale) and 3-channel (RGB) images.
  As a special case, \product's JPEG writer will accept 4-channel image
  data and silently drop the alpha channel while outputting.  Other
  channel count requests (i.e., anything other than 1, 3, and 4) will
  cause {\cf open()} to fail, since it is not possible to write a JFIF
  file with other than 1 or 3 channels.
\item Since JPEG/JFIF only supports 8 bits per channel, \product's
  JPEG/JFIF writer will silently convert to UINT8 upon output,
  regardless of requests to the contrary from the calling program.
\item \product's JPEG/JFIF reader and writer always operate in scanline
  mode and do not support tiled image input or output.
\end{itemize}



\vspace{.25in}

\section{JPEG-2000}
\label{sec:bundledplugins:jpeg2000}
\index{Jpeg 2000}

JPEG-2000 is a successor to the popular JPEG/JFIF format, that supports
better (wavelet) compression and a number of other extensions.  It's
geared toward photography.
JPEG-2000 files use the file extensions {\cf .jp2} or {\cf .j2k}.
The official JPEG-2000 format specification and other helpful info
may be found at \url{http://www.jpeg.org/JPEG2000.htm}.

JPEG-2000 is not yet widely used, so \product's support of it is 
preliminary.  In particular, we are not yet very good at handling
the metadata robustly.

%\subsubsection*{Attributes}
\vspace{.125in}

\noindent\begin{tabular}{p{1.75in}|p{0.5in}|p{3.0in}}
\ImageSpec Attribute & Type & JPEG-2000 header data or explanation \\
\hline
\qkws{jpeg2000:streamformat} & string & specifies the JPEG-2000
  stream format (\qkw{none} or \qkw{jpc})
\end{tabular}



\vspace{.25in}

\section{OpenEXR}
\label{sec:bundledplugins:openexr}
\index{OpenEXR}

OpenEXR is an image file format developed by Industrial Light \& Magic,
and subsequently open-sourced.  OpenEXR's strengths include support of
high dynamic range imagery ({\cf half} and {\cf float} pixels), tiled
images, explicit support of MIPmaps and cubic environment maps,
arbitrary metadata, and arbitrary numbers of color channels.  OpenEXR
files use the file extension {\cf .exr}.
The official OpenEXR site is \url{http://www.openexr.com/}.

%\subsubsection*{Attributes}
\vspace{.125in}

\noindent\begin{tabular}{p{1.75in}|p{0.5in}|p{3.0in}}
\ImageSpec Attribute & Type & OpenEXR header data or explanation \\
\hline
{\cf width}, {\cf height}, {\cf x}, {\cf y} & & {\cf dataWindow} \\
& & \\
{\cf\small full_width}, {\cf\small full_height}, {\cf\small full_x}, 
  {\cf\small full_y} & & {\cf displayWindow}.  \\
& & \\
\qkw{worldtocamera} & matrix & worldToCamera \\
\qkw{worldtoscreen} & matrix & worldToNDC \\
\qkw{ImageDescription} & string & comments \\
\qkw{Copyright} & string & owner \\
\qkw{DateTime} & string & capDate \\
\qkw{PixelAspectRatio} & float & pixelAspectRatio \\
\qkw{ExposureTime} & float & expTime \\
\qkw{FNumber} & float & aperture \\
\qkw{compression} & string & one of: \qkw{none}, \qkw{rle},
  \qkw{zip}, \qkw{piz}, \qkw{pxr24}, \qkw{b44}, \qkw{b44a},
  \qkw{dwaa}, or \qkw{dwab}.  If the
  writer receives a request for a compression type it does not
  recognize or is not supported by the version of OpenEXR on the system,
  it will use \qkw{zip} by default. \\
\qkw{textureformat} & string & set to \qkw{Plain Texture} for
  MIP-mapped OpenEXR files, \qkw{CubeFace Environment} or \qkw{Latlong
    Environment} for OpenEXR environment maps.  Non-environment
  non-MIP-mapped OpenEXR files will not set this attribute. \\
\qkw{wrapmodes} & string & wrapmodes \\
\qkw{smpte:TimeCode} & int[2] & SMPTE time code (vecsemantics will be
                                marked as TIMECODE) \\
\qkw{smpte:KeyCode} & int[7] & SMPTE key code (vecsemantics will be
                                marked as KEYCODE) \\
%\qkw{oiio:updirection} & string & Will be set to \qkw{y} for OpenEXR
% latlong environment maps to indicate that OpenEXR dictates a
% right-handed, ``$y$ is up'' coordinate system. \\
%\qkw{oiio:sampleborder} & int & Will be set to 1 for OpenEXR environment
% maps to indicate that OpenEXR dictates that boundary texels sample exactly
% on the texture border (pole, meridian, or cube edge).\\[2ex]
\qkw{openexr:lineOrder} & string & the OpenEXR lineOrder attribute
  (set to \qkws{increasingY}, \qkws{randomY}, or \qkws{decreasingY}).
 \\
\qkws{openexr:roundingmode} & int & the MIPmap rounding mode of the
  file. \\[2ex]
\qkws{openexr:dwaCompressionLevel} & float & compression level for
   dwaa or dwab compression (default: 45.0). \\
\emph{other} & & All other attributes will be added to the \ImageSpec by their
  name and apparent type.
\end{tabular}

\subsubsection*{Limitations}

\begin{itemize}
\item The OpenEXR format does not currently support multiple subimages,
  except for the special case of MIP-maps.
\item The OpenEXR format only supports HALF, FLOAT, and UINT32 pixel
  data.  \product's OpenEXR writer will silently convert data in formats
  (including the common UINT8 and UINT16 cases) to HALF data for output.
\end{itemize}


\vspace{.25in}

\section{PNG}
\label{sec:bundledplugins:png}
\index{PNG}

PNG (Portable Network Graphics) is an image file format developed by the
open source community as an alternative to the GIF, after Unisys started
enforcing patents allegedly covering techniques necessary to use GIF.
PNG files use the file extension {\cf .png}.

%\subsubsection*{Attributes}
\vspace{.125in}

\noindent\begin{tabular}{p{1.75in}|p{0.5in}|p{3.0in}}
\ImageSpec Attribute & Type & PNG header data or explanation \\
\hline
\qkw{ImageDescription} & string & Description \\
\qkw{Artist} & string & Author  \\
\qkw{DocumentName} & string & Title \\
\qkw{DateTime} & string & the timestamp in the PNG header \\
\qkw{PixelAspectRatio} & float & pixel aspect ratio \\
\qkw{XResolution} \qkw{YResolution}
  \qkw{ResolutionUnit} & & resolution and units from the PNG header. \\
\qkw{ICCProfile} & uint8[] & The ICC color profile \\
\end{tabular}

\subsubsection*{Limitations}

\begin{itemize}
\item PNG stupidly specifies that any alpha channel is ``unassociated''
  (i.e., that the color channels are not ``premultiplied'' by alpha).
  This is a disaster, since it results in bad loss of precision for
  alpha image compositing, and even makes it impossible to properly
  represent certain additive glows and other desirable pixel values.
  \product automatically associates alpha (i.e., multiplies colors by
  alpha) upon input and deassociates alpha (divides colors by alpha)
  upon output in order to properly conform to the OIIO convention (and
  common sense) that all pixel values passed through the OIIO APIs
  should use associated alpha.
\item PNG only supports UINT8 and UINT16 output; other requested formats
  will be automatically converted to one of these.
\end{itemize}


\vspace{.25in}

\section{PNM / Netpbm}
\label{sec:bundledplugins:pnm}
\index{PNM}

The Netpbm project, a.k.a.\ PNM (portable ``any'' map) defines PBM, PGM,
and PPM (portable bitmap, portable graymap, portable pixmap) files.
Without loss of generality, we will refer to these all collectively as
``PNM.''  These files have extensions {\cf .pbm}, {\cf .pgm}, and 
{\cf .ppm} and customarily correspond to bi-level bitmaps, 1-channel
grayscale, and 3-channel RGB files, respectively, or {\cf .pnm} for
those who reject the nonsense about naming the files depending on the
number of channels and bitdepth.

PNM files are not much good for anything, but because of their
historical significance and extreme simplicity (that causes many
``amateur'' programs to write images in these formats), \product
supports them.  PNM files do not support floating point images, anything
other than 1 or 3 channels, no tiles, no multi-image, no MIPmapping.
It's not a smart choice unless you are sending your images back to the
1980's via a time machine.

%\subsubsection*{Attributes}
\vspace{.125in}

\noindent\begin{tabular}{p{1.3in}|p{0.5in}|p{3.50in}}
\ImageSpec Attribute & Type & PNG header data or explanation \\
\hline
\qkw{oiio:BitsPerSample} & int & the true bits per sample of the file
  (1 for true PBM files, even though OIIO will report the {\cf format}
  as UINT8). \\
\qkw{pnm:binary} & int & nonzero if the file itself used the PNM
  binary format, 0 if it used ASCII.  The PNM writer honors this
  attribute in the \ImageSpec to determine whether to write an ASCII
  or binary file.
\end{tabular}



\vspace{.25in}

\section{PSD}
\label{sec:bundledplugins:psd}
\index{PSD}

% FIXME


\vspace{.25in}

\section{Ptex}
\label{sec:bundledplugins:ptex}
\index{Ptex}

Ptex is a special per-face texture format developed by Walt Disney
Feature Animation.  The format and software to read/write it are open
source, and available from \url{http://ptex.us/}.  Ptex files commonly
use the file extension {\cf .ptex}.

\product's support of Ptex is still incomplete.  We can read pixels from
Ptex files, but the \TextureSystem doesn't properly filter across face
boundaries when using it as a texture.  \product currently does not
write Ptex files at all.

%\subsubsection*{Attributes}
\vspace{.125in}

\noindent\begin{tabular}{p{1.75in}|p{0.5in}|p{3.0in}}
\ImageSpec Attribute & Type & Ptex header data or explanation \\
\hline
\qkw{ptex:meshType} & string & the mesh type, either
  \qkw{triangle} or \qkw{quad}. \\
\qkw{ptex:hasEdits} & int & nonzero if the Ptex file has edits. \\
\qkw{wrapmode} & string & the wrap mode as specified by the
  Ptex file. \\
\emph{other} & & Any other arbitrary metadata in the Ptex file will be stored
  directly as attributes in the \ImageSpec.
\end{tabular}



\vspace{.25in}

\section{RLA}
\label{sec:bundledplugins:rla}
\index{RLA}

RLA (Run-Length encoded, version A) is an early CGI renderer output format,
originating from Wavefront Advanced Visualizer and used primarily by software
developed at Wavefront.  RLA files commonly use the file extension {\cf .rla}.

%\subsubsection*{Attributes}
\vspace{.125in}

\noindent\begin{tabular}{p{1.75in}|p{0.5in}|p{3.0in}}
\ImageSpec Attribute & Type & RLA header data or explanation \\
\hline
{\cf width}, {\cf height}, {\cf x}, {\cf y} & & RLA ``active/viewable'' window. \\
& & \\
{\cf\small full_width}, {\cf\small full_height}, {\cf\small full_x}, 
  {\cf\small full_y} & & RLA ``full'' window.  \\
& & \\
\qkw{rla:FrameNumber} & int & frame sequence number. \\
\qkw{rla:Revision} & int & file format revision number, currently
  \qkw{0xFFFE}. \\
\qkw{rla:JobNumber} & int & job number ID of the file. \\
\qkw{rla:FieldRendered} & int & whether the image is a field-rendered
  (interlaced) one (\qkw{0} for false, non-zero for true). \\
\qkw{rla:FileName} & string & name under which the file was orignally saved. \\
\qkw{ImageDescription} & string & RLA ``Description'' of the image. \\
\qkw{Software} & string & name of software used to save the image. \\
\qkw{HostComputer} & string & name of machine used to save the image. \\
\qkw{Artist} & string & RLA ``UserName'': logon name of user who saved the image. \\
\qkw{rla:Aspect} & string & aspect format description string. \\
\qkw{rla:ColorChannel} & string & textual description of color channel data
  format (usually \qkw{rgb}). \\
\qkw{rla:Time} & string & description (format not standardized) of amount of
  time spent on creating the image. \\
\qkw{rla:Filter} & string & name of post-processing filter applied to the
  image. \\
\qkw{rla:AuxData} & string & textual description of auxiliary channel data
  format. \\
\qkw{rla:AspectRatio} & float & image aspect ratio. \\
\qkw{rla:RedChroma} & vec2 or vec3 of floats & red point XY (vec2) or XYZ
  (vec3) coordinates. \\
\qkw{rla:GreenChroma} & vec2 or vec3 of floats & green point XY (vec2) or XYZ
  (vec3) coordinates. \\
\qkw{rla:BlueChroma} & vec2 or vec3 of floats & blue point XY (vec2) or XYZ
  (vec3) coordinates. \\
\qkw{rla:WhitePoint} & vec2 or vec3 of floats & white point XY (vec2) or XYZ
  (vec3) coordinates. \\
\end{tabular}

\subsubsection*{Limitations}

\begin{itemize}
\item \product will only write 1 image to 1 file, multiple subimages
  are not supported by the writer (but are supported by the reader).
\end{itemize}



\vspace{.25in}

\section{SGI}
\label{sec:bundledplugins:sgi}
\index{SGI files}

The SGI image format was a simple raster format used long ago on SGI
machines.  SGI files use the file extensions {\cf sgi}, {\cf rgb}, 
{\cf rgba}, \qkw{bw}, \qkw{int}, and \qkw{inta}.

The SGI format is sometimes used for legacy apps, but has little merit
otherwise: no support for tiles, no MIPmaps, no multi-subimage, only 8-
and 16-bit integer pixels (no floating point), only 1-4 channels.

%\subsubsection*{Attributes}
\vspace{.125in}

\noindent\begin{tabular}{p{1.75in}|p{0.5in}|p{3.0in}}
\ImageSpec Attribute & Type & SGI header data or explanation \\
\hline
\qkw{ImageDescription} & string & image name \\
\qkw{Compression} & string & thee compression of the SGI file (\qkw{rle}, if
  RLE compression is used).
\end{tabular}



\vspace{.25in}

\section{Softimage PIC}
\label{sec:bundledplugins:pic}
\index{Softimage PIC}

Softimage PIC is an image file format used by the SoftImage 3D
application, and some other programs that needed to be compatible with
it.  Softimage files use the file extension {\cf .pic}.

The Softimage PIC format is sometimes used for legacy apps, but has
little merit otherwise, so currently \product only reads Softimage
files and is unable to write them.

%\subsubsection*{Attributes}
\vspace{.125in}

\noindent\begin{tabular}{p{1.75in}|p{0.5in}|p{3.0in}}
\ImageSpec Attribute & Type & PIC header data or explanation \\
\hline
\qkw{ImageDescription} & string & comment \\
\qkw{oiio:BitsPerSample} & int & the true bits per sample in the PIC file.
\end{tabular}



\vspace{.25in}

\section{RAW digital camera files}
\label{sec:bundledplugins:raw}
\index{RAW digital camera files}

A variety of digital camera ``raw'' formats are supported via this
plugin that is based on the LibRaw library ({\cf http://www.libraw.org/}).


% FIXME - fill in more docs here

\begin{comment}
%\subsubsection*{Attributes}
\vspace{.125in}

\noindent\begin{tabular}{p{1.75in}|p{0.5in}|p{3.0in}}
\ImageSpec Attribute & Type & PIC header data or explanation \\
\hline
\qkw{ImageDescription} & string & comment \\
\qkw{oiio:BitsPerSample} & int & the true bits per sample in the PIC file.
\end{tabular}
\end{comment}


\vspace{.25in}

\section{Targa}
\label{sec:bundledplugins:targa}
\index{Targa}

Targa (a.k.a.\ Truevision TGA) is an image file format with little merit
except that it is very simple and is used by many legacy applications.
Targa files use the file extension {\cf .tga}, or, much
more rarely, {\cf .tpic}.
The official Targa format specification may be found at\\
\url{http://www.dca.fee.unicamp.br/~martino/disciplinas/ea978/tgaffs.pdf}.

%\subsubsection*{Attributes}
\vspace{.125in}

\noindent\begin{tabular}{p{1.75in}|p{0.5in}|p{3.0in}}
\ImageSpec Attribute & Type & TGA header data or explanation \\
\hline
\qkw{ImageDescription} & string & comment \\
\qkw{Artist} & string & author \\
\qkw{DocumentName} & string & job name/ID \\
\qkw{Software} & string & software name \\
\qkw{DateTime} & string & TGA time stamp \\
\qkw{targa:JobTime} & string & TGA ``job time.'' \\
\qkw{Compression} & string & values of \qkw{none} and \qkw{rle} are
  supported.  The writer will use RLE compression if any unknown
  compression methods are requested. \\
\qkw{targa:ImageID} & string & Image ID \\
\qkw{PixelAspectRatio} & float & pixel aspect ratio \\
\qkw{oiio:BitsPerSample} & int & the true (in the file) bits per sample. \\
\end{tabular}
\\ 
\vspace{.25in}

If the TGA file contains a thumbnail, its dimensions will be
  stored in the attributes \qkw{thumbnail_width},
  \qkw{thumbnail_height}, and \qkw{thumbnail_nchannels}, and the
  thumbnail pixels themselves will be stored in \qkw{thumbnail_image}
  (as an array of UINT8 values, whose length is the total number of
  channel samples in the thumbnail).

\subsubsection*{Limitations}

\begin{itemize}
\item The Targa reader reserves enough memory for the entire image.
  Therefore it is not a good choice for high-performance image use such 
  as would be used for \ImageCache or \TextureSystem.
\item Targa files only support 8- and 16-bit unsigned integers (no
  signed, floating point, or HDR capabilities); the \product TGA writer
  will silently convert all output images to UINT8 (except if UINT16 is
  explicitly requested).
\item Targa only supports grayscale, RGB, and RGBA; the \product TGA
  writer will fail its call to {\cf open()} if it is asked create a file
  with more than 4 color channels.
\end{itemize}


\vspace{.25in}

\section{TIFF}
\label{sec:bundledplugins:tiff}
\index{TIFF}

TIFF (Tagged Image File Format) is a flexible file format created by
Aldus, now controlled by Adobe.  TIFF supports nearly everything anybody
could want in an image format (and has extactly the complexity you would
expect from such a requirement).
TIFF files commonly use the file extensions {\cf .tif} or, {\cf .tiff}.
Additionally, \product associates the following extensions with TIFF
files by default: {\cf .tx}, {\cf .env}, {\cf .sm}, {\cf .vsm}.

The official TIFF format specification may be found here:
\url{http://partners.adobe.com/public/developer/tiff/index.html} 
~ The most popular library for reading TIFF directly is {\cf libtiff},
available here: 
\url{http://www.remotesensing.org/libtiff/} ~ \product uses {\cf libtiff}
for its TIFF reading/writing.

We like TIFF a lot, especially since its complexity can be nicely hidden
behind OIIO's simple APIs.  It supports a wide variety of data formats
(though unfortunately not {\cf half}), an arbitrary number of channels,
tiles and multiple subimages (which makes it our preferred texture
format), and a rich set of metadata.

\product supports the vast majority of TIFF features, including: tiled
images (\qkw{tiled}) as well as scanline images; multiple subimages per
file (\qkw{multiimage}); MIPmapping (using multi-subimage; that means 
you can't use multiimage and MIPmaps simultaneously); data formats
8- 16, and 32 bit integer (both signed and unsigned), and 32- and 64-bit
floating point; palette images (will convert to RGB); ``miniswhite''
photometric mode (will convert to ``minisblack'').

The TIFF plugin attempts to support all the standard Exif, IPTC, and XMP
metadata if present.

%\subsubsection*{Attributes}
\vspace{.125in}

\noindent\begin{tabular}{p{2.0in}|p{0.5in}|p{2.75in}}
\ImageSpec Attribute & Type & TIFF header data or explanation \\
\hline
{\cf ImageSpec::x} & int & XPosition \\
{\cf ImageSpec::y} & int & YPosition \\
{\cf ImageSpec::full_width} & int & PIXAR\_IMAGEFULLWIDTH \\
{\cf ImageSpec::full_length} & int & PIXAR\_IMAGEFULLLENGTH \\
\qkw{ImageDescription} & string & ImageDescription \\
\qkw{DateTime} & string & DateTime \\
\qkw{Software} & string & Software \\
\qkw{Artist} & string & Artist \\
\qkw{Copyright} & string & Copyright \\
\qkw{Make} & string & Make \\
\qkw{Model} & string & Model \\
\qkw{DocumentName} & string & DocumentName \\
\qkw{HostComputer} & string & HostComputer \\
\qkws{XResultion} \qkws{YResolution} & float & XResolution, YResolution \\
\qkws{ResolutionUnit} & string & ResolutionUnit (\qkw{in} or
  \qkw{cm}). \\
\qkw{Orientation} & int & Orientation \\
\qkw{ICCProfile} & uint8[] & The ICC color profile \\
\qkw{textureformat} & string & {\cf PIXAR_TEXTUREFORMAT} \\
\qkw{wrapmodes} & string & {\cf PIXAR_WRAPMODES} \\
\qkw{fovcot} & float & {\cf PIXAR_FOVCOT} \\
\qkw{worldtocamera} & matrix & PIXAR\_MATRIX\_WORLDTOCAMERA \\
\qkw{worldtoscreen} & matrix & PIXAR\_MATRIX\_WORLDTOSCREEN\\
\qkw{comrpession} & string & based on TIFF Compression 
  (one of \qkw{none}, \qkw{lzw}, \qkw{ccittrle}, \qkw{zip}, \qkw{packbits}).\\
\qkw{tiff:compression} & int & the original integer code
  from the TIFF Compression tag.\\
\qkw{tiff:planarconfig} & string & PlanarConfiguration (\qkw{separate} or
  \qkw{contig}).  The \product TIFF writer will honor such a request in
  the \ImageSpec.\\
\qkwf{tiff:PhotometricInterpretation} & int & Photometric \\
\qkw{tiff:PageName} & string & PageName \\
\qkw{tiff:PageNumber} & int & PageNumber \\
\qkw{tiff:RowsPerStrip} & int & RowsPerStrip \\
\qkw{tiff:subfiletype} & 1 & SubfileType \\
\qkw{Exif:*} & & A wide variety of EXIF data are honored, and are all prefixed
  with \qkw{Exif:}.\\
\qkw{oiio:BitsPerSample} & int & The actual bits per sample in the file (may
  differ from {\cf ImageSpec::format}).\\
\qkw{oiio:UnassociatedAlpha} & int & Nonzero if the alpha channel
  contained ``unassociated'' alpha. \\
\end{tabular}

\subsubsection*{Limitations}

\product's TIFF reader and writer have some limitations you should be
aware of:
\begin{itemize}
\item No separate per-channel data formats (not supported by {\cf
  libtiff}).
\item Only multiples of 8 bits per pixel may be passed through
  \product's APIs, e.g., 1-, 2-, and 4-bits per pixel will be reported
  by OIIO as 8 bit images; 12 bits per pixel will be reported as 16,
  etc.  But the \qkw{oiio:BitsPerSample} attribute in the \ImageSpec
  will correctly report the original bit depth of the file.  Note that
  the TIFF specification itself does not support 16-bit floating point
  pixels ({\cf half} data).
\end{itemize}



\vspace{.25in}

\section{Webp}
\label{sec:bundledplugins:webp}
\index{WebP}

% FIXME


\vspace{.25in}

\section{Zfile}
\label{sec:bundledplugins:zfile}
\index{Zfile}

Zfile is a very simple format for writing a depth ($z$) image,
originally from Pixar's PhotoRealistic RenderMan but now supported by
many other renderers.  It's extremely minimal, holding only a width,
height, world-to-screen and camera-to-screen matrices, and uncompressed
float pixels of the z-buffer.
Zfile files use the file extension {\cf .zfile}.

%\subsubsection*{Attributes}
\vspace{.125in}

\noindent\begin{tabular}{p{1.75in}|p{0.5in}|p{3.0in}}
\ImageSpec Attribute & Type & Zfile header data or explanation \\
\hline
\qkw{worldtocamera} & matrix & NP \\
\qkw{worldtoscreen} & matrix & Nl \\
\end{tabular}



\index{Plugins!bundled|)}
\chapwidthend


%%%%%%%%%%%%%%%%%%%%%%%%%%%%%%%%%%%%%%%%


\begin{comment}

FOO () is an image file format.
Strengths.
FOO files use the file extension {\cf .foo}.

The official FOO format specification may be found at \url{} .

%\subsubsection*{Attributes}
\vspace{.125in}

\noindent\begin{tabular}{p{1.5in}|p{0.5in}|p{3.5in}}
\ImageSpec Attribute & Type & FOO header data or explanation \\
\hline
\end{tabular}

\subsubsection*{Limitations}

\begin{itemize}
\item blah
\end{itemize}

\end{comment}

\chapter{Cached Images}
\label{chap:imagecache}
\index{Image Cache|(}

\section{Image Cache Introduction and Theory of Operation}
\label{sec:imagecache:intro}

\ImageCache is a utility class that allows an application to read pixels
from a large number of image files while using a remarkably small amount
of memory and other resources.  Of course it is possible for an
application to do this directly using \ImageInput objects.  But
\ImageCache offers the following advantages:

\begin{itemize}
\item \ImageCache presents an even simpler user interface than
  \ImageInput --- the only supported operations are asking for an
  \ImageSpec describing a subimage in the file, retrieving for a block
  of pixels, and locking/reading/releasing individual tiles.  You refer
  to images by filename only; you don't need to keep track of individual
  file handles or \ImageInput objects.  You don't need to explicitly
  open or close files.

\item The \ImageCache is completely thread-safe; if multiple threads
  are accessing the same file, the \ImageCache internals will handle
  all the locking and resource sharing.

\item No matter how many image files you are accessing, the \ImageCache
  will maintain a reasonable number of simultaneously-open files,
  automatically closing files that have not been needed recently.

\item No matter how large the total pixels in all the image files you
  are dealing with are, the \ImageCache will use only a small amount of
  memory.  It does this by loading only the individual tiles requested,
  and as memory allotments are approached, automatically releasing the
  memory from tiles that have not been used recently.
\end{itemize}

In short, if you have an application that will need to read pixels from
many large image files, you can rely on \ImageCache to manage all the
resources for you.  It is reasonable to access thousands of image files
totalling hundreds of GB of pixels, efficiently and using a memory
footprint on the order of 50 MB.

\newpage
Below are some simple code fragments that shows \ImageCache in action:
\medskip

\begin{code}
    #include "OpenImageIO/imagecache.h"

    // Create an image cache and set some options
    ImageCache *cache = ImageCache::create ();
    cache->attribute ("max_memory_MB", 50.0);
    cache->attribute ("autotile", 64);

    // Get a block of pixels from a file.
    // (for brevity of this example, let's assume that 'size' is the
    // number of channels times the number of pixels in the requested region)
    float pixels[size];
    cache->get_pixels ("file1.jpg", 0, xbegin, xend, ybegin, yend,
                       zbegin, zend, TypeDesc::FLOAT, pixels);

    // Get information about a file
    ImageSpec spec;
    bool ok = cache->get_imagespec ("file2.exr", spec);
    if (ok)
        std::cout << "resolution is " << spec.width << "x" 
                  << "spec.height << "\n";

    // Request and hold a tile, do some work with its pixels, then release
    ImageCache::Tile *tile;
    tile = cache->get_tile ("file2.exr", 0, x, y, z);
    // The tile won't be freed until we release it, so this is safe:
    TypeDesc format;
    void *p = cache->tile_pixels (tile, format);
    // Now p points to the raw pixels of the tile, whose data format
    // is given by 'format'.
    cache->release_tile (tile);  
    // Now cache is permitted to free the tile when needed

    // Note that all files were referenced by name, we never had to open
    // or close any files, and all the resource and memory management 
    // was automatic.

    ImageCache::destroy (cache);
\end{code}

\newpage
\section{ImageCache API}
\label{sec:imagecache:api}

\subsection{Creating and destroying an image cache}
\label{sec:imagecache:api:createdestroy}

\ImageCache is an abstract API described as a pure virtual class.  The
actual internal implementation is not exposed through the external API
of \product.  Because of this, you cannot construct or destroy the
concrete implementation, so two static methods of \ImageCache are
provided:

\apiitem{static ImageCache *ImageCache::{\ce create} (bool shared=true)}

Creates a new \ImageCache and returns a pointer to it.  If 
{\cf shared} is {\cf true}, {\cf create()} will return a pointer
to a shared \ImageCache (so that multiple parts of an application
that request an \ImageCache will all end up with the same one).
If {\cf shared} is {\cf false}, a completely unique \ImageCache
will be created and returned.

\apiend

\apiitem{static void ImageCache::{\ce destroy} (ImageCache *x)}
Destroys an allocated \ImageCache, including freeing all system
resources that it holds.

This is necessary to ensure that the memory is freed in a way that
matches the way it was allocated within the library.  Note that simply
using {\cf delete} on the pointer will not always work (at least,
not on some platforms in which a DSO/DLL can end up using a different
allocator than the main program).

It is safe to destroy even a shared \ImageCache, as the implementation
of {\cf destroy()} will recognize a shared one and only truly release
its resources if it has been requested to be destroyed as many times as
shared \ImageCache's were created.
\apiend

\subsection{Setting options and limits for the image cache}
\label{sec:imagecache:api:attribute}

The following member functions of \ImageCache allow you to set
(and in some cases retrieve) options that control the overall
behavior of the image cache:

\apiitem{bool {\ce attribute} (const std::string \&name, TypeDesc type,
  const void *val)}
\indexapi{attribute}

Sets an attribute (i.e., a property or option) of the \ImageCache.
The {\cf name} designates the name of the attribute, {\cf type}
describes the type of data, and {\cf val} is a pointer to memory 
containing the new value for the attribute.

If the \ImageCache recognizes a valid attribute name that matches the
type specified, the attribute will be set to the new value and {\cf
  attribute()} will return {\cf true}.  If {\cf name} is not recognized
as a valid attribute name, or if the types do not match (e.g., {\cf
  type} is {\cf TypeDesc::FLOAT} but the named attribute is a string),
the attribute will not be modified, and {\cf attribute()} will return
{\cf false}.

Here are examples:

\begin{code}
      ImageCache *ts; 
      ...
      int maxfiles = 50;
      ts->attribute ("max_open_files", TypeDesc::INT, &maxfiles);

      const char *path = "/my/path";
      ts->attribute ("searchpath", TypeDesc::STRING, &path);
\end{code}

Note that when passing a string, you need to pass a pointer to the {\cf
  char*}, not a pointer to the first character.  (Rationale: for an {\cf
  int} attribute, you pass the address of the {\cf int}.  So for a
string, which is a {\cf char*}, you need to pass the address of the
string, i.e., a {\cf char**}).

The complete list of attributes can be found at the end of this section.

\apiend

\apiitem{bool {\ce attribute} (const std::string \&name, int val) \\
bool {\ce attribute} (const std::string \&name, float val) \\
bool {\ce attribute} (const std::string \&name, double val) \\
bool {\ce attribute} (const std::string \&name, const char *val) \\
bool {\ce attribute} (const std::string \&name, const std::string \& val)}
Specialized versions of {\cf attribute()} in which the data type is
implied by the type of the argument.

For example, the following are equivalent to the example above for the
general (pointer) form of {\cf attribute()}:

\begin{code}
      ts->attribute ("max_open_files", 50);
      ts->attribute ("searchpath", "/my/path");
\end{code}

\apiend


\apiitem{bool {\ce getattribute} (const std::string \&name, TypeDesc type,
  void *val)}
\indexapi{getattribute}

Gets the current value of an attribute of the \ImageCache.
The {\cf name} designates the name of the attribute, {\cf type}
describes the type of data, and {\cf val} is a pointer to memory 
where the user would like the value placed.

If the \ImageCache recognizes a valid attribute name that matches the
type specified, the attribute value will be stored at address {\cf val}
and {\cf attribute()} will return {\cf true}.  If {\cf name} is not recognized
as a valid attribute name, or if the types do not match (e.g., {\cf
  type} is {\cf TypeDesc::FLOAT} but the named attribute is a string),
no data will be written to {\cf val}, and {\cf attribute()} will return
{\cf false}.

Here are examples:

\begin{code}
      ImageCache *ts; 
      ...
      int maxfiles;
      ts->getattribute ("max_open_files", TypeDesc::INT, &maxfiles);

      const char *path;
      ts->getattribute ("searchpath", TypeDesc::STRING, &path);
\end{code}

Note that when passing a string, you need to pass a pointer to the {\cf
  char*}, not a pointer to the first character.  Also, the {\cf char*}
will end up pointing to characters owned by the \ImageCache; the
caller does not need to ever free the memory that contains the
characters.

The complete list of attributes can be found at the end of this section.


\apiend

\apiitem{bool {\ce getattribute} (const std::string \&name, int \&val) \\
bool {\ce getattribute} (const std::string \&name, float \&val) \\
bool {\ce getattribute} (const std::string \&name, double \&val) \\
bool {\ce getattribute} (const std::string \&name, char **val) \\
bool {\ce getattribute} (const std::string \&name, std::string \& val)}
Specialized versions of {\cf getattribute()} in which the data type is
implied by the type of the argument.

For example, the following are equivalent to the example above for the
general (pointer) form of {\cf getattribute()}:

\begin{code}
      int maxfiles;
      ts->getattribute ("max_open_files", &maxfiles);
      const char *path;
      ts->getattribute ("searchpath", &path);
\end{code}

\apiend


\subsubsection*{Image cache attributes}

Recognized attributes include the following:

\apiitem{int max_open_files}
The maximum number of file handles that the image cache will
hold open simultaneously.  (Default = 100)
\apiend

\apiitem{float max_memory_MB}
The maximum amount of memory (measured in MB) that the image cache
will use for its ``tile cache.'' (Default: 50.0 MB)
\apiend

\apiitem{string searchpath}
The search path for images: a colon-separated list of
directories that will be searched in order for any image name
that is not specified as an absolute path. (Default: no search path.)
\apiend

\apiitem{int autotile}
This attributes controls how the image cache deals with images that
are not ``tiled'' (i.e., are stored as scanlines). 

If {\cf autotile} is set to 0 (the default), an untiled image will be
treated as if it were a single tile of the resolution of the whole
image.  This is simple and fast, but can lead to poor cache behavior if
you are simultaneously accessing many large untiled images.

If {\cf autotile} is nonzero (e.g., 64 is a good recommended value), any
untiled images will be read and cached as if they were constructed in
tiles of size {\cf autotile} $\times$ {\cf autotile}.  This leads to
slightly more expensive disk access if you are using only a few
images, but if you are using many untiled images, the caching be much
more efficient.
\apiend

\apiitem{int automip}
If {\cf automip} is set to 0 (the default), an untiled single-subimage
file will only be able to utilize that single subimage.

If {\cf automip} is nonzero, any untiled, single-subimage
(un-MIP-mapped) images will have lower-resolution MIP-map levels
generated on-demand if pixels are requested from the lower-res subimages
(that don't really exist).  Essentially this makes the \ImageCache
pretends that the file is MIP-mapped even if it isn't.
\apiend

\apiitem{int forcefloat}
If set to nonzero, all image tiles will be converted to {\cf float} 
type when stored in the image cache.  This can be helpful especially
for users of \ImageBuf who want to simplify their image manipulations
to only need to consider {\cf float} data.

The default is zero, meaning that image pixels are not forced to
be {\cf float} when in cache.
\apiend

\apiitem{int accept_untiled}
When nonzero (the default), \ImageCache accepts untiled images as
usual.  When set to zero, \ImageCache will reject untiled images with
an error condition, as if the file could not be properly read.
This is sometimes helpful for applications that want to enforce use of
tiled images only.
\apiend

\apiitem{int failure_retries}
When an {\cf open()} or {\cf read_tile()} calls fails, pause and try
again, up to {\cf failure_retries} times before truly returning a
failure.  This is meant to address spooky disk or network failures.  The
default is zero, meaning that failures of open or tile reading will
immediately return as a failure.
\apiend

\bigskip

\subsection{Getting information about images}
\label{sec:imagecache:api:getimageinfo}
\label{sec:imagecache:api:getimagespec}

\apiitem{bool {\ce get_image_info} (ustring filename, ustring dataname, \\
  \bigspc\bigspc TypeDesc datatype, void *data)}
Retrieves information about the image named by {\cf filename}.
The {\cf dataname} is a keyword indcating what information should
be retrieved, {\cf datatype} is the type of data expected, and
{\cf data} points to caller-owned memory where the results should be
placed.  It is up to the caller to ensure that {\cf data} contains
enough space to hold an item of the requested {\cf datatype}.

The return value is {\cf true} if {\cf get_image_info()} is able
to find the requested {\cf dataname} and it matched the requested
{\cf datatype}.  If the requested data was not found, or was not
of the right data type, {\cf get_image_info()} will return {\cf false}.

Supported {\cf dataname} values include:

\begin{description}
\item[\spc] \spc
\vspace{-12pt} 
\item[\rm \kw{exists}] Return 1 if the file exists and
is an image format that OpenImageIO knows how to read, otherwise return
0.  The {\cf data} pointer is not used.

\item[\rm \kw{resolution}] The resolution of the image file, which
is an array of 2 integers (described as {\cf TypeDesc(INT,2)}).

\item[\rm \kw{texturetype}] A string describing the type of texture
of the given file, which describes how the texture may be used (also
which texture API call is probably the right one for it).
This currently may return one of: \qkw{unknown}, \qkw{Plain Texture},
\qkw{Volume Texture}, \qkw{Shadow}, 
or \qkw{Environment}.

\item[\rm \kw{textureformat}] A string describing the format of the
given file, which describes the kind of texture stored in the file.
This currently may return one of: \qkw{unknown}, \qkw{Plain Texture},
\qkw{Volume Texture}, \qkw{Shadow}, \qkw{CubeFace Shadow}, \qkw{Volume
  Shadow}, \qkw{LatLong Environment}, or \qkw{CubeFace Environment}.
Note that there are several kinds of shadows and environment maps,
all accessible through the same API calls.

\item[\rm \kw{channels}] The number of color channels in the file 
(an integer).

\item[\rm \kw{format}] The native data format of the pixels in the
  file (an integer, giving the {\cf TypeDesc::BASETYPE} of the data).
  Note that this is not necessarily the same as the data format stored
  in the image cache.

\item[\rm \kw{cachedformat}] The native data format of the pixels as
  stored in the image cache (an integer, giving the {\cf
    TypeDesc::BASETYPE} of the data).  Note that this is not necessarily
  the same as the native data format of the file.

\item[\rm \kw{viewingmatrix}] The viewing matrix, which is a
$4 \times 4$ matrix (an {\cf Imath::M44f}, described as {\cf
  TypeDesc(FLOAT,MATRIX)}).

\item[\rm \kw{projectionmatrix}] The projection matrix, which is a
$4 \times 4$ matrix (an {\cf Imath::M44f}, described as {\cf
  TypeDesc(FLOAT,MATRIX)}).

\item[Anything else] -- For all other data names, the
the metadata of the image file will be searched for an item that
matches both the name and data type.

\end{description}
\apiend

\apiitem{bool {\ce get_imagespec} (ustring filename, ImageSpec \&spec,
  int subimage=0)}

If the named image is found and able to be opened by an available
image format plugin, and the designated subimage exists, this function copies
its image specification for that subimage into {\cf spec} and returns
{\cf true}.  Otherwise, if the file is not found, could not be opened,
is not of a format readable by any plugin that could be found, or
the designated subimage did not exist in the file, the return value is
{\cf false} and {\cf spec} will not be modified.

\apiend

\apiitem{const ImageSpec * {\ce imagespec} (ustring filename, int subimage=0)}

If the named image is found and able to be opened by an available
image format plugin, and the designated subimage exists, this function
returns a pointer to an \ImageSpec that describes it.  Otherwise, if the
file is not found, could not be opened, is not of a format readable by
any plugin that could be find, or the designated subimage did
not exist in the file, the return value is NULL.

This method is much more efficient than {\cf get_imagespec()}, since it
just returns a pointer to the spec held internally by the \ImageCache
(rather than copying the spec to the user's memory).  However, the
caller must beware that the pointer is only valid as long as nobody
(even other threads) calls {\cf invalidate()} on the file, or {\cf
  invalidate_all()}, or destroys the \ImageCache.
\apiend

\apiitem{std::string {\ce resolve_filename} (const std::string \&filename)}
Returns the true path to the given file name, with searchpath logic
applied.
\apiend

\subsection{Getting pixels}
\label{sec:imagecache:api:getpixels}

\apiitem{bool {\ce get\_pixels} (ustring filename, int subimage, \\
         \bigspc int xbegin, int xend, int ybegin, int yend,
                            int zbegin, int zend, \\
         \bigspc TypeDesc format, void *result)}

Retrieve the rectangle of raw pixels
of the designated {\cf subimage}, storing the pixel values
beginning at the address specified by result.  The pixel values will be
converted to the type specified by {\cf format}.  It is up to the caller
to ensure that result points to an area of memory big enough to
accommodate the requested rectangle (taking into consideration its
dimensions, number of channels, and data format).
The rectangular region to be retrieved includes {\cf begin} but does not
include {\cf end} (much like STL begin/end usage).

\apiend

\subsection{Dealing with tiles}
\label{sec:imagecache:api:tiles}

\apiitem{ImageCache::Tile {\ce get_tile} (ustring filename, int subimage, \\
  \bigspc \bigspc int x, int y, int z)}
Find a tile given by an image {\cf filename}, {\cf subimage}, and pixel
coordinates.  An opaque pointer to the tile will be returned,
or {\cf NULL} if no such file (or tile within the file) exists or can
be read.  The tile will not be purged from the cache until 
after {\cf release_tile()} is called on the tile pointer.  This is
thread-safe.
\apiend

\apiitem{void {\ce release_tile} (ImageCache::Tile *tile)}
After finishing with a tile, {\cf release_tile()} will allow it to 
once again be purged from the tile cache if required.
\apiend

\apiitem{const void * {\ce tile_pixels} (ImageCache::Tile *tile,
  TypeDesc \&format)}
For a tile retrived by {\cf get_tile()}, return a pointer to the
pixel data itself, and also store in {\cf format} the data type that
the pixels are internally stored in (which may be different than
the data type of the pixels in the disk file).  This method should
only be called on a tile that has been requested by 
{\cf get_tile()} but has not yet been released with {\cf release_tile()}.
\apiend

\apiitem{void {\ce invalidate} (ustring filename)}
Invalidate any loaded tiles or open file handles associated with
the filename, so that any subsequent queries will be forced to
re-open the file or re-load any tiles (even those that were
previously loaded and would ordinarily be reused).  A client
might do this if, for example, they are aware that an image
being held in the cache has been updated on disk.  This is safe
to do even if other procedures are currently holding 
reference-counted tile pointers from the named image, but those 
procedures will not get updated pixels until they release the 
tiles they are holding.
\apiend

\apiitem{void {\ce invalidate_all} (bool force=false)}
Invalidate all loaded tiles and open file handles, so that any
subsequent queries will be forced to re-open the file or re-load any
tiles (even those that were previously loaded and would ordinarily be
reused).  A client might do this if, for example, they are aware that an
image being held in the cache has been updated on disk.  This is safe to
do even if other procedures are currently holding reference-counted tile
pointers from the named image, but those procedures will not get updated
pixels until they release the tiles they are holding.  If force is true,
everything will be invalidated, no matter how wasteful it is, but if
force is false, in actuality files will only be invalidated if their
modification times have been changed since they were first opened.
\apiend

\subsection{Errors and statistics}
\label{sec:imagecache:api:geterror}
\label{sec:imagecache:api:getstats}

\apiitem{std::string {\ce geterror} ()}
If any other API routines return {\cf false}, indicating that an error
has occurred, this routine will retrieve the error and clear the error
status.  If no error has occurred since the last time {\cf geterror()}
was called, it will return an empty string.
\apiend

\apiitem{std::string {\ce getstats} (int level=1)}
Returns a big string containing useful statistics about the \ImageCache
operations, suitable for saving to a file or outputting to the terminal.
The {\cf level} indicates the amount of detail in the statistics,
with higher numbers (up to a maximum of 5) yielding more and more
esoteric information.
\apiend


\index{Image Cache|)}

\chapwidthend

\chapter{Texture Access: {\cf TextureSystem}}
\label{chap:texturesystem}
\index{Texture System|(}

\def\TextureSystem{{\kw TextureSystem}\xspace}
\def\TextureOptions{{\kw TextureOptions}\xspace}
\def\TextureOptBatch{{\kw TextureOptBatch}\xspace}
\def\TextureOpt{{\kw TextureOpt}\xspace}


\section{Texture System Introduction and Theory of Operation}
\label{sec:texturesys:intro}

Coming soon.
FIXME

\section{Helper Classes}
\label{sec:texturesys:helperclasses}

\subsection{Imath}

The texture functinality of \product uses the excellent open source
{\cf Ilmbase} package's {\cf Imath} types when it requires 3D vectors
and transformation matrixes.  Specifically, we use {\cf Imath::V3f}
for 3D positions and directions, and {\cf Imath::M44f} for $4 \times 4$
transformation matrices.  To use these yourself, we recommend that you:

\begin{code}
    #include <OpenEXR/ImathVec.h>
    #include <OpenEXR/ImathMatrix.h>
\end{code}

Please refer to the {\cf Ilmbase} and {\cf OpenEXR}
documentation and header files for more complete information about
use of these types in your own application.  However, note that you
are not strictly required to use these classes in your application ---
{\cf Imath::V3f} has a memory layout identical to {\cf float[3]}
and {\cf Imath::M44f} has a memory layout identical to {\cf float[16]},
so as long as your own internal vectors and matrices have the same
memory layout, it's ok to just cast pointers to them when passing
as arguments to \TextureSystem methods.


\subsection{\TextureOpt}
\indexapi{TextureOpt}
\label{sec:textureopt}

\TextureOpt is a structure that holds many options controlling
single-point texture lookups.  Because each texture lookup API call takes
a reference to a \TextureOpt, the call signatures remain uncluttered
rather than having an ever-growing list of parameters, most of which
will never vary from their defaults.  Here is a brief description of
the data members of a \TextureOpt structure:

\apiitem{int firstchannel}
The beginning channel for the lookup.  For example, to retrieve just the blue
channel, you should have {\cf firstchannel} = 2 while passing {\cf nchannels} = 1
to the appropriate texture function.
\apiend

\apiitem{int subimage \\
ustring subimagename}
Specifies the subimage or face within the file to use for the texture lookup.
If {\cf subimagename} is set (it defaults to the empty string), it will
try to use the subimage that had a matching metadata
\qkw{oiio:subimagename}, otherwise the integer {\cf subimage} will be
used (which defaults to 0, i.e., the first/default subimage).  Nonzero
subimage indices only make sense for a texture file that supports
subimages or separate images per face (such as Ptex).  This will be
ignored if the file does not have multiple subimages or separate
per-face textures.
\apiend

\apiitem{Wrap swrap, twrap}
Specify the \emph{wrap mode} for 2D texture lookups (and 3D volume
texture lookups, using the additional {\cf rwrap} field).  These fields
are ignored for shadow and environment lookups.

These specify what happens when texture coordinates are found to be
outside the usual $[0,1]$ range over which the texture is defined.
{\cf Wrap} is an enumerated type that may take on any of the
following values:
\begin{description}
\item[\spc] \spc
\item[\rm \kw{WrapBlack}] The texture is black outside the [0,1] range.
\item[\rm \kw{WrapClamp}] The texture coordinates will be clamped to
  [0,1], i.e., the value outside [0,1] will be the same as the color
  at the nearest point on the border.
\item[\rm \kw{WrapPeriodic}] The texture is periodic, i.e., wraps back
  to 0 after going past 1.
\item[\rm \kw{WrapMirror}] The texture presents a mirror image at the
  edges, i.e., the coordinates go from 0 to 1, then back down to 0, then
  back up to 1, etc.
\item[\rm \kw{WrapDefault}] Use whatever wrap might be specified in the
  texture file itself, or some other suitable default (caveat emptor).
\end{description}

The wrap mode does not need to be identical in the $s$ and $t$
directions.
\apiend

\apiitem{float swidth, twidth}
For each direction, gives a multiplier for the derivatives.  Note that
a width of 0 indicates a point sampled lookup (assuming that blur is
also zero).  The default width is 1, indicating that the derivatives
should guide the amount of blur applied to the texture filtering (not
counting any additional \emph{blur} specified).
\apiend

\apiitem{float sblur, tblur}
For each direction, specifies an additional amount of pre-blur to apply
to the texture (\emph{after} derivatives are taken into account),
expressed as a portion of the width of the texture.  In other words,
blur = 0.1 means that the texture lookup should act as if the texture
was pre-blurred with a filter kernel with a width 1/10 the size of the
full image.  The default blur amount is 0, indicating a sharp texture
lookup.
\apiend

\apiitem{float fill}
Specifies the value that will be used for any color channels that are
requested but not found in the file.  For example, if you perform a
4-channel lookup on a 3-channel texture, the last channel will
get the fill value.  (Note: this behavior is affected by the
\qkw{gray_to_rgb} attribute described in 
Section~\ref{sec:texturesys:attributes}.)
\apiend

\apiitem{const float* missingcolor}
If not NULL, indicates that a missing or broken texture should \emph{not}
be treated as an error, but rather will simply return the supplied color
as the texture lookup color and {\cf texture()} will return {\cf true}.  
If the {\cf missingcolor} field is left at its default (a NULL pointer),
a missing or broken texture will be treated as an error and
{\cf texture()} will return {\cf false}.
Note: When not NULL, the data must point to \emph{nchannels} contiguous floats.
\apiend

\apiitem{float bias}
For shadow map lookups only, this gives the ``shadow bias'' amount.
\apiend

\apiitem{int samples}
For shadow map lookups only, the number of samples to use for the lookup.
\apiend

\apiitem{Wrap rwrap \\
float rblur, rwidth}
Specifies wrap, blur, and width for the third component of 3D volume texture
lookups.  These are not used for 2D texture lookups.
\apiend




\newpage
\section{TextureSystem Setup}
\label{sec:texturesys:api}

\subsection{Creating and destroying texture systems}
\label{sec:texturesys:api:createdestroy}

\TextureSystem is an abstract API described as a pure
virtual class.  The actual internal implementation is not exposed
through the external API of \product.  Because of this, you cannot
construct or destroy the concrete implementation, so two static
methods of \TextureSystem are provided:

\apiitem{static TextureSystem *TextureSystem::{\ce create} (bool share=true,\\
\bitspc\bigspc ImageCache *imagecache=nullptr)}
Creates a new \TextureSystem and returns a pointer to it.
If {\cf shared} is {\cf true}, the \TextureSystem returned will be a global
shared one, with a globally shared cache. If {\cf shared} is {\cf false}, a
new private \TextureSystem will be created, either with a newly created
private \ImageCache (if {\cf imagecache} is {\cf nullptr}), or with the
\ImageCache passed from (and owned by) the caller.
\apiend

\apiitem{static void TextureSystem::{\ce destroy} (TextureSystem *x, \\
\bigspc\bigspc bool teardown_imagecache=false)}
Destroys an allocated \TextureSystem, including freeing all system
resources that it holds (such as its underlying \ImageCache).

This is necessary to ensure that the memory is freed in a way that
matches the way it was allocated within the library.  Note that simply
using {\cf delete} on the pointer will not always work (at least,
not on some platforms in which a DSO/DLL can end up using a different
allocator than the main program).

If {\cf teardown_imagecache} is {\cf true}, and the \TextureSystem's
underlying \ImageCache is the \emph{shared} one, then that \ImageCache will
be thoroughly destroyed, not merely releasing the reference. \apiend

\subsection{Setting options and limits for the texture system}
\label{sec:texturesys:api:options}

The following member functions of \TextureSystem allow you to set
(and in some cases retrieve) options that control the overall
behavior of the texture system:

\apiitem{bool {\ce attribute} (string_view name, TypeDesc type,
  const void *val)}
\indexapi{attribute}

Sets an attribute (i.e., a property or option) of the \TextureSystem.
The {\cf name} designates the name of the attribute, {\cf type}
describes the type of data, and {\cf val} is a pointer to memory 
containing the new value for the attribute.

If the \TextureSystem recognizes a valid attribute name that matches the
type specified, the attribute will be set to the new value and {\cf
  attribute()} will return {\cf true}.  If {\cf name} is not recognized
as a valid attribute name, or if the types do not match (e.g., {\cf
  type} is {\cf TypeDesc::FLOAT} but the named attribute is a string),
the attribute will not be modified, and {\cf attribute()} will return
{\cf false}.

Here are examples:

\begin{code}
      TextureSystem *ts; 
      ...
      int maxfiles = 50;
      ts->attribute ("max_open_files", TypeDesc::INT, &maxfiles);

      const char *path = "/my/path";
      ts->attribute ("searchpath", TypeDesc::STRING, &path);
\end{code}

Note that when passing a string, you need to pass a pointer to the {\cf
  char*}, not a pointer to the first character.  (Rationale: for an {\cf
  int} attribute, you pass the address of the {\cf int}.  So for a
string, which is a {\cf char*}, you need to pass the address of the
string, i.e., a {\cf char**}).

The complete list of attributes can be found at the end of this section.

\apiend

\apiitem{bool {\ce attribute} (string_view name, int val) \\
bool {\ce attribute} (string_view name, float val) \\
bool {\ce attribute} (string_view name, double val) \\
bool {\ce attribute} (string_view name, string_view val)}
Specialized versions of {\cf attribute()} in which the data type is
implied by the type of the argument.

For example, the following are equivalent to the example above for the
general (pointer) form of {\cf attribute()}:

\begin{code}
      ts->attribute ("max_open_files", 50);
      ts->attribute ("searchpath", "/my/path");
\end{code}

\apiend


\apiitem{bool {\ce getattribute} (string_view name, TypeDesc type,
  void *val)}
\indexapi{getattribute}

Gets the current value of an attribute of the \TextureSystem.
The {\cf name} designates the name of the attribute, {\cf type}
describes the type of data, and {\cf val} is a pointer to memory 
where the user would like the value placed.

If the \TextureSystem recognizes a valid attribute name that matches the
type specified, the attribute value will be stored at address {\cf val}
and {\cf attribute()} will return {\cf true}.  If {\cf name} is not recognized
as a valid attribute name, or if the types do not match (e.g., {\cf
  type} is {\cf TypeDesc::FLOAT} but the named attribute is a string),
no data will be written to {\cf val}, and {\cf attribute()} will return
{\cf false}.

Here are examples:

\begin{code}
      TextureSystem *ts; 
      ...
      int maxfiles;
      ts->getattribute ("max_open_files", TypeDesc::INT, &maxfiles);

      const char *path;
      ts->getattribute ("searchpath", TypeDesc::STRING, &path);
\end{code}

Note that when passing a string, you need to pass a pointer to the {\cf
  char*}, not a pointer to the first character.  Also, the {\cf char*}
will end up pointing to characters owned by the \TextureSystem; the
caller does not need to ever free the memory that contains the
characters.

The complete list of attributes can be found at the end of this section.


\apiend

\apiitem{bool {\ce getattribute} (string_view name, int \&val) \\
bool {\ce getattribute} (string_view name, float \&val) \\
bool {\ce getattribute} (string_view name, double \&val) \\
bool {\ce getattribute} (string_view name, char **val) \\
bool {\ce getattribute} (string_view name, std::string \& val)}
Specialized versions of {\cf getattribute()} in which the data type is
implied by the type of the argument.

For example, the following are equivalent to the example above for the
general (pointer) form of {\cf getattribute()}:

\begin{code}
      int maxfiles;
      ts->getattribute ("max_open_files", &maxfiles);
      const char *path;
      ts->getattribute ("searchpath", &path);
\end{code}

\apiend


\subsubsection*{Texture system attributes}
\label{sec:texturesys:attributes}

Recognized attributes include the following:

\apiitem{int max_open_files \\
float max_memory_MB \\
string searchpath \\
string plugin_searchpath \\
int autotile \\
int autoscanline \\
int automip \\
int accept_untiled \\
int accept_unmipped \\
int failure_retries \\
int deduplicate \\
string substitute_image \\
int max_errors_per_file}

These attributes are all passed along to the underlying \ImageCache that
is used internally by the \TextureSystem.  Please consult the
\ImageCache attribute list in Section~\ref{sec:imagecache:api:attribute}
for explanations of these attributes.

\apiend

\apiitem{matrix worldtocommon}
The $4 \times 4$ matrix that provides the spatial transformation
from ``world'' to a ``common'' coordinate system.  This is used for
shadow map lookups, in which the shadow map itself encodes the
world coordinate system, but positions passed to {\cf shadow()} are
expressed in ``common'' coordinates.
\apiend

\apiitem{matrix commontoworld}
The $4 \times 4$ matrix that is the inverse of {\cf worldtocommon} ---
that is, it transforms points from ``common'' to ``world'' coordinates.

You do not need to set {\cf commontoworld} and {\cf worldtocommon}
separately; just setting either one will implicitly set the other, since
each is the inverse of the other.
\apiend

\apiitem{int gray_to_rgb}
If set to nonzero, texture lookups of single-channel (grayscale) 
images will replicate the sole channel's values into the next two
channels, making it behave like an RGB image that happens to have all
three channels with identical pixel values.  (Channels beyond the third
will get the ``fill'' value.)

The default value of zero means that all missing channels will get
the ``fill'' color.
\apiend

\apiitem{int max_tile_channels}
Sets the maximum number of color channels in a texture file for which all
channels will be loaded as cached tiles. Files with more than this number
of color channels will have only the requested subset loaded, in order
to save cache space (but at the possible wasted expense of separate tiles
that overlap their channel ranges). The default is 5.
\apiend

\apiitem{string latlong_up}
Sets the default ``up'' direction for latlong environment maps (only
applies if the map itself doesn't specify a format or is in a format
that explicitly requires a particular orientation).  The default is
\qkw{y}.  (Currently any other value will result in $z$ being ``up.'')
\apiend

\apiitem{int flip_t}
If nonzero, $t$ coordinates will be flipped ($1-t$) for texture lookups.
The default is 0.
\apiend

\apiitem{string options}
This catch-all is simply a comma-separated list of {\cf name=value}
settings of named options.  For example,
\begin{code}
        ic->attribute ("options", "max_memory_MB=512.0,autotile=1");
\end{code}
\apiend



\subsection{Opaque data for performance lookups}
\label{sec:texturesys:api:opaque}

\apiitem{Perthread * {\ce get_perthread_info} (Perthread *thread_info=NULL) \\
Perthread * {\ce create_perthread_info} () \\
void {\ce destroy_perthread_info} (Perthread *thread_info)}
\indexapi{get_perthread_info}
\indexapi{create_perthread_info} \indexapi{destroy_perthread_info}

The \TextureSystem implementation needs to maintain certain per-thread
state, and some \TextureSystem methods take an opaque {\cf Perthread} pointer
to this record. There are three options for how to deal with it:

1. Don't worry about it at all: don't use the methods that want {\cf
Perthread} pointers, or always pass {\cf NULL} for any {\cf Perthread*}
arguments, and \TextureSystem will do thread-specific-pointer retrieval as
necessary (though at some small cost).

2. If your app already stores per-thread information of its own, you may
call {\cf get_perthread_info(NULL)} to retrieve it for that thread, and then
pass it into the functions that allow it (thus sparing them the need and
expense of retrieving the thread-specific pointer). However, it is crucial
that this pointer not be shared between multiple threads. In this case,
the \TextureSystem manages the storage, which will automatically be released
when the thread terminates.

3. If your app also wants to manage the storage of the {\cf Perthread},
it can explicitly create one with {\cf create_perthread_info}, pass it around,
and eventually be responsible for destroying it with {\cf destroy_perthread_info}.
When managing the storage, the app may reuse the {\cf Perthread} for another
thread after the first is terminated, but still may not use the same
{\cf Perthread} for two threads running concurrently.
\apiend


\apiitem{TextureHandle * {\ce get_texture_handle} (ustring filename,\\
\bigspc\bigspc\bigspc  Perthread *thread_info=NULL)}
\indexapi{get_texture_handle}
Retrieve an opaque handle for fast texture lookups.  The optional opaque
pointer {\cf thread_info} is thread-specific information returned by
{\cf get_perthread_info()}.  Return {\cf NULL} if something has gone
horribly wrong.
\apiend

\apiitem{bool {\ce good} (TextureHandle *texture_handle)}
\indexapi{good}
Return true if the texture handle (previously returned by
{\cf get_texture_handle()}) is a valid image that can be subsequently read
or sampled.
\apiend


\newpage
\section{Texture Lookups -- single point}

\subsection{2D Texture Lookups}
\label{sec:texturesys:api:texture}

\apiitem{bool {\ce texture} (ustring filename, TextureOpt \&options,\\
\bigspc\spc                   float s, float t, float dsdx, float dtdx,\\
\bigspc\spc                   float dsdy, float dtdy, int nchannels, float *result),\\
\bigspc\spc                   float *dresultds=NULL, float *dresultdt=NULL)}
\indexapi{texture}

Perform a filtered 2D texture lookup on a position centered at 2D
coordinates ({\cf s}, {\cf t}) from the texture identified by
{\cf filename}, and using relevant texture {\cf options}.  The
{\cf nchannels} parameter determines the number of channels to retrieve
(e.g., 1 for a single value, 3 for an RGB triple, etc.).
The filtered results will be stored in {\cf result[0..nchannels-1]}.

We assume that this lookup will be part of an image that has pixel
coordinates {\cf x} and {\cf y}.  By knowing how {\cf s} and {\cf t}
change from pixel to pixel in the final image, we can properly
\emph{filter} or antialias the texture lookups.  This information is
given via derivatives {\cf dsdx} and {\cf dtdx} that define the change
in {\cf s} and {\cf t} per unit of {\cf x}, and {\cf dsdy} and {\cf
  dtdy} that define the change in {\cf s} and {\cf t} per unit of {\cf
  y}.  If it is impossible to know the derivatives, you may pass 0 for
them, but in that case you will not receive an antialiased texture lookup.

If the {\cf dresultds} and {\cf dresultdt} parameters are not {\cf NULL}
(the default), these specify locations in which to store the
\emph{derivatives} of the texture lookup, i.e., the change of the filtered
texture per unit of $s$ and $t$, respectively.  Each must point to at least
{\cf nchannels} contiguous floats.  If they are {\cf NULL}, the derivative
computations will not be performed.

Fields within {\cf options} that are honored for 2D texture lookups
include the following:

\vspace{-12pt}
\apiitem{int firstchannel}
\vspace{10pt}
The index of the first channel to look up from the texture.
\apiend

\vspace{-24pt}
\apiitem{int subimage}
\vspace{10pt}
The subimage or face within the file.
This will be ignored if the file does not have multiple subimages or
separate per-face textures.
\apiend

\vspace{-24pt}
\apiitem{Wrap swrap, twrap}
\vspace{10pt}
Specify the \emph{wrap mode} for each direction, one of: 
{\cf WrapBlack}, {\cf WrapClamp}, {\cf WrapPeriodic}, {\cf WrapMirror},
or {\cf WrapDefault}.
\apiend

\vspace{-24pt}
\apiitem{float swidth, twidth}
\vspace{10pt}
For each direction, gives a multiplier for the derivatives.
\apiend

\vspace{-24pt}
\apiitem{float sblur, tblur}
\vspace{10pt}
For each direction, specifies an additional amount of pre-blur to apply
to the texture (\emph{after} derivatives are taken into account),
expressed as a portion of the width of the texture.
\apiend

\vspace{-24pt}
\apiitem{float fill}
\vspace{10pt}
Specifies the value that will be used for any color channels that are
requested but not found in the file.  For example, if you perform a
4-channel lookup on a 3-channel texture, the last channel will
get the fill value.  (Note: this behavior is affected by the
\qkw{gray_to_rgb} attribute described in 
Section~\ref{sec:texturesys:attributes}.)
\apiend

\vspace{-24pt}
\apiitem{const float *missingcolor}
\vspace{10pt}
If not NULL, specifies the color that will be returned for missing or
broken textures (rather than being an error).
\apiend

This function returns {\cf true} upon success, or {\cf false} if the
file was not found or could not be opened by any available ImageIO
plugin.
\apiend


\apiitem{bool {\ce texture} (TextureHandle *texture_handle,
                          Perthread *thread_info, \\
\bigspc                   TextureOpt \&options, float s, float t, float dsdx, float dtdx,\\
\bigspc                   float dsdy, float dtdy, int nchannels, float *result,\\
\bigspc                   float *dresultds=NULL, float *dresultdt=NULL)}
A slightly faster {\cf texture} call for applications that are willing
to do the extra housekeeping of knowing the handle of the texture they
are accessing and the per-thread info for the curent thread.  These
may be retrieved by the {\cf get_texture_handle()} and 
{\cf get_perthread_info()} methods, respectively.
\apiend



%\newpage
\subsection{Volume Texture Lookups}
\label{sec:texturesys:api:texture3d}

\apiitem{bool {\ce texture3d} (ustring filename, TextureOpt \&options,\\
\bigspc\spc                    const Imath::V3f \&P, const Imath::V3f \&dPdx,\\
\bigspc\spc                    const Imath::V3f \&dPdy, const Imath::V3f \&dPdz,\\
\bigspc\spc                    int nchannels, float *result,\\
\bigspc\spc                   float *dresultds=NULL, float *dresultdt=NULL,\\
\bigspc\spc                   float *dresultdr=NULL)}
\indexapi{texture3d}

Perform a filtered 3D volumetric texture lookup on a position centered at
3D position {\cf P} from the texture identified by
{\cf filename}, and using relevant texture {\cf options}.  The filtered
results will be stored in {\cf result[0..nchannels-1]}.

We assume that this lookup will be part of an image that has pixel
coordinates {\cf x} and {\cf y} and depth {\cf z}.  
By knowing how {\cf P} changes from
pixel to pixel in the final image, and as we step in $z$ depth, we can properly \emph{filter} or
antialias the texture lookups.  This information is given via
derivatives {\cf dPdx}, {\cf dPdy}, and {\cf dPdz} that define the changes in {\cf P}
per unit of {\cf x}, {\cf y}, and {\cf z}, respectively.  If it is impossible to
know the derivatives, you may pass 0 for them, but in that case you will
not receive an antialiased texture lookup.

The {\cf P} coordinate and {\cf dPdx}, {\cf dPdy}, and {\cf dPdz}
derivatives are assumed to be in some kind of common global coordinate
system (usually \qkw{world} space) and will be automatically transformed
into volume local coordinates, if such a transormation is specified in
the volume file itself.

If the {\cf dresultds}, {\cf dresultdt}, and  {\cf dresultdr} parameters are
not {\cf NULL} (the default), these specify locations in which to store the
\emph{derivatives} of the texture lookup, i.e., the change of the filtered
texture per unit of $s$, $t$ and $r$, respectively.  Each must point to at least
{\cf nchannels} contiguous floats.  If they are {\cf NULL}, the derivative
computations will not be performed.

Fields within {\cf options} that are honored for 3D texture lookups
include the following:

\vspace{-12pt}
\apiitem{int firstchannel}
\vspace{10pt}
The index of the first channel to look up from the texture.
\apiend

\vspace{-24pt}
\apiitem{Wrap swrap, twrap, rwrap}
\vspace{10pt}
Specify the wrap modes for each direction, one of: 
{\cf WrapBlack}, {\cf WrapClamp}, {\cf WrapPeriodic}, {\cf WrapMirror},
or {\cf WrapDefault}.
\apiend

\vspace{-24pt}
\apiitem{float swidth, twidth, rwidth}
\vspace{10pt}
For each direction, gives a multiplier for the derivatives.
\apiend

\vspace{-24pt}
\apiitem{float sblur, tblur, rblur}
\vspace{10pt}
For each direction, specifies an additional amount of pre-blur to apply
to the texture (\emph{after} derivatives are taken into account),
expressed as a portion of the width of the texture.
\apiend

\vspace{-24pt}
\apiitem{float fill}
\vspace{10pt}
Specifies the value that will be used for any color channels that are
requested but not found in the file.  For example, if you perform a
4-channel lookup on a 3-channel texture, the last channel will
get the fill value.  (Note: this behavior is affected by the
\qkw{gray_to_rgb} attribute described in 
Section~\ref{sec:texturesys:attributes}.)
\apiend

\vspace{-24pt}
\apiitem{const float *missingcolor}
\vspace{10pt}
If not NULL, specifies the color that will be returned for missing or
broken textures (rather than being an error).
\apiend

\vspace{-24pt}
\apiitem{float time}
\vspace{10pt}
A time value to use if the volume texture specifies a time-varying
local transformation (default: 0).
\apiend

This function returns {\cf true} upon success, or {\cf false} if the
file was not found or could not be opened by any available ImageIO
plugin.

\apiend

\apiitem{bool {\ce texture3d} (TextureHandle *texture_handle,
                          Perthread *thread_info, \\
\bigspc\spc                   TextureOpt \&opt, const Imath::V3f \&P, const Imath::V3f \&dPdx,\\
\bigspc\spc                          const Imath::V3f \&dPdy, const Imath::V3f \&dPdz,\\
\bigspc\spc                          int nchannels, float *result, float *dresultds=NULL,\\
\bigspc\spc                          float *dresultdt=NULL, float *dresultdr=NULL)}
A slightly faster {\cf texture3d} call for applications that are willing
to do the extra housekeeping of knowing the handle of the texture they
are accessing and the per-thread info for the curent thread.  These
may be retrieved by the {\cf get_texture_handle()} and 
{\cf get_perthread_info()} methods, respectively.
\apiend


%\newpage
\subsection{Shadow Lookups}
\label{sec:texturesys:api:shadow}

\apiitem{bool {\ce shadow} (ustring filename, TextureOpt \&opt,\\
\bigspc                         const Imath::V3f \&P, const Imath::V3f \&dPdx,\\
\bigspc                         const Imath::V3f \&dPdy, int nchannels, float *result,\\
\bigspc                         float *dresultds=NULL, float *dresultdt=NULL)}
\indexapi{shadow}

Perform a shadow map lookup on a position centered at 3D
coordinate {\cf P} (in a designated ``common'' space) from the shadow map identified by
{\cf filename}, and using relevant texture {\cf options}.  The filtered
results will be stored in {\cf result[]}.

We assume that this lookup will be part of an image that has pixel
coordinates {\cf x} and {\cf y}.  By knowing how {\cf P} changes from
pixel to pixel in the final image, we can properly \emph{filter} or
antialias the texture lookups.  This information is given via
derivatives {\cf dPdx} and {\cf dPdy} that define the changes in {\cf P}
per unit of {\cf x} and {\cf y}, respectively.  If it is impossible to
know the derivatives, you may pass 0 for them, but in that case you will
not receive an antialiased texture lookup.

Fields within {\cf options} that are honored for 2D texture lookups
include the following:

\vspace{-12pt}
\apiitem{float swidth, twidth}
\vspace{10pt}
For each direction, gives a multiplier for the derivatives.
\apiend

\vspace{-24pt}
\apiitem{float sblur, tblur}
\vspace{10pt}
For each direction, specifies an additional amount of pre-blur to apply
to the texture (\emph{after} derivatives are taken into account),
expressed as a portion of the width of the texture.
\apiend

\vspace{-24pt}
\apiitem{float bias}
\vspace{10pt}
Specifies the amount of \emph{shadow bias} to use --- this effectively
ignores shadow occlusion that is closer than the bias amount to the
surface, helping to eliminate self-shadowing artifacts.
\apiend

\vspace{-24pt}
\apiitem{int samples}
\vspace{10pt}
Specifies the number of samples to use when evaluating the shadow map.
More samples will give a smoother, less noisy, appearance to the
shadows, but may also take longer to compute.
\apiend

This function returns {\cf true} upon success, or {\cf false} if the
file was not found or could not be opened by any available ImageIO
plugin.
\apiend

\apiitem{bool {\ce shadow} (TextureHandle *texture_handle,
                          Perthread *thread_info, \\
\bigspc                   TextureOpt \&opt, const Imath::V3f \&P,\\
\bigspc                         const Imath::V3f \&dPdx, const Imath::V3f \&dPdy,\\
\bigspc                         int nchannels, float *result,\\
\bigspc                         float *dresultds=NULL, float *dresultdt=NULL)}
A slightly faster {\cf shadow} call for applications that are willing
to do the extra housekeeping of knowing the handle of the texture they
are accessing and the per-thread info for the curent thread.  These
may be retrieved by the {\cf get_texture_handle()} and 
{\cf get_perthread_info()} methods, respectively.
\apiend


%\newpage
\subsection{Environment Lookups}
\label{sec:texturesys:api:environment}

\apiitem{bool {\ce environment} (ustring filename, TextureOpt \&options,\\
\bigspc\spc                      const Imath::V3f \&R, const Imath::V3f \&dRdx,\\
\bigspc\spc                      const Imath::V3f \&dRdy, int nchannels, float *result,\\
\bigspc\spc                     float *dresultds=NULL, float *dresultdt=NULL)}
\indexapi{environment}

Perform a filtered directional environment map lookup in the direction
of vector {\cf R}, from the texture identified by {\cf filename}, and
using relevant texture {\cf options}.  The filtered results will be
stored in {\cf result[]}.

We assume that this lookup will be part of an image that has pixel
coordinates {\cf x} and {\cf y}.  By knowing how {\cf R} changes from
pixel to pixel in the final image, we can properly \emph{filter} or
antialias the texture lookups.  This information is given via
derivatives {\cf dRdx} and {\cf dRdy} that define the changes in {\cf R}
per unit of {\cf x} and {\cf y}, respectively.  If it is impossible to
know the derivatives, you may pass 0 for them, but in that case you will
not receive an antialiased texture lookup.

Fields within {\cf options} that are honored for 3D texture lookups
include the following:

\vspace{-12pt}
\apiitem{int firstchannel}
\vspace{10pt}
The index of the first channel to look up from the texture.
\apiend

\vspace{-24pt}
\apiitem{float swidth, twidth}
\vspace{10pt}
For each direction, gives a multiplier for the derivatives.
\apiend

\vspace{-24pt}
\apiitem{float sblur, tblur}
\vspace{10pt}
For each direction, specifies an additional amount of pre-blur to apply
to the texture (\emph{after} derivatives are taken into account),
expressed as a portion of the width of the texture.
\apiend

\vspace{-24pt}
\apiitem{float fill}
\vspace{10pt}
Specifies the value that will be used for any color channels that are
requested but not found in the file.  For example, if you perform a
4-channel lookup on a 3-channel texture, the last channel will
get the fill value.  (Note: this behavior is affected by the
\qkw{gray_to_rgb} attribute described in 
Section~\ref{sec:texturesys:attributes}.)
\apiend

This function returns {\cf true} upon success, or {\cf false} if the
file was not found or could not be opened by any available ImageIO
plugin.
\apiend

\apiitem{bool {\ce environment} (TextureHandle *texture_handle,
                          Perthread *thread_info, \\
\bigspc\spc               TextureOpt \&opt, const Imath::V3f \&R,\\
\bigspc\spc               const Imath::V3f \&dRdx, onst Imath::V3f \&dRdy,\\
\bigspc\spc               int nchannels, float *result,\\
\bigspc\spc               float *dresultds=NULL, float *dresultdt=NULL)}
A slightly faster {\cf environment} call for applications that are willing
to do the extra housekeeping of knowing the handle of the texture they
are accessing and the per-thread info for the curent thread.  These
may be retrieved by the {\cf get_texture_handle()} and 
{\cf get_perthread_info()} methods, respectively.
\apiend



\section{Batched Texture Lookups}
\label{sec:texturesys:api:batched}
\index{SIMD} \index{batched texture lookups}

On CPU architectures with SIMD processing, texturing entire batches of
samples at once may provide a large speedup compared to texturing each
sample point individually. The batch size is fixed (for any build of
\product) and may be accessed with the following constant:

\apiitem{static const int {\ce Tex::BatchWidth}}
This constant specifies the batch size. This is fixed within any release
of \product, but may change from release to release and also may be
overridden at build time. A typical batch size is 16.
\apiend

All of the batched calls take a \emph{run mask}, which describes which
subset of ``lanes'' should be computed by the batched lookup:

\apiitem{typedef ... {\ce Tex::RunMask}}

The {\cf RunMask} is defined to be an integer large enough to hold at least
{\cf BatchWidth} bits. The least significant bit corresponds to the first
(i.e., {\cf [0]}) position of all batch arrays. For each position
{\cf i} in the batch, the bit identified by {\cf (1 << i)} controls whether
that position will be computed.

The defined constant {\cf RunMaskOn} contants the value with all bits
{\cf 0..BatchWidth-1} set to 1.
\apiend

\subsection{Batched Options}

\apiitem{class {\ce TextureOptBatch}}
\label{sec:textureoptbatch}

\TextureOptBatch is a structure that holds the options for doing an entire
batch of lookups from the same texture at once.
The members of \TextureOptBatch correspond to the similarly named members of
the single-point \TextureOpt, so we refer you to
Section~\ref{sec:textureopt} for detailed explanations, and this section
will only explain the differences between batched and single-point options.

\apiitem{int firstchannel \\
int subimage \\
ustring subimagename \\
Tex::Wrap swrap, twrap, rwrap \\
Tex::MipMode mipmode \\
Tex::InterpMode interpmode \\
int anisotropic \\
bool conservative_filter \\
float fill \\
const float *missingcolor } ~\\
These fields are all scalars --- a single value for each \TextureOptBatch
--- which means that the value of these options must be the same for every
texture sample point within a batch. If you have a number of texture lookups
to perform for the same texture, but they have (for example) differing wrap
modes or subimages from point to point, then you must split them into
separate batch calls.
\apiend

\apiitem{float sblur[Tex::BatchWidth] \\
float tblur[Tex::BatchWidth] \\
float rblur[Tex::BatchWidth] } ~\\
These arrays hold the $s$, and $t$ blur amounts, for each sample in the
batch, respectively. (And the $r$ blur amount, used only for volumetric
{\cf texture3d()} lookups.)
\apiend

\apiitem{float swidth[Tex::BatchWidth] \\
float twidth[Tex::BatchWidth] \\
float rwidth[Tex::BatchWidth] } ~\\
These arrays hold the $s$, and $t$ filtering width multiplier for
derivatives, for each sample in the batch, respectively. (And the $r$
multiplier, used only for volumetric {\cf texture3d()} lookups.)
\apiend
\apiend

\subsection{Batched Texture Lookup Calls}

\apiitem{bool {\ce texture} (ustring filename, TextureOptBatch \&options,\\
\bigspc                   Tex::RunMask mask, const float *s, const float *t,\\
\bigspc                   const float *dsdx, const float *dtdx,\\
\bigspc                   const float *dsdy, const float *dtdy,\\
\bigspc                   int nchannels, float *result,\\
\bigspc                   float *dresultds=nullptr, float *dresultdt=nullptr) \\[2ex]
bool {\ce texture} (TextureHandle *texture_handle,
                          Perthread *thread_info, \\
\bigspc                   TextureOptBatch \&options,\\
\bigspc                   Tex::RunMask mask, const float *s, const float *t,\\
\bigspc                   const float *dsdx, const float *dtdx,\\
\bigspc                   const float *dsdy, const float *dtdy,\\
\bigspc                   int nchannels, float *result,\\
\bigspc                   float *dresultds=nullptr, float *dresultdt=nullptr) \\
}

Perform filtered 2D texture lookups on a batch of positions from the
same texture, all at once.  The parameters {\cf s},
{\cf t}, {\cf dsdx}, {\cf dtdx}, and {\cf dsdy}, {\cf dtdy} are each
a pointer to {\cf [BatchSize]} values.  The {\cf mask} determines which
of those array elements to actually compute.

The various results are arranged as arrays that behave as if they were
declared

~~~~ {\cf  float result[channels][BatchSize]}

\noindent In other words, all the batch values for channel 0 are
adjacent, followed by all the batch values for channel 1, etc. (This is
``SOA'' order.)

This function returns {\cf true} upon success, or {\cf false} if the
file was not found or could not be opened by any available ImageIO
plugin.
\apiend

\apiitem{bool {\ce texture3d} (ustring filename, TextureOptBatch \&options,\\
\bigspc                        Tex::RunMask mask, const float *P, const float *dPdx,\\
\bigspc                        const float *dPdy, const float *dPdz,\\
\bigspc                        int nchannels, float *result, float *dresultds=nullptr,\\
\bigspc                        float *dresultdt=nullptr,float *dresultdr=nullptr)\\[2ex]
bool {\ce texture3d} (TextureHandle *texture_handle,
                      Perthread *thread_info, \\
\bigspc               TextureOptBatch \&options,\\
\bigspc               Tex::RunMask mask, const float *P, const float *dPdx,\\
\bigspc               const float *dPdy, const float *dPdz,\\
\bigspc               int nchannels, float *result, float *dresultds=nullptr,\\
\bigspc               float *dresultdt=nullptr, float *dresultdr=nullptr)}

Perform filtered 3D volumetric texture lookups on a batch of positions from
the same texture, all at once. The ``point-like'' parameters {\cf P}, {\cf
dPdx}, {\cf dPdy}, and {\cf dPdz} are each a pointers to arrays of
{\cf float value[3][BatchSize]}. That is, each one points to all the $x$ values
for the batch, immediately followed by all the $y$ values, followed by the
$z$ values.

The various results arrays are also arranged as arrays that behave as if
they were declared {\cf  float result[channels][BatchSize]}, where all the
batch values for channel 0 are adjacent, followed by all the batch values
for channel 1, etc.

This function returns {\cf true} upon success, or {\cf false} if the
file was not found or could not be opened by any available ImageIO
plugin.
\apiend

\begin{comment}
\apiitem{bool {\ce shadow} (ustring filename, TextureOptBatch \&options,\\
\bigspc                         Tex::RunMask mask, \\
\bigspc                         const float *P, const float *dPdx,\\
\bigspc                         const float *dPdy, int nchannels, float *result,\\
\bigspc                         float *dresultds=nullptr, float *dresultdt=nullptr)\\[2ex]
bool {\ce shadow} (TextureHandle *texture_handle,  Perthread *thread_info, \\
\bigspc                         TextureOptBatch \&options,\\
\bigspc                         Tex::RunMask mask, \\
\bigspc                         const float *P, const float *dPdx,\\
\bigspc                         const float *dPdy, int nchannels, float *result,\\
\bigspc                         float *dresultds=nullptr, float *dresultdt=nullptr)}

Perform filtered shadow map lookups on a batch of positions from
the same texture, all at once. The ``point-like'' parameters {\cf P},
{\cf dPdx}, and {\cf dPdy} are each a pointers to arrays of
{\cf float value[3][BatchSize]}. That is, each one points to all the $x$ values
for the batch, immediately followed by all the $y$ values, followed by the
$z$ values.

The various results arrays are also arranged as arrays that behave as if
they were declared {\cf  float result[channels][BatchSize]}, where all the
batch values for channel 0 are adjacent, followed by all the batch values
for channel 1, etc.

This function returns {\cf true} upon success, or {\cf false} if the
file was not found or could not be opened by any available ImageIO
plugin.
\apiend
\end{comment}

\apiitem{bool {\ce environment} (ustring filename, TextureOptBatch \&options,\\
\bigspc\spc                      Tex::RunMask mask, \\
\bigspc\spc                      const float *R, const float *dRdx,\\
\bigspc\spc                      const float *dRdy, int nchannels, float *result,\\
\bigspc\spc                      float *dresultds=nullptr, float *dresultdt=nullptr)\\[2ex]
bool {\ce environment} (TextureHandle *texture_handle,
                        Perthread *thread_info, \\
\bigspc\spc             TextureOptBatch \&options, Tex::RunMask mask, \\
\bigspc\spc             const float *R, const float *dRdx,\\
\bigspc\spc             const float *dRdy, int nchannels, float *result,\\
\bigspc\spc             float *dresultds=nullptr, float *dresultdt=nullptr)}

Perform filtered directional environment map lookups on a batch of positions
from the same texture, all at once. The ``point-like'' parameters {\cf R},
{\cf dRdx}, and {\cf dRdy} are each a pointers to arrays of
{\cf float value[3][BatchSize]}. That is, each one points to all the $x$ values
for the batch, immediately followed by all the $y$ values, followed by the
$z$ values.

Perform filtered directional environment map lookups on a collection of
directions all at once, which may be much more efficient than repeatedly
calling the single-point version of {\cf environment()}.  The parameters
{\cf R}, {\cf dRdx}, and {\cf dRdy} are now {\cf VaryingRef}'s that may
refer to either a single or an array of values, as are many the fields in
the {\cf options}.

The various results arrays are also arranged as arrays that behave as if
they were declared {\cf  float result[channels][BatchSize]}, where all the
batch values for channel 0 are adjacent, followed by all the batch values
for channel 1, etc.

This function returns {\cf true} upon success, or {\cf false} if the
file was not found or could not be opened by any available ImageIO
plugin.
\apiend


%\newpage
\section{Texture Metadata and Raw Texels}
\label{sec:texturesys:api:gettextureinfo}
\label{sec:texturesys:api:getimagespec}

\apiitem{bool {\ce get_texture_info} (ustring filename, int subimage, \\
\bigspc\spc\spc ustring dataname, TypeDesc datatype, void *data) \\
bool {\ce get_texture_info} (TextureHandle *texture_handle, \\
\bigspc\spc\spc Perthread *thread_info, int subimage, \\
\bigspc\spc\spc ustring dataname, TypeDesc datatype, void *data)}

Retrieves information about the texture, either named by {\cf filename} or
specified by an opaque handle returned by {\cf get_texture_handle()}.
The {\cf dataname} is a keyword indcating what information should
be retrieved, {\cf datatype} is the type of data expected, and
{\cf data} points to caller-owned memory where the results should be
placed.  It is up to the caller to ensure that {\cf data} contains
enough space to hold an item of the requested {\cf datatype}.

The return value is {\cf true} if {\cf get_texture_info()} is able to answer
the query -- that is, find the requested {\cf dataname} for the texture and
it matched the requested {\cf datatype}.  If the requested data was not
found, or was not of the right data type, {\cf get_texture_info()} will
return {\cf false}. Except for the \qkw{exists} and \qkw{udim} queries, file that does not
exist or could not be read properly as an image also constitutes a query
failure that will return {\cf false}.

Supported {\cf dataname} values include:

\begin{description}
\item[\spc] \spc \vspace{-12pt} 

\item[\rm \kw{exists}] Stores the value 1 (as an {\cf int} if the file
exists and is an image format that \product can read, or 0 if the file
does not exist, or could not be properly read as a texture. Note that
unlike all other queries, this query will ``succeed'' (return {\cf true})
even if the file does not exist.

\item[\rm \kw{udim}] Stores the value 1 (as an {\cf int}) if the file
is a ``virtual UDIM'' or texture atlas file (as described in
Section~\ref{sec:texturesys:udim}) or 0 otherwise.

\item[\rm \kw{subimages}] The number of subimages/faces in the file, as an integer.

\item[\rm \kw{resolution}] The resolution of the texture file, which
is an array of 2 integers (described as {\cf TypeDesc(INT,2)}).

\item[\rm \kw{resolution} (int[3])] The 3D resolution of the texture file, which
is an array of 3 integers (described as {\cf TypeDesc(INT,3)})  The
third value will e 1 unless it's a volumetric (3D) image.

\item[\rm \kw{miplevels}] The number of MIPmap levels for the specified
subimage (an integer).

\item[\rm \kw{texturetype}] A string describing the type of texture
of the given file, which describes how the texture may be used (also
which texture API call is probably the right one for it).
This currently may return one of: \qkw{unknown}, \qkw{Plain Texture},
\qkw{Volume Texture}, \qkw{Shadow}, 
or \qkw{Environment}.

\item[\rm \kw{textureformat}] A string describing the format of the
given file, which describes the kind of texture stored in the file.
This currently may return one of: \qkw{unknown}, \qkw{Plain Texture},
\qkw{Volume Texture}, \qkw{Shadow}, \qkw{CubeFace Shadow}, \qkw{Volume
  Shadow}, \qkw{LatLong Environment}, or \qkw{CubeFace Environment}.
Note that there are several kinds of shadows and environment maps,
all accessible through the same API calls.

\item[\rm \kw{channels}] The number of color channels in the file 
(an integer).

\item[\rm \kw{format}] The native data format of the pixels in the
  file (an integer, giving the {\cf TypeDesc::BASETYPE} of the data).
  Note that this is not necessarily the same as the data format stored
  in the image cache.

\item[\rm \kw{cachedformat}] The native data format of the pixels as
  stored in the image cache (an integer, giving the {\cf
    TypeDesc::BASETYPE} of the data).  Note that this is not necessarily
  the same as the native data format of the file.

\item[\rm \kw{datawindow}] 
Returns the pixel data window of the image, which is either an array of 4
integers (returning xmin, ymin, xmax, ymax) or an array of 6 integers
(returning xmin, ymin, zmin, xmax, ymax, zmax). The $z$ values may be useful
for 3D/volumetric images; for 2D images they will be 0).

\item[\rm \kw{displaywindow}] 
Returns the display (a.k.a.\ full) window of the image, which is either an
array of 4 integers (returning xmin, ymin, xmax, ymax) or an array of 6
integers (returning xmin, ymin, zmin, xmax, ymax, zmax). The $z$ values may
be useful for 3D/volumetric images; for 2D images they will be 0).

\item[\rm \kw{worldtocamera}] The viewing matrix, which is a $4 \times 4$
matrix (an {\cf Imath::M44f}, described as {\cf TypeDesc(FLOAT,MATRIX)}),
giving the world-to-camera 3D transformation matrix that was used when  the
image was created. Generally, only rendered images will have this.

\item[\rm \kw{worldtoscreen}] The projection matrix, which is a $4 \times 4$
matrix (an {\cf Imath::M44f}, described as {\cf TypeDesc(FLOAT,MATRIX)}),
giving the matrix that projected points from world space into a 2D screen
coordinate system where $x$ and $y$ range from $-1$ to $+1$.  Generally,
only rendered images will have this.

\item[\rm \kw{averagecolor}] If available in the metadata (generally only
for files that have been processed by {\cf maketx}), this will return the
average color of the texture (into an array of floats).

\item[\rm \kw{averagealpha}] If available in the metadata (generally only
for files that have been processed by {\cf maketx}), this will return the
average alpha value of the texture (into a float).

\item[\rm \kw{constantcolor}] If the metadata (generally only for files that
have been processed by {\cf maketx}) indicates that the texture has the same
values for all pixels in the texture, this will retrieve the constant color
of the texture (into an array of floats). A non-constant image (or one that
does not have the special metadata tag identifying it as a constant texture)
will fail this query (return false).

\item[\rm \kw{constantalpha}] If the metadata indicates that the texture has
the same values for all pixels in the texture, this will retrieve the
constant alpha value of the texture (into a float). A non-constant image (or
one that does not have the special metadata tag identifying it as a constant
texture) will fail this query (return false).

\item[\rm \kw{stat:tilesread}] Number of tiles read from this file ({\cf int64}).

\item[\rm \kw{stat:bytesread}] Number of bytes of uncompressed pixel data read
from this file ({\cf int64}).

\item[\rm \kw{stat:redundant_tiles}] Number of times a tile was read, where
the same tile had been rad before. ({\cf int64}).

\item[\rm \kw{stat:redundant_bytesread}] Number of bytes (of uncompressed pixel
data) in tiles that were read redundantly. ({\cf int64}).

\item[\rm \kw{stat:redundant_bytesread}] Number of tiles read from this file ({\cf int}).

\item[\rm \kw{stat:timesopened}] Number of times this file was opened ({\cf int}).

\item[\rm \kw{stat:iotime}] Time (in seconds) spent on all I/O for this file ({\cf float}).

\item[\rm \kw{stat:mipsused}] Stores 1 if any MIP levels beyond the highest
resolution were accesed, otherwise 0. ({\cf int})

\item[\rm \kw{stat:is_duplicate}] Stores 1 if this file was a duplicate of
another image, otherwise 0. ({\cf int})

\item[Anything else] -- For all other data names, the
the metadata of the image file will be searched for an item that
matches both the name and data type.

\end{description}
\apiend

\apiitem{bool {\ce get_imagespec} (ustring filename, int subimage, ImageSpec \&spec) \\
bool {\ce get_imagespec} (TextureHandle *texture_handle, Perthread *thread_info,\\
  \bigspc\bigspc int subimage, ImageSpec \&spec)}

If the image (specified by either name or handle)
is found and able to be opened by an available
image format plugin, this function copies its image specification into
{\cf spec} and returns {\cf true}.  Otherwise, if the file is not
found, could not be opened, or is not of a format readable by any
plugin that could be found, the return value is {\cf false}.
\apiend


\apiitem{const ImageSpec * {\ce imagespec} (ustring filename, int subimage) \\
const ImageSpec * {\ce imagespec} (TextureHandle *texture_handle, \\
\bigspc\bigspc Perthread *thread_info, int subimage)}

If the named image is found and able to be opened by an available
image format plugin, and the designated subimage exists, this function
returns a pointer to an \ImageSpec that describes it.  Otherwise, if the
file is not found, could not be opened, is not of a format readable by
any plugin that could be find, or the designated subimage did
not exist in the file, the return value is NULL.

This method is much more efficient than {\cf get_imagespec()}, since it
just returns a pointer to the spec held internally by the underlying \ImageCache
(rather than copying the spec to the user's memory).  However, the
caller must beware that the pointer is only valid as long as nobody
(even other threads) calls {\cf invalidate()} on the file, or {\cf
  invalidate_all()}, or destroys the \TextureSystem.
\apiend

\apiitem{bool {\ce get_texels} (ustring filename, TextureOpt \&options, int miplevel, \\
\bigspc                       int xbegin, int xend, int ybegin, int yend,\\
\bigspc                       int zbegin, int zend, int chbegin, int chend,\\
\bigspc                       TypeDesc format, void *result) \\
bool {\ce get_texels} (TextureHandle *texture_handle, PerThread *thread_info, \\
\bigspc                       Perthread *thread_info, TextureOpt \&options, int miplevel, \\
\bigspc                       int xbegin, int xend, int ybegin, int yend,\\
\bigspc                       int zbegin, int zend, int chbegin, int chend,\\
\bigspc                       TypeDesc format, void *result)}

For a texture identified by either name or handle,
retrieve a rectangle of raw unfiltered texels at the named MIP-map level, storing
the texel values beginning at the address specified by result.
Note that the face/subimage is communicated through {\kw options.subimage}.
The texel values will be converted to the type specified by
format.  It is up to the caller to ensure that result points to
an area of memory big enough to accommodate the requested
rectangle (taking into consideration its dimensions, number of
channels, and data format).  The rectangular region to be
retrieved includes {\cf begin} but does not include {\cf end} (much
like STL begin/end usage).
Requested pixels that are not part of the valid pixel data region of the
image file will be filled with zero values.

Fields within {\cf options} that are honored for raw texel retieval
include the following:

\vspace{-12pt}
\apiitem{int subimage}
\vspace{10pt}
The subimage to retrieve.
\apiend

% FIXME -- we should support this
%\vspace{-24pt}
%\apiitem{Wrap swrap, twrap}
%\vspace{10pt}
%Specify the \emph{wrap mode} for each direction, one of: 
%{\cf WrapBlack}, {\cf WrapClamp}, {\cf WrapPeriodic}, {\cf WrapMirror},
%or {\cf WrapDefault}.
%\apiend

\vspace{-24pt}
\apiitem{float fill}
\vspace{10pt}
Specifies the value that will be used for any color channels that are
requested but not found in the file.  For example, if you perform a
4-channel lookup on a 3-channel texture, the last channel will
get the fill value.  (Note: this behavior is affected by the
\qkw{gray_to_rgb} attribute described in 
Section~\ref{sec:texturesys:attributes}.)
\apiend

Return true if the file is found and could be opened by an
available ImageIO plugin, otherwise return false.

\apiend

\apiitem{std::string {\ce resolve_filename} (const std::string \&filename)}
Returns the true path to the given file name, with searchpath logic
applied.
\apiend

\section{Miscellaneous -- Statistics, errors, flushing the cache}
\label{sec:texturesys:api:geterror}
\label{sec:texturesys:api:getstats}
\label{sec:texturesys:api:resetstats}
\label{sec:texturesys:api:invalidate}

\apiitem{std::string {\ce geterror} ()}
\index{error checking}
If any other API routines return {\cf false}, indicating that an
error has occurred, this routine will retrieve the error and clear
the error status.  If no error has occurred since the last time
{\cf geterror()} was called, it will return an empty string.
\apiend

\apiitem{std::string {\ce getstats} (int level=1, bool icstats=true)}
Returns a big string containing useful statistics about the \ImageCache
operations, suitable for saving to a file or outputting to the terminal.
The {\cf level} indicates the amount of detail in the statistics,
with higher numbers (up to a maximum of 5) yielding more and more
esoteric information.  If {\cf icstats} is true, the returned string
will also contain all the statistics of the underlying \ImageCache,
but if false will only contain texture-specific statistics.
\apiend

\apiitem{void {\ce reset_stats} ()}
Reset most statistics to be as they were with a fresh
\ImageCache.  Caveat emptor: this does not flush the cache
itelf, so the resulting statistics from the next set of texture
requests will not match the number of tile reads, etc., that
would have resulted from a new \ImageCache.
\apiend

\apiitem{void {\ce invalidate} (ustring filename)}
Invalidate any loaded tiles or open file handles associated with
the filename, so that any subsequent queries will be forced to
re-open the file or re-load any tiles (even those that were
previously loaded and would ordinarily be reused).  A client
might do this if, for example, they are aware that an image
being held in the cache has been updated on disk.  This is safe
to do even if other procedures are currently holding 
reference-counted tile pointers from the named image, but those 
procedures will not get updated pixels until they release the 
tiles they are holding.
\apiend

\apiitem{void {\ce invalidate_all} (bool force=false)}
Invalidate all loaded tiles and open file handles, so that any
subsequent queries will be forced to re-open the file or re-load any
tiles (even those that were previously loaded and would ordinarily be
reused).  A client might do this if, for example, they are aware that an
image being held in the cache has been updated on disk.  This is safe to
do even if other procedures are currently holding reference-counted tile
pointers from the named image, but those procedures will not get updated
pixels until they release the tiles they are holding.  If force is true,
everything will be invalidated, no matter how wasteful it is, but if
force is false, in actuality files will only be invalidated if their
modification times have been changed since they were first opened.
\apiend

\apiitem{void {\ce close} (ustring filename) \\
void {\ce close_all} ()}
\NEW % 1.9
Close any open file handles associated with a named file, or for all
files, but do not invalidate any image spec information or pixels
associated with the files.  A client might do this in order to
release OS file handle resources, or to make it safe for other
processes to modify cached files.
\apiend

\subsection{UDIM and texture atlases}
\label{sec:texturesys:udim}
The {\cf texture()} call supports virtual filenames that expand per lookup
for UDIM and other tiled texture atlas techniques. The substitutions will
occur if the texture filename initially passed to {\cf texture()} does not
exist as a concrete file and contains one or more of the following
substrings:

\medskip

%\noindent
\begin{tabular}{p{0.75in} p{4.75in}}
{\cf <UDIM>} & 1001 + \emph{utile} + \emph{vtile}*10 \\
{\cf <u>} & \emph{utile} \\
{\cf <v>} & \emph{vtile} \\
{\cf <U>} & \emph{utile} + 1 \\
{\cf <V>} & \emph{vtile} + 1 \\
\end{tabular}

\medskip

\noindent where the tile numbers are derived from the input $u,v$ texture
coordinates as follows:

\begin{code}
    // Each unit square of texture is a different tile
    utile = max (0, int(u));
    vtile = max (0, int(v));
    // Re-adjust the texture coordinates to the offsets within the tile
    u = u - utile;
    v = v - vtile;
\end{code}
\smallskip

\noindent Example:

\begin{code}
    ustring filename ("paint.<UDIM>.tif");
    float s = 1.4, t = 3.8;
    texsys->texture (filename, s, t, ...);
\end{code}

\noindent will retrieve from file \qkw{paint.1032.tif} at coordinates $(0.4,0.8)$.

\smallskip

Please note that most other calls, including most queries for {\cf
get_texture_info()}, will fail with one of these special filenames, since
it's not a real file and the system doesn't know which concrete file you it
corresponds to in the absence of specific texture coordinates.


\index{Texture System|)}

\chapwidthend


\chapter{Python Bindings}

\chapter{Image Buffer}

\part{Image Utilities}

\chapter{{\kw oiiotool}: the OIIO Swiss Army Knife}
\label{chap:oiiotool}
\indexapi{oiiotool}

\section{Overview}


The \oiiotool program will read images (from any file format for which
an \ImageInput plugin can be found), perform various operations on them,
and write images (in any format for which an \ImageOutput plugin can be
found).

The \oiiotool utility is invoked as follows:

\medskip

\hspace{0.25in} \oiiotool \emph{args}

\medskip

\oiiotool maintains an \emph{image stack}, with the top image in the
stack also called the \emph{current image}.  The stack begins containing
no images.

\oiiotool arguments consist of image names, or commands.  When an
image name is encountered, that image is pushed on the stack and becomes
the new \emph{current image}.

Most other commands either alter the current image (replacing it with
the alteration), or in some cases will pull more than one image off the
stack (such as the current image and the next item on the stack) and
then push a new image.




\section{\oiiotool Tutorial / Recipes}

This section will give quick examples of common uses of \oiiotool to get
you started.  They should be fairly intuitive, but you can read the
subsequent sections of this chapter for all the details on every
command.

\subsection*{Printing information about images}

To print the name, format, resolution, and data type of an image
(or many images):

\begin{code}
    oiiotool --info *.tif
\end{code}

\noindent To also print the full metadata about each input image, use both
{\cf --info} and {\cf -v}:

\begin{code}
    oiiotool --info -v *.tif
\end{code}

\noindent To print info about all subimages and/or MIP-map levels of each
input image, use the {\cf -a} flag:

\begin{code}
    oiiotool --info -v -a mipmap.exr
\end{code}

\noindent To print statistics giving the minimum, maximum, average, and
standard deviation of each channel of an image, as well as other
information about the pixels:

\begin{code}
    oiiotool --stats img_2012.jpg
\end{code}

\noindent The {\cf --info}, {\cf --stats}, {\cf -v}, and {\cf -a} flags may
be used in any combination.


\subsection*{Converting between file formats}

It's a snap to convert among image formats supported by \product
(i.e., for which \ImageInput and \ImageOutput plugins can be found).
The \oiiotool utility will simply infer the file format from the
file extension. The following example converts a PNG image to JPEG:

\begin{code}
    oiiotool lena.png -o lena.jpg
\end{code}

The first argument ({\cf lena.png}) is a filename, causing \oiiotool to
read the file and makes it the current image.  The {\cf -o} command
outputs the current image to the filename specified by the next
argument.

Thus, the above command should be read to mean, ``Read {\cf lena.png}
into the current image, then output the current image as {\cf lena.jpg}
(using whatever file format is traditionally associated with the {\cf
  .jpg} extension).''


\subsection*{Comparing two images}

To print a report of the differences between two images of the same
resolution:

\begin{code}
    oiiotool old.tif new.tif --diff
\end{code}

\noindent If you also want to save an image showing just the differences:

\begin{code}
    oiiotool old.tif new.tif --diff --sub --abs -o diff.tif
\end{code}

This looks complicated, but it's really simple: read {\cf old.tif},
read {\cf new.tif} (pushing {\cf old.tif} down on the image stack),
report the differences between them, subtract {\cf new.tif} from 
{\cf old.tif} and replace them both with the difference image,
replace that with its absolute value, then save that image to 
{\cf diff.tif}.


\subsection*{Changing the data format or bit depth}

Just use the {\cf -d} option to specify a pixel data format for all
subsequent outputs.  For example, assuming that {\cf in.tif} uses 16-bit
unsigned integer pixels, the following will convert it to an 8-bit
unsigned pixels:

\begin{code}
    oiiotool in.tif -d uint8 -o out.tif
\end{code}


\subsection*{Changing the compression}

The following command converts writes a TIFF file, specifically using
LZW compression:

\begin{code}
    oiiotool in.tif --compression lzw -o compressed.tif
\end{code}

The following command writes its results as a JPEG file at a 
compression quality of 50 (pretty severe compression):

\begin{code}
    oiiotool big.jpg --quality 50 -o small.jpg
\end{code}



\subsection*{Converting between scanline and tiled images}

Convert a scanline file to a tiled file with $16 \times 16$ tiles:

\begin{code}
    oiiotool s.tif --tile 16 16 -o t.tif
\end{code}

\noindent Convert a tiled file to scanline:

\begin{code}
    oiiotool t.tif --scanline -o s.tif
\end{code}



\subsection*{Adding captions or metadata}

\begin{code}
    oiiotool foo.jpg --caption "Hawaii vacation" -o bar.jpg
    oiiotool foo.jpg --keyword "volcano,lava" -o bar.jpg
\end{code}


\subsection*{Resize an image}

\begin{code}
    oiiotool original.tif --resize 200% -o big.tif
    oiiotool original.tif --resize 25% -o small.tif
    oiiotool original.tif --resize 1024x768 -o specific.tif
\end{code}


\subsection*{Color convert an image}

This command linearizes a JPEG assumed to be in sRGB, saving as
an HDRI OpenEXR file:

\begin{code}
    oiiotool photo.jpg --colorconvert sRGB linear -o output.exr
\end{code}

\noindent And the other direction:

\begin{code}
    oiiotool render.exr --colorconvert linear sRGB -o fortheweb.png
\end{code}

\noindent This converts between two named color spaces (presumably
defined by your facility's OpenColorIO configuration):

\begin{code}
    oiiotool in.dpx --colorconvert lg10 lnf -o out.exr
\end{code}



\section{\oiiotool commands: general}

\apiitem{--help}
Prints usage information to the terminal.
\apiend

\apiitem{-v}
Verbose status messages --- print out more information about what
\oiiotool is doing at every step.
\apiend

\apiitem{-q}
Quet mode --- print out less information about what \oiiotool is doing
(only errors).
\apiend

\apiitem{-a}
Performs all operations on all subimages and/or MIPmap levels of each
input image.  Without {\cf -a}, generally each input image will really
only read the top-level MIPmap of the first subimage of the file.
\apiend

\apiitem{--info}
Prints information about each input image as it is read.  If verbose mode
is turned on ({\cf -v}), all the metadata for the image is printed.
If verbose mode is not turned on, only the resolution and data format
are printed.
\apiend

\apiitem{--stats}
Prints detailed statistical information about each input image as it is
read.
\apiend

\apiitem{--diff}
This command computes the difference of the current image and the next
image on the stack, and prints a report of those differences (how
many pixels differed, the maximum amount, etc.).  This command does not
alter the image stack.
\apiend

\apiitem{--no-clobber}
Sets ``no clobber'' mode, in which existing images on disk will never be 
overridden, even if the {\cf -o} command specifies that file.
\apiend

\apiitem{--threads \emph{n}}
Use \emph{n} execution threads if it helps to speed up image operations.
The default (also if $n=0$) is to use as many threads as there are cores
present in the hardware.
\apiend

\begin{comment}
\apiitem{--inplace}
Causes the output to \emph{replace} the input file, rather than create a
new file with a different name.

Without this flag, \oiiotool expects two file names, which will
be used to specify the input and output files, respectively.

But when {\cf --inplace} option is used, any number of file names $\ge 1$ may
be specified, and the image conversion commands are applied to each file
in turn, with the output being saved under the original file name.  This
is useful for applying the same conversion to many files.  

For example, the following example will add the caption ``Hawaii
vacation'' to all JPEG files in the current directory:

\begin{code}
        oiiotool --inplace --adjust-time --caption "Hawaii vacation" *.jpg
\end{code}
\apiend
\end{comment}


\section{\oiiotool commands: reading and writing images}

The commands described in this section read images, write images,
or control the way that subsequent images will be written upon output.

\apiitem{\rm \emph{filename}}
If a command-line option is the name of an image file, that file will
be read and will become the new \emph{current image}, with the previous
current image pushed onto the image stack.
\apiend

\apiitem{-o \rm \emph{filename}}
Outputs the current image to the named file.  This does not remove the
current image, it merely saves a copy of it.
\apiend

\apiitem{-d {\rm \emph{datatype}}}

Attempts to set the pixel data type of all subsequent outputs.  Valid
choices are: {\cf uint8}, {\cf sint8}, 
{\cf uint16}, {\cf sint16}, {\cf  half}, {\cf float}, {\cf double}.

The types {\cf uint10} and {\cf uint12} may be used to request 10- or
12-bit unsigned integers.  If the output file format does not support
them, {\cf uint16} will be substituted.

If the {\cf -d} option is not supplied, the output data type will
be the same as the data format of the input files, if possible.

In any case, if the output file type does not support the requested
data type, it will instead use whichever supported data type results
in the least amount of precision lost.
\apiend

% FIXME -- no it doesn't!
%\apiitem{-g {\rm \emph{gamma}}}
%Applies a gamma correction of $1/\mathrm{gamma}$ to the pixels as they
%are output.
%\apiend

%\apiitem{--sRGB}
%Explicitly tags the image as being in sRGB color space.  Note that this
%does not alter pixel values, it only marks which color space those
%values refer to (and only works for file formats that understand such
%things).  An example use of this command is if you have an image 
%that is not explicitly marked as being in any particular color space,
%but you know that the values are sRGB.
%\apiend

\apiitem{--scanline}
Requests that subsequent output files be scanline-oriented, if scanline
orientation is supported by the output file format.  By default, the
output file will be scanline if the input is scanline, or tiled if the
input is tiled.
\apiend

\apiitem{--tile {\rm \emph{x}} {\rm \emph{y}}}
Requests that subsequent output files be tiled, with the given $x \times y$ 
tile size, if tiled images are supported by the output format.
By default, the output file will take on the tiledness and tile size
of the input file.
\apiend

\apiitem{--compression {\rm \emph{method}}}
Sets the compression method for subsequent output images.  Each
\ImageOutput plugin will have its own set of methods that it supports.
By default, the output image will use the same compression technique as
the input image (assuming it is supported by the output format,
otherwise it will use the default compression method of the output
plugin).  
\apiend

\apiitem{--quality {\rm \emph{q}}}
Sets the compression quality, on a 1--100 floating-point scale.
This only has an effect if the particular compression method supports
a quality metric (as JPEG does).
\apiend

\apiitem{--planarconfig {\rm \emph{config}}}
Sets the planar configuration of subsequent outputs (if supported by
their formats).  Valid choices are: {\cf config} for contiguous (or
interleaved) packing of channels in the file (e.g., RGBRGBRGB...), 
{\cf separate} for separate channel planes (e.g.,
RRRR...GGGG...BBBB...), or {\cf default} for the default choice for the
given format.  This command will be ignored for output files whose 
file format does not support the given choice.
\apiend

\apiitem{--adjust-time}
When this flag is present, after writing each output, the resulting
file's modification time will be adjusted to match any \qkw{DateTime}
metadata in the image.  After doing this, a directory listing will show
file times that match when the original image was created or captured,
rather than simply when \oiiotool was run.  This has no effect on
image files that don't contain any \qkw{DateTime} metadata.
\apiend

\apiitem{--noautocrop}
For subsequent outputs, do \emph{not} automatically crop images whose
formats don't support separate pixel data and full/display windows.
Without this, the default is that outputs will be cropped or padded with
black as necessary when written to formats that don't support the
concepts of pixel data windows and full/display windows.  This is a
non-issue for file formats that support these concepts, such as OpenEXR.
\apiend

\section{\oiiotool commands that change the current image metadata}

This section describes \oiiotool commands that alter the metadata
of the current image, but do not alter its pixel values.  Only the
current (i.e., top of stack) image is affected, not any images further
down the stack.

If the {\cf -a} flag has previously been set, these commands apply to
all subimages or MIPmap levels of the current top image.  Otherwise,
they only apply to the highest-resolution MIPmap level of the first
subimage of the current top image.

\apiitem{--attrib {\rm \emph{name value}}}
Adds or replaces metadata with the given \emph{name} to have the 
specified \emph{value}.

It will try to infer the type of the metadata from the value: if the
value contains only numerals (with optional leading minus sign), it will
be saved as {\cf int} metadata; if it also contains a decimal point, it
will be saved as {\cf float} metadata; otherwise, it will be saved as
a {\cf string} metadata.
\apiend

\apiitem{--sattrib {\rm \emph{name value}}}
Adds or replaces metadata with the given \emph{name} to have the 
specified \emph{value}, forcing it to be interpreted as a {\cf string}.
This is helpful if you want to set a {\cf string} metadata to a value
that the {\cf --attrib} command would normally interpret as a number.
\apiend

\apiitem{--caption {\rm \emph{text}}}
Sets the image metadata \qkw{ImageDescription}.
This has no effect if the output image format does not support some kind
of title, caption, or description metadata field.
Be careful to enclose \emph{text} in quotes if you want your caption to
include spaces or certain punctuation!
\apiend

\apiitem{--keyword {\rm \emph{text}}}
Adds a keyword to the image metadata \qkw{Keywords}.  Any existing
keywords will be preserved, not replaced, and the new keyword will not
be added if it is an exact duplicate of existing keywords.  This has no
effect if the output image format does not support some kind of keyword
field.  

Be careful to enclose \emph{text} in quotes if you want your keyword to
include spaces or certain punctuation.  For image formats that have only
a single field for keywords, \OpenImageIO will concatenate the keywords,
separated by semicolon (`;'), so don't use semicolons within your
keywords.
\apiend

\apiitem{--clear-keywords}
Clears all existing keywords in the current image.
\apiend

\apiitem{--attrib {\rm \emph{name text}}}
Sets the named image metadata attribute to a string given by
\emph{text}.  For example, you could explicitly set the IPTC location
metadata fields with:

\begin{code}
        oiiotool --attrib "IPTC:City" "Berkeley" in.jpg out.jpg
\end{code}
\apiend

\apiitem{--orientation {\rm \emph{orient}}}
Explicitly sets the image's \qkw{Orientation} metadata to a numeric
value (see Section~\ref{metadata:orientation} for the numeric codes).
This only changes the metadata field that specifies
how the image should be displayed, it does NOT alter the pixels
themselves, and so has no effect for image formats that don't
support some kind of orientation metadata.
\apiend

\apiitem{--rotcw \\
--rotccw \\
--rot180}
Adjusts the image's \qkw{Orientation} metadata by rotating it $90^\circ$
clockwise, $90^\circ$ degrees counter-clockwise, or $180^\circ$,
respectively, compared to its current setting.  This only changes the
metadata field that specifies how the image should be displayed, it does
NOT alter the pixels themselves, and so has no effect for image formats
that don't support some kind of orientation metadata.
\apiend

\apiitem{--origin {\rm \emph{offset}}}
Set the pixel data window origin.  The offset is in the form
\begin{code}
     [+-]x[+-]y
\end{code}
\noindent Examples: 
\begin{code}
    --origin +20+10           x=20, y=10
    --origin +0-40            x=0, y=-40
\end{code}
\apiend

\apiitem{--fullsize {\rm \emph{size}}}
Set the display/full window size and/or offset.  The size is in the
form 
\\ \emph{width}\,{\cf x}\,\emph{height}{\cf [+-]}\emph{xoffset}{\cf
  [+-]}\emph{yoffset} \\
If either the offset or resolution is omitted, it will remain
unchanged.

\noindent Examples: 

\begin{tabular}{p{2in} p{4in}}
    {\cf --fullsize 1920x1080}  &      resolution w=1920, h=1080, offset unchanged \\
    {\cf --fullsize -20-30} &          resolution unchanged, x=-20, y=-30 \\
    {\cf --fullsize 1024x768+100+0}  & resolution w=1024, h=768, offset
    x=100, y=0
\end{tabular}

\apiend

\apiitem{--fullpixels}
Set the full/display window range to exactly cover the pixel data window.
\apiend



\section{\oiiotool commands that make new images}

\apiitem{--create {\rm \emph{size channels}}}

Create new black image with the given size and number of channels,
pushing it onto the image stack and making it the new current image.

The \emph{size} is in the form
\\ \emph{width}\,{\cf x}\,\emph{height}{\cf [+-]}\emph{xoffset}{\cf
  [+-]}\emph{yoffset} \\
If the offset is omitted, it will be $x=0,y=0$.

\noindent Examples:

\begin{tabular}{p{2in} p{4in}}
    {\cf --create 1920x1080 3}  &      RGB with w=1920, h=1080, x=0, y=0 \\
    {\cf --create 1024x768+100+0 4}  & RGBA with w=1024, h=768, x=100, y=0
\end{tabular}
\apiend

\apiitem{--unmip}
If the current image is MIP-mapped, discard all but the top level
(i.e., replacing the current image with a new image consisting of only the
highest-resolution level).
\apiend

\apiitem{--selectmip {\rm \emph{level}}}
If the current image is MIP-mapped, replace the current image with a new
image consisting of only the given \emph{level} of the MIPmap.
Level 0 is the highest resolution version, level 1 is the next-lower
resolution version, etc.
\apiend

\apiitem{--subimage {\rm \emph{n}}}
If the current image has multiple subimages, replace the current image
with a new image consisting of only the given subimage.
\apiend

\apiitem{--add}
Replace the \emph{two} top images with a new image that is the sum of
those images.
\apiend

\apiitem{--sub}
Replace the \emph{two} top images with a new image that is the difference
between the next-to-top and the top image.
\apiend

\apiitem{--abs}
Replace the current image with a new image that has each pixel
consisting of the \emph{absolute value} of he old pixel value.
\apiend

\apiitem{--flip}
Replace the current image with a new image that is flipped vertically,
with the top scanline becoming the bottom, and vice versa.
\apiend

\apiitem{--flop}
Replace the current image with a new image that is flopped horizontally,
with the leftmost column becoming the rightmost, and vice versa.
\apiend

\apiitem{--flipflop}
Replace the current image with a new image that is both flipped and
flopped, which is the same as a 180 degree rotation.
\apiend

\apiitem{--crop {\rm \emph{size}}}
Replace the current image with a new copy with the given \emph{size},
cropping old pixels no longer needed, padding black pixels where they
previously did not exist in the old image, and adjusting the offsets
if requested.

The size is in the form 
\\ \spc\spc \emph{width}\,{\cf x}\,\emph{height}{\cf [+-]}\emph{xoffset}{\cf
  [+-]}\emph{yoffset}
\\ or~~~~ \spc \emph{xmin,ymin,xmax,ymax} \\

\noindent Examples: 

\begin{tabular}{p{2in} p{4in}}
    {\cf --crop 100x120+35+40}  &      resolution w=100, h=120, offset x=35, y=40 \\
    {\cf --crop 35,40,134,159}  &      resolution w=100, h=120, offset x=35, y=40
\end{tabular}
\apiend

\apiitem{--croptofull}
Replace the current image with a new image that is ropped or padded
as necessary to make the pixel data window exactly cover
the full/display window.
\apiend

\apiitem{--resize {\rm \emph{size}}}
Replace the current image with a new image that is resized to the 
given pixel data resolution.  The size is in the form 
\\ \spc\spc \emph{width}\,{\cf x}\,\emph{height}
\\ or~~~~ \spc \emph{scale}{\verb|%|} \\

\noindent Examples: 

\begin{tabular}{p{2in} p{4in}}
    {\cf --resize 1024x768}  &     new resolution w=100, h=120, offset x=35, y=40 \\
    {\cf --resize 50{\verb|%|}}  & reduce resolution by 50\verb|%| \\
    {\cf --resize 300{\verb|%|}}  & increase resolution by 3x
\end{tabular}

\apiend

\apiitem{--pop}
Pop the image stack, discarding the current image and thereby
making the next image on the stack into the new current image.
\apiend

\apiitem{--dup}
Duplicate the current image and push the duplicate on the stack.
Note that this results in both the current and the next image 
on the stack being identical copies.
\apiend


\section{\oiiotool commands for color management}

\apiitem{--iscolorspace {\rm \emph{colorspace}}}
Alter the metadata of the current image so that it thinks its pixels
are in the named color space.  This does not alter the pixels of the
image, it only changes \oiiotool's understanding of what color
space those those pixels are in.
\apiend

\apiitem{--tocolorspace {\rm \emph{tospace}}}
Replace the current image with a new image whose pixels are transformed
from their existing color space (as best understood or guessed by OIIO)
into the named \emph{tospace}.
\apiend

\apiitem{--tocolorspace {\rm \emph{fromspace tospace}}}
Replace the current image with a new image whose pixels are transformed
from the named \emph{fromspace} color space into the named
\emph{tospace} (disregarding any notion it may have previously had 
about the color space of the current image).
\apiend




\chapter{The {\kw iv} Image Viewer}
\label{chap:iv}
\indexapi{iv}

The {\cf iv} program is a great interactive image viewer.  Because {\cf
  iv} is built on top on \product, it can display images of any formats
readable by \ImageInput plugins on hand.

\medskip

More documentation on this later.

\chapter{Getting Image information With {\kw iinfo}}
\label{chap:iinfo}
\indexapi{iinfo}

%\section{Overview}

The {\cf iinfo} program will print either basic information (name,
resolution, format) or detailed information (including all metadata)
found in images.  Because {\cf iinfo} is built on top on \product, it
will print information about images of any formats readable by
\ImageInput plugins on hand.



\section{Using {\cf iinfo}}

The {\cf iinfo} utility is invoked as follows:

\bigskip

\hspace{0.25in} {\cf iinfo} [\emph{options}] \emph{filename} ...

\medskip

Where \emph{filename} (and any following strings) names the image
file(s) whose information should be printed.  The image files may be of
any format recognized by \product (i.e., for which \ImageInput plugins
are available).

In its most basic usage, it simply prints the resolution, number of
channels, pixel data type, and file format type of each of the
files listed:

\begin{code}
    $ iinfo img_6019m.jpg grid.tif lenna.png

    img_6019m.jpg : 1024 x  683, 3 channel, uint8 jpeg
    grid.tif      :  512 x  512, 3 channel, uint8 tiff
    lenna.png     :  120 x  120, 4 channel, uint8 png
\end{code}

% $

The {\cf -s} flag also prints the uncompressed sizes of each image
file, plus a sum for all of the images:

\begin{code}
    $ iinfo -s img_6019m.jpg grid.tif lenna.png

    img_6019m.jpg : 1024 x  683, 3 channel, uint8 jpeg (2.00 MB)
    grid.tif      :  512 x  512, 3 channel, uint8 tiff (0.75 MB)
    lenna.png     :  120 x  120, 4 channel, uint8 png (0.05 MB)
    Total size: 2.81 MB
\end{code}

% $

The {\cf -v} option turns on \emph{verbose mode}, which exhaustively
prints all metadata about each image:

\begin{code}
    $ iinfo -v img_6019m.jpg

    img_6019m.jpg : 1024 x  683, 3 channel, uint8 jpeg
        channel list: R, G, B
        Color space: sRGB
        ImageDescription: "Family photo"
        Make: "Canon"
        Model: "Canon EOS DIGITAL REBEL XT"
        Orientation: 1 (normal)
        XResolution: 72
        YResolution: 72
        ResolutionUnit: 2 (inches)
        DateTime: "2008:05:04 19:51:19"
        Exif:YCbCrPositioning: 2
        ExposureTime: 0.004
        FNumber: 11
        Exif:ExposureProgram: 2 (normal program)
        Exif:ISOSpeedRatings: 400
        Exif:DateTimeOriginal: "2008:05:04 19:51:19"
        Exif:DateTimeDigitized: "2008:05:04 19:51:19"
        Exif:ShutterSpeedValue: 7.96579 (1/250 s)
        Exif:ApertureValue: 6.91887 (f/11)
        Exif:ExposureBiasValue: 0
        Exif:MeteringMode: 5 (pattern)
        Exif:Flash: 16 (no flash, flash supression)
        Exif:FocalLength: 27 (27 mm)
        Exif:ColorSpace: 1
        Exif:PixelXDimension: 2496
        Exif:PixelYDimension: 1664
        Exif:FocalPlaneXResolution: 2855.84
        Exif:FocalPlaneYResolution: 2859.11
        Exif:FocalPlaneResolutionUnit: 2 (inches)
        Exif:CustomRendered: 0 (no)
        Exif:ExposureMode: 0 (auto)
        Exif:WhiteBalance: 0 (auto)
        Exif:SceneCaptureType: 0 (standard)
        Keywords: "Carly; Jack"
\end{code}

% $

If the input file has multiple subimages, extra information summarizing
the subimages will be printed:

\begin{code}
    $ iinfo img_6019m.tx

    img_6019m.tx : 1024 x 1024, 3 channel, uint8 tiff (11 subimages)

    $ iinfo -v img_6019m.tx

    img_6019m.tx : 1024 x 1024, 3 channel, uint8 tiff
        11 subimages: 1024x1024 512x512 256x256 128x128 64x64 32x32 16x16 8x8 4x4 2x2 1x1 
        channel list: R, G, B
        tile size: 64 x 64
        ...
\end{code}

Furthermore, the {\cf -a} option will print information about all 
individual subimages:

\begin{code}
    $ iinfo -a ../sample-images/img_6019m.tx

    img_6019m.tx : 1024 x 1024, 3 channel, uint8 tiff (11 subimages)
     subimage  0: 1024 x 1024, 3 channel, uint8 tiff
     subimage  1:  512 x  512, 3 channel, uint8 tiff
     subimage  2:  256 x  256, 3 channel, uint8 tiff
     subimage  3:  128 x  128, 3 channel, uint8 tiff
     subimage  4:   64 x   64, 3 channel, uint8 tiff
     subimage  5:   32 x   32, 3 channel, uint8 tiff
     subimage  6:   16 x   16, 3 channel, uint8 tiff
     subimage  7:    8 x    8, 3 channel, uint8 tiff
     subimage  8:    4 x    4, 3 channel, uint8 tiff
     subimage  9:    2 x    2, 3 channel, uint8 tiff
     subimage 10:    1 x    1, 3 channel, uint8 tiff


    $ iinfo -v -a img_6019m.tx
    img_6019m.tx : 1024 x 1024, 3 channel, uint8 tiff
        11 subimages: 1024x1024 512x512 256x256 128x128 64x64 32x32 16x16 8x8 4x4 2x2 1x1 
     subimage  0: 1024 x 1024, 3 channel, uint8 tiff
        channel list: R, G, B
        tile size: 64 x 64
        ...
     subimage  1:  512 x  512, 3 channel, uint8 tiff
        channel list: R, G, B
        ...
    ...
\end{code}


\section{{\cf iinfo} command-line options}

\apiitem{--help}
Prints usage information to the terminal.
\apiend

\apiitem{-v}
Verbose output --- prints all metadata of the image files.
\apiend

\apiitem{-a}
Print information about all subimages in the file(s).
\apiend

\apiitem{-f}
Print the filename as a prefix to every line.  For example,

\begin{code}
    $ iinfo -v -f img_6019m.jpg

    img_6019m.jpg : 1024 x  683, 3 channel, uint8 jpeg
    img_6019m.jpg : channel list: R, G, B
    img_6019m.jpg : Color space: sRGB
    img_6019m.jpg : ImageDescription: "Family photo"
    img_6019m.jpg : Make: "Canon"
    ...
\end{code}
%$
\apiend

\apiitem{-m {\rm \emph{pattern}}}
Match the \emph{pattern} (specified as an extended regular expression)
against data metadata field names and print only data fields whose names
match.  The default is to print all data fields found in the file (if
{\cf -v} is given).

For example,
\begin{code}
    $ iinfo -v -f -m ImageDescription test*.jpg

    test3.jpg :     ImageDescription: "Birthday party"
    test4.jpg :     ImageDescription: "Hawaii vacation"
    test5.jpg :     ImageDescription: "Bob's graduation"
    test6.jpg :     ImageDescription: <unknown>
\end{code}
%$
\apiend

Note: the {\cf -m} option is probably not very useful without also using
the {\cf -v} and {\cf -f} options.

\apiitem{--md5}
Displays an MD5 digest of the pixel data of the image (and of each
subimage if combined with the {\cf -a} flag).
\apiend

\apiitem{-s}
Show the image sizes, including a sum of all the listed images.
\apiend


\chapter{Converting Image Formats With {\kw iconvert}}
\label{chap:iconvert}
\indexapi{iconvert}

\section{Overview}

The {\cf iconvert} program will read an image (from any file format for
which an \ImageInput plugin can be found) and then write the image to a
new file (in any format for which an \ImageOutput plugin can be found).
In the process, {\cf iconvert} can optionally change the file format or
data format (for example, converting floating-point data to 8-bit
integers), apply gamma correction, switch between tiled and scanline
orientation, or alter or add certain metadata to the image.

The {\cf iconvert} utility is invoked as follows:

\medskip

\hspace{0.25in} {\cf iconvert} [\emph{options}] \emph{input} \emph{output}

\medskip

Where \emph{input} and \emph{output} name the input image and desired
output filename.  The image files may be of any format recognized by
\product (i.e., for which \ImageInput plugins are available).  The file
format of the output image will be inferred from the file extension of
the output filename (e.g., \qkw{foo.tif} will write a TIFF file).


\section{{\cf iconvert} Recipes}

This section will give quick examples of common uses of {\cf iconvert}.

\subsection*{Converting between file formats}

It's a snap to converting among image formats supported by \product
(i.e., for which \ImageInput and \ImageOutput plugins can be found).
The {\cf iconvert} utility will simply infer the file format from the
file extension. The following example converts a PNG image to JPEG:

\begin{code}
    iconvert lena.png lena.jpg
\end{code}

\subsection*{Changing the data format or bit depth}

Just use the {\cf -d} option to specify a pixel data format.  For
example, assuming that {\cf in.tif} uses 16-bit unsigned integer
pixels, the following will convert it to an 8-bit unsigned pixels:

\begin{code}
    iconvert -d uint8 in.tif out.tif
\end{code}

\subsection*{Changing the compression}

The following command converts writes a TIFF file, specifically using
LZW compression:

\begin{code}
    iconvert --compression lzw in.tif out.tif
\end{code}

The following command writes its results as a JPEG file at a 
compression quality of 50 (pretty severe compression):

\begin{code}
    iconvert --quality 50 big.jpg small.jpg
\end{code}

\subsection*{Gamma-correcting an image}

The following gamma-corrects the pixels, raising all pixel
values to $x^{1/2.2}$ upon writing:

\begin{code}
    iconvert -g 2.2 in.tif out.tif
\end{code}

\subsection*{Converting between scanline and tiled images}

Convert a scanline file to a tiled file with $16 \times 16$ tiles:

\begin{code}
    iconvert --tile 16 16 s.tif t.tif
\end{code}

\noindent Convert a tiled file to scanline:

\begin{code}
    iconvert --scanline t.tif s.tif
\end{code}


\section{{\cf iconvert} command-line options}

\apiitem{--help}
Prints usage information to the terminal.
\apiend

\apiitem{-v}
Verbose status messages.
\apiend

\apiitem{-d {\rm \emph{datatype}}}

Attempt to sets the output pixel data type to one of: {\cf uint8}, 
{\cf sint8}, {\cf uint16}, {\cf sint16}, {\cf half}, {\cf float}, 
{\cf double}.

If the {\cf -d} option is not supplied, the output data type will
be the same as the data format of the input file.

In either case, the output file format itself (implied by the file
extension of the output filename) may trump the request if the file
format simply does not support the requested data type.
\apiend

\apiitem{-g {\rm \emph{gamma}}}
Applies a gamma correction of $1/\mathrm{gamma}$ to the pixels as they
are output.
\apiend

\apiitem{--tile {\rm \emph{x}} {\rm \emph{y}}}
Requests that the output file be tiled, with the given $x \times y$ 
tile size, if tiled images are supported by the output format.
By default, the output file will take on the tiledness and tile size
of the input file.
\apiend

\apiitem{--scanline}
Requests that the output file be scanline-oriented (even if the input
file was tile-oriented), if scanline orientation is supported by the
output file format.  By default, the output file will be scanline
if the input is scanline, or tiled if the input is tiled.
\apiend

\apiitem{--compression {\rm \emph{method}}}
Sets the compression method for the output image.  Each \ImageOutput
plugin will have its own set of methods that it supports.

By default, the output image will use the same compression technique as
the input image (assuming it is supported by the output format,
otherwise it will use the default compression method of the output
plugin).  
\apiend

\apiitem{--quality {\rm \emph{q}}}
Sets the compression quality, on a 1--100 floating-point scale.
This only has an effect if the particular compression method supports
a quality metric (as JPEG does).
\apiend

\chapter{Searching Image Metadata With {\kw igrep}}
\label{chap:igrep}
\indexapi{igrep}

%\section{Overview}

The {\cf igrep} program search one or more image files for metadata
that match a string or regular expression.



\section{Using {\cf igrep}}

The {\cf igrep} utility is invoked as follows:

\bigskip

\hspace{0.25in} {\cf igrep} [\emph{options}] \emph{pattern} \emph{filename} ...

\medskip

Where \emph{pattern} is a POSIX.2 regular expression (just like the
Unix/Linux {\cf grep(1)} command), and \emph{filename} (and any
following names) specify images or directories that should be searched.
An image file will ``match'' if any of its metadata contains values
contain substring that are recognized regular expression.  The image
files may be of any format recognized by \product (i.e., for which
\ImageInput plugins are available).

Example:

\begin{code}
    $ igrep Jack *.jpg 
    bar.jpg: Keywords = Carly; Jack
    foo.jpg: Keywords = Jack
    test7.jpg: ImageDescription = Jack on vacation
\end{code}

% $



\section{{\cf igrep} command-line options}

\apiitem{--help}
Prints usage information to the terminal.
\apiend

\apiitem{-d}
Print directory names as it recurses.  This only happens if the {\cf -r}
option is also used.
\apiend

\apiitem{-E}
Interpret the pattern as an extended regular expression (just like
{\cf egrep} or {\cf grep -E}).
\apiend

\apiitem{-f}
Match the expression against the filename, as well as the metadata
within the file.
\apiend

\apiitem{-i}
Ignore upper/lower case distinctions.  Without this flag, the expression
matching will be case-sensitive.
\apiend

\apiitem{-l}
Simply list the matching files by name, surpressing the normal output
that would include the metadata name and values that matched.
For example:

\begin{code}
    $ igrep Jack *.jpg 
    bar.jpg: Keywords = Carly; Jack
    foo.jpg: Keywords = Jack
    test7.jpg: ImageDescription = Jack on vacation

    $ igrep -l Jack *.jpg
    bar.jpg
    foo.jpg
    test7.jpg
\end{code}

\apiend

\apiitem{-r}
Recurse into directories.  If this flag is present, any files specified
that are directories will have any image file contained therein to be
searched for a match (an so on, recursively).
\apiend

\apiitem{-v}
Invert the sense of matching, to select image files that \emph{do not}
match the expression.
\apiend


\chapter{Comparing Images With {\kw idiff}}
\label{chap:idiff}
\indexapi{idiff}

\section{Overview}
The {\cf idiff} program compares two images, printing a report about how
different they are and optionally producing a third image that records
the pixel-by-pixel differences between them.  There are a variety of
options and ways to compare (absolute pixel difference, various
thresholds for warnings and errors, and also an optional perceptual
difference metric).

Because {\cf idiff} is built on top on \product, it can compare two
images of any formats readable by \ImageInput plugins on hand.  They may
have any (or different) file formats, data formats, etc.

\section{Using {\cf idiff}}

The {\cf idiff} utility is invoked as follows:

\bigskip

\hspace{0.25in} {\cf idiff} [\emph{options}] \emph{image1} \emph{image2}

\medskip

Where \emph{input1} and \emph{input2} are the names of two image files
that should be compared.  They may be of any format recognized by
\product (i.e., for which image-reading plugins are available).

If the two input images are not the same resolutions, or do not have the
same number of channels, the comparison will return FAILURE immediately
and will not attempt to compare the pixels of the two images.  If 
they are the same dimensions, the pixels of the two images will be
compared, and a report will be printed including the mean and maximum
error, how many pixels were above the warning and failure thresholds,
and whether the result is {\cf PASS}, {\cf WARNING}, or {\cf FAILURE}.
For example:

\begin{code}
    $ idiff a.jpg b.jpg

    Comparing "a.jpg" and "b.jpg"
      Mean error = 0.00450079
      RMS error = 0.00764215
      Peak SNR = 42.3357
      Max error  = 0.254902 @ (700, 222, B)
      574062 pixels (82.1%) over 1e-06
      574062 pixels (82.1%) over 1e-06
    FAILURE
\end{code}

% $

The ``mean error'' is the average difference (per channel, per pixel).
The ``max error'' is the largest difference in any pixel channel,
and will point out on which pixel and channel it was found.
It will also give a count of how many pixels were above the warning
and failure thresholds.

The metadata of the two images (e.g., the comments) are not currently
compared; only differences in pixel values are taken into consideration.

\subsection*{Raising the thresholds}

By default, if any pixels differ between the images, the comparison
will fail.  You can allow \emph{some} differences to still pass by
raising the failure thresholds.  The following example will allow
images to pass the comparison test, as long as no more than 10\%
of the pixels differ by 0.004 (just above a 1/255 threshold):

\begin{code}
    idiff -fail 0.004 -failpercent 10 a.jpg b.jpg
\end{code}

But what happens if a just a few pixels are very different?  Maybe you
want that to fail, also.  The following adjustment will fail if at least
10\% of pixels differ by 0.004, or if \emph{any} pixel differs by
more than 0.25:

\begin{code}
    idiff -fail 0.004 -failpercent 10 -hardfail 0.25 a.jpg b.jpg
\end{code}

If none of the failure criteria are met, and yet some pixels are 
still different, it will still give a WARNING.  But you can also
raise the warning threshold in a similar way:

\begin{code}
    idiff -fail 0.004 -failpercent 10 -hardfail 0.25 \
             -warn 0.004 -warnpercent 3 a.jpg b.jpg
\end{code}

\noindent The above example will PASS as long as fewer than 3\%
of pixels differ by more than 0.004.  If it does, it will be a
WARNING as long as no more than 10\% of pixels differ by 0.004
and no pixel differs by more than 0.25, otherwise it is a FAILURE.

%\subsection*{Perceptual differences}


\subsection*{Output a difference image}

Ordinary text output will tell you how many pixels failed or were
warnings, and which pixel had the biggest difference.  But sometimes
you need to see visually where the images differ.  You can get
{\cf idiff} to save an image of the differences between the two input
images:

\begin{code}
    idiff -o diff.tif -abs a.jpg b.jpg
\end{code}

The {\cf -abs} flag saves the absolute value of the differences
(i.e., all positive values or zero).  If you omit the {\cf -abs},
pixels in which {\cf a.jpg} have smaller values than {\cf b.jpg}
will be negative in the difference image (be careful in this case
of using a file format that doesn't support negative values).

You can also scale the difference image with the {\cf -scale},
making them easier to see.  And the {\cf -od} flag can be used
to output a difference image only if the comparison fails, but 
not if the images pass within the designated threshold (thus
saving you the trouble and space of saving a black image).


\section{{\cf idiff} Reference}

The various command-line options are discussed below:

\subsection*{General options}

\apiitem{--help}
Prints usage information to the terminal.
\apiend

\apiitem{-v}
Verbose output --- more detail about what it finds when comparing
images.  (Currently, there is no extra info to print.)
\apiend

\apiitem{-a}
Compare all subimages.  Without this flag, only the first subimage
of each file will be compared.
\apiend


\subsection*{Thresholds and comparison options}

\apiitem{-fail {\rm \emph{A}} \\
-failpercent {\rm \emph{B}} \\
-hardfail {\rm \emph{C}}}

Sets the threshold for {\cf FAILURE}: if more than \emph{B}\% of pixels
(on a 0-100 floating point scale) are greater than \emph{A} different,
or if \emph{any} pixels are more than \emph{C} different.  The defaults
are to fail if more than 0\% (any) pixels differ by more than 0.00001
(1e-6), and \emph{C} is infinite.
\apiend

\apiitem{-warn {\rm \emph{A}} \\
-warnpercent {\rm \emph{B}} \\
-hardwarn {\rm \emph{C}}}

Sets the threshold for {\cf WARNING}: if more than \emph{B}\% of pixels
(on a 0-100 floating point scale) are greater than \emph{A} different,
or if \emph{any} pixels are more than \emph{C} different.  The defaults
are to warn if more than 0\% (any) pixels differ by more than 0.00001
(1e-6), and \emph{C} is infinite.
\apiend

\apiitem{-p}
Does an additional test on the images to attempt to see if they are
\emph{perceptually} different (whether you are likely to discern
a difference visually), using Hector Yee's metric.  If this option
is enabled, the statistics will additionally show a report on how
many pixels failed the perceptual test, and the test overall will
fail if more than the ``fail percentage'' failed the perceptual test.
\apiend

\subsection*{Difference image output}

\apiitem{-o {\rm \emph{outputfile}}}
Outputs a \emph{difference image} to the designated file.
This difference image pixels consist are each of the value of the
corresponding pixel from \emph{image1} minus the value of the
pixel \emph{image2}.  

The file extension of the output file is used to determine the file
format to write (e.g., \qkw{out.tif} will write a TIFF file,
\qkw{out.jpg} will write a JPEG, etc.).  The data format of the output
file will be format of whichever of the two input images has higher
precision (or the maximum precision that the designated output format is
capable of, if that is less than either of the input imges).

Note that pixels whose value is lower in \emph{image1} than in
\emph{image2}, this will result in negative pixels (which may be clamped
to zero if the image format does not support negative values)), unless
the {\cf -abs} option is also used.  
\apiend

\apiitem{-abs}
Will cause the output image to consist of the \emph{absolute value}
of the difference between the two input images (so all values in the
difference image $\ge 0$).
\apiend

\apiitem{-scale {\rm \emph{factor}}}
Scales the values in the difference image by the given (floating point)
factor.  The main use for this is to make small actual differences more
visible in the resulting difference image by giving a large scale factor.
\apiend

\apiitem{-od}
Causes a difference image to be produce \emph{only} if the image
comparison fails.  That is, even if the {\cf -o} option is used,
images that are within the comparison threshold will not write out
a useless black (or nearly black) difference image.
\apiend

\subsection*{Process return codes}

The {\cf idiff} program will return a code that can be used by scripts
to indicate the results:

\medskip

\begin{tabular}{p{0.3in} p{5in}}
0 & OK: the images match within the warning and error
thresholds. \\
1 & Warning: the errors differ a little, but within error thresholds. \\ 
2 & Failure: the errors differ a lot, outside error thresholds. \\
3 & The images weren't the same size and couldn't be compared. \\
4 & File error: could not find or open input files, etc.
\end{tabular}

\begin{code}
\end{code}

\chapter{Making Tiled MIP-Map Texture Files With {\cf maketx}}
\label{chap:maketx}
\indexapi{maketx}

\section{Overview}

The \maketx program will read an image (from any file format for
which an \ImageInput plugin can be found) and then write it in a form
in which it will have high performance when used by \TextureSystem
(Chapter~\ref{chap:texturesystem}).  This involves converting it to
tiled (versus scanline) orientation, writing multiple subimages at
different resolutions (MIP-map), and setting a variety of header or
metadata fields appropriately for texture maps.

The \maketx utility is invoked as follows:

\medskip

\hspace{0.25in} {\cf maketx} [\emph{options}] \emph{input}... -o \emph{output}

\medskip

Where \emph{input} and \emph{output} name the input image and desired
output filename.  The input files may be of any image format recognized by
\product (i.e., for which \ImageInput plugins are available).  The file
format of the output image will be inferred from the file extension of
the output filename (e.g., \qkw{foo.tif} will write a TIFF file).


\section{{\cf maketx} command-line options}

\apiitem{--help}
Prints usage information to the terminal.
\apiend

\apiitem{-v}
Verbose status messages, including runtime statistics and timing.
\apiend

\apiitem{-o {\rm \emph{outputname}}}
Sets the name of the output texture.
\apiend

\apiitem{--threads \emph{n}}
Use \emph{n} execution threads if it helps to speed up image operations.
The default (also if $n=0$) is to use as many threads as there are cores
present in the hardware.
\apiend

\apiitem{--format {\rm \emph{formatname}}}
Specifies the image format of the output file (e.g., ``tiff'',
``OpenEXR'', etc.).  If {\cf --format} is not used, \maketx will 
guess based on the file extension of the output filename; if it
is not a recognized format extension, TIFF will be used by default.
\apiend

\apiitem{-d {\rm \emph{datatype}}}
Attempt to sets the output pixel data type to one of: {\cf uint8}, 
{\cf sint8}, {\cf uint16}, {\cf sint16}, {\cf half}, {\cf float}, 
{\cf double}.

If the {\cf -d} option is not supplied, the output data type will
be the same as the data format of the input file.

In either case, the output file format itself (implied by the file
extension of the output filename) may trump the request if the file
format simply does not support the requested data type.
\apiend

\apiitem{--tile {\rm \emph{x}} {\rm \emph{y}}}
Specifies the tile size of the output texture.  If not specified,
\maketx will make $64 \times 64$ tiles.
\apiend

\apiitem{--separate}
Forces ``separate'' (e.g., RRR...GGG...BBB) packing of channels in the
output file.  Without this option specified, ``contiguous'' (e.g.,
RGBRGBRGB...) packing of channels will be used for those file formats
that support it.
\apiend

\apiitem{--compression {\rm \emph{method}}}
\NEW Sets the compression method for the output image (the default is to try
to use \qkw{zip} compression, if it is available).
\apiend

\apiitem{--update}
Ordinarily, textures are created unconditionally (which could take
several seconds for large input files if read over a network) and will
be stamped with the current time.

The {\cf --update} option enables \emph{update mode}I if the output file
already exists and has the same time stamp as the input file, the
texture will not be recreated.  If the output file does not exist or has
a different time than the input file, then the texture will be created
be given the time stamp of the input file.
\apiend

\apiitem{--wrap {\rm \emph{wrapmode}} \\
--swrap {\rm \emph{wrapmode}} --twrap {\rm \emph{wrapmode}}}
Sets the default \emph{wrap mode} for the texture, which determines
the behavior when the texture is sampled outside the $[0,1]$ range.
Valid wrap modes are: {\cf black}, {\cf clamp}, {\cf periodic},
{\cf mirror}.  The default, if none is set, is {\cf black}.  The
{\cf --wrap} option sets the wrap mode in both directions
simultaneously, while the {\cf --swrap} and {\cf --twrap} may be used to
set them individually in the $s$ (horizontal) and $t$ (vertical)
diretions.

Although this sets the default wrap mode for a texture, note that
the wrap mode may have an override specified in the texture lookup
at runtime.
\apiend

\apiitem{--resize}
Causes the highest-resolution level of the MIP-map to be a
power-of-two resolution in each dimension
(by rounding up the resolution of the input image).  There is no
good reason to do this for the sake of OIIO's texture system, but 
some users may require it in order to create MIP-map images
that are compatible with both OIIO and other texturing systems that
require power-of-2 textures.
\apiend

\apiitem{--nomipmap}
Causes the output to \emph{not} be MIP-mapped, i.e., only will have
the highest-resolution level.
\apiend

\apiitem{--nchannels {\rm \emph{n}}}
Sets the number of output channels.  If \emph{n} is less than the 
number of channels in the input image, the extra channels will simply
be ignored.  If \emph{n} is greater than the number of channels in the
input image, the additional channels will be filled with 0 values.
\apiend

\apiitem{--checknan}
Checks every pixel of the input image to ensure that no NaN or Inf
values are present.  If such non-finite pixel values are found, 
an error message will be printed and {\cf maketx} will terminate without
writing the output image (returning an error code).
\apiend

\apiitem{--fixnan {\rm \emph{streategy}}}
Repairs any pixels in the input image that contained {\cf NaN} or 
{\cf Inf} values (hereafter referred to collectively as ``nonfinite'').
If \emph{strategy} is {\cf black}, nonfinite values will be replaced
with {\cf 0}.  If \emph{strategy} is {\cf box3}, nonfinite values will
be replaced by the average of all the finite values within a $3 \times 3$
region surrounding the pixel.
\apiend

%\apiitem{--ingamma {\rm \emph{value}} \\
%--outgamma {\rm \emph{value}}}
%Not currently implemented
%\apiend

\apiitem{--Mcamera {\rm \emph{...16 floats...}} \\
--Mscreen {\rm \emph{...16 floats...}}}
Sets the camera and screen matrices (sometimes called {\cf Nl} and
{\cf NP}, respectively, by some renderers) in the texture file, 
overriding any such matrices that may be in the input image (and would
ordinarily be copied to the output texture).
\apiend

\apiitem{--hash}
Computes a SHA-1 hash on the input file's pixels and embeds this hash
in the ``ImageDescription'' metadata of the output texture.  This is
useful in helping the \TextureSystem identify duplicate textures at
runtime.
\apiend

\apiitem{--prman-metadata}
Causes metadata \qkw{PixarTextureFormat} to be set, which is useful if
you intend to create an OpenEXR texture or environment map that can be
used with PRMan as well as OIIO.
\apiend

\apiitem{--constant-color-detect}
Detects images in which all pixels are identical, and outputs the
texture at a reduced resolution equal to the tile size, rather than
filling endless tiles with the same constant color.  That is, by
substituting a low-res texture for a high-res texture if it's a constant
color, you could save a lot of save disk space, I/O, and texture cache size.
It also sets the \qkw{ImageDescription} to contain a
special message of the form \qkw{ConstantColor=[r,g,...]}.  
\apiend

\apiitem{--monochrome-detect}
Detects multi-channel images in which all color components are
identical, and outputs the texture as a single-channel image instead.
That is, it changes RGB images that are gray into single-channel gray
scale images.

Use with caution!  This is a great optimization if such textures will
only have their first channel accessed, but may cause unexpected behavior
if the ``client'' application will attempt to access those other
channels that will no longer exist.
\apiend

\apiitem{--opaque-detect}
Detects images that have a designated alpha channel for which the alpha value
for all pixels is 1.0 (fully opaque), and omits the alpha channel from
the output texture.  So, for example, an RGBA input texture where A=1
for all pixels will be output just as RGB.  The purpose is to save disk
space, texture I/O bandwidth, and texturing time for those textures
where alpha was present in the input, but clearly not necessary.

Use with caution!  This is a great optimization only if your use of such
textures will assume that missing alpha channels are equivalent to
textures whose alpha is 1.0 everywhere.
\apiend

\apiitem{--ignore-unassoc}
\NEW Ignore any header tags in the input images that indicate that the
input has ``unassociated'' alpha.  When this option is used, color
channels with unassociated alpha will not be automatically multiplied
by alpha to turn them into associated alpha. This is also a good way
to fix input images that really are associated alpha, but whose headers
incorrectly indicate that they are unassociated alpha. 
\apiend

\apiitem{--prman}
PRMan is will crash in strange ways if given textures that don't have
its quirky set of tile sizes and other specific metadata.  If you want
\maketx to generate textures that may be used with either \OpenImageIO
or PRMan, you should use the {\cf --prman} option, which will set
several options to make PRMan happy, overriding any contradictory
settings on the command line or in the input texture.  

Specifically, this option sets the tile size (to 64x64 for 8 bit,
64x32 for 16 bit integer, and 32x32 for float or {\cf half} images),
uses ``separate'' planar configuration ({\cf --separate}), and sets
PRMan-specific metadata ({\cf --prman-metadata}).  It also outputs 
sint16 textures if uint16 is requested (because PRMan for some reason
does not accept true uint16 textures).

\OpenImageIO will happily accept textures that conform to PRMan's
expectations, but not vice versa.  But \OpenImageIO's \TextureSystem
has better performance with textures that use \maketx's default settings
rather than these oddball choices.  You have been warned!
\apiend

\apiitem{--oiio}
This sets several options that we have determined are the 
optimal values for \OpenImageIO's \TextureSystem, overriding any
contradictory settings on the command line or in the input texture.

Specifically, this is the equivalent to using \\
 {\cf --separate --tile 64 64 --hash}.
\apiend

\apiitem{--colorconvert {\rm \emph{inspace outspace}}}
Convert the color space of the input image from \emph{inspace} to
\emph{tospace}.  If OpenColorIO is installed and finds a valid
configuration, it will be used for the color conversion.  If OCIO
is not enabled (or cannot find a valid configuration, OIIO will at
least be able to convert among linear, sRGB, and Rec709.
\apiend

\apiitem{--unpremult}
When undergoing color some conversions, it is helpful to
``un-premultiply'' the alpha before converting color channels, and then
re-multiplying by alpha.  Caveat emptor -- if you don't know exactly
when to use this, you probably shouldn't be using it at all.
\apiend


\apiitem{--mipimage {\rm \emph{filename}}}
Specifies the name of an image file to use as a custom MIP-map level, 
instead of simply downsizing the last one.  This option may be used
multiple times to specify multiple levels.  For example:
\begin{code}
    maketx 256.tif --miplevel 128.tif --miplevel 64.tif -o out.tx
\end{code}
This will make a texture with the first MIP level taken from {\cf 256.tif},
the second level from {\cf 128.tif}, the third from {\cf 64.tif}, and
then subsequent levels will be the usual downsizings of {\cf 64.tif}.
\apiend

% --shadow --shadcube
% --volshad --envlatl --envcube --lightprobe --latl2envcube --vertcross
% --fov
% --opaquewidth


\begin{comment}

\section{{\cf maketx} Recipes}

% FIXME

This section will give quick examples of common uses of {\cf maketx}.

\subsection*{Converting between file formats}

It's a snap to converting among image formats supported by \product
(i.e., for which \ImageInput and \ImageOutput plugins can be found).
The {\cf maketx} utility will simply infer the file format from the
file extension. The following example converts a PNG image to JPEG:

\begin{code}
    maketx lena.png lena.jpg
\end{code}

\end{comment}



\part{Appendices}
\begin{appendix}

%\include{typedesc}
\chapter{Building OpenImageIO}

\chapter{Metadata conventions}
\label{chap:stdmetadata}


The \ImageSpec class, described thoroughly in
Section~\ref{sec:ImageSpec}, provides the basic description of an image
that are essential across all formats --- resolution, number of
channels, pixel data format, etc.  Individual images may have additional
data, stored as name/value pairs in the {\cf extra_attribs} field.
Though literally \emph{anything} can be stored in {\cf extra_attribs}
--- it's specifically designed for format- and user-extensibility ---
this chapter establishes some guidelines and lays out all of the field
names that \product understands.


\section{Description of the image}

\apiitem{"ImageDescription" : string}
The image description, title, caption, or comments.
\apiend

\apiitem{"Keywords" : string}

Semicolon-separated keywords describing the contents of the image.
(Semicolons are used rather than commas because of the common case
of a comma being part of a keyword itself, e.g., ``Kurt Vonnegut, Jr.''
or ``Washington, DC.'')
\apiend

\apiitem{"Artist" : string}
The artist, creator, or owner of the image.
\apiend

\apiitem{"Copyright" : string}
Any copyright notice or owner of the image.
\apiend

\apiitem{"DateTime" : string}
The creation date of the image, in the following format: {\cf YYYY:MM:DD
  HH:MM:SS} (exactly 19 characters long, not including a terminating
NULL).  For example, 7:30am on Dec
31, 2008 is encoded as \qkw{2008:12:31 07:30:00}.
\apiend

\apiitem{"DocumentName" : string}
The name of an overall document that this image is a part of.
\apiend

\apiitem{"Software" : string}
The software that was used to create the image.
\apiend

\apiitem{"HostComputer" : string}
The name or identity of the computer that created the image.
\apiend

\section{Display hints}

\apiitem{"Orientation" : int}
By default, image pixels are ordered from the top of the display to the
bottom, and within each scanline, from left to right (i.e., the same
ordering as English text and scan progression on a CRT).  But the
\qkw{Orientation} field can suggest that it should be displayed with
a different orientation, according to the TIFF/EXIF conventions:

\begin{tabular}{p{0.3in} p{4in}}
1 & normal (top to bottom, left to right)  \\
2 & flipped horizontally (top to botom, right to left)  \\
3 & rotate $180^\circ$ (bottom to top, right to left) \\
4 & flipped vertically (bottom to top, left to right)  \\
5 & transposed (left to right, top to bottom) \\
6 & rotated $90^\circ$ clockwise (right to left, top to bottom) \\
7 & transverse (right to left, bottom to top) \\
8 & rotated $90^\circ$ counter-clockwise (left to right, bottom to top) \\
\end{tabular}
\apiend

\apiitem{"PixelAspectRatio" : float}
The aspect ratio ($x/y$) of the individual pixels, with square pixels
being 1.0 (the default).
\apiend

\apiitem{"XResolution" : float \\
"YResolution" : float \\
"ResolutionUnit" : string}
The number of horizontal ($x$) and vertical ($y$) pixels per 
resolution unit.  This ties the image to a physical size (where 
applicable, such as with a scanned image, or an image that will
eventually be printed).

Different file formats may dictate different resolution units.
For example, the TIFF ImageIO plugin supports \qkw{none}, \qkw{in}, 
and \qkw{cm}.
\apiend

\section{Disk file format info/hints}

\apiitem{"BitsPerSample" : int}
Number of bits per sample \emph{in the file}.  

Note that this may not match the reported {\cf ImageSpec::format}, if
the plugin is translating from an unsupported format.  For example, if a
file stores 4 bit grayscale per channel, the {\cf "BitsPerSample"} may
be 4 but the {\cf format} field may be {\cf TypeDesc::UINT8} (because
the \product APIs do not support fewer than 8 bits per sample).
\apiend

\apiitem{"planarconfig" : string}
\qkw{contig} indicates that the file has contiguous pixels (RGB RGB
RGB...), whereas \qkw{separate} indicate that the file stores each
channel separately (RRR...GGG...BBB...).

Note that only contiguous pixels are transmitted through the \product
APIs, but this metadata indicates how it is (or should be) stored in the
file, if possible.
\apiend

\apiitem{"compression" : string}
Indicates the type of compression the file uses.  Supported compression
modes will vary from \ImageInput plugin to plugin, and each plugin
should document the modes it supports.  If {\cf ImageInput::open} is
called with an \ImageSpec that specifies an compression mode not
supported by that \ImageInput, it will choose a reasonable default.
As an example, the TIFF \ImageInput plugin supports \qkw{none},
\qkw{lzw}, \qkw{ccittrle}, \qkw{zip} (the default), \qkw{packbits}.
\apiend

\apiitem{"CompressionQuality" : int}
Indicates the quality of compression to use (0--100), for those 
plugins and compression methods that allow a variable amount of 
compression, with higher numbers indicating higher image fidelity.
\apiend

\section{Photographs or scanned images}

The following metadata items are specific to photos or captured images.

\apiitem{"Make" : string}
For captured or scanned image, the make of the camera or scanner.
\apiend

\apiitem{"Model" : string}
For captured or scanned image, the model of the camera or scanner.
\apiend

\apiitem{"ExposureTime" : float}
The exposure time (in seconds) of the captured image.
\apiend

\apiitem{"FNumber" : float}
The f/stop of the camera when it captured the image.
\apiend

\section{Texture Information}

Several standard metadata are very helpful for images that are intended
to be used as textures (especially for \product's \TextureSystem).

\apiitem{"textureformat" : string}
The kind of texture that this image is intended to be.  We suggest the
following names:

\noindent \begin{tabular}{p{1.75in} p{3.25in}}
\qkw{Plain Texture} & Ordinary 2D texture \\
\qkw{Volume Texture} & 3D volumetric texture \\
\qkw{Shadow} & Ordinary $z$-depth shadow map \\
\qkw{CubeFace Shadow} & Cube-face shadow map \\
\qkw{Volume Shadow} & Volumetric (``deep'') shadow map \\
\qkw{LatLong Environment} & Latitude-longitude (rectangular) environment
map \\
\qkw{CubeFace Environment} & Cube-face environment map \\
\end{tabular}
\apiend

\apiitem{"wrapmodes" : string}
Give the intended texture \emph{wrap mode} indicating what happens with
texture coordinates outside the $[0...1]$ range.  We suggest the
following names: \qkw{black}, \qkw{periodic}, \qkw{clamp}, \qkw{mirror}.
If the wrap mode is different in each direction, they should simply be
separated by a comma.  For example, \qkw{black} means black wrap in both
directions, whereas \qkw{clamp,periodic} means to clamp in $u$ and be
periodic in $v$.
\apiend

\apiitem{"fovcot" : float}
The cotangent ($x/y$) of the field of view of the original image (which
may not be the same as the aspect ratio of the pixels of the texture,
which may have been resized).
\apiend

\apiitem{"worldtocamera" : matrix44}
For shadow maps or rendered images this item (of type {\cf TypeDesc::PT_MATRIX})
is the world-to-camera matrix describing the camera position.
\apiend

\apiitem{"worldtoscreen" : matrix44}
For shadow maps or rendered images this item (of type {\cf TypeDesc::PT_MATRIX})
is the world-to-screen matrix describing the full projection of the 3D
view onto a $[-1...1] \times [-1...1]$ 2D domain.
\apiend

\apiitem{"updirection" : string}
For environment maps, indicates which direction is ``up'' (valid values
are \qkw{y} or \qkw{z}), to disambiguate conventions for environment map
orientation.
\apiend


\section{Exif metadata}

% FIXME -- unsupported/undocumented: ExifVersion, FlashpixVersion,
% ComponentsConfiguration, MakerNote, UserComment, RelatedSoundFile,
% OECF, SubjectArea, SpatialFrequencyResponse, 
% CFAPattern, DeviceSettingDescription
%
% SubjectLocation -- unsupported, but we could do it

The following Exif metadata tags correspond to items in the ``standard''
set of metadata.

\medskip

\begin{tabular}{p{1.5in} p{3.5in}}
{\bf Exif tag} & {\bf \product metadata convention} \\
\hline
ColorSpace & (stored in {\cf ImageSpec::Linearity}) \\
ExposureTime & \qkw{ExposureTime} \\
FNumber & \qkw{FNumber} \\
\end{tabular}

\medskip

The other remaining Exif metadata tags all include the ``Exif:'' prefix
to keep it from clashing with other names that may be used for other
purposes.

\apiitem{"Exif:ExposureProgram" : int}
The exposure program used to set exposure when the picture was taken:
\medskip

\begin{tabular}{p{0.3in} p{4in}}
0 & unknown \\
1 & manual \\
2 & normal program \\
3 & aperture priority \\
4 & shutter priority \\
5 & Creative program (biased toward depth of field) \\
6 & Action program (biased toward fast shutter speed) \\
7 & Portrait mode (closeup photo with background out of focus) \\
8 & Landscape mode (background in focus)
\end{tabular}
\apiend

\apiitem{"Exif:SpectralSensitivity" : string}
The camera's spectral sensitivity, using the ASTM conventions.
\apiend

\apiitem{"Exif:ISOSpeedRatings" : int}
The ISO speed and ISO latitude of the camera as specified in ISO 12232.
\apiend

%\apiitem{"Exif:OECF",	TIFF_NOTYPE }	 // skip it
%\apiitem{"Exif:ExifVersion",	TIFF_NOTYPE }	 // skip it

\apiitem{"Exif:DateTimeOriginal" : string}
Date and time that the original image data was generated (in
\qkw{YYYY:MM:DD HH:MM:SS} format).
\apiend

\apiitem{"Exif:DateTimeDigitized" : string}
Date and time that the image was stored as digital data (in
\qkw{YYYY:MM:DD HH:MM:SS} format).
\apiend

%\apiitem{"Exif:ComponentsConfiguration",TIFF_UNDEFINED }
\apiitem{"Exif:CompressedBitsPerPixel" : float }
The compression mode used, measured in compressed bits per pixel.
\apiend

\apiitem{"Exif:ShutterSpeedValue" : float }
Shutter speed, in APEX units: $-\log_2 (\mathit{exposure time})$
\apiend

\apiitem{"Exif:ApertureValue" : float }
Aperture, in APEX units: $2 \log_2 (\mathit{fnumber})$
\apiend

\apiitem{"Exif:BrightnessValue" : float }
Brightness value, assumed to be in the range of $-99.99$ -- $99.99$.
\apiend

\apiitem{"Exif:ExposureBiasValue" : float }
Exposure bias, assumed to be in the range of $-99.99$ -- $99.99$.
\apiend

\apiitem{"Exif:MaxApertureValue" : float }
Smallest F number of the lens, in APEX units: $2 \log_2 (\mathit{fnumber})$
\apiend

\apiitem{"Exif:SubjectDistance" : float }
Distance to the subject, in meters.
\apiend

\apiitem{"Exif:MeteringMode" : int}
The metering mode:

\medskip

\begin{tabular}{p{0.3in} p{4in}}
0 & unknown \\
1 & average \\
2 & center-weighted average \\
3 & spot \\
4 & multi-spot \\
5 & pattern \\
6 & partial \\
255 & other
\end{tabular}
\apiend

\apiitem{"Exif:LightSource" : int}
The kind of light source:

\medskip

\begin{tabular}{p{0.3in} p{4in}}
0 & unknown \\
1 & daylight \\
2 & tungsten (incandescent light) \\
4 & flash \\
9 & fine weather \\
10 & cloudy weather \\
11 & shade \\
12 & daylight fluorescent (D 5700-7100K) \\
13 & day white fluorescent (N 4600-5400K) \\
14 & cool white fuorescent (W 3900 - 4500K) \\
15 & white fluorescent (WW 3200 - 3700K) \\
17 & standard light A \\
18 & standard light B \\
19 & standard light C \\
20 & D55 \\
21 & D65 \\
22 & D75 \\
23 & D50 \\
24 & ISO studio tungsten \\
255 & other light source
\end{tabular}
\apiend

\apiitem{"Exif:Flash" int}
A sum of:
\smallskip

\begin{tabular}{p{0.3in} p{4in}}
1 & if the flash fired \\
\hline
0 & no strobe return detection function \\
4 & strobe return light was not detected \\
6 & strobe return light was detected \\
\hline
8 & compulsary flash firing \\
16 & compulsary flash supression \\
24 & auto-flash mode \\
\hline 
32 & no flash function (0 if flash function present) \\
\hline
64 & red-eye reduction supported (0 if no red-eye reduction mode) 
\end{tabular}

\apiend

\apiitem{"Exif:FocalLength" : float }
Actual focal length of the lens, in mm.
\apiend

%\apiitem{"Exif:SubjectArea",TIFF_NOTYPE } // skip
%\apiitem{"Exif:MakerNote",TIFF_NOTYPE } // skip it
%\apiitem{"Exif:UserComment",TIFF_NOTYPE }// skip it

\apiitem{"Exif:SubsecTime" : string}
Fractions of a second to augment the \qkw{DateTime} (expressed
as text of the digits to the right of the decimal).
\apiend

\apiitem{"Exif:SubsecTimeOriginal" : string}
Fractions of a second to augment the \qkw{Exif:DateTimeOriginal} (expressed
as text of the digits to the right of the decimal).
\apiend

\apiitem{"Exif:SubsecTimeDigitized" : string}
Fractions of a second to augment the \qkw{Exif:DateTimeDigital} (expressed
as text of the digits to the right of the decimal).
\apiend

%\apiitem{"Exif:FlashPixVersion",TIFF_NOTYPE }

\apiitem{"Exif:PixelXDimension" : int \\
"Exif:PixelYDimension" : int }
The $x$ and $y$ dimensions of the valid pixel area.
FIXME -- better explanation?
\apiend

%\apiitem{"Exif:RelatedSoundFile", TIFF_NOTYPE }// skip

\apiitem{"Exif:FlashEnergy" : float }
Strobe energy when the image was captures, measured in Beam Candle Power
Seconds (BCPS).
\apiend

%\apiitem{"Exif:SpatialFrequencyResponse",TIFF_NOTYPE }

\apiitem{"Exif:FocalPlaneXResolution" : float \\
"Exif:FocalPlaneYResolution" : float \\
"Exif:FocalPlaneResolutionUnit" : int} 
The number of pixels in the $x$ and $y$ dimension, per resolution unit.
The code for resolution units is: 2 for inches.
% FIXME? units?
\apiend


%\apiitem{"Exif:SubjectLocation" : int} // FIXME: short[2]

\apiitem{"Exif:ExposureIndex" : float }
The exposure index selected on the camera.
\apiend

\apiitem{"Exif:SensingMethod" : int}
The image sensor type on the camra:
\medskip

\begin{tabular}{p{0.3in} p{4in}}
1 & undefined \\
2 & one-chip color area sensor \\
3 & two-chip color area sensor \\
4 & three-chip color area sensor \\
5 & color sequential area sensor \\
7 & trilinear sensor \\
8 & color trilinear sensor 
\end{tabular}
\apiend

\apiitem{"Exif:FileSource" : int}
Set to 3, if captured by a digital camera, otherwise it should not be
present.
\apiend

\apiitem{"Exif:SceneType" : int}
Set to 1, if a directly-photographed image, otherwise it should not be
present.
\apiend

%\apiitem{"Exif:CFAPattern",TIFF_NOTYPE }

\apiitem{"Exif:CustomRendered" : int}
Set to 0 for a normal process, 1 if some custom processing has been
performed on the image data.
\apiend

\apiitem{"Exif:ExposureMode" : int}
The exposure mode:
\medskip

\begin{tabular}{p{0.3in} p{4in}}
0 & auto \\
1 & manual \\
2 & auto-bracket
\end{tabular}
\apiend

\apiitem{"Exif:WhiteBalance" : int}
Set to 0 for auto white balance, 1 for manual white balance.
\apiend

\apiitem{"Exif:DigitalZoomRatio" : float }
The digital zoom ratio used when the image was shot.
\apiend

\apiitem{"Exif:FocalLengthIn35mmFilm" : int}
The equivalent focal length of a 35mm camera, in mm.
\apiend

\apiitem{"Exif:SceneCaptureType" : int}
The type of scene that was shot:
\medskip

\begin{tabular}{p{0.3in} p{4in}}
0 & standard \\
1 & landscape \\
2 & portrait \\
3 & night scene
\end{tabular}
\apiend

\apiitem{"Exif:GainControl" : float }
The degree of overall gain adjustment:
\medskip

\begin{tabular}{p{0.3in} p{4in}}
0 & none \\
1 & low gain up \\
2 & high gain up \\
3 & low gain down \\
4 & high gain down
\end{tabular}
\apiend

\apiitem{"Exif:Contrast" : int}
The direction of contrast processing applied by the camera:
\medskip

\begin{tabular}{p{0.3in} p{4in}}
0 & normal \\
1 & soft \\
2 & hard
\end{tabular}
\apiend

\apiitem{"Exif:Saturation" : int}
The direction of saturation processing applied by the camera:
\medskip

\begin{tabular}{p{0.3in} p{4in}}
0 & normal \\
1 & low saturation \\
2 & high saturation
\end{tabular}
\apiend

\apiitem{"Exif:Sharpness" : int}
The direction of sharpness processing applied by the camera:
\medskip

\begin{tabular}{p{0.3in} p{4in}}
0 & normal \\
1 & soft \\
2 & hard
\end{tabular}
\apiend

%\apiitem{"Exif:DeviceSettingDescription",TIFF_NOTYPE }

\apiitem{"Exif:SubjectDistanceRange" : int}
The distance to the subject:
\medskip

\begin{tabular}{p{0.3in} p{4in}}
0 & unknown \\
1 & macro \\
2 & close \\
3 & distant
\end{tabular}
\apiend

\apiitem{"Exif:ImageUniqueID" : string}
A unique identifier for the image, as 16 ASCII hexidecimal digits 
representing a 128-bit number.
\apiend


\section{GPS Exif metadata}

The following GPS-related Exif metadata tags correspond to items in the
``standard'' set of metadata.

%\apiitem{"GPS:VersionID" : int}
%\apiend

\apiitem{"GPS:LatitudeRef" : string}
Whether the \qkw{GPS:Latitude} tag refers to north or south: \qkw{N} or 
\qkw{S}.
\apiend

\apiitem{"GPS:Latitude" : float[3]}
The degrees, minutes, and seconds of latitude (see also \qkw{GPS:LatitudeRef}).
\apiend

\apiitem{"GPS:LongitudeRef" : string}
Whether the \qkw{GPS:Longitude} tag refers to east or west: \qkw{E} or 
\qkw{W}.
\apiend

\apiitem{"GPS:Longitude" : float[3]}
The degrees, minutes, and seconds of longitude (see also 
\qkw{GPS:LongitudeRef}).
\apiend

\apiitem{"GPS:AltitudeRef" : string}
A value of 0 indicates that the altitude is above sea level, 1 indicates
below sea level.
\apiend

\apiitem{"GPS:Altitude" : float}
Absolute value of the altitude, in meters, relative to sea level 
(see \qkw{GPS:AltitudeRef} for whether it's above or below sea level).
\apiend

\apiitem{"GPS:TimeStamp" : float[3]}
Gives the hours, minutes, and seconds, in UTC.
\apiend

\apiitem{"GPS:Satellites" : string}
Information about what satellites were visible.
\apiend

\apiitem{"GPS:Status" : string}
\qkw{A} indicates a measurement in progress, \qkw{V} indicates
measurement interoperability.
\apiend

\apiitem{"GPS:MeasureMode" : string}
\qkw{2} indicates a 2D measurement, \qkw{3} indicates a 3D measurement.
\apiend

\apiitem{"GPS:DOP" : float}
Data degree of precision.
\apiend

\apiitem{"GPS:SpeedRef" : string}
Indicates the units of the related \qkw{GPS:Speed} tag: 
\qkw{K} for km/h, \qkw{M} for miles/h, \qkw{N} for knots.
\apiend

\apiitem{"GPS:Speed" : float}
Speed of the GPS receiver (see \qkw{GPS:SpeedRef} for the units).
\apiend

\apiitem{"GPS:TrackRef" : string}
Describes the meaning of the \qkw{GPS:Track} field: \qkw{T} for true
direction, \qkw{M} for magnetic direction.
\apiend

\apiitem{"GPS:Track" : float}
Direction of the GPS receiver movement (from 0--359.99).  The
related \qkw{GPS:TrackRef} indicate whether it's true or magnetic.
\apiend

\apiitem{"GPS:ImgDirectionRef" : string}
Describes the meaning of the \qkw{GPS:ImgDirection} field: \qkw{T} for true
direction, \qkw{M} for magnetic direction.
\apiend

\apiitem{"GPS:ImgDirection" : float}
Direction of the image when captured (from 0--359.99).  The
related \qkw{GPS:ImgDirectionRef} indicate whether it's true or magnetic.

\apiend

\apiitem{"GPS:MapDatum" : string}
The geodetic survey data used by the GPS receiver.
\apiend

\apiitem{"GPS:DestLatitudeRef" : string}
Whether the \qkw{GPS:DestLatitude} tag refers to north or south: \qkw{N} or 
\qkw{S}.
\apiend

\apiitem{"GPS:DestLatitude" : float[3]}
The degrees, minutes, and seconds of latitude of the destination (see also 
\qkw{GPS:DestLatitudeRef}).
\apiend

\apiitem{"GPS:DestLongitudeRef" : string}
Whether the \qkw{GPS:DestLongitude} tag refers to east or west: \qkw{E} or 
\qkw{W}.
\apiend

\apiitem{"GPS:DestLongitude" : float[3]}
The degrees, minutes, and seconds of longitude of the destination (see also 
\qkw{GPS:DestLongitudeRef}).
\apiend

\apiitem{"GPS:DestBearingRef" : string}
Describes the meaning of the \qkw{GPS:DestBearing} field: \qkw{T} for true
direction, \qkw{M} for magnetic direction.
\apiend

\apiitem{"GPS:DestBearing" : float}
Bearing to the destination point (from 0--359.99).  The
related \qkw{GPS:DestBearingRef} indicate whether it's true or magnetic.
\apiend

\apiitem{"GPS:DestDistanceRef" : string}
Indicates the units of the related \qkw{GPS:DestDistance} tag: 
\qkw{K} for km, \qkw{M} for miles, \qkw{N} for knots.
\apiend

\apiitem{"GPS:DestDistance" : float}
Distance to the destination (see \qkw{GPS:DestDistanceRef} for the units).
\apiend

\apiitem{"GPS:ProcessingMethod" : string}
Processing method information.
\apiend

\apiitem{"GPS:AreaInformation" : string}
Name of the GPS area.
\apiend

\apiitem{"GPS:DateStamp" : string}
Date according to the GPS device, in format \qkw{YYYY:MM:DD}.
\apiend

\apiitem{"GPS:Differential" : int}
If 1, indicates that differential correction was applied.
\apiend



\section{IPTC metadata}

The IPTC (International Press Telecommunications Council) publishes
conventions for storing image metadata, and this standard is growing
in popularity and is commonly used in photo-browsing programs to
record captions and keywords.

The following IPTC metadata items correspond exactly to metadata in the
\product conventions, so it is recommended that you use the standards
and that plugins supporting IPTC metadata respond likewise:

\medskip

\begin{tabular}{p{1.5in} p{3.5in}}
{\bf IPTC tag} & {\bf \product metadata convention} \\
\hline
Caption & \qkw{ImageDescription} \\[0.5ex]
Keyword & IPTC keywords should be concatenated, separated by semicolons
({\cf ;}), and stored as the \qkw{Keywords} attribute. \\[0.5ex]
ExposureTime & \qkw{ExposureTime} \\[0.5ex]
CopyrightNotice & \qkw{Copyright} \\[0.5ex]
Creator & \qkw{Artist} \\
\end{tabular}

\medskip

The remainder of IPTC metadata fields should use the following names,
prefixed with ``IPTC:'' to avoid conflicts with other plugins or
standards.

\apiitem{"IPTC:ObjectName" : string}
The name of the object in the picture.
\apiend

% \apiitem{  25, "Keywords" : string}
%\apiend
% FIXME

\apiitem{"IPTC:Instructions" : string}
Special instructions for handling the image.
\apiend

%\apiitem{"IPTC:Creator" : string}
%The creator of the image.  This is optinal and, If present, 
%is expected to be the same as the data in the standard \qkw{Artist} field.
%\apiend

\apiitem{"IPTC:AuthorsPosition" : string}
The job title or position of the creator of the image.
\apiend

\apiitem{"IPTC:City" : string \\
"IPTC:State" : string \\
"IPTC:Country" : string}
The city, state, and country of the location of the image.
\apiend

\apiitem{"IPTC:Headline" : string}
Any headline that is meant to accompany the image.
\apiend

\apiitem{"IPTC:Provider" : string}
The provider of the image, or credit line.
\apiend

\apiitem{"IPTC:Source" : string}
The source of the image.
\apiend

%\apiitem{"IPTC:CopyrightNotice" : string}
%The copyright notice for the image.  This is optinal and, If present, 
%is expected to be the same as the data in the standard \qkw{Copyright} field.
%\apiend

\apiitem{"IPTC:Contact" : string}
The contact information for the image (possibly including name, address,
email, etc.).
\apiend

%\apiitem{"IPTC:Caption", string }
%The caption, abstract, or description of the image.
%This is optional and, if present, is expected to be the same as the
%data in the standard \qkw{ImageDescription} field.
%\apiend

\apiitem{"IPTC:CaptionWriter" : string}
The name of the person who wrote the caption or description of the image.
\apiend


\section{Extension conventions}

To avoid conflicts with other plugins, or with any additional standard
metadata names that may be added in future verions of \product, it is
strongly advised that writers of new plugins should prefix their
metadata with the name of the format, much like the \qkw{Exif:}
and \qkw{IPTC:} metadata.


\chapwidthend



%\include{header}
\chapter{Glossary}

\begin{description}

\item[Channel] One of several data values persent in each pixel.
  Examples include red, green, blue, alpha, etc.  The data in one
  channel of a pixel may be represented by a single number, whereas the
  pixel as a whole requires one number for each channel.

\item[Client] A client (as in ``client application'') is a program or
  library that uses \product or any of its constituent libraries.

\item[Data format] The type of numerical representation used to
  store a piece of data.  Examples include 8-bit unsigned integers,
  32-bit floating-point numbers, etc.

\item[Image File Format] The specification and data layout of an
  image on disk.  For example, TIFF, JPEG/JFIF, OpenEXR, etc.

\item[Metadata] Data about data.  As used in \product, this means
  Information about an image, beyond describing the values of the pixels
  themselves.  Examples include the name of the artist that created the
  image, the date that an image was scanned, the camera settings used
  when a photograph was taken, etc.

\item[Native data format] The \emph{data format} used in the disk file
  representing an image.  Note that with \product, this may be different
  than the data format used by an application to store the image
  in the computer's RAM.

\item[Pixel] One pixel element of an image, consisting of one number
  describing each \emph{channel} of data at a particular location in an
  image.

\item[Scanline] A single horizontal row of pixels of an image.  See also
  \emph{tile}.

\item[Scanline Image] An image whose data layout on disk is organized by
  breaking the image up into horizontal scanlines, typically with the
  ability to read or write an entire scanline at once.  See also
  \emph{tiled image}.

\item[Tile] A rectangular region of pixels of an image.  A rectangular
  tile is more spatially coherent than a scanline that stretches across
  the entire image --- that is, a pixel's neighbors are most likely in
  the same tile, whereas a pixel in a scanline image will typically have
  most of its immediate neighbors on different scanlines (requiring
  additional scanline reads in order to access them).

\item[Tiled Image] An image whose data layout on disk is organized by
  breaking the image up into rectangular regions of pixels called
  \emph{tiles}.  All the pixels in a tile can be read or written at
  once, and individual tiles may be read or written separately from
  other tiles.

\item[Volume Image] A 3-D set of pixels that has not only horizontal and
  vertical dimensions, but also a "depth" dimension.

\end{description}

\chapwidthend

\end{appendix}

\backmatter

%\bibliographystyle{alpha}	%% Select for [GH95]
\bibliographystyle{apalike}    %% Select for (Gritz and Hahn, 1995)
%\addcontentsline{toc}{chapter}{Bibliography}
%\bibliography{bmrtbib}

\addcontentsline{toc}{chapter}{Index}
\printindex

\end{document}


% Canonical figure
%\begin{figure}[ht]
%\noindent
%\includegraphics[width=5in]{Figures/bredow/foo} 
%\caption{Caption
%\label{fig:foo}}
%\end{figure}
