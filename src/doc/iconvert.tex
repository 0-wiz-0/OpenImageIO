\chapter{Converting Image Formats With {\kw iconvert}}
\label{chap:iconvert}
\indexapi{iconvert}

\section{Overview}

The {\cf iconvert} program will read an image (from any file format for
which an \ImageInput plugin can be found) and then write the image to a
new file (in any format for which an \ImageOutput plugin can be found).
In the process, {\cf iconvert} can optionally change the file format or
data format (for example, converting floating-point data to 8-bit
integers), apply gamma correction, switch between tiled and scanline
orientation, or alter or add certain metadata to the image.

The {\cf iconvert} utility is invoked as follows:

\medskip

\hspace{0.25in} {\cf iconvert} [\emph{options}] \emph{input} \emph{output}

\medskip

Where \emph{input} and \emph{output} name the input image and desired
output filename.  The image files may be of any format recognized by
\product (i.e., for which \ImageInput plugins are available).  The file
format of the output image will be inferred from the file extension of
the output filename (e.g., \qkw{foo.tif} will write a TIFF file).


\section{{\cf iconvert} Recipes}

This section will give quick examples of common uses of {\cf iconvert}.

\subsection*{Converting between file formats}

It's a snap to converting among image formats supported by \product
(i.e., for which \ImageInput and \ImageOutput plugins can be found).
The {\cf iconvert} utility will simply infer the file format from the
file extension. The following example converts a PNG image to JPEG:

\begin{code}
    iconvert lena.png lena.jpg
\end{code}

\subsection*{Changing the data format or bit depth}

Just use the {\cf -d} option to specify a pixel data format.  For
example, assuming that {\cf in.tif} uses 16-bit unsigned integer
pixels, the following will convert it to an 8-bit unsigned pixels:

\begin{code}
    iconvert -d uint8 in.tif out.tif
\end{code}

\subsection*{Changing the compression}

The following command converts writes a TIFF file, specifically using
LZW compression:

\begin{code}
    iconvert --compression lzw in.tif out.tif
\end{code}

The following command writes its results as a JPEG file at a 
compression quality of 50 (pretty severe compression):

\begin{code}
    iconvert --quality 50 big.jpg small.jpg
\end{code}

\subsection*{Gamma-correcting an image}

The following gamma-corrects the pixels, raising all pixel
values to $x^{1/2.2}$ upon writing:

\begin{code}
    iconvert -g 2.2 in.tif out.tif
\end{code}

\subsection*{Converting between scanline and tiled images}

Convert a scanline file to a tiled file with $16 \times 16$ tiles:

\begin{code}
    iconvert --tile 16 16 s.tif t.tif
\end{code}

\noindent Convert a tiled file to scanline:

\begin{code}
    iconvert --scanline t.tif s.tif
\end{code}


\section{{\cf iconvert} command-line options}

\apiitem{--help}
Prints usage information to the terminal.
\apiend

\apiitem{-v}
Verbose status messages.
\apiend

\apiitem{-d {\rm \emph{datatype}}}

Attempt to sets the output pixel data type to one of: {\cf uint8}, 
{\cf sint8}, {\cf uint16}, {\cf sint16}, {\cf half}, {\cf float}, 
{\cf double}.

If the {\cf -d} option is not supplied, the output data type will
be the same as the data format of the input file.

In either case, the output file format itself (implied by the file
extension of the output filename) may trump the request if the file
format simply does not support the requested data type.
\apiend

\apiitem{-g {\rm \emph{gamma}}}
Applies a gamma correction of $1/\mathrm{gamma}$ to the pixels as they
are output.
\apiend

\apiitem{--tile {\rm \emph{x}} {\rm \emph{y}}}
Requests that the output file be tiled, with the given $x \times y$ 
tile size, if tiled images are supported by the output format.
By default, the output file will take on the tiledness and tile size
of the input file.
\apiend

\apiitem{--scanline}
Requests that the output file be scanline-oriented (even if the input
file was tile-oriented), if scanline orientation is supported by the
output file format.  By default, the output file will be scanline
if the input is scanline, or tiled if the input is tiled.
\apiend

\apiitem{--compression {\rm \emph{method}}}
Sets the compression method for the output image.  Each \ImageOutput
plugin will have its own set of methods that it supports.

By default, the output image will use the same compression technique as
the input image (assuming it is supported by the output format,
otherwise it will use the default compression method of the output
plugin).  
\apiend

\apiitem{--quality {\rm \emph{q}}}
Sets the compression quality, on a 1--100 floating-point scale.
This only has an effect if the particular compression method supports
a quality metric (as JPEG does).
\apiend
