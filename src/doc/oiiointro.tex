\chapter{Introduction}
\label{chap:oiiointro}



Welcome to \product!

\bigskip

\section{Overview}

\product provides simple but powerful \ImageInput and \ImageOutput APIs
that abstract the reading and writing of 2D image file formats.  They
don't support every possible way of encoding images in memory, but for a
reasonable and common set of desired functionality, they provide an
exceptionally easy way for an application using the APIs support a wide
--- and extensible --- selection of image formats without knowing the
details of any of these formats.

Concrete instances of these APIs, each of which implements the ability
to read and/or write a different image file format, are stored as
plugins (i.e., dynamic libraries, DLL's, or DSO's) that are loaded at
runtime.  The \product distribution contains such plugins for several
popular formats.  Any user may create conforming plugins that implement
reading and writing capabilities for other image formats, and any
application that uses \product would be able to use those plugins.

The library also implements the helper class {\kw ImageBuf}, which is a
handy way to store and manipulate images in memory.  {\kw ImageBuf}
itself uses \ImageInput and \ImageOutput for its file I/O, and therefore
is also agnostic as to image file formats.

The {\kw ImageCache} class transparently manages a cache so that it can
access truly vast amounts of image data (thousands of image files
totaling tens of GB) very efficiently using only a tiny amount (tens of
megabytes at most) of runtime memory.  Additionally, a {\kw
  TextureSystem} class provides filtered MIP-map texture lookups, atop
the nice caching behavior of {\kw ImageCache}.

Finally, the \product distribution contains several utility programs
that operate on images, each of which is built atop \ImageInput and
\ImageOutput, and therefore may read or write any image file type for
which an appropriate plugin is found at runtime.  Paramount among these
utilities is {\fn iv}, a really fantastic and powerful image viewing
application.  Additionally, there are programs for converting images
among different formats, comparing image data between two images, 
and examining image metadata.

All of this is released as ``open source'' software using the very
permissive BSD license.  So you should feel free to use any or all of
\product in your own software, whether it is private or public, open
source or proprietary, free or commercial.  You may also modify it on
your own.  You are also encouraged to contribute to the continued
development of \product and to share any improvements that you make on
your own, though you are by no means required to do so.

\section{Simplifying Assumptions}

\product is not the only image library in the world.  Certainly there
are many fine libraries that implement a single image format (including
the excellent {\fn libtiff}, {\fn jpeg-6b}, and {\fn OpenEXR} that
\product itself relies on).  Many libraries attempt to present a uniform
API for reading and writing multiple image file formats.  Most of these
support a fixed set of image formats, though a few of these
also attempt to provide an extensible set by using the plugin approach.

But in our experience, these libraries are all flawed in one or more
ways: (1) They either support only a few formats, or many formats but
with the majority of them somehow incomplete or incorrect.  (2) Their
APIs are not sufficiently expressive as to handle all the image features
we need (such as tiled images, which is critical for our texture
library).  (3) Their APIs are \emph{too complete}, trying to handle
every possible permutation of image format features, and as a result
are horribly complicated.

The third sin is the most severe, and is almost always the main problem
at the end of the day.  Even among the many open source image libraries
that rely on extensible plugins, we have not found one that is both
sufficiently flexible and has APIs anywhere near as simple to understand
and use as those of \product.

Good design is usually a matter of deciding what \emph{not} to do, and
\product is no exception.  We achieve power and elegance only by
making simplifying assumptions.  Among them:

\begin{itemize}
  \item \product only deals with ordinary 2D images, and to a limited
    extent 3D volumes, and image files that contain multiple (but
    finite) independent images within them.  \product {\bf~ does not deal
      with motion picture files.}  At least, not currently.

  \item Pixel data are 8- 16- or 32-bit int (signed or unsigned), 16-
    32- or 64-bit float.  NOTHING ELSE.  No $<8$ bit images, or pixels
    boundaries that aren't byte boundaries.  Files with $<8$ bits will
    appear to the client as 8-bit unsigned grayscale images.

  \item Only fully elaborated, non-compressed data are accepted
    and returned by the API.  Compression or special encodings are
    handled entirely within an \product plugin.

  \item Color space is grayscale or RGB.  Non-spectral data, such as
    XYZ, CMYK, or YUV, are converted to RGB upon reading.\

  \item All color channels have the same data format.  Upon read, an
    \ImageInput ought to convert all channels to the one with the highest
    precision in the file.

  \item All image channels in a subimage are sampled at the same
    resolution.  For file formats that allow some channels to be
    subsampled, they will be automatically up-sampled to the highest
    resolution channel in the subimage.

  \item Color information is always in the order R, G, B, and the alpha
    channel, if any, always follows RGB, and z channel (if any) always
    follows alpha.  So if a file actually stores ABGR, the plugin is
    expected to rearrange it as RGBA.

\end{itemize}

It's important to remember that these restrictions apply to data passed
through the APIs, not to the files themselves.  It's perfectly fine to
have an \product plugin that supports YUV data, or 4 bits per channel, or
any other exotic feature.  You could even write a movie-reading
\ImageInput (despite \product's claims of not supporting movies) and
make it look to the client like it's just a series of images within the
file.  It's just that all the nonconforming details are handled entirely
within the \product plugin and are not exposed through the main \product
APIs.


\subsection*{Historical Origins}

\product is the evolution of concepts and tools I've been working on 
for two decades.

In the 1980's, every program I wrote that output images would have a
simple, custom format and viewer.  I soon graduated to using a standard
image file format (TIFF) with my own library implementation.  Then I
switched to Sam Leffler's stable and complete {\fn libtiff}.

In the mid-to-late-1990's, I worked at Pixar as one of the main
implementors of PhotoRealistic RenderMan, which had \emph{display
  drivers} that consisted of an API for opening files and outputting
pixels, and a set of DSO/DLL plugins that each implement image output
for each of a dozen or so different file format.  The plugins all
responded to the same API, so the renderer itself did not need to know
how to the details of the image file formats, and users could (in
theory, but rarely in practice) extend the set of output image formats
the renderer could use by writing their own plugins.

This was the seed of a good idea, but PRMan's display driver plugin API
was abstruse and hard to use.  So when I started Exluna in 2000, Matt
Pharr, Craig Kolb, and I designed a new API for image output for our own
renderer, Entropy.  This API, called ``ExDisplay,'' was C++, and much
simpler, clearer, and easier to use than PRMan's display drivers.

NVIDIA's Gelato (circa 2002), whose early work was done by myself, Dan
Wexler, Jonathan Rice, and Eric Enderton, had an API
called ``ImageIO.''  ImageIO was 
\emph{much} more powerful and descriptive than ExDisplay, and had an
API for \emph{reading} as well as writing images.  Gelato was not only
``format agnostic'' for its image output, but also for its
image input (textures, image viewer, and other image utilities).
We released the API specification and headers (though not the
library implementation) using the BSD open source license, firmly
repudiating any notion that the API should be specific to NVIDIA or
Gelato.

For Gelato 3.0 (circa 2007), we refined ImageIO again (by this time,
Philip Nemec was also a major influence, in addition to Dan, Eric, and
myself\footnote{Gelato as a whole had many other contributors; those
  I've named here are the ones I recall contributing to the design or
  implementation of the ImageIO APIs}).  This revision was not a major
overhaul but more of a fine tuning.  Our ideas were clearly approaching
stability.  But, alas, the Gelato project was canceled before Gelato 3.0
was released, and despite our prodding, NVIDIA executives would not open
source the full ImageIO code and related tools.

After I left NVIDIA, I was determined to recreate this work once
again -- and ONLY once more -- and release it as open source from the
start.  Thus, \product was born.  I started with the existing Gelato
ImageIO specification and headers (which were BSD licensed all along),
and made some more refinements since I had to rewrite the entire
implementation from scratch anyway.  I think the additional changes are
all improvements.  This is the software you have in your hands today.


\subsection*{Acknowledgments}

\begin{comment}
The direct precursor to \product was Gelato's ImageIO, which was
co-designed and implemented by Larry Gritz, Dan Wexler, Jonathan Rice, Eric
Enderton, and Philip Nemec.

Big thanks to our bosses at NVIDIA for allowing us to share the API spec
and headers under the BSD license.  And thanks to their inability to
open source their own implementation in a timely manner, I was forced to
create this clearly superior descendant.
\end{comment}

\product incorporates or depends upon several other open source
packages:

\begin{itemize}
\item {\cf libtiff} \copyright 1988-1997 Sam Leffler and 1991-1997 Silicon
Graphics, Inc. \\ {\cf http://libtiff.org}
\item {\cf jpeg-6b} \copyright 1991-1998, Thomas G. Lane.  {\cf http://www.ijg.org}
\item OpenEXR, Ilmbase, and Half \copyright 2006, Industrial Light \& Magic.\\
{\cf http://www.openexr.com}
\item {\cf zlib} \copyright 1995-2005 Jean-loup Gailly and Mark Adler. 
{\cf http://www.zlib.net}
\item {\cf libpng} \copyright 1998-2008 Glenn Randers-Pehrson, et al.  
{\cf http://www.libpng.org}
\item The {\cf atomic<>} template in {\cf src/include/tbb} is from
Intel's Thread Building Blocks, \copyright 2005--2008 Intel Corporation.
{\cf http://www.threadingbuildingblocks.org/}
\item The SHA-1 implemenation we use is public domain by
Dominik Reichl \\ {\cf http://www.dominik-reichl.de/}
\item Boost {\cf http://www.boost.org}
\item GLEW \copyright 2002-2007 Milan Ikits, et al. 
{\cf http://glew.sourceforge.net}
\item Jasper \copyright 2001-2006 Michael David Adams, et al. \\
{\cf http://www.ece.uvic.ca/~mdadams/jasper/}
\end{itemize}

These other packages are all distributed under licenses that allow them
to be used by and distributed with \product.

\chapwidthend
