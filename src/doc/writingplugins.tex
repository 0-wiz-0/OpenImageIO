\chapter{Writing ImageIO Plugins}
\label{chap:writingplugins}



\section{Plugin Introduction}
\label{sec:pluginintro}

As explained in Chapters~\ref{chap:imageinput} and
\ref{chap:imageoutput}, the ImageIO library does not know how to read or
write any particular image formats, but rather relies on plugins located
and loaded dynamically at run-time.  This set of plugins, and therefore
the set of image file formats that \product or its clients can read and
write, is extensible without needing to modify \product itself.  

This chapter explains how to write your own \product plugins.  We will
first explain separately how to write image file readers and writers,
then tie up the loose ends of how to build the plugins themselves.

\section{Image Readers}
\label{sec:pluginreaders}

A plugin that reads a particular image file format must implement a
\emph{subclass} of \ImageInput (described in
Chapter~\ref{chap:imageinput}).  This is actually very straightforward
and consists of the following steps, which we will illustrate with a
real-world example of writing a JPEG/JFIF plug-in.

\begin{enumerate}
\item Read the base class definition from {\fn imageio.h}.  It may also
  be helpful to just use the {\cf OpenImageIO} namespace, just to 
  make your code a little less verbose.

  \begin{code}
    #include "imageio.h"
    using namespace OpenImageIO;
  \end{code}

\item Declare three public items:

  \begin{enumerate}
    \item An integer called \emph{name}{\cf _imageio_version} that identifies
      the version of the ImageIO protocol implemented by the plugin,
      defined in {\fn imageio.h} as the constant {\cf OPENIMAGEIO_PLUGIN_VERSION}.
      This allows the library to be sure it is not loading a plugin
      that was compiled against an incompatible version of \product.
    \item A function named \emph{name}{\cf _input_imageio_create} that
      takes no arguments and returns a new instance of your \ImageInput
      subclass.  (Note that \emph{name} is the name of your format,
      and must match the name of the plugin itself.)
    \item An array of {\cf char *} called \emph{name}{\cf _input_extensions}
      that contains the list of file extensions that are likely to indicate
      a file of the right format.  The list is terminated by a {\cf NULL}
      pointer.
  \end{enumerate}

  All of these items must be inside an `{\cf extern "C"}' block in order
  to avoid name mangling by the C++ compiler.  Depending on your
  compiler, you may need to use special commands to dictate that the
  symbols will be exported in the DSO; we provide a special {\cf
  DLLEXPORT} macro for this purpose, defined in {\fn export.h}.

  Putting this all together, we get the following for our JPEG example:

  \begin{code}
    extern "C" {
        DLLEXPORT int jpeg_imageio_version = OPENIMAGEIO_PLUGIN_VERSION;
        DLLEXPORT JpgInput *jpeg_input_imageio_create () {
            return new JpgInput;
        }
        DLLEXPORT const char *jpeg_input_extensions[] = {
            "jpg", "jpe", "jpeg", NULL
        };
    };
  \end{code}

\item The definition and implementation of an \ImageInput subclass for
  this file format.  It must publicly inherit \ImageInput, and must
  overload the following methods which are ``pure virtual'' in the
  \ImageInput base class:

  \begin{enumerate}
    \item {\cf format_name()} should return the name of the format, which
      ought to match the name of the plugin and by convention is
      strictly lower-case and contains no whitespace.
    \item {\cf open()} should open the file and return true, or should
      return false if unable to do so (including if the file was found
      but turned out not to be in the format that your plugin is trying
      to implement).
    \item {\cf close()} should close the file, if open.
    \item {\cf read_native_scanline} should read a single scanline from
      the file into the address provided, uncompressing it but
      keeping it in its native data format without any translation.
    \item The virtual destructor, which should {\cf close()} if the file
      is still open, addition to performing any other tear-down activities.
  \end{enumerate}
  
  Additionally, your \ImageInput subclass may optionally choose to
  overload any of the following methods, which are defined in the
  \ImageInput base class and only need to be overloaded if the default
  behavior is not appropriate for your plugin:

  \begin{enumerate}
    \item[(f)] {\cf seek_subimage()}, only if your format supports
      reading multiple subimages within a single file.
    \item[(g)] {\cf read_native_tile()}, only if your format supports
      reading tiled images.
  \end{enumerate}

  Here is how the class definition looks for our JPEG example.  Note
  that the JPEG/JFIF file format does not support multiple subimages
  or tiled images.

  \begin{code}
    class JpgInput : public ImageInput {
     public:
        JpgInput () { init(); }
        virtual ~JpgInput () { close(); }
        virtual const char * format_name (void) const { return "jpeg"; }
        virtual bool open (const std::string &name, ImageSpec &spec);
        virtual bool read_native_scanline (int y, int z, void *data);
        virtual bool close ();
     private:
        FILE *m_fd;
        bool m_first_scanline;
        struct jpeg_decompress_struct m_cinfo;
        struct jpeg_error_mgr m_jerr;

        void init () { m_fd = NULL; }
    };
  \end{code}
\end{enumerate}

Your subclass implementation of {\cf open()}, {\cf close()}, and {\cf
  read_native_scanline()} are the heart of an \ImageInput
implementation.  (Also {\cf read_native_tile()} and {\cf
  seek_subimage()}, for those image formats that support them.)

The remainder of this section simply lists the full implementation of
our JPEG reader, which relies heavily on the open source {\fn jpeg-6b}
library to perform the actual JPEG decoding.

\includedcode{../jpeg.imageio/jpeginput.cpp}




\section{Image Writers}
\label{sec:pluginwriters}

A plugin that writes a particular image file format must implement a
\emph{subclass} of \ImageOutput (described in
Chapter~\ref{chap:imageoutput}).  This is actually very straightforward
and consists of the following steps, which we will illustrate with a
real-world example of writing a JPEG/JFIF plug-in.

\begin{enumerate}
\item Read the base class definition from {\fn imageio.h}, just as
  with an image reader (see Section~\ref{sec:pluginreaders}).

\item Declare three public items:

  \begin{enumerate}
    \item An integer called \emph{name}{\cf _imageio_version} that identifies
      the version of the ImageIO protocol implemented by the plugin,
      defined in {\fn imageio.h} as the constant {\cf OPENIMAGEIO_PLUGIN_VERSION}.
      This allows the library to be sure it is not loading a plugin
      that was compiled against an incompatible version of \product.
      Note that if your plugin has both a reader and writer and they
      are compiled as separate modules (C++ source files), you don't
      want to declare this in \emph{both} modules; either one is fine.
    \item A function named \emph{name}{\cf _output_imageio_create} that
      takes no arguments and returns a new instance of your \ImageOutput
      subclass.  (Note that \emph{name} is the name of your format,
      and must match the name of the plugin itself.)
    \item An array of {\cf char *} called \emph{name}{\cf _output_extensions}
      that contains the list of file extensions that are likely to indicate
      a file of the right format.  The list is terminated by a {\cf NULL}
      pointer.
  \end{enumerate}

  All of these items must be inside an `{\cf extern "C"}' block in order
  to avoid name mangling by the C++ compiler.  Depending on your
  compiler, you may need to use special commands to dictate that the
  symbols will be exported in the DSO; we provide a special {\cf
  DLLEXPORT} macro for this purpose, defined in {\fn export.h}.

  Putting this all together, we get the following for our JPEG example:

  \begin{code}
    extern "C" {
        DLLEXPORT int jpeg_imageio_version = OPENIMAGEIO_PLUGIN_VERSION;
        DLLEXPORT JpgOutput *jpeg_output_imageio_create () {
            return new JpgOutput;
        }
        DLLEXPORT const char *jpeg_input_extensions[] = {
            "jpg", "jpe", "jpeg", NULL
        };
    };
  \end{code}

\item The definition and implementation of an \ImageOutput subclass for
  this file format.  It must publicly inherit \ImageOutput, and must
  overload the following methods which are ``pure virtual'' in the
  \ImageOutput base class:

  \begin{enumerate}
    \item {\cf format_name()} should return the name of the format, which
      ought to match the name of the plugin and by convention is
      strictly lower-case and contains no whitespace.
    \item {\cf supports()} should return {\cf true} if its argument
      names a feature supported by your format plugin, {\cf false} if it
      names a feature not supported by your plugin.  See
      Section~\ref{sec:supportsfeaturelist} for the list of feature
      names.
    \item {\cf open()} should open the file and return true, or should
      return false if unable to do so (including if the file was found
      but turned out not to be in the format that your plugin is trying
      to implement).
    \item {\cf close()} should close the file, if open.
    \item {\cf write_scanline} should write a single scanline to
      the file, translating from internal to native data format and
      handling strides properly.
    \item The virtual destructor, which should {\cf close()} if the file
      is still open, addition to performing any other tear-down activities.
  \end{enumerate}
  
  Additionally, your \ImageOutput subclass may optionally choose to
  overload any of the following methods, which are defined in the
  \ImageOutput base class and only need to be overloaded if the default
  behavior is not appropriate for your plugin:

  \begin{enumerate}
    \item[(g)] {\cf write_tile()}, only if your format supports
      writing tiled images.
    \item[(h)] {\cf write_rectangle()}, only if your format supports
      writing arbitrary rectangles.
    \item[(i)] {\cf write_image()}, only if you have a more clever
      method of doing so than the default implementation that calls
      {\cf write_scanline()} or {\cf write_tile()} repeatedly.
  \end{enumerate}

  Here is how the class definition looks for our JPEG example.  Note
  that the JPEG/JFIF file format does not support multiple subimages
  or tiled images.

  \begin{code}
    class JpgOutput : public ImageOutput {
     public:
        JpgOutput () { init(); }
        virtual ~JpgOutput () { close(); }
        virtual const char * format_name (void) const { return "jpeg"; }
        virtual bool supports (const std::string &property) const { return false; }
        virtual bool open (const std::string &name, const ImageSpec &spec,
                           bool append=false);
        virtual bool write_scanline (int y, int z, TypeDesc format,
                                     const void *data, stride_t xstride);
        bool close ();
     private:
        FILE *m_fd;
        std::vector<unsigned char> m_scratch;
        struct jpeg_compress_struct m_cinfo;
        struct jpeg_error_mgr m_jerr;

        void init () { m_fd = NULL; }
    };
  \end{code}
\end{enumerate}

Your subclass implementation of {\cf open()}, {\cf close()}, and {\cf
  write_scanline()} are the heart of an \ImageOutput implementation.
(Also {\cf write_tile()}, for those image formats that support tiled
output.)

An \ImageOutput implementation must properly handle all data formats and
strides passed to {\cf write_scanline()} or {\cf write_tile()}, unlike
an \ImageInput implementation, which only needs to read scanlines or
tiles in their native format and then have the super-class handle the
translation.  But don't worry, all the heavy lifting can be accomplished
with the following helper functions provided as protected member
functions of \ImageOutput:

\apiitem{const void * {\ce to_native_scanline} (TypeDesc format,
                                    const void *data, \\ \bigspc stride_t xstride,
                                    std::vector<unsigned char>
                                    \&scratch);}

Convert a full scanline of pixels (pointed to by \emph{data}) with the
given \emph{format} and strides into contiguous pixels in the native
format (described by the \ImageSpec returned by the {\cf spec()} member
function).  The location of the newly converted data is returned, which
may either be the original \emph{data} itself if no data conversion was
necessary and the requested layout was contiguous (thereby avoiding
unnecessary memory copies), or may point into memory allocated within
the \emph{scratch} vector passed by the user.  In either case, the
caller doesn't need to worry about thread safety or freeing any
allocated memory (other than eventually destroying the scratch vector).
\apiend

\apiitem{const void * {\ce to_native_tile} (TypeDesc format, const
  void *data,\\ \bigspc
                                stride_t xstride, stride_t ystride, stride_t zstride,\\ \bigspc
                                std::vector<unsigned char> \&scratch);}

Convert a full tile of pixels (pointed to by \emph{data}) with the given
\emph{format} and strides into contiguous pixels in the native format
(described by the \ImageSpec returned by the {\cf spec()} member
function).  The location of the newly converted data is returned, which
may either be the original \emph{data} itself if no data conversion was
necessary and the requested layout was contiguous (thereby avoiding
unnecessary memory copies), or may point into memory allocated within
the \emph{scratch} vector passed by the user.  In either case, the
caller doesn't need to worry about thread safety or freeing any
allocated memory (other than eventually destroying the scratch vector).

\apiend

\apiitem{const void * {\ce to_native_rectangle} (int xmin, int xmax, int
  ymin, int ymax, \\ \bigspc
                                     int zmin, int zmax, 
                                     TypeDesc format, const void
                                     *data, \\ \bigspc
                                     stride_t xstride, stride_t ystride,
                                     stride_t zstride, \\ \bigspc
                                     std::vector<unsigned char> \&scratch);}

Convert a rectangle of pixels (pointed to by \emph{data}) with the given
\emph{format}, dimensions, and strides into contiguous pixels in the
native format (described by the \ImageSpec returned by the {\cf spec()}
member function).  The location of the newly converted data is returned,
which may either be the original \emph{data} itself if no data
conversion was necessary and the requested layout was contiguous
(thereby avoiding unnecessary memory copies), or may point into memory
allocated within the \emph{scratch} vector passed by the user.  In
either case, the caller doesn't need to worry about thread safety or
freeing any allocated memory (other than eventually destroying the
scratch vector).

\apiend

\bigskip
\bigskip

\noindent
The remainder of this section simply lists the full implementation of
our JPEG writer, which relies heavily on the open source {\fn jpeg-6b}
library to perform the actual JPEG encoding.

\includedcode{../jpeg.imageio/jpegoutput.cpp}


\section{Building ImageIO Plugins}
\label{sec:buildingplugins}

FIXME -- spell out how to compile and link plugins on each of the major
platforms.


\chapwidthend
