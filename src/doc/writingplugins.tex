\chapter{Writing ImageIO Plugins}
\label{chap:writingplugins}



\section{Plugin Introduction}
\label{sec:pluginintro}

As explained in Chapters~\ref{chap:imageinput} and
\ref{chap:imageoutput}, the ImageIO library does not know how to read or
write any particular image formats, but rather relies on plugins located
and loaded dynamically at run-time.  This set of plugins, and therefore
the set of image file formats that \product or its clients can read and
write, is extensible without needing to modify \product itself.  

This chapter explains how to write your own \product plugins.  We will
first explain separately how to write image file readers and writers,
then tie up the loose ends of how to build the plugins themselves.

\section{Image Readers}
\label{sec:pluginreaders}

A plugin that reads a particular image file format must implement a
\emph{subclass} of \ImageInput (described in
Chapter~\ref{chap:imageinput}).  This is actually very straightforward
and consists of the following steps, which we will illustrate with a
real-world example of writing a JPEG/JFIF plug-in.

\begin{enumerate}
\item Read the base class definition from {\fn imageio.h}.  It may also
  be helpful to enclose the contents of your plugin in the same
  namespace that the \product library uses:

  \begin{code}
    #include <OpenImageIO/imageio.h>
    OIIO_PLUGIN_NAMESPACE_BEGIN

    ... everything else ...

    OIIO_PLUGIN_NAMESPACE_END
  \end{code}

\item Declare these public items:

  \begin{enumerate}
    \item An integer called \emph{name}{\cf _imageio_version} that identifies
      the version of the ImageIO protocol implemented by the plugin,
      defined in {\fn imageio.h} as the constant {\cf OIIO_PLUGIN_VERSION}.
      This allows the library to be sure it is not loading a plugin
      that was compiled against an incompatible version of \product.
    \item An function named \emph{name}{\cf _imageio_library_version} that identifies
      the underlying dependent library that is responsible for reading or
      writing the format (it may return {\cf NULL} to indicate that there is
      no dependent library being used for this format).
    \item A function named \emph{name}{\cf _input_imageio_create} that
      takes no arguments and returns a new instance of your \ImageInput
      subclass.  (Note that \emph{name} is the name of your format,
      and must match the name of the plugin itself.)
    \item An array of {\cf char *} called \emph{name}{\cf _input_extensions}
      that contains the list of file extensions that are likely to indicate
      a file of the right format.  The list is terminated by a {\cf NULL}
      pointer.
  \end{enumerate}

  All of these items must be inside an `{\cf extern "C"}' block in order
  to avoid name mangling by the C++ compiler, and we provide handy
  macros {\cf OIIO_PLUGIN_EXPORTS_BEGIN} and {\cf OIIO_PLUGIN_EXPORTS_END}
  to make this easy.  Depending on your
  compiler, you may need to use special commands to dictate that the
  symbols will be exported in the DSO; we provide a special {\cf
  OIIO_EXPORT} macro for this purpose, defined in {\fn export.h}.

  Putting this all together, we get the following for our JPEG example:

  \begin{code}
    OIIO_PLUGIN_EXPORTS_BEGIN
        OIIO_EXPORT int jpeg_imageio_version = OIIO_PLUGIN_VERSION;
        OIIO_EXPORT JpgInput *jpeg_input_imageio_create () {
            return new JpgInput;
        }
        OIIO_EXPORT const char *jpeg_input_extensions[] = {
            "jpg", "jpe", "jpeg", "jif", "jfif", "jfi", NULL
        };
        OIIO_EXPORT const char* jpeg_imageio_library_version () {
          #define STRINGIZE2(a) #a
          #define STRINGIZE(a) STRINGIZE2(a)
          #ifdef LIBJPEG_TURBO_VERSION
            return "jpeg-turbo " STRINGIZE(LIBJPEG_TURBO_VERSION);
          #else
            return "jpeglib " STRINGIZE(JPEG_LIB_VERSION_MAJOR) "." \\
                    STRINGIZE(JPEG_LIB_VERSION_MINOR);
          #endif
        }
    OIIO_PLUGIN_EXPORTS_END
  \end{code}

\item The definition and implementation of an \ImageInput subclass for
  this file format.  It must publicly inherit \ImageInput, and must
  overload the following methods which are ``pure virtual'' in the
  \ImageInput base class:

  \begin{enumerate}
    \item {\cf format_name()} should return the name of the format, which
      ought to match the name of the plugin and by convention is
      strictly lower-case and contains no whitespace.
    \item {\cf open()} should open the file and return true, or should
      return false if unable to do so (including if the file was found
      but turned out not to be in the format that your plugin is trying
      to implement).
    \item {\cf close()} should close the file, if open.
    \item {\cf read_native_scanline} should read a single scanline from
      the file into the address provided, uncompressing it but
      keeping it in its native data format without any translation.
    \item The virtual destructor, which should {\cf close()} if the file
      is still open, addition to performing any other tear-down activities.
  \end{enumerate}
  
  Additionally, your \ImageInput subclass may optionally choose to
  overload any of the following methods, which are defined in the
  \ImageInput base class and only need to be overloaded if the default
  behavior is not appropriate for your plugin:

  \begin{enumerate}
    \item[(f)] {\cf supports()}, only if your format supports any of
      the optional features described in
      Section~\ref{sec:inputsupportsfeaturelist}.
    \item[(g)] {\cf valid_file()}, if your format has a way to
      determine if a file is of the given format in a way that is less
      expensive than a full {\cf open()}.
    \item[(h)] {\cf seek_subimage()}, only if your format supports
      reading multiple subimages within a single file.
    \item[(i)] {\cf read_native_scanlines()}, only if your format has a speed
      advantage when reading multiple scanlines at once.  If you do not
      supply this function, the default implementation will simply call
      {\cf read_scanline()} for each scanline in the range.
    \item[(j)] {\cf read_native_tile()}, only if your format supports
      reading tiled images.
    \item[(k)] {\cf read_native_tiles()}, only if your format supports
      reading tiled images and there is a speed advantage when reading
      multiple tiles at once.  If you do not supply this function, the
      default implementation will simply call {\cf read_native_tile()} for each
      tile in the range.
    \item[(l)] ``Channel subset'' versions of {\cf read_native_scanlines()}
      and/or {\cf read_native_tiles()}, only if your format has a more
      efficient means of reading a subset of channels.  If you do not
      supply these methods, the default implementation will simply use
      {\cf read_native_scanlines()} or {\cf read_native_tiles()} to read
      into a temporary all-channel buffer and then copy the channel
      subset into the user's buffer.
    \item[(m)] {\cf read_native_deep_scanlines()} and/or 
      {\cf read_native_deep_tiles()}, only if your format supports
      ``deep'' data images.
  \end{enumerate}

  Here is how the class definition looks for our JPEG example.  Note
  that the JPEG/JFIF file format does not support multiple subimages
  or tiled images.

  \begin{code}
    class JpgInput final : public ImageInput {
     public:
        JpgInput () { init(); }
        virtual ~JpgInput () { close(); }
        virtual const char * format_name (void) const override { return "jpeg"; }
        virtual bool open (const std::string &name, ImageSpec &spec) override;
        virtual bool read_native_scanline (int y, int z, void *data) override;
        virtual bool close () override;
     private:
        FILE *m_fd;
        bool m_first_scanline;
        struct jpeg_decompress_struct m_cinfo;
        struct jpeg_error_mgr m_jerr;

        void init () { m_fd = NULL; }
    };
  \end{code}
\end{enumerate}

Your subclass implementation of {\cf open()}, {\cf close()}, and {\cf
  read_native_scanline()} are the heart of an \ImageInput
implementation.  (Also {\cf read_native_tile()} and {\cf
  seek_subimage()}, for those image formats that support them.)

The remainder of this section simply lists the full implementation of
our JPEG reader, which relies heavily on the open source {\fn jpeg-6b}
library to perform the actual JPEG decoding.

\tinyincludedcode{../jpeg.imageio/jpeginput.cpp}




\section{Image Writers}
\label{sec:pluginwriters}

A plugin that writes a particular image file format must implement a
\emph{subclass} of \ImageOutput (described in
Chapter~\ref{chap:imageoutput}).  This is actually very straightforward
and consists of the following steps, which we will illustrate with a
real-world example of writing a JPEG/JFIF plug-in.

\begin{enumerate}
\item Read the base class definition from {\fn imageio.h}, just as
  with an image reader (see Section~\ref{sec:pluginreaders}).

\item Declare three public items:

  \begin{enumerate}
    \item An integer called \emph{name}{\cf _imageio_version} that identifies
      the version of the ImageIO protocol implemented by the plugin,
      defined in {\fn imageio.h} as the constant {\cf OIIO_PLUGIN_VERSION}.
      This allows the library to be sure it is not loading a plugin
      that was compiled against an incompatible version of \product.
      Note that if your plugin has both a reader and writer and they
      are compiled as separate modules (C++ source files), you don't
      want to declare this in \emph{both} modules; either one is fine.
    \item A function named \emph{name}{\cf _output_imageio_create} that
      takes no arguments and returns a new instance of your \ImageOutput
      subclass.  (Note that \emph{name} is the name of your format,
      and must match the name of the plugin itself.)
    \item An array of {\cf char *} called \emph{name}{\cf _output_extensions}
      that contains the list of file extensions that are likely to indicate
      a file of the right format.  The list is terminated by a {\cf NULL}
      pointer.
  \end{enumerate}

  All of these items must be inside an `{\cf extern "C"}' block in order
  to avoid name mangling by the C++ compiler, and we provide handy
  macros {\cf OIIO_PLUGIN_EXPORTS_BEGIN} and {\cf OIIO_PLUGIN_EXPORTS_END}
  to mamke this easy.  Depending on your
  compiler, you may need to use special commands to dictate that the
  symbols will be exported in the DSO; we provide a special {\cf
  OIIO_EXPORT} macro for this purpose, defined in {\fn export.h}.

  Putting this all together, we get the following for our JPEG example:

  \begin{code}
    OIIO_PLUGIN_EXPORTS_BEGIN
        OIIO_EXPORT int jpeg_imageio_version = OIIO_PLUGIN_VERSION;
        OIIO_EXPORT JpgOutput *jpeg_output_imageio_create () {
            return new JpgOutput;
        }
        OIIO_EXPORT const char *jpeg_input_extensions[] = {
            "jpg", "jpe", "jpeg", NULL
        };
    OIIO_PLUGIN_EXPORTS_END
  \end{code}

\item The definition and implementation of an \ImageOutput subclass for
  this file format.  It must publicly inherit \ImageOutput, and must
  overload the following methods which are ``pure virtual'' in the
  \ImageOutput base class:

  \begin{enumerate}
    \item {\cf format_name()} should return the name of the format, which
      ought to match the name of the plugin and by convention is
      strictly lower-case and contains no whitespace.
    \item {\cf supports()} should return {\cf true} if its argument
      names a feature supported by your format plugin, {\cf false} if it
      names a feature not supported by your plugin.  See
      Section~\ref{sec:supportsfeaturelist} for the list of feature
      names.
    \item {\cf open()} should open the file and return true, or should
      return false if unable to do so (including if the file was found
      but turned out not to be in the format that your plugin is trying
      to implement).
    \item {\cf close()} should close the file, if open.
    \item {\cf write_scanline} should write a single scanline to
      the file, translating from internal to native data format and
      handling strides properly.
    \item The virtual destructor, which should {\cf close()} if the file
      is still open, addition to performing any other tear-down activities.
  \end{enumerate}
  
  Additionally, your \ImageOutput subclass may optionally choose to
  overload any of the following methods, which are defined in the
  \ImageOutput base class and only need to be overloaded if the default
  behavior is not appropriate for your plugin:

  \begin{enumerate}
    \item[(g)] {\cf write_scanlines()}, only if your format supports
      writing scanlines and you can get a performance improvement when
      outputting multiple scanlines at once.  If you don't supply
      {\cf write_scanlines()}, the default implementation will simply
      call {\cf write_scanline()} separately for each scanline in the
      range.
    \item[(h)] {\cf write_tile()}, only if your format supports
      writing tiled images.
    \item[(i)] {\cf write_tiles()}, only if your format supports
      writing tiled images and you can get a performance improvement
      when outputting multiple tiles at once.  If you don't supply
      {\cf write_tiles()}, the default implementation will simply
      call {\cf write_tile()} separately for each tile in the range.
    \item[(j)] {\cf write_rectangle()}, only if your format supports
      writing arbitrary rectangles.
    \item[(k)] {\cf write_image()}, only if you have a more clever
      method of doing so than the default implementation that calls
      {\cf write_scanline()} or {\cf write_tile()} repeatedly.
    \item[(l)] {\cf write_deep_scanlines()} and/or 
      {\cf write_deep_tiles()}, only if your format supports
      ``deep'' data images.
  \end{enumerate}

  It is not strictly required, but certainly appreciated, if a file format
  does not support tiles, to nonetheless accept an \ImageSpec that specifies
  tile sizes by allocating a full-image buffer in {\cf open()}, providing an
  implementation of {\cf write_tile()} that copies the tile of data to the
  right spots in the buffer, and having {\cf close()} then call 
  {\cf write_scanlines} to process the buffer now that the image has been
  fully sent.

  Here is how the class definition looks for our JPEG example.  Note
  that the JPEG/JFIF file format does not support multiple subimages
  or tiled images.

  \begin{code}
    class JpgOutput final : public ImageOutput {
     public:
        JpgOutput () { init(); }
        virtual ~JpgOutput () { close(); }
        virtual const char * format_name (void) const override { return "jpeg"; }
        virtual int supports (string_view property) const override { return false; }
        virtual bool open (const std::string &name, const ImageSpec &spec,
                           bool append=false) override;
        virtual bool write_scanline (int y, int z, TypeDesc format,
                                     const void *data, stride_t xstride) override;
        bool close ();
     private:
        FILE *m_fd;
        std::vector<unsigned char> m_scratch;
        struct jpeg_compress_struct m_cinfo;
        struct jpeg_error_mgr m_jerr;

        void init () { m_fd = NULL; }
    };
  \end{code}
\end{enumerate}

Your subclass implementation of {\cf open()}, {\cf close()}, and {\cf
  write_scanline()} are the heart of an \ImageOutput implementation.
(Also {\cf write_tile()}, for those image formats that support tiled
output.)

An \ImageOutput implementation must properly handle all data formats and
strides passed to {\cf write_scanline()} or {\cf write_tile()}, unlike
an \ImageInput implementation, which only needs to read scanlines or
tiles in their native format and then have the super-class handle the
translation.  But don't worry, all the heavy lifting can be accomplished
with the following helper functions provided as protected member
functions of \ImageOutput that convert a scanline, tile, or rectangular
array of values from one format to the native format(s) of the file.

\apiitem{const void * {\ce to_native_scanline} (TypeDesc format, const void *data, \\
                \bigspc stride_t xstride, std::vector<unsigned char> \&scratch, \\
                                    \bigspc unsigned int dither=0,
                                    int yorigin=0, int zorigin=0)}

Convert a full scanline of pixels (pointed to by \emph{data}) with the
given \emph{format} and strides into contiguous pixels in the native
format (described by the \ImageSpec returned by the {\cf spec()} member
function).  The location of the newly converted data is returned, which
may either be the original \emph{data} itself if no data conversion was
necessary and the requested layout was contiguous (thereby avoiding
unnecessary memory copies), or may point into memory allocated within
the \emph{scratch} vector passed by the user.  In either case, the
caller doesn't need to worry about thread safety or freeing any
allocated memory (other than eventually destroying the scratch vector).
\apiend

\apiitem{const void * {\ce to_native_tile} (TypeDesc format, const void *data,\\
                            \bigspc stride_t xstride, stride_t ystride, stride_t zstride,\\
                            \bigspc std::vector<unsigned char> \&scratch,
                            unsigned int dither=0, \\ \bigspc int xorigin=0,
                             int yorigin=0, int zorigin=0)}

Convert a full tile of pixels (pointed to by \emph{data}) with the given
\emph{format} and strides into contiguous pixels in the native format
(described by the \ImageSpec returned by the {\cf spec()} member
function).  The location of the newly converted data is returned, which
may either be the original \emph{data} itself if no data conversion was
necessary and the requested layout was contiguous (thereby avoiding
unnecessary memory copies), or may point into memory allocated within
the \emph{scratch} vector passed by the user.  In either case, the
caller doesn't need to worry about thread safety or freeing any
allocated memory (other than eventually destroying the scratch vector).

\apiend

\apiitem{const void * {\ce to_native_rectangle} (int xbegin, int xend, \\
                                    \bigspc int ybegin, int yend,
                                     int zbegin, int zend, \\ \bigspc
                                     TypeDesc format, const void
                                     *data, \\ \bigspc
                                     stride_t xstride, stride_t ystride,
                                     stride_t zstride, \\ \bigspc
                                     std::vector<unsigned char> \&scratch,
                            unsigned int dither=0, \\ \bigspc int xorigin=0,
                             int yorigin=0, int zorigin=0)}

Convert a rectangle of pixels (pointed to by \emph{data}) with the given
\emph{format}, dimensions, and strides into contiguous pixels in the
native format (described by the \ImageSpec returned by the {\cf spec()}
member function).  The location of the newly converted data is returned,
which may either be the original \emph{data} itself if no data
conversion was necessary and the requested layout was contiguous
(thereby avoiding unnecessary memory copies), or may point into memory
allocated within the \emph{scratch} vector passed by the user.  In
either case, the caller doesn't need to worry about thread safety or
freeing any allocated memory (other than eventually destroying the
scratch vector).

\apiend

For {\cf float} to 8 bit integer conversions only, if {\cf dither} parameter
is nonzero, random dither will be added to reduce quantization banding
artifacts; in this case, the specific nonzero {\cf dither} value is used as
a seed for the hash function that produces the per-pixel dither amounts, and
the optional {\cf origin} parameters help it to align the pixels to the
right position in the dither pattern.


\bigskip
\bigskip

\noindent
The remainder of this section simply lists the full implementation of
our JPEG writer, which relies heavily on the open source {\fn jpeg-6b}
library to perform the actual JPEG encoding.

\tinyincludedcode{../jpeg.imageio/jpegoutput.cpp}


\section{Tips and Conventions}
\label{sec:plugintipsconventions}

\product's main goal is to hide all the pesky details of individual file
formats from the client application.  This inevitably leads to various
mismatches between a file format's true capabilities and requests that
may be made through the \product APIs.  This section outlines
conventions, tips, and rules of thumb that we recommend for image file
support.

\subsection*{Readers}
\begin{itemize}
\item If the file format stores images in a non-spectral color space
  (for example, YUV), the reader should automatically convert to RGB to
  pass through the OIIO APIs.  In such a case, the reader should signal
  the file's true color space via a \qkw{Foo:colorspace} attribute in
  the \ImageSpec.
\item ``Palette'' images should be automatically converted by the reader
  to RGB.
\item If the file supports thumbnail images in its header, the reader
  should store the thumbnail dimensions in attributes
  \qkw{thumbnail_width}, \qkw{thumbnail_height}, and
  \qkw{thumbnail_nchannels} (all of which should be {\cf int}), and the
  thumbnail pixels themselves in \qkw{thumbnail_image} as an array of
  channel values (the array length is the total number of channel
  samples in the thumbnail).
\end{itemize}

\subsection*{Writers}

The overall rule of thumb is: try to always ``succeed'' at writing the
file, outputting the closest approximation of the user's data as
possible.  But it is permissible to fail the {\cf open()} call if it is
clearly nonsensical or there is no possible way to output a decent
approximation of the user's data.  Some tips:

\begin{itemize}
\item If the client application requests a data format not directly
  supported by the file type, silently write the supported data format
  that will result in the least precision or range loss.
\item It is customary to fail a call to {\cf open()} if the \ImageSpec
  requested a number of color channels plainly not supported by the
  file format.  As an exception to this rule, it is permissible for a
  file format that does not support alpha channels to silently drop
  the fourth (alpha) channel of a 4-channel output request.
\item If the app requests a \qkw{Compression} not supported by the file
  format, you may choose as a default any lossless compression
  supported.  Do not use a lossy compression unless you are fairly 
  certain that the app wanted a lossy compression.
\item If the file format is able to store images in a non-spectral color
  space (for example, YUV), the writer may accept a \qkw{Foo:colorspace}
  attribute in the \ImageSpec as a request to automatically convert and
  store the data in that format (but it will always be passed as RGB
  through the OIIO APIs).
\item If the file format can support thumbnail images in its header, and
  the \ImageSpec contain attributes \qkw{thumbnail_width},
  \qkw{thumbnail_height}, \qkw{thumbnail_nchannels}, and
  \qkw{thumbnail_image}, the writer should attempt to store the
  thumbnail if possible.
\end{itemize}



\section{Building ImageIO Plugins}
\label{sec:buildingplugins}

FIXME -- spell out how to compile and link plugins on each of the major
platforms.


\chapwidthend
